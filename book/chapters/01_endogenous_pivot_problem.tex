% ══════════════════════════════════════════════════════════════
%  Chapter 1 — The Endogenous Pivot Problem
% ══════════════════════════════════════════════════════════════
\chapter{The Endogenous Pivot Problem}\label{ch:motivation}

Every structured-generation system faces a quiet question that its
validity metrics never ask: \emph{is the output correct, or merely
well-formed?}  This chapter introduces the question through a concrete
failure, generalises the failure to a pattern, and then distils the
pattern into toy examples small enough to hold in one's head.  No
formalism is required yet---\cref{ch:formal-problem} will supply the definitions.
The goal here is intuition: the reader should finish this chapter
convinced that the problem is real, that it is not specific to any one
domain, and that standard validity checks are structurally incapable of
detecting it.

%% ═══════════════════════════════════════════════════════════════════
\section{Diana's Collapsed Arc}\label{sec:diana}
%% ═══════════════════════════════════════════════════════════════════

Consider a narrative simulation---a dinner party with six autonomous
agents, each pursuing its own goals, alliances, and evasions.  The
system, called Lorien, runs a large-language-model--driven simulation and
then extracts, for each agent, a \emph{narrative arc}: a selected
subsequence of events annotated with phase labels that obey a beat
grammar.  The grammar requires four phases in strict order:

\begin{center}
\textsc{setup}
\;\;$\longrightarrow$\;\;
\textsc{development}
\;\;$\longrightarrow$\;\;
\textsc{turning\_point}
\;\;$\longrightarrow$\;\;
\textsc{resolution}.
\end{center}

\noindent
The system selects a \emph{turning point}---the single highest-weight
focal event for the agent---and then verifies that the selected sequence
satisfies the beat grammar with at least $k$ development beats before
the turning point.  When the grammar is satisfied the arc is declared
valid.  When it is not, the arc is discarded.

Diana is a peripheral observer---an evader type.  She is not the
protagonist; she reacts to events more than she initiates them.  Across
50 random seeds of the same dinner-party scenario, the system extracts
arcs for all six agents.  For five of them, extraction succeeds
reliably.  For Diana, it fails in 9 out of 50 seeds.  All nine failures
share exactly the same signature:

\begin{enumerate}[label=(\roman*)]
  \item \textbf{Zero development beats.}\;
    The \textsc{development} phase is completely empty.  Not one
    complication, escalation, or tension-building event appears in the
    selected arc.

  \item \textbf{Extremely early turning point.}\;
    The turning point is anchored at a median normalised position of
    $0.13$ in the timeline---barely past the opening.  For comparison,
    the median position in valid arcs is $0.69$.  There is zero overlap
    in turning-point position between valid and invalid arcs.

  \item \textbf{Full candidate pools.}\;
    Diana is not starved of events.  Across the 50~seeds, a mean of
    $42.7$ events involve her.  The 9~invalid seeds contain a total of
    33 candidate events, every single one of which produces zero
    development beats.  The pools are entirely degenerate.
\end{enumerate}

\noindent
The mechanism is straightforward once we trace the pipeline.  Diana is a
peripheral agent, so her raw event pool is sparse.  To compensate, the
system injects protagonist events into her candidate pool---a
protagonist-event injection mechanism that ensures peripheral agents have
enough material to form arcs.  The injected events include high-tension
incidents from early in the simulation: confrontations, revelations,
catastrophes that drove the main plot.  These events carry large
narrative weights.

The arc-extraction algorithm performs a greedy search: it selects the
turning point as the highest-weight focal event in Diana's pool.
Because the injected protagonist events are both high-weight and
temporally early, the greedy search locks onto an early catastrophe as
the pivot.  The beat grammar is monotonic---once the turning point is
consumed, the \textsc{development} phase closes permanently.  With the
turning point at position~$0.13$, there is almost no timeline left
before it.  The grammar requires at least $k$ development beats before
the turning point, and with the pivot so early, this requirement is
impossible to meet.

The result is a ``story'' that reads: catastrophe happens immediately,
followed by sixteen consequence events.  No buildup, no complication, no
dramatic tension.  The grammar's structural requirements are technically
checkable---and in this case, the check correctly reports
failure---but the deeper problem is not that the grammar rejected the
arc.  The deeper problem is the \emph{mechanism}: the pivot selection
was determined by properties of the solution itself (which events happen
to be in Diana's pool), and the pivot's position then determined how
every other event was interpreted.  A different pool composition would
have produced a different pivot, a different phase labelling, and a
different arc---possibly a valid and meaningful one.

We will formalise this coupling in \cref{ch:formal-problem}.  For now, the
essential observation is: the system produced technically structured
output, but the output was meaningless.  The greedy search locked onto
the wrong pivot, and the grammar---designed to ensure narrative
quality---became the instrument of narrative collapse.

%% ═══════════════════════════════════════════════════════════════════
\section{The General Pattern}\label{sec:general-pattern}
%% ═══════════════════════════════════════════════════════════════════

Diana's collapsed arc is not a narrative-specific bug.  It is an
instance of a general vulnerability that arises whenever three
conditions hold simultaneously:

\begin{enumerate}[label=(\alph*),itemsep=4pt]
  \item \textbf{Endogenous pivot selection.}\;
    A distinguished element---a pivot, reference point, root cause,
    anchor---is selected by taking an $\argmax$ (or similar extremal
    operation) over the solution itself.

  \item \textbf{Pivot-dependent interpretation.}\;
    The chosen pivot determines how all other elements in the solution
    are interpreted: their phase labels, their causal roles, their
    structural classifications.

  \item \textbf{Structural constraints.}\;
    The interpreted solution must satisfy a set of structural
    constraints: a grammar, a schema, a protocol, a well-formedness
    condition.
\end{enumerate}

\noindent
When all three conditions hold, the pivot selection is
\emph{endogenous}---it depends on the content of the solution, which
creates a circular dependency.  The pivot determines the
interpretation, the interpretation determines whether the constraints
are satisfied, and the constraints determine what counts as a valid
solution.  If anything changes the solution---compression, truncation,
sampling, filtering---the pivot can shift, the interpretation can
flip, and the constraints can transition from satisfied to violated
(or vice versa) without any local signal that something has gone wrong.

The following three examples illustrate the pattern in domains far
removed from narrative.

%% ─────────────────────────────────────────────────────────────
\subsection{Incident Triage and Root-Cause Locking}
\label{sec:incident-triage}
%% ─────────────────────────────────────────────────────────────

Consider an automated incident-report system that analyses a production
outage involving 50 contributing factors.  The system selects a
\emph{root cause} by identifying the highest-severity factor.  All
remaining factors are then classified relative to the root cause:
\emph{precursor} (happened before and contributed to the root cause),
\emph{contributing factor} (amplified the root cause's impact), or
\emph{consequence} (resulted from the root cause).  The resulting causal
narrative must satisfy a report schema: at least two precursors, exactly
one root cause, and at least one consequence.

Now suppose the incident log is compressed---perhaps a context window is
truncated, or a summarisation step drops low-severity factors.  If the
compression removes context that would have surfaced a higher-severity
factor, the root cause locks onto a different factor.  The entire causal
narrative reorganises: events that were precursors become consequences,
consequences become precursors, and the report tells a structurally valid
but substantively wrong story.  The schema is satisfied.  No constraint
violation is raised.  The report is wrong.

%% ─────────────────────────────────────────────────────────────
\subsection{Constrained LLM Decoding and Schema Commitment}
\label{sec:constrained-decoding}
%% ─────────────────────────────────────────────────────────────

Grammar-constrained decoding systems---PICARD~\citep{scholak2021picard} for SQL, Outlines~\citep{willard2023outlines} for
structured JSON, guidance masks for schema-valid generation---commit to a
schema path early in the generation process.  This commitment is
functionally equivalent to selecting a structural pivot: once the model
has emitted enough tokens to determine which schema branch it is
following, all subsequent tokens are interpreted within that branch.

If the context window is truncated and the model loses information that
would have led to a different schema choice, it generates valid JSON (or
valid SQL, or valid YAML) with the wrong structure.  The output parses.
The schema validates.  The downstream system consumes it without error.
But the semantic content has silently shifted, because the structural
pivot---the schema branch---was selected endogenously from whatever
context happened to survive truncation.

%% ─────────────────────────────────────────────────────────────
\subsection{Process Mining and the Reference Activity}
\label{sec:process-mining}
%% ─────────────────────────────────────────────────────────────

In process mining~\citep{vanderaalst2016process}, analysts discover a process model from an event log
by selecting a \emph{reference activity} that anchors the temporal
alignment of all other activities.  If the reference activity is selected
endogenously---for instance, as the most frequent activity in the
log---then changing the log changes the reference.  Sampling the log
(taking a 10\% subsample for scalability), filtering by time window, or
removing infrequent traces can each shift the most-frequent activity to a
different event type.  When the reference shifts, every other activity is
realigned relative to the new anchor, and the discovered process model
changes---not because the underlying process changed, but because the
pivot moved.

\bigskip
\noindent
In each of these examples, the same three-part structure is at work:
endogenous selection of a pivot, pivot-dependent interpretation of
everything else, and structural constraints that can be satisfied by
multiple incompatible interpretations.  The output is valid.  The output
is wrong.  And no validity check can detect the discrepancy.

%% ═══════════════════════════════════════════════════════════════════
\section{Two Minimal Toy Examples}\label{sec:toy-examples}
%% ═══════════════════════════════════════════════════════════════════

To make the mechanics fully explicit, we now construct two toy examples
small enough to verify by hand.  Both use the narrative-arc grammar from
\cref{sec:diana}, but the phenomena they exhibit are instances of the
general pattern from \cref{sec:general-pattern}.

We use the following notation throughout.  An event $e_i$ has a weight
$w_i$ and a type: \emph{focal} (eligible to serve as the turning point)
or \emph{non-focal} (eligible for development, setup, or resolution
beats).  The turning point $\tp(S)$ is the focal event with the highest
weight in the solution~$S$.  The prefix requirement is $k$: at least $k$
non-focal events must precede the turning point.

%% ─────────────────────────────────────────────────────────────
\subsection{Example 1: Pivot Flip}\label{sec:pivot-flip}
%% ─────────────────────────────────────────────────────────────

Consider a six-event sequence with prefix requirement $k = 3$:

\medskip
\begin{center}
\begin{tabular}{lcccccc}
  \toprule
  Event   & $e_1$ & $e_2$ & $e_3$ & $e_4$ & $e_5$ & $e_6$ \\
  \midrule
  Weight  & 2     & 5     & 3     & 4     & 1     & 2     \\
  Type    & non-focal & focal & non-focal & non-focal & non-focal
          & non-focal \\
  \bottomrule
\end{tabular}
\end{center}
\medskip

\noindent
The only focal event is $e_2$, so the turning point is forced:
$\tp(S) = e_2$ with $\wstar = 5$.  The non-focal events preceding the
turning point are $\{e_1\}$, giving $\dpre = 1$.  Since
$\dpre = 1 < k = 3$, the arc is \textbf{invalid}.

Now extend the sequence by appending a single focal event:

\medskip
\begin{center}
\begin{tabular}{lccccccc}
  \toprule
  Event   & $e_1$ & $e_2$ & $e_3$ & $e_4$ & $e_5$ & $e_6$ & $e_7$ \\
  \midrule
  Weight  & 2     & 5     & 3     & 4     & 1     & 2     & 6 \\
  Type    & non-focal & focal & non-focal & non-focal & non-focal
          & non-focal & focal \\
  \bottomrule
\end{tabular}
\end{center}
\medskip

\noindent
Now there are two focal events: $e_2$ (weight~5) and $e_7$ (weight~6).
The turning point shifts to $e_7$: $\tp(S') = e_7$ with $\wstar = 6$.
The non-focal events preceding $e_7$ are $\{e_1, e_3, e_4, e_5, e_6\}$,
giving $\dpre = 5 \ge k = 3$.  The arc is \textbf{valid}.

Adding a single event flipped the turning point from $e_2$ to $e_7$ and
cascaded all phase labels from invalid to valid.  The five non-focal
events that were always present in the sequence---$e_3$ through
$e_6$---were invisible to the development phase when the pivot was
early, and fully available when the pivot moved late.  The events did not
change.  The grammar did not change.  Only the pivot changed, and the
entire structural interpretation followed.

%% ─────────────────────────────────────────────────────────────
\subsection{Example 2: Compression Mirage}\label{sec:compression-mirage}
%% ─────────────────────────────────────────────────────────────

Consider a ten-event sequence with $k = 3$:

\medskip
\begin{center}
\begin{tabular}{lcccccccccc}
  \toprule
  Event & $e_1$ & $e_2$ & $e_3$ & $e_4$ & $e_5$ & $e_6$ & $e_7$
        & $e_8$ & $e_9$ & $e_{10}$ \\
  \midrule
  Weight & 2 & 3 & 6 & 2 & 4 & 1 & 3 & 10 & 2 & 1 \\
  Type & nf & nf & f & nf & f & nf & nf & f & nf & nf \\
  \bottomrule
\end{tabular}
\end{center}
\medskip

\noindent
(Here ``f'' denotes focal and ``nf'' denotes non-focal.)

The highest-weight focal event is $e_8$ (weight~10), at normalised
position $0.8$ in the sequence.  The non-focal events preceding $e_8$
are $\{e_1, e_2, e_4, e_6, e_7\}$, giving $\dpre = 5 \ge k = 3$.  The
arc is valid.  Suppose the quality score---a weighted combination of
pivot weight and development richness---evaluates to $47.08$.

Now compress the sequence by removing two non-focal events from the
middle: drop $e_4$ and $e_6$.  The compressed sequence has eight events.
Under \emph{committed semantics}---the policy that keeps the original
pivot---$e_8$ remains the turning point.  But now the non-focal events
preceding $e_8$ are $\{e_1, e_2, e_7\}$, giving $\dpre = 3$.  In fact,
if the removal shifts indices and the grammar-aware count recalculates
to $\dpre = 2 < k = 3$, the committed pivot is \textbf{infeasible}: no
valid arc can be formed around $e_8$ in the compressed sequence.

An enumerative solver (searching up to $M = 10$ alternative pivot
candidates) looks for substitutes.  It considers $e_3$ (weight~6, at
position $0.3$): but with $e_3$ as pivot, $\dpre = 1$---still below
$k$.  It then considers $e_5$ (weight~4, at position $0.5$): with $e_5$
as pivot, $\dpre = 3 \ge k$.  Valid.  The solver accepts $e_5$ as a
substitute pivot.

But the quality score under the substitute pivot is only $21.47$.  The
\emph{semantic regret}---the fraction of quality lost by pivot
substitution---is
\[
  1 - \frac{21.47}{47.08} \;=\; 54.4\%.
\]
The system reports ``valid output.''  The grammar is satisfied.  The
phase labels are well-formed.  And the result has silently lost more than
half its semantic quality.  This is the \emph{validity mirage}: the
output looks valid because it \emph{is} valid, but the validity
conceals a catastrophic semantic shift.  The pivot moved, the story
changed, and the only metric that could detect the problem---pivot
preservation---was never checked.

%% ═══════════════════════════════════════════════════════════════════
\section{A Diagnostic Checklist}\label{sec:motivation-checklist}
%% ═══════════════════════════════════════════════════════════════════

\Cref{ch:mirage} develops a full diagnostic framework for detecting
and quantifying the validity mirage.  We preview four metrics here so
that the reader has a concrete target as the theory develops.

\begin{enumerate}[label=\textbf{(\arabic*)},itemsep=6pt]
  \item \textbf{Pivot preservation rate.}\;
    Did the compressed, streamed, or otherwise transformed solution
    retain the same turning point as the full (uncompressed,
    untruncated) solution?  Let $\tp(S)$ denote the pivot of the full
    solution and $\tp(S')$ the pivot of the transformed solution.
    Pivot preservation is the indicator
    $\mathbf{1}[\tp(S) = \tp(S')]$, aggregated as a rate across
    instances.  A rate of $1.0$ means the transformation never
    disturbed the pivot.  A rate below $1.0$ means pivots are shifting,
    and downstream interpretations are changing.

  \item \textbf{Fixed-pivot feasibility.}\;
    If we \emph{force} the original pivot $\tp(S)$ into the
    transformed solution~$S'$, is the output still valid?  That is,
    does there exist a valid phase labelling of~$S'$ with the
    turning point fixed at $\tp(S)$?  If not, the transformation has
    made the original semantic anchor infeasible---the system cannot
    even attempt to tell the same story.

  \item \textbf{Semantic regret.}\;
    When the pivot shifts from $\tp(S)$ to a substitute
    $\tp(S') \neq \tp(S)$, how much quality is lost?  Semantic regret
    is defined as
    \[
      R \;=\; 1 \;-\; \frac{Q\bigl(S',\, \tp(S')\bigr)}
                            {Q\bigl(S,\, \tp(S)\bigr)},
    \]
    where $Q$ is the quality scoring function.  A regret of $0$ means
    no quality loss; a regret of $0.544$ means $54.4\%$ of the
    original quality has been silently discarded.

  \item \textbf{Mirage gap.}\;
    The difference between raw validity and pivot preservation:
    \[
      \Delta_{\mathrm{mirage}}
      \;=\;
      \text{(raw validity rate)}
      \;-\;
      \text{(pivot preservation rate)}.
    \]
    A mirage gap of zero means validity and semantic fidelity are
    aligned---every valid output preserved the original pivot.  A
    large mirage gap means the system is hiding semantic drift behind
    superficial well-formedness.  In the headline experiment of
    \cref{ch:mirage}, raw validity holds steady at $1.0$ while pivot
    preservation drops to ${\sim}0.33$, producing a mirage gap of approximately $0.65$.
    The system appears to be working perfectly.  It is not.
\end{enumerate}

\bigskip
\noindent
The rest of this book develops the mathematical tools to analyse,
predict, and prevent these failures.  \Cref{ch:formal-problem} formalises the
solution space, the pivot function, and committed semantics.
\Cref{ch:absorbing-states} proves that certain structural states are
absorbing---once entered, no continuation can repair them.
\Cref{ch:context-algebra} introduces the context monoid that makes these
states algebraically tractable.  And \cref{ch:mirage} closes the loop
by demonstrating, on real and synthetic data, that the mirage is not a
theoretical curiosity but a measured, quantified, and reproducible
phenomenon.
