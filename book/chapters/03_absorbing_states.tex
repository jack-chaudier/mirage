% ══════════════════════════════════════════════════════════════
%  Chapter 3 — Absorbing States in Greedy Search
% ══════════════════════════════════════════════════════════════
\chapter{Absorbing States in Greedy Search}\label{ch:absorbing-states}

The greedy policy introduced in \cref{ch:formal-problem} selects a turning point by
taking the $\argmax$ over focal-actor weights and then fills the remaining
narrative phases around it.  This chapter shows that a simple counting
argument---comparing the number of development-eligible events against the
grammar's prefix requirement---yields an exact impossibility boundary.
Below this boundary, the greedy policy produces \emph{zero} valid
sequences; the product of the event graph with the phase grammar enters an
absorbing state from which no continuation can reach acceptance.

We proceed in stages.  \Cref{sec:dev-eligible} formalises the
development-eligible set and its count~$\jdev$.
\Cref{sec:phase-grammar-dfa} specifies the phase grammar as a
deterministic finite automaton and proves its monotonicity lemma.
\Cref{sec:product-automaton} constructs the product automaton that couples
the greedy walk with the grammar.
\Cref{sec:prefix-impossibility} states and proves the central result,
\cref{thm:prefix-impossibility}, in five self-contained steps.
\Cref{sec:absorbing-remarks} collects important remarks on the theorem's
scope.  \Cref{sec:empirical-absorbing} presents the empirical evidence,
and \cref{sec:exercises-absorbing} offers exercises.

% ──────────────────────────────────────────────────────────────
\section{Development-Eligible Events}\label{sec:dev-eligible}
% ──────────────────────────────────────────────────────────────

Throughout, let $G = (V, E, a, t, w)$ be an event graph on vertex set~$V$,
where $a(v)$ is the actor of event~$v$, $t(v)$ its timestamp, and $w(v)$
its narrative weight.  Fix a focal actor~$a^{*}$ and write
\[
  e^{*}
  \;=\;
  \argmax_{v \,:\, a(v) = a^{*}} w(v)
\]
for the \emph{forced turning point}---the event selected by the greedy
policy as the structural pivot.  Let $k \ge 1$ denote the grammar's
\emph{prefix requirement}: the minimum number of \textsc{development}
symbols that must be consumed before \textsc{turning\_point}.

\begin{definition}[Development-eligible set]\label{def:dev-eligible}
  Given the event graph~$G$, turning point~$e^{*}$, and prefix
  requirement~$k$, the \emph{candidate pool} is
  $P = V \setminus \{e^{*}\}$.  The \emph{development-eligible set} is
  \[
    D
    \;=\;
    \bigl\{\,
      v \in P
      \;\bigm|\;
      t(v) < t(e^{*})
      \;\text{and $v$ can receive the \textsc{development} label
              under the phase grammar}
    \,\bigr\}.
  \]
  The \emph{development-eligible count} is $\jdev = |D|$.
\end{definition}

The two conditions in the definition deserve emphasis.
Condition~(i), $t(v) < t(e^{*})$, restricts attention to events that
precede the turning point in the timeline; events after the turning point
cannot contribute \textsc{development} symbols under the monotonic grammar
(\cref{lem:dfa-monotonicity}).  Condition~(ii) encodes the
classifier--policy interaction: whether a given event \emph{can} receive
the \textsc{development} label depends on the classifier in use.
(The exclusion $v \neq e^{*}$ is already captured by the requirement
$v \in P = V \setminus \{e^{*}\}$.)

\begin{remark}[Classifier dependence of~$\jdev$]\label{rem:classifier-dep}
  The count~$\jdev$ depends on the classifier--policy interaction.
  Under a \emph{position-based classifier}, any event before the turning
  point in the timeline is eligible for \textsc{development}, so
  $\jdev = |\{v \in P : t(v) < t(e^{*})\}|$.  Under a
  \emph{grammar-aware classifier}, only events that the DFA can label as
  \textsc{development}---for instance, events belonging to non-focal
  actors---are eligible, and $\jdev$ may be strictly smaller.  This
  distinction is the source of Class~B failures analysed in
  \cref{ch:taxonomy}.
\end{remark}

\begin{example}[Position-based vs.\ grammar-aware counting]%
\label{ex:jdev-comparison}
  Consider a sequence of $n = 10$ events in which the turning
  point~$e^{*}$ occupies position~7 (in temporal order).  Among the six
  events preceding~$e^{*}$, two belong to the focal actor~$a^{*}$ and
  four belong to other actors.

  \begin{enumerate}[label=(\roman*)]
    \item \textbf{Position-based classifier.}
      Every event before the turning point is eligible:
      $\jdev = 6$.

    \item \textbf{Grammar-aware classifier}
      (requiring \textsc{development} events to be non-focal).
      Only the four non-focal events qualify:
      $\jdev = 4$.
  \end{enumerate}

  \noindent
  If the prefix requirement is $k = 5$, the position-based classifier
  gives $\jdev = 6 \ge k$ (no impossibility predicted), while the
  grammar-aware classifier gives $\jdev = 4 < k$
  (\cref{thm:prefix-impossibility} applies: greedy validity is exactly
  zero).
\end{example}

% ──────────────────────────────────────────────────────────────
\section{The Phase Grammar DFA}\label{sec:phase-grammar-dfa}
% ──────────────────────────────────────────────────────────────

We model the narrative phase grammar as a deterministic finite automaton
$\mathcal{A} = (Q,\, \Sigma,\, \delta,\, q_0,\, F)$ defined as follows.

\begin{definition}[Phase grammar DFA]\label{def:phase-dfa}
  \leavevmode
  \begin{itemize}[itemsep=4pt]
    \item \textbf{States.}\quad
      $Q = \{S_{\textsc{setup}},\;
             S_{\textsc{dev}},\;
             S_{\textsc{tp}},\;
             S_{\textsc{res}},\;
             S_{\textsc{reject}}\}$.

    \item \textbf{Alphabet.}\quad
      $\Sigma = \{\textsc{setup},\;
                   \textsc{development},\;
                   \textsc{turning\_point},\;
                   \textsc{resolution}\}$.

    \item \textbf{Initial state.}\quad $q_0 = S_{\textsc{setup}}$.

    \item \textbf{Accepting states.}\quad $F = \{S_{\textsc{res}}\}$.

    \item \textbf{Transition function~$\delta$.}\quad
      The transitions are \emph{monotonic}: phase labels must appear in
      non-decreasing order
      $\textsc{setup} \prec \textsc{development} \prec
       \textsc{turning\_point} \prec \textsc{resolution}$.
      Formally, the non-reject transitions are:

      \medskip
      \begin{center}
      \begin{tabular}{lll}
        \toprule
        \textbf{Current state} & \textbf{Symbol} & \textbf{Next state} \\
        \midrule
        $S_{\textsc{setup}}$  & \textsc{setup}          & $S_{\textsc{setup}}$ \\
        $S_{\textsc{setup}}$  & \textsc{development}    & $S_{\textsc{dev}}$   \\
        $S_{\textsc{setup}}$  & \textsc{turning\_point} & $S_{\textsc{tp}}$    \\
        $S_{\textsc{dev}}$    & \textsc{development}    & $S_{\textsc{dev}}$   \\
        $S_{\textsc{dev}}$    & \textsc{turning\_point} & $S_{\textsc{tp}}$    \\
        $S_{\textsc{tp}}$     & \textsc{resolution}     & $S_{\textsc{res}}$   \\
        $S_{\textsc{res}}$    & \textsc{resolution}     & $S_{\textsc{res}}$   \\
        \bottomrule
      \end{tabular}
      \end{center}
      \medskip

      \noindent
      Every $(q, \sigma)$ pair not listed above transitions to
      $S_{\textsc{reject}}$.  The state~$S_{\textsc{reject}}$ is
      \emph{absorbing}: $\delta(S_{\textsc{reject}}, \sigma)
      = S_{\textsc{reject}}$ for every $\sigma \in \Sigma$.

    \item \textbf{Prefix requirement.}\quad
      The automaton must consume at least $k$ copies of the symbol
      \textsc{development} before consuming \textsc{turning\_point} for
      the run to be valid.  Equivalently, one may augment the state space
      with a counter $c \in \{0, 1, \ldots, k\}$ and allow the transition
      $S_{\textsc{dev}} \xrightarrow{\textsc{turning\_point}} S_{\textsc{tp}}$
      only when $c \ge k$.  For readability, we keep the counter implicit
      and treat the prefix requirement as a side condition on acceptance.
  \end{itemize}
\end{definition}

The key structural property of this DFA is that phases advance
monotonically and can never retreat.  The following lemma isolates the
consequence that matters most for the impossibility theorem.

\begin{lemma}[DFA Monotonicity]\label{lem:dfa-monotonicity}
  Any state reached after consuming \textsc{turning\_point} is absorbing
  with respect to \textsc{development} transitions.  Formally, if
  \[
    \delta^{*}(q_0,\; \sigma_1 \cdots \sigma_m)
    \;\in\;
    \{S_{\textsc{tp}},\; S_{\textsc{res}}\},
  \]
  then for every continuation $\sigma_{m+1} \cdots \sigma_n$ with
  $\sigma_j = \textsc{development}$ for some $j \in \{m{+}1, \ldots, n\}$,
  we have
  \[
    \delta^{*}(q_0,\; \sigma_1 \cdots \sigma_n)
    \;=\;
    S_{\textsc{reject}}.
  \]
\end{lemma}

\begin{proof}
  We argue by inspection of the transition table in
  \cref{def:phase-dfa}.

  \emph{Case 1: the DFA is in $S_{\textsc{tp}}$.}\;
  The only symbol that leads to a non-reject state is \textsc{resolution},
  which transitions to~$S_{\textsc{res}}$.  Consuming \textsc{development}
  in $S_{\textsc{tp}}$ yields $S_{\textsc{reject}}$.

  \emph{Case 2: the DFA is in $S_{\textsc{res}}$.}\;
  The only symbol that keeps the DFA in a non-reject state is
  \textsc{resolution} (self-loop on $S_{\textsc{res}}$).
  Consuming \textsc{development} in $S_{\textsc{res}}$ yields
  $S_{\textsc{reject}}$.

  \emph{Case 3: the DFA is in $S_{\textsc{reject}}$.}\;
  By definition, $S_{\textsc{reject}}$ is absorbing:
  $\delta(S_{\textsc{reject}}, \sigma) = S_{\textsc{reject}}$ for all
  $\sigma \in \Sigma$.

  \medskip\noindent
  In every case, once \textsc{turning\_point} has been consumed, the DFA
  is in $S_{\textsc{tp}}$, $S_{\textsc{res}}$, or $S_{\textsc{reject}}$.
  From any of these states, consuming \textsc{development} leads
  (immediately or eventually) to~$S_{\textsc{reject}}$, from which
  acceptance is impossible.  Therefore the DFA never returns
  to~$S_{\textsc{dev}}$ after processing \textsc{turning\_point}.
\end{proof}

% ──────────────────────────────────────────────────────────────
\section{The Product Automaton}\label{sec:product-automaton}
% ──────────────────────────────────────────────────────────────

The impossibility proof couples the greedy policy's event-selection order
with the grammar DFA.  The vehicle for this coupling is a \emph{product
automaton}~$\mathcal{P}$.

\begin{definition}[Product automaton]\label{def:product-automaton}
  Let $G = (V, E, a, t, w)$ be an event graph with $|V| = n$, and let
  $\mathcal{A} = (Q, \Sigma, \delta, q_0, F)$ be the phase grammar DFA
  (\cref{def:phase-dfa}).  Linearise the event graph by the greedy
  policy's selection order: events are processed in decreasing weight
  order, $w(v_1) \ge w(v_2) \ge \cdots \ge w(v_n)$, with ties broken by
  a fixed deterministic rule (e.g., earliest timestamp first).

  The \emph{product automaton} is
  $\mathcal{P} = G \times \mathcal{A}$, with:
  \begin{itemize}[itemsep=3pt]
    \item \textbf{State space.}\;
      Pairs $(i,\, q)$ where $i \in \{0, 1, \ldots, n\}$ indexes the
      current position in the greedy selection order and
      $q \in Q$ is the DFA state.

    \item \textbf{Initial state.}\;
      $(0,\, q_0) = (0,\, S_{\textsc{setup}})$.

    \item \textbf{Transitions.}\;
      At position~$i$, the greedy policy selects event~$v_{i+1}$ and
      assigns it the ``best'' phase label~$\sigma_{i+1} \in \Sigma$ that
      keeps the DFA on a path toward acceptance.  The product automaton
      transitions from $(i, q)$ to
      $(i{+}1,\; \delta(q, \sigma_{i+1}))$.

    \item \textbf{Acceptance.}\;
      The product automaton accepts if and only if it reaches a state
      $(n, q_f)$ with $q_f \in F = \{S_{\textsc{res}}\}$ and the prefix
      requirement (at least $k$ \textsc{development} symbols before
      \textsc{turning\_point}) is satisfied.
  \end{itemize}
\end{definition}

Note that the greedy walk selects events in weight order, but phase
labels are assigned relative to the turning point's timestamp, which is
fixed at selection time.  An event classified as \textsc{development}
during greedy selection always has timestamp earlier than the turning
point's timestamp, by the event-ordering convention of
\cref{def:event-ordering}.

The greedy policy imposes one hard constraint on the walk
through~$\mathcal{P}$:

\begin{quote}
  \textbf{Turning-point pre-selection.}\;
  The event $e^{*} = \argmax_{v : a(v)=a^{*}} w(v)$ is pre-selected as
  the turning point.  When the walk reaches the position~$i^{*}$ at
  which $e^{*}$ appears in the weight-sorted order, the label must be
  $\sigma_{i^{*}} = \textsc{turning\_point}$.
\end{quote}

Between positions~$0$ and~$i^{*} - 1$, the greedy policy assigns each
event the best available label---typically \textsc{setup} or
\textsc{development}---that advances the DFA toward the prefix
requirement.  At position~$i^{*}$, the DFA consumes
\textsc{turning\_point} and transitions to~$S_{\textsc{tp}}$.  From
position~$i^{*} + 1$ onward, the only productive label is
\textsc{resolution}, by \cref{lem:dfa-monotonicity}.

The product automaton makes the proof's state-space argument explicit:
every run of the greedy policy corresponds to a unique path
through~$\mathcal{P}$, and the question of whether the greedy policy can
produce a valid sequence reduces to whether there exists an accepting path
in~$\mathcal{P}$.

% ──────────────────────────────────────────────────────────────
\section{The Prefix-Constraint Impossibility Theorem}%
\label{sec:prefix-impossibility}
% ──────────────────────────────────────────────────────────────

We are now in a position to state and prove the chapter's main result.

\begin{theorem}[Prefix-Constraint Impossibility]%
\label{thm:prefix-impossibility}
  Let $G$ be an event graph, $e^{*}$ the forced turning point, $k$ the
  prefix requirement, and $\jdev$ the development-eligible count
  (\cref{def:dev-eligible}).  If
  \[
    \jdev \;<\; k,
  \]
  then the greedy policy produces zero valid sequences.
\end{theorem}

\begin{proof}
  The proof proceeds in five steps.

  \medskip
  \noindent\textbf{Step 1 (Product automaton).}\;
  Construct the product automaton $\mathcal{P} = G \times \mathcal{A}$ as
  in \cref{def:product-automaton}.  Every run of the greedy policy traces
  a unique path through the state space
  $\{0, \ldots, n\} \times Q$.  The greedy policy produces a valid
  sequence if and only if this path ends in an accepting state~$(n, q_f)$
  with $q_f \in F$ and the prefix requirement satisfied.

  \medskip
  \noindent\textbf{Step 2 (Turning-point pre-selection).}\;
  The greedy policy pre-selects
  $e^{*} = \argmax_{v : a(v)=a^{*}} w(v)$
  as the turning point.  Let $i^{*}$ be the position of~$e^{*}$ in the
  weight-sorted selection order.  At step~$i^{*}$, the policy assigns the
  label $\sigma_{i^{*}} = \textsc{turning\_point}$, and the DFA
  transitions to~$S_{\textsc{tp}}$:
  \[
    \delta(q_{i^{*}-1},\; \textsc{turning\_point})
    \;=\;
    S_{\textsc{tp}},
  \]
  where $q_{i^{*}-1}$ is the DFA state upon entering step~$i^{*}$
  (necessarily $S_{\textsc{setup}}$ or $S_{\textsc{dev}}$ if the run has
  not already reached $S_{\textsc{reject}}$).

  \medskip
  \noindent\textbf{Step 3 (Post-TP absorption).}\;
  By the DFA Monotonicity Lemma (\cref{lem:dfa-monotonicity}), once the
  DFA enters $S_{\textsc{tp}}$, it can never return to~$S_{\textsc{dev}}$.
  Any subsequent \textsc{development} symbol sends the DFA
  to~$S_{\textsc{reject}}$, from which acceptance is impossible.
  Therefore, \emph{all} \textsc{development} symbols must be consumed
  \emph{before} step~$i^{*}$.

  \medskip
  \noindent\textbf{Step 4 (Counting development slots).}\;
  The events available to receive \textsc{development} labels before
  step~$i^{*}$ are exactly the members of the development-eligible
  set~$D$ from \cref{def:dev-eligible}.  Each such event can contribute at
  most one \textsc{development} symbol to the run.  Therefore, the maximum
  number of \textsc{development} symbols consumed before
  \textsc{turning\_point} is
  \[
    \#\{\textsc{development}\text{ before } \textsc{turning\_point}\}
    \;\le\;
    |D|
    \;=\;
    \jdev.
  \]

  \medskip
  \noindent\textbf{Step 5 (Prefix violation).}\;
  The grammar requires at least $k$ \textsc{development} symbols before
  \textsc{turning\_point} for the run to satisfy the prefix requirement.
  By Step~4, at most $\jdev$ such symbols can appear.  If $\jdev < k$,
  then
  \[
    \#\{\textsc{development}\text{ before } \textsc{turning\_point}\}
    \;\le\;
    \jdev
    \;<\;
    k.
  \]
  The prefix requirement is therefore violated on \emph{every} path
  through~$\mathcal{P}$.  No path can reach an accepting state with a
  satisfied prefix requirement.  Equivalently, the product automaton
  enters an absorbing region---no matter how the remaining events are
  labelled, acceptance is unreachable.

  \medskip\noindent
  Since every run of the greedy policy corresponds to a path
  through~$\mathcal{P}$, and no such path is accepting when $\jdev < k$,
  the greedy policy produces zero valid sequences.
\end{proof}

% ──────────────────────────────────────────────────────────────
\section{Important Remarks}\label{sec:absorbing-remarks}
% ──────────────────────────────────────────────────────────────

\begin{remark}[Converse is false]\label{rem:converse-false}
  The condition $\jdev \ge k$ does \emph{not} guarantee that the greedy
  policy succeeds.  Having enough development-eligible events is
  \emph{necessary} for validity but far from \emph{sufficient}.  The
  greedy policy may still fail for independent reasons: it may mis-order
  events so that the DFA enters $S_{\textsc{reject}}$ through a label
  conflict (Class~C failure), or it may select a turning point that is
  semantically incoherent even though the grammar is satisfied (Class~D
  failure).  Both of these failure modes are analysed in
  \cref{ch:taxonomy}.

  \Cref{thm:prefix-impossibility} therefore provides a
  \emph{sufficient condition for failure}, not a necessary one.  The
  failure boundary $\jdev = k$ is sharp in one direction only: below it,
  validity is exactly zero; above it, validity is possible but not
  guaranteed.
\end{remark}

\begin{remark}[Focal-only counting produces false positives]%
\label{rem:focal-only}
  A common error is to count only focal-actor events (those with
  $a(v) = a^{*}$) as development-eligible, ignoring events from other
  actors.  This under-counts the development-eligible set: if an event
  $v$ with $a(v) \neq a^{*}$ can receive the \textsc{development} label,
  it belongs in~$D$.

  Focal-only counting yields $\jdev^{\text{focal}} \le \jdev$, which may
  trigger a false positive prediction of impossibility: the focal-only
  count may satisfy $\jdev^{\text{focal}} < k$ even when $\jdev \ge k$.
  The correct count uses \emph{all} events eligible for
  \textsc{development}, regardless of actor.  This point is validated by
  the test assertions in the repository's test suite, which confirm that
  multi-actor events contribute to the development-eligible pool.
\end{remark}

\begin{remark}[Bridge to $\dpre$]\label{rem:bridge-dpre}
  In the \emph{unit-mass regime}---where each event contributes exactly
  one unit of development content---the development-eligible count
  coincides with the pre-pivot development mass:
  $\jdev = \dpre$.  The absorbing-state condition $\jdev < k$ then
  becomes $\dpre < k$, which is precisely the \emph{absorbing predicate}
  $\absorb$ from the context algebra developed in \cref{ch:absorbing-ideal}.

  This bridge connects the combinatorial argument of the present chapter
  with the algebraic framework: the absorbing ideal of
  \cref{ch:absorbing-ideal} is the algebraic closure of the impossibility region
  identified here.  When event masses are non-uniform, $\jdev$ and
  $\dpre$ may diverge, and the algebraic formulation
  (via~$\dpre$) provides the more refined characterisation.
\end{remark}

% ──────────────────────────────────────────────────────────────
\section{Empirical Validation}\label{sec:empirical-absorbing}
% ──────────────────────────────────────────────────────────────

The theoretical boundary $\jdev = k$ admits clean empirical verification.
We summarise the main findings~\citep{gaffney2026absorbing}; full experimental details appear in the
repository.

\paragraph{The boundary heatmap.}
\Cref{fig:kj-heatmap} displays the greedy policy's validity rate as a
function of the prefix requirement~$k$ (horizontal axis) and the
development-eligible count~$\jdev$ (vertical axis), aggregated over
$11{,}400$ event-graph instances.  The $k$--$\jdev$ plane divides into
three clearly delineated regions:

\begin{enumerate}[label=(\roman*)]
  \item \textbf{Impossibility zone} ($\jdev < k$, below the diagonal).\;
    Greedy validity is exactly $0.000$ across all $11{,}400$ instances in
    this region, confirming \cref{thm:prefix-impossibility} without
    exception.

  \item \textbf{Stochastic zone}
    ($k \le \jdev \lesssim 5.31k + 12.23$).\;
    Validity rises stochastically from~$0$ toward~$1.0$.  In this band,
    having enough development-eligible events is necessary but not
    sufficient; other failure modes (Classes~C and~D) depress the success
    rate.

  \item \textbf{High-validity zone}
    ($\jdev \gtrsim 5.31k + 12.23$).\;
    The 95\% success envelope lies approximately along the line
    $\jdev \approx 5.31k + 12.23$.  Above this envelope, the greedy
    policy almost always succeeds.
\end{enumerate}

\begin{figure}[t]
  \centering
  \includegraphics[width=0.75\textwidth]{figures/kj_heatmap.pdf}
  \caption{%
    Greedy validity rate in the $k$--$\jdev$ plane
    ($11{,}400$~instances).  The diagonal $\jdev = k$ (dashed white line)
    is the impossibility boundary of
    \cref{thm:prefix-impossibility}: every cell below the diagonal has
    validity exactly~$0.000$.  The solid white curve marks the 95\%
    success envelope, approximately $\jdev \approx 5.31k + 12.23$.
    Three regions emerge: the \emph{impossibility zone} (below
    diagonal), the \emph{stochastic zone} (between diagonal and
    envelope), and the \emph{high-validity zone} (above envelope).%
  }\label{fig:kj-heatmap}
\end{figure}

\paragraph{Exogenous turning-point control experiment.}
A natural question is whether the impossibility barrier is an artifact of
the grammar alone or whether it depends on the greedy policy's specific
method of turning-point selection.  To test this, we repeat the experiment
with an \emph{exogenous} turning point: instead of selecting
$e^{*} = \argmax_{v : a(v)=a^{*}} w(v)$, we fix the turning point at the median-timestamp focal event (rather than the max-weight focal event).

Under exogenous assignment, the diagonal barrier persists
\emph{structurally}---the proof of \cref{thm:prefix-impossibility} does
not depend on \emph{how} $e^{*}$ is chosen, only on the relationship
between $\jdev$ and~$k$ once $e^{*}$ is fixed.  However, the
\emph{probability} that a randomly generated instance falls below the
diagonal changes dramatically:
$P(\jdev < k)$ drops from $13.6\%$ under the greedy (argmax) policy to
$0.0\%$ under exogenous assignment.

The absorbing state is therefore activated by the \emph{interaction} of
weight-based turning-point selection with temporal front-loading of high-weight
events---not by the grammar alone.  When the turning point is the
highest-weight focal event, it tends to sit early in the weight-sorted
order, leaving few events before it in the timeline.  The grammar's prefix
requirement then becomes unsatisfiable.  Exogenous assignment breaks this
coupling, eliminating the pathway into the impossibility zone.

% ──────────────────────────────────────────────────────────────
\section{Exercises}\label{sec:exercises-absorbing}
% ──────────────────────────────────────────────────────────────

\begin{exercise}\label{exer:minimal-graph}
  Construct a minimal event graph with $n = 8$ events where
  \cref{thm:prefix-impossibility} predicts failure (i.e., $\jdev < k$
  for $k = 3$) and verify by exhaustive enumeration that no valid
  sequence exists under the greedy policy.

  \emph{Hint.}\;  Choose weights so that $e^{*}$ (the maximum-weight
  focal event) is at position~2 in the timeline, leaving at most two
  events before it.  Then $\jdev \le 2 < 3 = k$.  To verify, enumerate
  all possible label assignments to the eight events and confirm that
  every assignment either violates the phase grammar's monotonicity, fails
  the prefix requirement, or both.
\end{exercise}

\begin{exercise}\label{exer:relax-k}
  Show that relaxing the prefix requirement from $k = 3$ to $k = 1$
  changes the location of the impossibility boundary but not the structure
  of the proof.
  \begin{enumerate}[label=(\alph*)]
    \item State the new impossibility condition.
    \item Identify which step of the proof changes and which steps remain
          identical.
    \item Give an example of an event graph that is impossible under
          $k = 3$ but valid under $k = 1$.
  \end{enumerate}

  \emph{Solution sketch.}\;
  The new impossibility condition is $\jdev < 1$, i.e., $\jdev = 0$:
  there are no development-eligible events before the turning point.
  Steps~1--4 of the proof are identical; only Step~5 changes, replacing
  the bound $\jdev < 3$ with $\jdev < 1$.  Any event graph with
  $1 \le \jdev < 3$ is impossible under $k = 3$ but potentially valid
  under $k = 1$.
\end{exercise}

\begin{exercise}\label{exer:product-size}
  Prove that the product automaton~$\mathcal{P}$ from
  \cref{def:product-automaton} has at most $|V| \times |Q|$ states,
  giving an $O(n)$ bound on the state-space size (since $|Q| = 5$ is
  constant).

  \emph{Solution.}\;
  The state space of~$\mathcal{P}$ is
  $\{0, 1, \ldots, n\} \times Q$, which has $(n+1) \times |Q|$
  elements.  Since $|Q| = 5$ is a fixed constant independent of the
  event graph,
  \[
    |\text{States of } \mathcal{P}|
    \;=\;
    (n+1) \cdot 5
    \;=\;
    5n + 5
    \;\in\;
    O(n).
  \]
  Each step of the walk through~$\mathcal{P}$ processes one event and
  performs a constant-time DFA transition, so the entire walk takes
  $O(n)$ time.  The proof of \cref{thm:prefix-impossibility} thus
  operates on a linear-size state space, and the counting argument in
  Steps~4--5 requires no search---only a comparison between $\jdev$ and
  $k$, both computable in $O(n)$ time by a single pass over the event
  graph. \qed
\end{exercise}

\begin{exercise}\label{exer:non-monotonic}
  Consider a \emph{modified} phase grammar in which \textsc{development}
  events may appear after \textsc{turning\_point}.  Specifically, add the
  transitions
  \[
    \delta(S_{\textsc{tp}}, \textsc{development}) = S_{\textsc{tp}}
    \quad\text{and}\quad
    \delta(S_{\textsc{res}}, \textsc{development}) = S_{\textsc{res}}
  \]
  to the DFA of \cref{def:phase-dfa}.
  \begin{enumerate}[label=(\alph*)]
    \item Does \cref{thm:prefix-impossibility} still hold under this
          modified grammar?
    \item Identify the exact step in the proof that breaks and explain
          why.
    \item Give a concrete event graph where $\jdev < k$ under the
          original grammar's definition of~$D$, but the modified grammar
          admits a valid sequence.
  \end{enumerate}

  \emph{Solution sketch.}\;
  \begin{enumerate}[label=(\alph*)]
    \item No. \Cref{thm:prefix-impossibility} does \emph{not} hold under
          the modified grammar.

    \item Step~3 of the proof relies on the DFA Monotonicity Lemma
          (\cref{lem:dfa-monotonicity}), which states that no
          \textsc{development} symbol can be consumed after
          \textsc{turning\_point}.  The modified grammar explicitly
          violates this property: \textsc{development} is accepted in
          both $S_{\textsc{tp}}$ and $S_{\textsc{res}}$.  With
          monotonicity broken, \textsc{development} symbols can be
          consumed \emph{after} the turning point, so the count of
          development slots is no longer bounded by the pre-TP
          count~$\jdev$.  Step~4's upper bound becomes invalid, and
          Step~5's conclusion no longer follows.

    \item Consider an event graph with $n = 6$, $k = 3$,
          $e^{*}$ at position~2 in the timeline (so $\jdev = 1 < 3$).
          Under the original grammar, the greedy policy cannot produce a
          valid sequence.  Under the modified grammar, one can place one
          \textsc{development} event before $e^{*}$ and two
          \textsc{development} events after it, satisfying the prefix
          requirement of $k = 3$ total \textsc{development} symbols
          (now allowed both before and after the turning point).
          \qed
  \end{enumerate}
\end{exercise}
