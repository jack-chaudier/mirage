\chapter{Context Algebra}\label{ch:context-algebra}

This chapter develops the algebraic foundation on which the rest of the
theory is built~\citep{gaffney2026mirage}.  We introduce the \emph{context element}---a compact
summary of the structural state of a block of events---and define an
associative composition operator that merges adjacent blocks while
faithfully tracking the position of the global pivot.  The resulting
monoid is the engine that powers both the streaming left-fold
implementation and the parallel holographic tree.

%% ═══════════════════════════════════════════════════════════════════
\section{The Context Element}\label{sec:context-element}
%% ═══════════════════════════════════════════════════════════════════

Every contiguous block of events can be compressed into a triple that
records three structural quantities: where the strongest focal event
sits, how much development capacity the block contains in total, and how
much of that capacity falls before the strongest focal event.  These are
the only three numbers required to compose blocks correctly.

\begin{definition}[Context element]\label{def:context-element}
A \textbf{context element} is a triple
\[
  C \;=\; \bigl(\,\wstar,\; \dtotal,\; \dpre\,\bigr),
\]
where the components are defined as follows.
\begin{enumerate}[label=(\roman*)]
  \item $\wstar \in \R \cup \{-\infty\}$ is the \textbf{local pivot
    key}: the maximum weight among all focal-actor events in the block.
    If the block contains no focal events, $\wstar = -\infty$.
  \item $\dtotal \in \N$ is the \textbf{total development capacity}: the
    count of non-focal events in the block.  Each non-focal event
    contributes one unit of development capacity regardless of its
    weight or timestamp.
  \item $\dpre \in \N$ is the \textbf{pre-pivot development capacity}:
    the count of non-focal events whose timestamp strictly precedes the
    timestamp of the local pivot (the focal event attaining $\wstar$).
    When $\wstar = -\infty$ (no focal events), $\dpre$ is defined to be
    zero.
\end{enumerate}
\end{definition}

The intuition behind each component is direct:

\begin{itemize}
  \item \textbf{Pivot key $\wstar$.}\; The pivot is the focal event that
    the endogenous selection mechanism has chosen---the strongest signal
    in the block.  Its weight $\wstar$ determines whether this block's
    pivot can survive composition with adjacent blocks: if a neighbour
    contains a focal event with a strictly larger weight, the global
    pivot will shift to that neighbour.

  \item \textbf{Total development $\dtotal$.}\; The total count of
    non-focal events measures the raw amount of development capacity
    that the block contributes.  Under composition this quantity simply
    adds, since every non-focal event remains non-focal regardless of
    where the pivot falls.

  \item \textbf{Pre-pivot development $\dpre$.}\; This is the
    structurally subtle component.  The feasibility condition for the
    endogenous pivot (typically requiring at least $k$ non-focal events
    before the pivot) depends on how many non-focal events appear
    \emph{before} the global pivot in the combined sequence.  The
    quantity $\dpre$ records this count relative to the block's own
    local pivot, but as we shall see in \cref{sec:endogenous-composition},
    composition may promote additional events into the pre-pivot
    region when the global pivot shifts rightward.
\end{itemize}

\begin{example}[Concrete context element]\label{ex:context-element}
Consider a block of seven events arranged in temporal order:
\[
  \underbrace{e_1}_{\text{non-focal}},\;\;
  \underbrace{e_2}_{\substack{\text{focal}\\w=3}},\;\;
  \underbrace{e_3}_{\text{non-focal}},\;\;
  \underbrace{e_4}_{\text{non-focal}},\;\;
  \underbrace{e_5}_{\substack{\text{focal}\\w=7}},\;\;
  \underbrace{e_6}_{\text{non-focal}},\;\;
  \underbrace{e_7}_{\text{non-focal}}.
\]
The focal events are $e_2$ (weight~3) and $e_5$ (weight~7).  The local
pivot is $e_5$ since $7 > 3$, giving $\wstar = 7$.

The non-focal events are $e_1, e_3, e_4, e_6, e_7$, so
$\dtotal = 5$.

Among those five non-focal events, the ones with timestamps strictly
preceding $e_5$ are $e_1, e_3, e_4$---three events---giving
$\dpre = 3$.

Therefore the context element for this block is
\[
  C = (7,\; 5,\; 3).
\]
\end{example}

%% ═══════════════════════════════════════════════════════════════════
\section{Endogenous Composition}\label{sec:endogenous-composition}
%% ═══════════════════════════════════════════════════════════════════

We now define the binary operation that composes two adjacent context
elements---the left block $C_A$ followed in time by the right block
$C_B$---into a single context element representing the concatenated
sequence.

\begin{definition}[Endogenous composition $\opendo$]
\label{def:endogenous-composition}
Let $C_A = (\wstar_A,\; \dtotal_A,\; \dpre[{,A}])$ and
$C_B = (\wstar_B,\; \dtotal_B,\; \dpre[{,B}])$ be context elements.
Their \textbf{endogenous composition} is
\[
  C_A \opendo C_B
  \;=\;
  \bigl(\,\wstar,\;\dtotal,\;\dpre\,\bigr),
\]
where
\begin{align}
  \wstar   &= \max\!\bigl(\wstar_A,\; \wstar_B\bigr),
  \label{eq:comp-wstar}\\[4pt]
  \dtotal  &= \dtotal_A + \dtotal_B,
  \label{eq:comp-dtotal}\\[4pt]
  \dpre    &=
  \begin{cases}
    \dpre[{,A}]                   & \text{if } \wstar_A \ge \wstar_B, \\[3pt]
    \dtotal_A + \dpre[{,B}]       & \text{if } \wstar_B > \wstar_A.
  \end{cases}
  \label{eq:comp-dpre}
\end{align}
\end{definition}

The rules for $\wstar$ and $\dtotal$ are immediate: the global pivot key
is the maximum over both blocks, and total development capacity is
additive.  The rule for $\dpre$ requires more careful thought.

\begin{remark}[The pivot-shift insight]\label{rem:pivot-shift}
The $\dpre$ rule in \cref{eq:comp-dpre} encodes the following critical
structural insight.  There are two regimes:

\begin{enumerate}
  \item \textbf{Pivot stays in the left block}
    ($\wstar_A \ge \wstar_B$).\;
    The global pivot is the same event that was the local pivot of block
    $A$.  The non-focal events before the global pivot are exactly those
    that were before $A$'s local pivot.  Nothing in block $B$ can
    contribute to the pre-pivot count because the entire right block
    sits \emph{after} the pivot.  Hence $\dpre = \dpre[{,A}]$.

  \item \textbf{Pivot shifts to the right block}
    ($\wstar_B > \wstar_A$).\;
    The global pivot now lives inside block $B$.  Every event in block
    $A$---focal or non-focal---has a timestamp that precedes the global
    pivot.  In particular, \emph{all} $\dtotal_A$ non-focal events in
    block $A$ are now in the pre-pivot region.  Within block $B$ itself,
    the events before $B$'s local pivot are counted by $\dpre[{,B}]$.
    Therefore the total pre-pivot development count is
    $\dtotal_A + \dpre[{,B}]$.
\end{enumerate}
This asymmetry---the left block's \emph{entire} development capacity
gets promoted upon a rightward pivot shift, but the right block
contributes nothing additional when the pivot stays left---is the source
of non-commutativity (see \cref{sec:non-commutativity}).
\end{remark}

\begin{example}[Pivot shifts right]\label{ex:comp-right-shift}
Let $A = (3, 5, 2)$ and $B = (7, 4, 1)$.  Since $\wstar_B = 7 >
\wstar_A = 3$, the pivot shifts to block $B$.  Applying
\cref{def:endogenous-composition}:
\begin{align*}
  \wstar   &= \max(3, 7) = 7, \\
  \dtotal  &= 5 + 4 = 9, \\
  \dpre    &= \dtotal_A + \dpre[{,B}] = 5 + 1 = 6.
\end{align*}
Therefore $A \opendo B = (7, 9, 6)$.

The interpretation: all five of $A$'s non-focal events now sit before
the global pivot (which is inside $B$), contributing $5$ to the
pre-pivot count.  Within $B$, only one non-focal event precedes $B$'s
local pivot, contributing~$1$.  The total pre-pivot development capacity
is~$6$.
\end{example}

\begin{example}[Pivot stays left]\label{ex:comp-left-stays}
Now let $A = (7, 4, 1)$ and $B = (3, 5, 2)$.  Since $\wstar_A = 7 \ge
\wstar_B = 3$, the pivot stays in block $A$:
\begin{align*}
  \wstar   &= \max(7, 3) = 7, \\
  \dtotal  &= 4 + 5 = 9, \\
  \dpre    &= \dpre[{,A}] = 1.
\end{align*}
Therefore $A \opendo B = (7, 9, 1)$.

Here the global pivot remains in $A$, where only one non-focal event
preceded the local pivot.  The entire right block $B$ lies after the
pivot, so none of $B$'s events contribute to $\dpre$.
\end{example}

%% ═══════════════════════════════════════════════════════════════════
\section{Associativity}\label{sec:associativity}
%% ═══════════════════════════════════════════════════════════════════

The central algebraic property of $\opendo$ is associativity: we may
parenthesise a three-way composition in either order and obtain the same
result.  This is what allows us to decompose a long event sequence into
arbitrary contiguous blocks, compose each block independently, and then
combine them---the hallmark of a parallel-friendly reduction.

\begin{proposition}[Associativity of $\opendo$]\label{prop:associativity}
For any context elements $A$, $B$, and $C$,
\[
  (A \opendo B) \opendo C
  \;=\;
  A \opendo (B \opendo C).
\]
\end{proposition}

\begin{proof}
Write $A = (\wstar_A, d_A, p_A)$, $B = (\wstar_B, d_B, p_B)$, and
$C = (\wstar_C, d_C, p_C)$, where we abbreviate $\dtotal$ and $\dpre$
as $d$ and $p$ for readability.

The $\wstar$ component of both sides equals
$\max(\wstar_A, \wstar_B, \wstar_C)$ by associativity of $\max$.  The
$\dtotal$ component of both sides equals $d_A + d_B + d_C$ by
associativity of addition.  It remains to verify the $\dpre$ component.

We proceed by exhaustive case analysis on which block contains the
global pivot.

\bigskip
\noindent\textbf{Case~1: Global pivot in $A$}
($\wstar_A \ge \wstar_B$ and $\wstar_A \ge \wstar_C$).

\smallskip
\emph{Left-associated: $(A \opendo B) \opendo C$.}\;
Let $AB = A \opendo B$.  Since $\wstar_A \ge \wstar_B$, we have
$AB = (\wstar_A,\; d_A + d_B,\; p_A)$.
Then $AB \opendo C$: since $\wstar_A \ge \wstar_C$, the result is
$(\wstar_A,\; d_A + d_B + d_C,\; p_A)$.

\smallskip
\emph{Right-associated: $A \opendo (B \opendo C)$.}\;
Let $BC = B \opendo C$.  Two subcases arise:

\begin{enumerate}[label=(\alph*)]
  \item $\wstar_B \ge \wstar_C$:\; $BC = (\wstar_B,\; d_B + d_C,\; p_B)$.
    Then $A \opendo BC$: since $\wstar_A \ge \wstar_B$, the result is
    $(\wstar_A,\; d_A + d_B + d_C,\; p_A)$. \checkmark

  \item $\wstar_C > \wstar_B$:\; $BC = (\wstar_C,\; d_B + d_C,\;
    d_B + p_C)$.
    Then $A \opendo BC$: since $\wstar_A \ge \wstar_C$, the result is
    $(\wstar_A,\; d_A + d_B + d_C,\; p_A)$. \checkmark
\end{enumerate}
In both subcases the $\dpre$ component is $p_A$, matching the
left-associated result.

\bigskip
\noindent\textbf{Case~2: Global pivot in $B$}
($\wstar_B > \wstar_A$ and $\wstar_B \ge \wstar_C$).

\smallskip
\emph{Left-associated: $(A \opendo B) \opendo C$.}\;
Since $\wstar_B > \wstar_A$,
$AB = (\wstar_B,\; d_A + d_B,\; d_A + p_B)$.
Then $AB \opendo C$: since $\wstar_B \ge \wstar_C$, the result is
$(\wstar_B,\; d_A + d_B + d_C,\; d_A + p_B)$.

\smallskip
\emph{Right-associated: $A \opendo (B \opendo C)$.}\;
Since $\wstar_B \ge \wstar_C$,
$BC = (\wstar_B,\; d_B + d_C,\; p_B)$.
Then $A \opendo BC$: since $\wstar_B > \wstar_A$, the result is
$(\wstar_B,\; d_A + d_B + d_C,\; d_A + p_B)$. \checkmark

\medskip
\noindent\textbf{Key insight (why Case~2 is subtle).}\;
The quantity $d_A$ appears in $\dpre$ on both sides because it gets
``promoted'' into the pre-pivot region exactly once---when the global
pivot moves from the left sub-expression into block $B$.  In the
left-associated computation, this promotion happens during the
$A \opendo B$ step.  In the right-associated computation, it happens
during the $A \opendo BC$ step.  Either way, $d_A$ is promoted exactly
once and combined with $p_B$, yielding the same final
$\dpre = d_A + p_B$.  The single-promotion invariant is what makes
associativity non-obvious here, and it is the reason the operator is
\emph{not} commutative (see \cref{sec:non-commutativity}): the
direction of the pivot shift determines whether promotion occurs.

\bigskip
\noindent\textbf{Case~3: Global pivot in $C$}
($\wstar_C > \wstar_A$ and $\wstar_C > \wstar_B$).

\smallskip
\emph{Left-associated: $(A \opendo B) \opendo C$.}\;
First form $AB = A \opendo B$.  Regardless of whether $\wstar_A \ge
\wstar_B$ or $\wstar_B > \wstar_A$, the total development of $AB$ is
$d_{AB} = d_A + d_B$.
Then $AB \opendo C$: since $\wstar_C > \wstar_{AB}$, the pivot shifts
right, giving
\[
  \dpre = d_{AB} + p_C = d_A + d_B + p_C.
\]
The result is $(\wstar_C,\; d_A + d_B + d_C,\; d_A + d_B + p_C)$.

\smallskip
\emph{Right-associated: $A \opendo (B \opendo C)$.}\;
Since $\wstar_C > \wstar_B$,
$BC = (\wstar_C,\; d_B + d_C,\; d_B + p_C)$.
Then $A \opendo BC$: since $\wstar_C > \wstar_A$, the pivot shifts
right, giving
\[
  \dpre = d_A + (d_B + p_C) = d_A + d_B + p_C.
\]
The result is $(\wstar_C,\; d_A + d_B + d_C,\; d_A + d_B + p_C)$.
\checkmark

\medskip
All three cases yield identical results under both parenthesisations.
Since every possible ordering of $\wstar_A, \wstar_B, \wstar_C$ falls
into exactly one of these cases (with ties handled by the
$\ge$\,/\,$>$ structure), the proof is complete.
\end{proof}

We consolidate the result with a numerical verification of the subtle
Case~2.

\begin{example}[Numerical verification of Case~2]\label{ex:assoc-case2}
Let $A = (3, 5, 2)$, $B = (7, 4, 1)$, $C = (2, 3, 1)$.  The global
pivot is in $B$ since $\wstar_B = 7 > \wstar_A = 3$ and
$\wstar_B = 7 \ge \wstar_C = 2$.

\smallskip
\emph{Left-associated.}\;
$AB = A \opendo B$: since $7 > 3$, we get
$AB = (7,\; 5+4,\; 5+1) = (7, 9, 6)$.
Then $AB \opendo C$: since $7 \ge 2$, the result is
$(7,\; 9+3,\; 6) = (7, 12, 6)$.

\smallskip
\emph{Right-associated.}\;
$BC = B \opendo C$: since $7 \ge 2$, we get
$BC = (7,\; 4+3,\; 1) = (7, 7, 1)$.
Then $A \opendo BC$: since $7 > 3$, the result is
$(7,\; 5+7,\; 5+1) = (7, 12, 6)$.

\smallskip
Both sides yield $(7, 12, 6)$, confirming associativity. \qed
\end{example}

%% ═══════════════════════════════════════════════════════════════════
\section{Identity Element}\label{sec:identity}
%% ═══════════════════════════════════════════════════════════════════

\begin{proposition}[Identity element for $\opendo$]
\label{prop:identity}
The element
\[
  e \;=\; (-\infty,\; 0,\; 0)
\]
is a two-sided identity for $\opendo$: for every context element $C$,
\[
  C \opendo e = C
  \qquad\text{and}\qquad
  e \opendo C = C.
\]
\end{proposition}

\begin{proof}
Let $C = (\wstar, d, p)$ be an arbitrary context element.

\medskip
\noindent\textbf{Right identity: $C \opendo e = C$.}\;
Applying \cref{def:endogenous-composition} with $C_A = C$ and
$C_B = e = (-\infty, 0, 0)$:
\begin{align*}
  \wstar'  &= \max(\wstar,\; -\infty) = \wstar, \\
  \dtotal' &= d + 0 = d, \\
  \dpre'   &= p
    \quad\text{(since $\wstar \ge -\infty$, the first case of
    \cref{eq:comp-dpre} applies)}.
\end{align*}
Therefore $C \opendo e = (\wstar, d, p) = C$. \checkmark

\medskip
\noindent\textbf{Left identity: $e \opendo C = C$.}\;
Applying \cref{def:endogenous-composition} with $C_A = e = (-\infty, 0, 0)$
and $C_B = C$:
\begin{align*}
  \wstar'  &= \max(-\infty,\; \wstar) = \wstar, \\
  \dtotal' &= 0 + d = d.
\end{align*}
For $\dpre'$, we consider two subcases.  If $\wstar = -\infty$, then
$\wstar_A = \wstar_B = -\infty$, so $\wstar_A \ge \wstar_B$ and
$\dpre' = \dpre[{,A}] = 0 = p$ (since $p = 0$ when $\wstar = -\infty$
by \cref{def:context-element}).  If $\wstar > -\infty$, then
$\wstar_B = \wstar > -\infty = \wstar_A$, so the second case of
\cref{eq:comp-dpre} applies:
\[
  \dpre' = \dtotal_A + \dpre[{,B}] = 0 + p = p.
\]
In either subcase, $\dpre' = p$.  Therefore $e \opendo C = (\wstar, d, p) = C$. \checkmark
\end{proof}

\begin{theorem}[Context monoid]\label{thm:context-monoid}
The set of context elements equipped with endogenous composition and
identity element $e$ forms a monoid:
\[
  \bigl(\,\mathcal{C},\;\opendo,\;e\,\bigr)
  \quad\text{is a monoid.}
\]
That is, $\opendo$ is associative (\cref{prop:associativity}), and $e = (-\infty, 0, 0)$
is a two-sided identity (\cref{prop:identity}).
\end{theorem}

%% ═══════════════════════════════════════════════════════════════════
\section{Non-Commutativity}\label{sec:non-commutativity}
%% ═══════════════════════════════════════════════════════════════════

\begin{remark}[Non-commutativity of $\opendo$]\label{rem:non-commutativity}
The operator $\opendo$ is \textbf{not} commutative.  We have already
seen a concrete counterexample: in \cref{ex:comp-right-shift,ex:comp-left-stays},
the same two elements composed in opposite orders give different results.
Specifically, with $A = (3, 5, 2)$ and $B = (7, 4, 1)$:
\[
  A \opendo B = (7, 9, 6),
  \qquad
  B \opendo A = (7, 9, 1).
\]
These differ in their $\dpre$ components: $6 \neq 1$.

The reason is structural.  When $\wstar_B > \wstar_A$ and $B$ is to the
right of $A$, \emph{all} of $A$'s development capacity ($\dtotal_A = 5$)
gets promoted into the pre-pivot region, yielding $\dpre = 5 + 1 = 6$.
But when $B$ is to the \emph{left} of $A$ (i.e., we compute
$B \opendo A$), the pivot is already in the left block, so only $B$'s
own pre-pivot count matters: $\dpre = \dpre[{,B}] = 1$.

In other words, temporal ordering matters.  The left block's entire
development capacity is promoted when the pivot shifts rightward, but no
symmetric promotion occurs when the pivot remains in the left block.
This asymmetry is an inherent feature of the endogenous context
structure---it reflects the physical fact that events before the pivot
and events after the pivot play fundamentally different roles.
\end{remark}

%% ═══════════════════════════════════════════════════════════════════
\section{Systems Implications}\label{sec:systems-implications}
%% ═══════════════════════════════════════════════════════════════════

The algebraic properties established in this chapter have direct
consequences for the computational implementation.

\begin{remark}[Parallel tree reduction]\label{rem:parallel-tree}
Because $\opendo$ is strictly associative
(\cref{prop:associativity}), the context reduction over a sequence of
$n$ events can be parallelised as a balanced binary tree reduction with
$O(\log n)$ depth.  At each level of the tree, adjacent pairs of context
elements are composed independently, halving the number of elements.
After $\lceil \log_2 n \rceil$ levels, a single root element remains.

The implementation in
\texttt{src/holographic\_tree.py}\footnote{Specifically the
\texttt{HolographicContextTree} class, which maintains a binomial carry
forest of composed context elements.}\ realises this parallel structure.
Each call to \texttt{append()} triggers $O(\log n)$ compositions in the
worst case, maintaining the invariant that the forest encodes the full
context of all events seen so far.  The root summary can be queried at
any time via \texttt{get\_root\_summary()}.
\end{remark}

\begin{remark}[Streaming composition]\label{rem:streaming}
Associativity also enables a simple streaming protocol: process events
one at a time, composing each new singleton context element into a
running accumulator via a left fold.  The accumulator at any point
equals the context element for the entire event history observed so far.

The code path in \texttt{src/tropical\_semiring.py} implements this
strategy directly.  The function \texttt{build\_tropical\_context()}
performs a left fold over the event sequence, while the holographic tree
in \texttt{src/holographic\_tree.py} implements the parallel tree
variant.  Both produce identical results for any input sequence---a fact
verified empirically by \texttt{test\_05\_holographic\_exactness.py}.
The left fold has $O(n)$ sequential depth but requires only $O(1)$
working memory (a single accumulator); the tree has $O(\log n)$ depth
but requires $O(\log n)$ memory for the forest.  The choice between them
is a classical space--parallelism trade-off, and associativity guarantees
that the choice cannot affect correctness.
\end{remark}

%% ═══════════════════════════════════════════════════════════════════
\section{Exercises}\label{sec:context-algebra-exercises}
%% ═══════════════════════════════════════════════════════════════════

\begin{exercise}\label{exer:basic-composition}
Let $A = (10, 3, 1)$ and $B = (5, 6, 4)$.  Compute $A \opendo B$ and
$B \opendo A$.  Verify that the two results differ, and explain which
component differs and why.
\end{exercise}

\begin{exercise}\label{exer:tied-pivots}
Let $A = (5, 2, 1)$, $B = (5, 3, 2)$, and $C = (5, 4, 3)$, where all
three pivot keys are tied at $\wstar = 5$.  The tie-breaking rule in
\cref{def:endogenous-composition} is that the left block wins when
$\wstar_A \ge \wstar_B$ (i.e., the first case of \cref{eq:comp-dpre}
applies on ties).

\begin{enumerate}[label=(\alph*)]
  \item Compute $(A \opendo B) \opendo C$.  What is $\dpre$ in each
    intermediate and final result?
  \item Compute $A \opendo (B \opendo C)$.  Verify that the final
    result matches part~(a).
  \item Explain in words why tie-breaking in favour of the left block is
    essential for associativity.
\end{enumerate}
\end{exercise}

\begin{exercise}\label{exer:identity-power}
Let $C$ be any context element with $\wstar > -\infty$.  Prove that for
any $n \ge 1$,
\[
  C \opendo \underbrace{e \opendo e \opendo \cdots \opendo e}_{n \text{
  copies}} = C.
\]
\emph{Hint.}\; Use the identity property (\cref{prop:identity}) and
induction, or argue directly from the definition.
\end{exercise}

\begin{exercise}\label{exer:no-focal}
Consider a sequence of events in which \emph{every} event is non-focal
(there are no focal-actor events at all).  Let $C$ be the context
element obtained by composing the singleton elements for this sequence.

\begin{enumerate}[label=(\alph*)]
  \item What are $\wstar$, $\dtotal$, and $\dpre$ for the resulting
    context element $C$?
  \item A context element is called \emph{feasible} at threshold $k$ if
    $\dpre \ge k$.  Can the all-non-focal context element ever be
    feasible?  Explain.
  \item What does this say about the necessity of focal events for the
    endogenous pivot mechanism?
\end{enumerate}
\end{exercise}
