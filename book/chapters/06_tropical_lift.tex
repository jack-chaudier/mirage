\chapter{The Tropical Lift}\label{ch:tropical-lift}

The monoid developed in \cref{ch:context-algebra} compresses any
contiguous block of events into a triple $(\wstar, \dtotal, \dpre)$.
This triple tracks a single pivot---the strongest focal event---and
records how much development capacity sits before it.  The triple is
enough to test feasibility at a fixed threshold $k$ after the fact, but
it discards information: we learn the \emph{best} pivot's pre-pivot
count yet know nothing about the second-best, or about pivots that
fall just short of feasibility.

This chapter lifts the monoid into a richer representation---a
\emph{tropical context}---that replaces the scalar $\wstar$ with a
weight vector indexed by pre-pivot development count.  The resulting
structure is a faithful extension of the original monoid: it reproduces
exactly the same $\wstar$ and $\dpre$, while simultaneously answering
feasibility queries at every threshold from $0$ to $k$ in a single
pass.  The composition rule for weight vectors turns out to be a
simple shift-and-max operation---the tropical semiring structure from
which the chapter takes its name.\footnote{For background on tropical
semirings, see e.g.\ Maclagan and Sturmfels, \emph{Introduction to
Tropical Geometry}, AMS, 2015.}

%% ═══════════════════════════════════════════════════════════════════
\section{From Monoid to Weight Vectors}\label{sec:monoid-to-vectors}
%% ═══════════════════════════════════════════════════════════════════

Recall the context element $C = (\wstar, \dtotal, \dpre)$ from
\cref{def:context-element}.  The scalar $\wstar$ is the weight of the
best pivot in the block, and $\dpre$ tells us how many non-focal events
precede it.  If $\dpre \ge k$, the pivot is feasible.  But what if
$\dpre = k - 1$?  The monoid offers no recourse: the best pivot fell
one short, and we have no record of whether a slightly weaker pivot
might have $\dpre \ge k$.

The motivation for the tropical lift is exactly this: we want to know
the best pivot weight achievable with \emph{exactly} $j$ pre-pivot
development events, for every $j$ from $0$ to $k$.

\begin{definition}[Tropical context]\label{def:tropical-context}
A \textbf{tropical context} is a pair
\[
  T \;=\; (W,\; \dtotal),
\]
where
\begin{enumerate}[label=(\roman*)]
  \item $W \in (\R \cup \{-\infty\})^{k+1}$ is the \textbf{weight
    vector}.  The entry $W[j]$ for $j = 0, 1, \ldots, k$ is the
    maximum weight among all focal events in the block whose pre-pivot
    development count equals exactly $j$.  If no such focal event
    exists, $W[j] = -\infty$.

  \item $\dtotal \in \N$ is the \textbf{total development capacity}:
    the count of non-focal events in the block, exactly as in
    \cref{def:context-element}.
\end{enumerate}
The weight vector is indexed from $0$ (a pivot with no preceding
non-focal events) up to $k$ (a pivot with at least $k$ preceding
non-focal events, where counts beyond $k$ are capped at slot $k$).
\end{definition}

The connection to the original monoid is immediate.

\begin{remark}[Recovering the monoid]\label{rem:recovering-monoid}
Given a tropical context $T = (W, \dtotal)$, the monoid's best pivot
weight and its pre-pivot count are
\[
  \wstar = \max_{0 \le j \le k} W[j],
  \qquad
  \dpre = \min\!\Bigl(\argmax_{0 \le j \le k} W[j],\; k\Bigr).
\]
The tropical representation is strictly richer than the triple: it
records the best pivot at \emph{every} slot, not just the globally
best one.
\end{remark}

Feasibility also takes a clean form.

\begin{remark}[Feasibility]\label{rem:feasibility}
A tropical context $T$ is \textbf{feasible} at threshold $k$ if and
only if $W[k] > -\infty$.  That is, there exists at least one focal
event in the block with $k$ or more non-focal events preceding it.
\end{remark}

%% ═══════════════════════════════════════════════════════════════════
\section{Lifting Single Events}\label{sec:lifting-events}
%% ═══════════════════════════════════════════════════════════════════

Every individual event is lifted into the tropical context by the
factory function \texttt{from\_event}.

\begin{definition}[Singleton tropical context]\label{def:from-event}
Given an event $e$ and a threshold $k$, the \textbf{singleton tropical
context} is $T_e = (W_e, \dtotal_e)$ defined as follows.
\begin{enumerate}[label=(\roman*)]
  \item If $e$ is \textbf{focal} (a pivot candidate):
    \[
      W_e[0] = w(e), \qquad
      W_e[j] = -\infty \;\text{ for } j = 1, \ldots, k, \qquad
      \dtotal_e = 0.
    \]
    A focal event occupies slot $0$: it is a pivot with zero
    development events before it.

  \item If $e$ is \textbf{non-focal} (a development event):
    \[
      W_e[j] = -\infty \;\text{ for all } j = 0, \ldots, k, \qquad
      \dtotal_e = 1.
    \]
    A non-focal event contributes one unit of development capacity but
    cannot serve as a pivot.
\end{enumerate}
\end{definition}

\begin{example}[Focal singleton]\label{ex:focal-singleton}
Let $e$ be a focal event with weight $7.5$ and let $k = 3$.  Then
\[
  T_e = \bigl(\,[7.5,\; -\infty,\; -\infty,\; -\infty],\;\; 0\bigr).
\]
The pivot sits in slot $0$ with weight $7.5$; no development capacity
exists.
\end{example}

\begin{example}[Non-focal singleton]\label{ex:nonfocal-singleton}
Let $e$ be a non-focal event and let $k = 3$.  Then
\[
  T_e = \bigl(\,[-\infty,\; -\infty,\; -\infty,\; -\infty],\;\; 1\bigr).
\]
All weight-vector entries are $-\infty$ (no pivot is available), and
the block contributes one unit of development capacity.
\end{example}

%% ═══════════════════════════════════════════════════════════════════
\section{Tropical Composition}\label{sec:tropical-composition}
%% ═══════════════════════════════════════════════════════════════════

This section presents the central algorithm.  Given two tropical
contexts representing adjacent blocks---the left block $T_A$ followed
in time by the right block $T_B$---we compute their composition.

\begin{definition}[Tropical composition $\otimes$]
\label{def:tropical-composition}
Let $T_A = (W_A, d_A)$ and $T_B = (W_B, d_B)$ be tropical contexts
with the same threshold $k$.  Their \textbf{tropical composition} is
\[
  T_A \otimes T_B \;=\; (W_{\mathrm{result}},\; d_A + d_B),
\]
where $W_{\mathrm{result}}$ is computed by the following
shift-and-max rule.  For each result slot $j = 0, 1, \ldots, k$:
\begin{equation}\label{eq:shift-and-max}
  W_{\mathrm{result}}[j]
  \;=\;
  \begin{cases}
    W_A[j]
      & \text{if } j < d_A, \\[4pt]
    \max\!\bigl(W_A[j],\; W_B[j - d_A]\bigr)
      & \text{if } j \ge d_A.
  \end{cases}
\end{equation}
\end{definition}

The logic is as follows.  A pivot from block $A$ that already had $j$
pre-pivot development events keeps its slot: those $j$ events are still
before it in the concatenated sequence.  A pivot from block $B$ that
had $j'$ pre-pivot development events within $B$ acquires all $d_A$ of
$A$'s non-focal events as additional predecessors, so it shifts from
slot $j'$ to slot $j' + d_A$.  If $j' + d_A > k$, the shifted slot is
capped at $k$ (both slots $j' + d_A$ and $k$ indicate ``at least $k$
pre-pivot events'').

This is precisely the operational meaning of the $\dpre$ promotion
from \cref{eq:comp-dpre} in \cref{ch:context-algebra}: when a pivot
resides in block $B$, all of $A$'s development capacity becomes
pre-pivot.

\begin{remark}[Capping at $k$]\label{rem:capping}
The formulation in \cref{eq:shift-and-max} handles capping implicitly:
because the weight vector has only $k+1$ entries indexed $0$ through
$k$, a pivot from $B$ at slot $j'$ with $j' + d_A > k$ maps to
result slot $k$ via
$W_{\mathrm{result}}[k] = \max(W_{\mathrm{result}}[k],\; W_B[j'])$.
The implementation iterates over the entries of $W_B$ and writes to
$\min(j' + d_A,\; k)$, which is equivalent.
\end{remark}

\begin{example}[Tropical composition]\label{ex:tropical-composition}
Let $k = 3$, and consider the two tropical contexts
\begin{align*}
  T_A &= \bigl(\,[5.0,\; 3.0,\; -\infty,\; -\infty],\;\; d_A = 2\bigr),\\
  T_B &= \bigl(\,[7.0,\; -\infty,\; 4.0,\; -\infty],\;\; d_B = 3\bigr).
\end{align*}
We compute $T_A \otimes T_B$ slot by slot:
\begin{align*}
  j = 0 &\colon\; j < d_A = 2, \quad
    W_{\mathrm{result}}[0] = W_A[0] = 5.0. \\[3pt]
  j = 1 &\colon\; j < d_A = 2, \quad
    W_{\mathrm{result}}[1] = W_A[1] = 3.0. \\[3pt]
  j = 2 &\colon\; j \ge d_A, \quad
    W_{\mathrm{result}}[2] = \max\!\bigl(W_A[2],\; W_B[2 - 2]\bigr)
    = \max(-\infty,\; 7.0) = 7.0. \\[3pt]
  j = 3 &\colon\; j \ge d_A, \quad
    W_{\mathrm{result}}[3] = \max\!\bigl(W_A[3],\; W_B[3 - 2]\bigr)
    = \max(-\infty,\; -\infty) = -\infty.
\end{align*}
The result is
\[
  T_A \otimes T_B
  = \bigl(\,[5.0,\; 3.0,\; 7.0,\; -\infty],\;\; d = 5\bigr).
\]
\textbf{Interpretation.}\; The best pivot with zero pre-pivot
development events has weight $5.0$ (from $A$, slot~0).  With exactly
two pre-pivot events, the best weight is $7.0$ (from $B$, shifted by
$d_A = 2$).  Slot~3 is $-\infty$, so no feasible pivot exists at
threshold $k = 3$.
\end{example}

\begin{proposition}[Associativity of tropical composition]
\label{prop:tropical-associativity}
For any tropical contexts $T_A$, $T_B$, and $T_C$ with common
threshold $k$,
\[
  (T_A \otimes T_B) \otimes T_C
  \;=\;
  T_A \otimes (T_B \otimes T_C).
\]
\end{proposition}

\begin{proof}
The $\dtotal$ component of both sides equals $d_A + d_B + d_C$ by
associativity of addition.  It remains to verify the weight vector.

A pivot originating in block $X \in \{A, B, C\}$ at internal slot $j_X$
ends up in result slot $s(j_X)$ defined by
\[
  s_A(j_A) = j_A, \qquad
  s_B(j_B) = j_B + d_A, \qquad
  s_C(j_C) = j_C + d_A + d_B,
\]
all capped at $k$.  These slot assignments are determined by the
block's position in the concatenated sequence and are independent of
parenthesisation.  At each result slot $j$, both parenthesisations
compute $\max$ over the same set of contributing pivots from $A$, $B$,
and $C$, so the result is identical.

In detail, consider the left-associated computation.  Let
$T_{AB} = T_A \otimes T_B$ with $d_{AB} = d_A + d_B$.  Then in
$T_{AB} \otimes T_C$, a pivot from $C$ at internal slot $j_C$ shifts
to $\min(j_C + d_{AB},\, k) = \min(j_C + d_A + d_B,\, k)$.  A pivot
from $A$ at slot $j_A$ was placed at slot $j_A$ in $T_{AB}$ and
remains at slot $j_A$ in the final result.  A pivot from $B$ at slot
$j_B$ was placed at slot $\min(j_B + d_A,\, k)$ in $T_{AB}$ and
remains there.

The right-associated computation yields the same slot assignments:
$T_{BC} = T_B \otimes T_C$ places a $C$-pivot at $\min(j_C + d_B, k)$
and a $B$-pivot at slot $j_B$.  Then $T_A \otimes T_{BC}$ leaves
$A$-pivots at $j_A$, shifts $B$-pivots to $\min(j_B + d_A, k)$, and
shifts $C$-pivots to $\min(j_C + d_B + d_A, k)$.

Since both parenthesisations produce the same slot assignment for every
pivot and take $\max$ at each slot, the weight vectors agree.
\end{proof}

%% ═══════════════════════════════════════════════════════════════════
\section{Building Context from a Sequence}\label{sec:building-context}
%% ═══════════════════════════════════════════════════════════════════

With the composition operator in hand, computing the tropical context
for an entire event sequence is a left fold.

\begin{algorithm}[t]
\caption{Build tropical context from an event sequence.}
\label{alg:build-tropical}
\begin{algorithmic}[1]
\Procedure{BuildTropicalContext}{events, $k$}
  \State $T \gets \Call{Empty}{k}$
    \Comment{$W = [-\infty, \ldots, -\infty]$, $\dtotal = 0$}
  \For{each event $e$ in temporal order}
    \State $T \gets T \otimes \Call{FromEvent}{e, k}$
  \EndFor
  \State \Return $T$
\EndProcedure
\end{algorithmic}
\end{algorithm}

\Cref{alg:build-tropical} is the direct implementation of the left
fold.  The code path is
\texttt{src/tropical\_semiring.py:\allowbreak build\_tropical\_context()},
which iterates over events and composes each singleton into the
accumulator via \texttt{compose\_tropical()}.
We now trace through a complete example to build intuition.

\begin{example}[Full step-by-step construction]
\label{ex:full-construction}
Let $k = 3$ and consider the sequence of five events:
\[
  e_1\;\text{(non-focal)}, \quad
  e_2\;\text{(focal, $w = 4$)}, \quad
  e_3\;\text{(non-focal)}, \quad
  e_4\;\text{(focal, $w = 7$)}, \quad
  e_5\;\text{(non-focal)}.
\]

\medskip\noindent
\textbf{Initialisation.}\;
$T = ([-\infty,\; -\infty,\; -\infty,\; -\infty],\;\; d = 0)$.

\medskip\noindent
\textbf{Step~1: append $e_1$ (non-focal).}\;
$T_{e_1} = ([-\infty,\; -\infty,\; -\infty,\; -\infty],\;\; d = 1)$.
Composing $T \otimes T_{e_1}$: the left block has $d = 0$, so $B$'s
entries shift by $0$.  Both weight vectors are all $-\infty$.
\[
  T = ([-\infty,\; -\infty,\; -\infty,\; -\infty],\;\; d = 1).
\]

\medskip\noindent
\textbf{Step~2: append $e_2$ (focal, $w = 4$).}\;
$T_{e_2} = ([4,\; -\infty,\; -\infty,\; -\infty],\;\; d = 0)$.
Composing with $d_A = 1$:
\begin{align*}
  j = 0 &\colon\; j < 1, \quad W[0] = -\infty. \\
  j = 1 &\colon\; j \ge 1, \quad W[1] = \max(-\infty,\; T_{e_2}.W[0])
    = \max(-\infty, 4) = 4. \\
  j = 2 &\colon\; j \ge 1, \quad W[2] = \max(-\infty,\; T_{e_2}.W[1])
    = \max(-\infty, -\infty) = -\infty. \\
  j = 3 &\colon\; \text{similarly } -\infty.
\end{align*}
\[
  T = ([-\infty,\; 4,\; -\infty,\; -\infty],\;\; d = 1).
\]
Event $e_2$ (weight $4$) lands in slot $1$ because one non-focal event
($e_1$) precedes it.

\medskip\noindent
\textbf{Step~3: append $e_3$ (non-focal).}\;
$T_{e_3} = ([-\infty,\; -\infty,\; -\infty,\; -\infty],\;\; d = 1)$.
Composing with $d_A = 1$: $W_A$ entries stay, and $W_B$ is all
$-\infty$, so no new pivots appear.  Only $\dtotal$ increases.
\[
  T = ([-\infty,\; 4,\; -\infty,\; -\infty],\;\; d = 2).
\]

\medskip\noindent
\textbf{Step~4: append $e_4$ (focal, $w = 7$).}\;
$T_{e_4} = ([7,\; -\infty,\; -\infty,\; -\infty],\;\; d = 0)$.
Composing with $d_A = 2$:
\begin{align*}
  j = 0 &\colon\; j < 2, \quad W[0] = -\infty. \\
  j = 1 &\colon\; j < 2, \quad W[1] = 4. \\
  j = 2 &\colon\; j \ge 2, \quad W[2] = \max(-\infty,\; T_{e_4}.W[0])
    = \max(-\infty, 7) = 7. \\
  j = 3 &\colon\; j \ge 2, \quad W[3] = \max(-\infty,\; T_{e_4}.W[1])
    = \max(-\infty, -\infty) = -\infty.
\end{align*}
\[
  T = ([-\infty,\; 4,\; 7,\; -\infty],\;\; d = 2).
\]
Event $e_4$ (weight $7$) lands in slot $2$: two non-focal events
($e_1, e_3$) precede it in the sequence.

\medskip\noindent
\textbf{Step~5: append $e_5$ (non-focal).}\;
$T_{e_5} = ([-\infty,\; -\infty,\; -\infty,\; -\infty],\;\; d = 1)$.
Composing with $d_A = 2$: no new pivots; $\dtotal$ increments.
\[
  T = ([-\infty,\; 4,\; 7,\; -\infty],\;\; d = 3).
\]

\medskip\noindent
\textbf{Final result.}\;
$T = ([-\infty,\; 4,\; 7,\; -\infty],\;\; d = 3)$.  Since
$W[3] = -\infty$, the context is \textbf{not feasible} at $k = 3$.
The best weight overall is $\max(W) = 7$, achieved at slot $2$.  This
means the best pivot ($e_4$, weight $7$) has exactly $2$ pre-pivot
development events.

If the threshold were $k = 2$, the context would be feasible with
$W[2] = 7$.
\end{example}

\begin{remark}[Brute-force verification]\label{rem:brute-force}
\Cref{ex:full-construction} can be verified by direct enumeration.
The focal events and their non-focal predecessor counts are:
\begin{itemize}
  \item $e_2$ (weight $4$): preceded by $e_1$ (non-focal), so
    $\dpre = 1$.
  \item $e_4$ (weight $7$): preceded by $e_1, e_3$ (non-focal), so
    $\dpre = 2$.
\end{itemize}
Therefore $W[1] = 4$ and $W[2] = 7$, with all other slots $-\infty$.
This matches the left-fold result exactly.
\end{remark}

%% ═══════════════════════════════════════════════════════════════════
\section{Best Feasible Pivot Search}\label{sec:best-feasible-pivot}
%% ═══════════════════════════════════════════════════════════════════

With the weight vector in hand, the feasibility query reduces to an
index lookup.

\begin{definition}[Best feasible pivot]\label{def:best-feasible}
Given a tropical context $T = (W, \dtotal)$ and threshold $k$, the
\textbf{best feasible pivot weight} is
\[
  \wstar_{\mathrm{feas}} = W[k].
\]
If $W[k] = -\infty$, no feasible pivot exists.  If $W[k] > -\infty$,
the best feasible pivot has weight $W[k]$ and at least $k$ non-focal
predecessors.
\end{definition}

To recover the \emph{identity} of the best feasible pivot (not just
its weight), the algorithm must trace back through the composition to
find which event contributed $W[k]$.  In the streaming setting, this
requires maintaining the event reference alongside the weight.  In the
holographic tree setting (\cref{rem:parallel-tree} in
\cref{ch:context-algebra}), the tree's structure enables efficient
pivot-block lookup.

\begin{remark}[Top-$m$ search]\label{rem:top-m}
For applications requiring the top $m$ feasible pivots rather than
just the best, the function
\texttt{focal\_pivots\_with\_prefix} enumerates all focal events
together with their prefix development counts.  Filtering to those with
prefix count $\ge k$ and sorting by weight yields the top-$m$ list.
This brute-force enumeration runs in $O(n \log n)$ time (dominated by
sorting) and is used for validation purposes.
\end{remark}

%% ═══════════════════════════════════════════════════════════════════
\section{Subsumption of the Original Monoid}\label{sec:subsumption}
%% ═══════════════════════════════════════════════════════════════════

The tropical context is designed as a strict generalisation of the
original monoid.  We now state the precise subsumption relationship
and report its empirical validation.

\begin{proposition}[Tropical subsumption]\label{prop:subsumption}
Let $T = (W, \dtotal)$ be the tropical context obtained by left-fold
composition over an event sequence, and let
$C = (\wstar, \dtotal, \dpre)$ be the original monoid context computed
over the same sequence.  Then
\begin{align}
  \max_{0 \le j \le k} W[j] &= \wstar,
  \label{eq:subsumption-wstar}\\[4pt]
  \argmax_{0 \le j \le k} W[j] &= \min(\dpre,\; k),
  \label{eq:subsumption-dpre}
\end{align}
where $\argmax$ returns the smallest index achieving the maximum in
the case of ties.
\end{proposition}

\begin{proof}[Proof sketch]
Both the monoid and the tropical context process the same event
sequence by left fold.  The monoid tracks only the globally strongest
pivot (breaking ties in favour of the leftmost), while the tropical
context tracks the strongest pivot \emph{at each slot}.  The globally
strongest pivot occupies exactly one slot, say slot $j^*$, so
$\max(W) = W[j^*] = \wstar$.

For \cref{eq:subsumption-dpre}, the slot $j^*$ records the number of
non-focal predecessors of the global pivot, capped at $k$.  The
monoid's $\dpre$ is the uncapped count, so
$j^* = \min(\dpre, k)$.
\end{proof}

\begin{remark}[Empirical validation]\label{rem:empirical-subsumption}
The test suite \texttt{test\_03\_monoid\_subsumption.py} validates
\cref{prop:subsumption} across $200$ random seeds per configuration,
spanning a range of sequence lengths, focal ratios, and thresholds $k$.
Across all configurations, zero violations are observed with a
numerical tolerance of $10^{-12}$.  The tropical context faithfully
reproduces the original monoid's outputs.
\end{remark}

The tropical context is strictly richer than the monoid: it records
feasibility information at every slot simultaneously.  This per-slot
resolution is what enables the contract-guarded compression developed
in \cref{ch:absorbing-ideal}.

%% ═══════════════════════════════════════════════════════════════════
\section{Complexity Analysis}\label{sec:tropical-complexity}
%% ═══════════════════════════════════════════════════════════════════

\begin{proposition}[Complexity of tropical operations]
\label{prop:tropical-complexity}
The following complexity bounds hold:
\begin{enumerate}[label=(\roman*)]
  \item \textbf{Single composition.}\; Computing $T_A \otimes T_B$
    requires $O(k)$ time: one pass over the $k + 1$ result slots.

  \item \textbf{Left-fold construction.}\; Building the tropical
    context for a sequence of $n$ events via
    \cref{alg:build-tropical} requires $O(n \cdot k)$ time: $n$
    compositions, each $O(k)$.

  \item \textbf{Holographic tree.}\; The tree variant maintains
    $O(\log n)$ composition depth, achieving $O(k \log n)$ amortised
    cost per event insertion and $O(n \cdot k)$ total build cost.
\end{enumerate}
\end{proposition}

\begin{remark}[Practical cost]\label{rem:practical-cost}
In the narrative application, the threshold $k$ is typically small
($1 \le k \le 5$).  For such values the $O(k)$ factor is a small
constant, and the left-fold construction is effectively $O(n)$.  The
holographic tree's $O(\log n)$ query depth becomes relevant for
interactive or streaming applications where individual updates must
be fast relative to the total sequence length.
\end{remark}

%% ═══════════════════════════════════════════════════════════════════
\section{Implementation Notes}\label{sec:implementation-notes}
%% ═══════════════════════════════════════════════════════════════════

\begin{remark}[Sentinel value for $-\infty$]
\label{rem:sentinel}
The implementation in \texttt{src/tropical\_semiring.py} represents
the weight vector $W$ as a \texttt{numpy} array of floating-point
numbers.  The value $-\infty$ is encoded as Python's
\texttt{float("-inf")}.  The composition function
\texttt{compose\_tropical} implements the shift-and-max rule from
\cref{def:tropical-composition}: it copies $W_A$ into the result, then
iterates over entries of $W_B$, shifting each by $d_A$ and capping at
$k$.  This is equivalent to \cref{eq:shift-and-max}.
\end{remark}

\begin{remark}[Factory methods]\label{rem:factory}
The \texttt{TropicalContext} class provides three factory methods:
\begin{itemize}
  \item \texttt{empty(k)}: returns the identity element
    $([-\infty, \ldots, -\infty],\; 0)$.
  \item \texttt{from\_event(e, k)}: implements
    \cref{def:from-event}---for focal events, $W[0] = w(e)$ and
    $\dtotal = 0$; for non-focal events, all entries of $W$ are
    $-\infty$ and $\dtotal = 1$.
  \item \texttt{compose(other)}: delegates to
    \texttt{compose\_tropical}.
\end{itemize}
The left fold in \texttt{build\_tropical\_context} iterates over the
event sequence, composing each singleton into the accumulator.  The
ground-truth validator \texttt{brute\_force\_tropical\_context}
enumerates all focal events, counts their non-focal predecessors
directly, and populates the weight vector by brute force---providing
an independent reference for testing.
\end{remark}

%% ═══════════════════════════════════════════════════════════════════
\section{Exercises}\label{sec:tropical-lift-exercises}
%% ═══════════════════════════════════════════════════════════════════

\begin{exercise}\label{exer:tropical-compose-by-hand}
Let $k = 2$ and consider the two tropical contexts
\[
  T_A = \bigl(\,[6.0,\; -\infty,\; 2.0],\;\; d_A = 1\bigr),
  \qquad
  T_B = \bigl(\,[8.0,\; 3.0,\; -\infty],\;\; d_B = 2\bigr).
\]
\begin{enumerate}[label=(\alph*)]
  \item Compute $T_A \otimes T_B$ using the shift-and-max rule
    (\cref{eq:shift-and-max}).
  \item Is the result feasible at $k = 2$?
  \item What are $\wstar$ and $\dpre$ for the result?  Verify that they
    match what the original monoid would produce.
\end{enumerate}
\end{exercise}

\begin{exercise}\label{exer:identity-tropical}
Show that the empty tropical context $T_{\varepsilon} = ([-\infty,
\ldots, -\infty],\; 0)$ is a two-sided identity for $\otimes$.  That
is, for any tropical context $T$,
\[
  T \otimes T_{\varepsilon} = T
  \qquad\text{and}\qquad
  T_{\varepsilon} \otimes T = T.
\]
\emph{Hint.}\; For the left-identity case, note that $d_A = 0$ in the
shift-and-max rule.
\end{exercise}

\begin{exercise}\label{exer:tropical-fold-trace}
Let $k = 2$ and consider the event sequence:
\[
  e_1\;\text{(focal, $w = 5$)},\quad
  e_2\;\text{(non-focal)},\quad
  e_3\;\text{(focal, $w = 3$)},\quad
  e_4\;\text{(non-focal)}.
\]
Trace through the left-fold construction (\cref{alg:build-tropical})
step by step.  State the tropical context after each event is
appended.  Verify the final result against brute-force enumeration of
the focal events and their predecessor counts.
\end{exercise}

\begin{exercise}\label{exer:all-nonfocal-tropical}
Let all $n$ events in a sequence be non-focal.  What is the tropical
context produced by \cref{alg:build-tropical}?  Compare this to the
monoid context from \cref{exer:no-focal} in
\cref{ch:context-algebra} and confirm that the subsumption relation
(\cref{prop:subsumption}) holds.
\end{exercise}

\begin{exercise}\label{exer:associativity-numerical}
Let $k = 2$ and define
\begin{align*}
  T_A &= \bigl(\,[4.0,\; -\infty,\; -\infty],\;\; d_A = 1\bigr),\\
  T_B &= \bigl(\,[-\infty,\; -\infty,\; -\infty],\;\; d_B = 1\bigr),\\
  T_C &= \bigl(\,[9.0,\; -\infty,\; -\infty],\;\; d_C = 0\bigr).
\end{align*}
Compute $(T_A \otimes T_B) \otimes T_C$ and
$T_A \otimes (T_B \otimes T_C)$ separately.  Verify that the results
are identical and that the final context is feasible at $k = 2$.
\end{exercise}
