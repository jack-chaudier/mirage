% ══════════════════════════════════════════════════════════════
%  Chapter 7 — The Absorbing Ideal and Compression Contracts
% ══════════════════════════════════════════════════════════════
\chapter{The Absorbing Ideal and Compression Contracts}
\label{ch:absorbing-ideal}

The preceding chapters have established two independent lines of
reasoning.  On the algebraic side, \cref{ch:context-algebra} gave us the
endogenous context monoid $(\mathcal{C},\opendo)$ with strict
associativity and a logarithmic parallel reduction.  On the structural
side, \cref{ch:absorbing-states} showed that prefix-deficient states are
absorbing under greedy semantics---once a sequence has committed to a
deficient prefix, no continuation can rescue it.

This chapter unifies the two lines.  We extend the context element with
a \emph{commitment flag}, define a richer composition operation
$\opcommit$ that respects commitment, and prove that the set of
absorbing elements forms a \emph{left ideal} in the resulting monoid.
We then turn to the practical question: which operations can push an
element across the absorbing boundary?  The answer, confirmed both by
theory and by experiment, is that \emph{compression is the unique
closure-breaking operation}.  We conclude by formulating a
\emph{no-absorption contract} that every compression map must satisfy to
preserve algebraic closure.

% ══════════════════════════════════════════════════════════════
\section{Extended Context Elements}
\label{sec:extended-ctx}
% ══════════════════════════════════════════════════════════════

In \cref{ch:context-algebra} a context element was a triple
$(\wstar,\dtotal,\dpre)$.  To model systems that have irrevocably
selected a pivot, we adjoin a single bit.

\begin{definition}[Extended context element]
\label{def:extended-ctx}
An \emph{extended context element} is a quadruple
\[
  \bar C \;=\; (\wstar,\;\dtotal,\;\dpre,\;\kappa)
\]
where $\wstar \in \R$ is the weight of the focal event (running-max
pivot weight), $\dtotal \in \N$ is the total development count (number
of non-focal events in the block), $\dpre \in \N$ is the pre-pivot
development count (non-focal events that precede the current pivot), and
$\kappa \in \{0,1\}$ is the \emph{commitment flag}:
\begin{itemize}[itemsep=3pt]
  \item $\kappa = 0$: \textbf{uncommitted}.  The pivot identity is
        provisional; a future suffix may contain an event of higher weight
        that replaces the current pivot.
  \item $\kappa = 1$: \textbf{committed}.  The pivot identity is locked.
        No event in any future suffix can replace it, regardless of weight.
\end{itemize}
We write $\bar{\mathcal{C}}$ for the set of all extended context
elements.
\end{definition}

\begin{remark}[Operational interpretation of commitment]
\label{rem:commitment-operational}
The commitment flag models any mechanism by which a system makes an
irrevocable decision about pivot identity.  Three common sources of
commitment arise in practice:
\begin{enumerate}[label=(\roman*),itemsep=2pt]
  \item \emph{Streaming assignment.}  A streaming processor that emits
        labels on the fly has, at each step, assigned roles (pre-pivot
        development, pivot, post-pivot material) based on the current
        running-max pivot.  Once a label has been emitted, the assignment
        is irrevocable: setting $\kappa = 1$ records this fact.
  \item \emph{Compressed context.}  A solver that compresses its context
        window typically discards events that were ``redundant'' relative
        to a particular pivot.  The compressed representation is
        thereafter committed to that pivot, because the discarded events
        are no longer available to support a different one.
  \item \emph{Checkpoint publication.}  Any system that publishes an
        intermediate result---a partial narrative, a tentative
        ranking---has committed to the pivot underlying that result.
        Rolling back the pivot would invalidate the published output.
\end{enumerate}
In each case, the system has passed a point of no return.  The
commitment flag is the algebraic encoding of that irreversibility.
\end{remark}

% ══════════════════════════════════════════════════════════════
\section{Committed Composition}
\label{sec:committed-composition}
% ══════════════════════════════════════════════════════════════

With extended context elements in hand, we define the composition
operation that respects commitment.

\begin{definition}[Committed composition $\opcommit$]
\label{def:committed-composition}
Given extended context elements
\[
  \bar C_A = (\wstar_A,\; d_A,\; \dpre{,A},\; \kappa_A),
  \qquad
  \bar C_B = (\wstar_B,\; d_B,\; \dpre{,B},\; \kappa_B),
\]
their \emph{committed composition}
$\bar C_A \opcommit \bar C_B = (\wstar,\;\dtotal,\;\dpre,\;\kappa)$ is
defined by three cases.

\medskip
\noindent\textbf{Case 1.}  $\kappa_A = 1$ \emph{(left block committed).}
\[
  \wstar = \wstar_A, \qquad
  \dtotal = d_A + d_B, \qquad
  \dpre = \dpre{,A}, \qquad
  \kappa = 1.
\]
The committed pivot in $A$ is preserved unconditionally.  Block~$B$
contributes to the total development count but cannot override the
pivot, regardless of $\wstar_B$.

\medskip
\noindent\textbf{Case 2.}  $\kappa_A = 0$ and $\wstar_A \ge \wstar_B$
\emph{(left uncommitted, left pivot dominates).}
\[
  \wstar = \wstar_A, \qquad
  \dtotal = d_A + d_B, \qquad
  \dpre = \dpre{,A}, \qquad
  \kappa = \kappa_A = 0.
\]
This coincides with the endogenous composition $\opendo$: the
higher-weight pivot stays in $A$.

\medskip
\noindent\textbf{Case 3.}  $\kappa_A = 0$ and $\wstar_B > \wstar_A$
\emph{(left uncommitted, right pivot dominates).}
\[
  \wstar = \wstar_B, \qquad
  \dtotal = d_A + d_B, \qquad
  \dpre = d_A + \dpre{,B}, \qquad
  \kappa = \kappa_B.
\]
The pivot shifts to $B$.  All of $A$'s events become pre-pivot
development material and are added to $\dpre{,B}$.  The commitment
status of the result inherits from $B$.
\end{definition}

\begin{remark}[Comparison with $\opendo$]
\label{rem:commit-vs-endo}
The critical difference between $\opcommit$ and $\opendo$ lies
exclusively in Case~1.  Under $\opendo$, there is no Case~1: if
$\wstar_B > \wstar_A$, the pivot always shifts to $B$.  Under
$\opcommit$, the commitment flag blocks this shift.  The pivot in $A$ is
preserved even when $B$ contains a strictly higher-weight event.

Cases~2 and~3 of $\opcommit$ are identical to the two cases of
$\opendo$.  Committed composition therefore extends endogenous
composition: when $\kappa_A = 0$, the two operations agree.
\end{remark}

\begin{example}[Committed composition locks a suboptimal pivot]
\label{ex:committed-suboptimal}
Let $\bar C_A = (3,\, 5,\, 1,\, 1)$ and $\bar C_B = (10,\, 4,\, 2,\, 0)$.

\medskip
\noindent\emph{Under $\opendo$:}  Since $\wstar_B = 10 > 3 = \wstar_A$,
the pivot shifts to $B$.  The result is
\[
  \bar C_A \opendo \bar C_B
    = (10,\; 5+4,\; 5+2,\; 0)
    = (10,\; 9,\; 7,\; 0).
\]
The stronger pivot in $B$ has been selected, and all five of $A$'s
events now contribute to the pre-pivot development count.

\medskip
\noindent\emph{Under $\opcommit$:}  Since $\kappa_A = 1$, Case~1
applies regardless of weights.  The result is
\[
  \bar C_A \opcommit \bar C_B
    = (3,\; 5+4,\; 1,\; 1)
    = (3,\; 9,\; 1,\; 1).
\]
The system remains locked to the weight-3 pivot in $A$, ignoring the
weight-10 event in $B$.  The pre-pivot count stays at $\dpre = 1$.  If
$k = 3$ (the grammar's prefix requirement), then $\dpre = 1 < 3$ and
the committed result is absorbing---a situation that $\opendo$ would
have avoided by shifting to the stronger pivot.
\end{example}

% ══════════════════════════════════════════════════════════════
\section{The Absorbing Predicate}
\label{sec:absorbing-predicate}
% ══════════════════════════════════════════════════════════════

We now give the algebraic characterisation of states from which the
grammar's prefix requirement can never be satisfied.

\begin{definition}[Absorbing predicate]
\label{def:absorbing-predicate}
Fix a grammar prefix requirement $k \in \N$.  The \emph{absorbing
predicate} $\absorb$ is the subset of $\bar{\mathcal{C}}$ defined by
\[
  \bar C \in \absorb
  \quad\Longleftrightarrow\quad
  \dpre(\bar C) < k.
\]
An element $\bar C$ satisfying $\bar C \in \absorb$ is said to be
\emph{absorbing}: its pre-pivot development count is insufficient to
meet the prefix requirement.
\end{definition}

\begin{remark}[Connection to the impossibility theorem]
\label{rem:absorb-connection}
The absorbing predicate is the algebraic generalisation of the condition
that caused greedy failure in \cref{ch:absorbing-states}.  In that chapter, the
relevant quantity was $\jdev < k$: the development index of the current
pivot was too small for the grammar to be satisfiable.  Here, $\dpre < k$
plays exactly the same role, lifted from the sequential setting into the
algebraic framework.

The key insight is that $\jdev$ was defined relative to a specific
sequence position, whereas $\dpre$ is a property of an abstract context
element.  This abstraction is what allows the absorbing condition to
compose: we can reason about the absorption status of a composite block
without unfolding it into its constituent events.
\end{remark}

% ══════════════════════════════════════════════════════════════
\section{Absorbing Left Ideal}
\label{sec:absorbing-left-ideal}
% ══════════════════════════════════════════════════════════════

The main algebraic result of this chapter is that the absorbing
predicate defines a left ideal in the committed monoid.  Informally:
once a system has committed to an absorbing state, no suffix can rescue
it.

\begin{proposition}[Absorbing left ideal]
\label{prop:absorbing-left-ideal}
Let $k \in \N$ be the grammar prefix requirement.  If\/ $\bar C_A \in
\bar{\mathcal{C}}$ satisfies $\kappa_A = 1$ and $\bar C_A \in \absorb$
(i.e., $\dpre{,A} < k$), then for every $\bar C_D \in
\bar{\mathcal{C}}$,
\[
  \bar C_A \opcommit \bar C_D \;\in\; \absorb.
\]
That is, the set $\{\,\bar C \in \bar{\mathcal{C}} : \kappa = 1 \text{
and } \dpre < k\,\}$ is a left ideal of\/
$(\bar{\mathcal{C}},\opcommit)$.
\end{proposition}

\begin{proof}
Since $\kappa_A = 1$, Case~1 of \cref{def:committed-composition}
applies unconditionally, regardless of the values $\wstar_D$, $d_D$,
$\dpre{,D}$, and $\kappa_D$ carried by the suffix.  The committed
composition rule yields:
\[
  \bar C_A \opcommit \bar C_D
    \;=\;
    \bigl(\wstar_A,\;\; d_A + d_D,\;\; \dpre{,A},\;\; 1\bigr).
\]
We verify each component of the result against the absorbing predicate
and the left ideal requirement.

\medskip
\noindent\emph{Pre-pivot count is preserved.}\quad
The pre-pivot development count of the result is $\dpre{,A}$.  The
suffix $\bar C_D$ does not contribute to $\dpre$: because the pivot is
locked in $A$, no event in $D$ can become the new pivot, and therefore
no event in $D$ can change the set of events that precede the pivot.
The pre-pivot count is determined entirely by the committed left block.

\medskip
\noindent\emph{The absorbing predicate is satisfied.}\quad
By hypothesis, $\dpre{,A} < k$.  Since the pre-pivot count of the
result equals $\dpre{,A}$, the result also has $\dpre < k$.  Therefore
$\bar C_A \opcommit \bar C_D \in \absorb$.

\medskip
\noindent\emph{The commitment flag is preserved.}\quad
The result has $\kappa = 1$.  Combined with $\dpre < k$, this shows
that the result lies in the set
$\{\,\bar C : \kappa = 1 \text{ and } \dpre < k\,\}$, confirming that
the set is closed under right-multiplication by arbitrary elements
of $\bar{\mathcal{C}}$.

\medskip
Since $\bar C_D$ was arbitrary, the set of committed absorbing elements
is a left ideal of $(\bar{\mathcal{C}},\opcommit)$.
\end{proof}

\begin{remark}[Interpretation]
\label{rem:left-ideal-interpretation}
\Cref{prop:absorbing-left-ideal} is the algebraic formalisation of an
operationally devastating fact: \emph{once committed to a bad state, you
cannot recover}.  No matter how rich, how well-structured, or how long
the suffix is, it cannot repair the pre-pivot deficiency because the
commitment flag prevents the pivot from moving.  The absorbing elements
form a ``black hole'' in the monoid: any element that falls in---and is
committed---can never escape.

This should be contrasted with the sequential impossibility theorem of
\cref{ch:absorbing-states}, which established the same conclusion for a
specific greedy algorithm.  \Cref{prop:absorbing-left-ideal} is
stronger: it applies to \emph{any} process that respects committed
composition, regardless of the algorithm used to generate the suffix.
\end{remark}

% ══════════════════════════════════════════════════════════════
\section{Escape Under Endogenous Semantics}
\label{sec:escape-endo}
% ══════════════════════════════════════════════════════════════

The absorbing left ideal depends crucially on commitment.  Without it,
escape is possible---and understanding when escape occurs is essential
for the validity mirage analysis of \cref{ch:mirage}.

\begin{remark}[Escape from absorption without commitment]
\label{rem:escape-endo}
Suppose $\bar C_A$ has $\dpre{,A} < k$ (so $\bar C_A \in \absorb$) but
$\kappa_A = 0$ (uncommitted).  Under $\opendo$, there is no Case~1: the
pivot can shift if a higher-weight event appears on the right.  If the
suffix $\bar C_D$ has $\wstar_D > \wstar_A$ and
$\dpre{,D} \ge k - d_A$, then Case~3 of the composition applies:
\[
  \bar C_A \opendo \bar C_D
    = (\wstar_D,\;\; d_A + d_D,\;\; d_A + \dpre{,D},\;\; \kappa_D).
\]
The new pre-pivot count is $d_A + \dpre{,D}$.  If
$d_A + \dpre{,D} \ge k$, then $\bar C_A \opendo \bar C_D \notin
\absorb$: the composite has escaped absorption.
\end{remark}

\begin{example}[Absorption escape under $\opendo$]
\label{ex:absorption-escape}
Let $k = 3$ and consider
\[
  \bar C_A = (3,\, 2,\, 0,\, 0).
\]
Since $\dpre{,A} = 0 < 3 = k$, we have $\bar C_A \in \absorb$.  Now
let
\[
  \bar C_D = (10,\, 5,\, 4,\, 0).
\]
Under endogenous composition, $\wstar_D = 10 > 3 = \wstar_A$, so the
pivot shifts to $D$:
\[
  \bar C_A \opendo \bar C_D
    = (10,\; 2+5,\; 2+4,\; 0)
    = (10,\; 7,\; 6,\; 0).
\]
The pre-pivot count is $6 \ge 3 = k$: the composite has escaped
absorption.

\medskip
\noindent Under committed composition with $\kappa_A = 1$, the same
inputs yield
\[
  \bar C_A \opcommit \bar C_D
    = (3,\; 7,\; 0,\; 1).
\]
The pre-pivot count remains $0 < 3$: still absorbed.  Commitment
prevents escape.
\end{example}

\begin{remark}[Algebraic foundation of the validity mirage]
\label{rem:mirage-foundation}
The contrast between committed and uncommitted semantics is the
algebraic foundation of the \emph{validity mirage} developed in
\cref{ch:mirage}~\citep{gaffney2026mirage}.

An enumerative solver with beam width $M > 1$ explores multiple pivot
candidates simultaneously.  Each candidate corresponds to a different
factorisation of the sequence into pre-pivot, pivot, and post-pivot
material.  The solver's search process effectively operates under
$\opendo$ semantics: it can ``escape'' absorption by substituting a
different pivot---one that was not the running-max focal event of the
original prefix, but that appeared in a different branch of the beam.

The resulting output may well be valid: every grammar constraint is
satisfied, and every structural requirement is met.  But the pivot
identity has changed.  The solution that the solver returns is not the
solution that the original prefix was building toward.  Semantics have
drifted silently, and no constraint violation has been raised.  This is
the mirage: the output \emph{looks} correct because validity is
preserved, yet the underlying semantic intent has been abandoned.
\end{remark}

% ══════════════════════════════════════════════════════════════
\section{The No-Absorption Compression Contract}
\label{sec:compression-contract}
% ══════════════════════════════════════════════════════════════

The left ideal theorem (\cref{prop:absorbing-left-ideal}) tells us that
committed absorption is permanent.  The natural follow-up question is:
which operations can \emph{create} absorption where none existed before?
The empirical answer, detailed in \cref{sec:empirical-validation}, is
that compression is the unique culprit.  This section formulates the
contract that prevents it.

\begin{definition}[No-absorption contract]
\label{def:no-absorption-contract}
Fix a grammar prefix requirement $k \in \N$.  A \emph{compression map}
$\mu \colon \mathcal{E} \to \mathcal{E}$ (where $\mathcal{E}$ denotes
the space of event blocks) satisfies the \emph{no-absorption contract}
if for every event block $B$,
\[
  \dtotal\bigl(\mu(B)\bigr)
  \;\ge\;
  \min\bigl(\dtotal(B),\; k\bigr).
\]
\end{definition}

\begin{remark}[Operational interpretation]
\label{rem:contract-interpretation}
The contract permits compression to remove redundant development
capacity---events beyond what is needed for the prefix requirement---but
forbids it from reducing the development count below the survival
threshold $k$.  Concretely:
\begin{itemize}[itemsep=3pt]
  \item If the original block has $\dtotal(B) \ge k$, then the
        compressed block must also have $\dtotal(\mu(B)) \ge k$.
        Compression may discard the excess, but it must retain at least
        $k$ development events.
  \item If the original block has $\dtotal(B) < k$, then the contract
        requires only $\dtotal(\mu(B)) \ge \dtotal(B)$: the compression
        must not make the situation worse, but it is not required to
        conjure development events that were never there.
\end{itemize}
\end{remark}

\begin{remark}[Why $\dtotal$ and not $\dpre$]
\label{rem:why-dtotal}
The contract is stated in terms of $\dtotal$ rather than $\dpre$ for a
fundamental reason: the compression map does not know which event will
ultimately serve as the pivot.

At compression time, the pivot identity may still be provisional
($\kappa = 0$), and a future suffix may shift the pivot to a different
event.  The pre-pivot count $\dpre$ is defined relative to a specific
pivot; if the pivot changes, $\dpre$ changes with it.  By contrast,
$\dtotal$ counts \emph{all} non-focal events in the block, regardless
of their position relative to any particular pivot.

Preserving $\dtotal \ge k$ ensures that no matter where the pivot
ultimately lands, there is sufficient development material available for
the prefix requirement to be satisfied.  A contract based on $\dpre$
would protect only the current pivot assignment and would become
vacuous---or actively harmful---if the pivot subsequently shifted.
\end{remark}

\paragraph{From definition to algorithm.}
\Cref{def:no-absorption-contract} specifies the \emph{minimal} contract: a
guard on $\dtotal$ that ensures enough development capacity survives
compression to meet the prefix requirement~$k$.
\Cref{alg:contract-compress} below implements a \emph{stronger} guard
that additionally verifies suffix-feasibility---i.e., that the tropical
weight at rank~$k$ is preserved after each candidate removal.  The
stronger guard is not required by the theory (the absorbing-ideal results
hold with the minimal contract) but is advisable in practice, since it
catches near-absorbing states that the minimal $\dtotal \ge k$ contract
would pass.

\begin{algorithm}[t]
\caption{Contract-guarded compression}
\label{alg:contract-compress}
\begin{algorithmic}[1]
\Require Event block $\mathit{events}$, prefix requirement $k$,
         retention fraction $\mathit{retention} \in (0,1]$,
         random seed $\mathit{seed}$
\Ensure Compressed event block satisfying the no-absorption contract
\State $\mathit{target} \gets
       \lceil \mathit{retention} \times |\mathit{events}| \rceil$
\State $\mathit{removable} \gets$ non-focal events in
       $\mathit{events}$, shuffled by $\mathit{seed}$
\State $\mathit{removed} \gets 0$
\For{each candidate event $e$ in $\mathit{removable}$}
  \If{$|\mathit{events}| - \mathit{removed} \le \mathit{target}$}
    \State \textbf{break} \Comment{Reached retention target}
  \EndIf
  \State $\mathit{ctx\_without} \gets
         \Call{BuildTropicalContext}{\mathit{events} \setminus \{e\},\; k}$
  \If{$\mathit{ctx\_without}.\mathbf{W}[k]
      \ge \mathit{original}.\mathbf{W}[k]$}
    \Comment{Feasibility preserved}
    \State Remove $e$ from $\mathit{events}$;\quad
           $\mathit{removed} \gets \mathit{removed} + 1$
  \Else
    \State Block $e$ \Comment{Removal would violate contract}
  \EndIf
\EndFor
\State \Return remaining events
\end{algorithmic}
\end{algorithm}

\Cref{alg:contract-compress} presents the contract-guarded compression
procedure.  The algorithm iterates over non-focal events in a
seed-determined random order, greedily attempting to remove each one.
Before removing an event, it rebuilds the tropical context
(\cref{ch:tropical-lift}) for the reduced block and checks whether the
tropical weight at rank $k$ is preserved.  If removal would reduce
$\mathbf{W}[k]$ below its original value, the event is blocked: it is
essential for maintaining the prefix requirement.

The greedy removal order (controlled by the random seed) means that the
algorithm is not guaranteed to achieve the \emph{maximum} compression
ratio consistent with the contract.  However, it is guaranteed to
\emph{satisfy} the contract: every removal is individually checked, and
no removal that would violate feasibility is permitted.  The retention
parameter provides a hard floor on the number of events retained,
offering an additional safety margin beyond the algebraic contract.
The overhead of the contract guard is bounded: each candidate removal
requires one call to \textsc{BuildTropicalContext} at cost $O(n \cdot k)$,
and at most $O(n)$ candidates are tested, giving $O(n^2 k)$ worst-case
cost---acceptable for the moderate block sizes typical of streaming
applications.

% ══════════════════════════════════════════════════════════════
\section{Empirical Validation}
\label{sec:empirical-validation}
% ══════════════════════════════════════════════════════════════

The algebraic theory of the preceding sections makes precise
predictions.  We now report the experimental results that validate those
predictions and, crucially, identify the one operation that breaks them.

% ── Algebraic core ────────────────────────────────────────────
\subsection{Algebraic Core (Experiment~51)}
\label{subsec:exp51}

\Cref{ch:context-algebra} established that $\opendo$ is strictly associative.
The same property extends to $\opcommit$: committed composition
inherits associativity from the case analysis, because Case~1 is
idempotent (committed blocks remain committed) and the remaining cases
reduce to $\opendo$.

\emph{Pairwise exactness.}\quad
Across 240 randomly generated pairs of extended context elements, the
pairwise composition $\bar C_A \opcommit \bar C_B$ was computed and
compared against a reference implementation that evaluates the full
event-level sequence.  Result: \textbf{0 violations out of 240 cases.}

\emph{Associativity.}\quad
Across 80 randomly generated triples $(\bar C_A, \bar C_B, \bar C_C)$,
the two bracketings $(\bar C_A \opcommit \bar C_B) \opcommit \bar C_C$
and $\bar C_A \opcommit (\bar C_B \opcommit \bar C_C)$ were compared
component-wise.  Result: \textbf{0 violations out of 80 cases.}

These results validate the monoid structure computationally.

% ── Absorbing-ideal closure ───────────────────────────────────
\subsection{Absorbing-Ideal Closure (Experiment~56)}
\label{subsec:exp56}

To test \cref{prop:absorbing-left-ideal} empirically, we generated 96
test cases, each consisting of a committed absorbing element $\bar C_A$
(with $\kappa_A = 1$ and $\dpre{,A} < k$) composed on the right with a
randomly generated suffix $\bar C_D$.  In every case, we checked whether
the result $\bar C_A \opcommit \bar C_D$ satisfied $\dpre < k$.

\medskip
\noindent
Result: \textbf{0 violations out of 96 cases.}  The left ideal property
holds exactly across the full test suite.

% ── Closure diagnostics ──────────────────────────────────────
\subsection{Closure Diagnostics (Experiment~57)}
\label{subsec:exp57}

Which elementary operations preserve the monoid's algebraic structure,
and which break it?  Experiment~57 tested each operation independently
for closure violations.  The results are summarised in
\cref{tab:closure-diagnostics}.

\begin{table}[ht]
\centering
\caption{Closure violation rates by operation (Experiment~57).
         Each operation was tested on a suite of randomly generated
         extended context elements.  A violation occurs when the
         operation produces an output that crosses the absorbing
         boundary: a non-absorbing input yields an absorbing output, or
         vice versa.}
\label{tab:closure-diagnostics}
\begin{tabular}{@{}lc@{}}
\toprule
\textbf{Operation} & \textbf{Violation rate} \\
\midrule
Composition ($\opcommit$)           & 0.000 \\
Compression                         & 0.133 \\
Pivot update                        & 0.000 \\
Split-at-point                      & 0.000 \\
Absorbing escape under compression  & 0.033 \\
\bottomrule
\end{tabular}
\end{table}

The results are striking.  Every operation except compression has a
violation rate of exactly zero: composition, pivot update, and
split-at-point all preserve the algebraic structure perfectly.
Compression alone violates closure, at a rate of 13.3\%.  Furthermore,
3.3\% of compressions cause an \emph{absorbing escape}: a committed
absorbing element is compressed into a non-absorbing one, or---more
dangerously---a non-absorbing element is compressed into an absorbing
one.

\textbf{Conclusion:} Compression is the \emph{unique closure-breaking
operation} in the context algebra.  This is the key empirical result of
the chapter.  Every other elementary operation preserves the monoid
structure, but naive compression can push elements across the absorbing
boundary.  This finding motivates the no-absorption contract of
\cref{def:no-absorption-contract}: it is precisely the guard needed to
tame the one operation that breaks closure.

% ── Absorption escape rates ──────────────────────────────────
\subsection{Absorption Escape Rates}
\label{subsec:escape-rates}

\begin{figure}[ht]
\centering
\includegraphics[width=0.85\textwidth]{%
  figures/test_04_absorption_escape_rates.png}
\caption{Absorption escape rates under committed versus uncommitted
         semantics across generator configurations.  Under committed
         semantics ($\opcommit$), the escape rate is zero: the left
         ideal property holds.  Under uncommitted semantics ($\opendo$),
         escape is possible and occurs at measurable rates that depend
         on the weight distribution and block length.}
\label{fig:absorption-escape-rates}
\end{figure}

\Cref{fig:absorption-escape-rates} displays the absorption escape rates
for committed and uncommitted semantics across a range of generator
configurations.  The committed escape rate is uniformly zero, confirming
\cref{prop:absorbing-left-ideal}.  The uncommitted escape rate varies
with the generator configuration but is consistently positive,
illustrating the escape mechanism described in \cref{sec:escape-endo}.
The gap between the two curves is the algebraic signature of
commitment-induced irreversibility.

% ══════════════════════════════════════════════════════════════
\section{Exercises}
\label{sec:ch7-exercises}
% ══════════════════════════════════════════════════════════════

\begin{exercise}[The absorbing ideal is not two-sided]
\label{exer:not-two-sided}
Show that the absorbing ideal of \cref{prop:absorbing-left-ideal} is
\emph{not} a two-sided ideal.  That is, construct extended context
elements $\bar C_A$ and $\bar C_D$ such that $\bar C_A \in \absorb$
with $\kappa_A = 1$, yet $\bar C_D \opcommit \bar C_A \notin \absorb$.

\emph{Hint:}  Choose $\bar C_D$ with $\kappa_D = 0$ and a pivot of
sufficiently high weight so that if $\bar C_D$ is on the left and
uncommitted, the pivot stays in $D$.  If $\dpre{,D} \ge k$, the
composite inherits the non-absorbing pre-pivot count from $D$.
\end{exercise}

\begin{exercise}[Closure of absorbing elements under committed composition]
\label{exer:absorbing-closure}
Prove that under committed semantics, the set of committed absorbing
elements is closed under $\opcommit$ from the left.  Specifically, show
that if $\bar C_A \in \absorb$ with $\kappa_A = 1$ and $\bar C_B \in
\absorb$ with $\kappa_B = 1$, then
$\bar C_A \opcommit \bar C_B \in \absorb$ with commitment flag
$\kappa = 1$.
\end{exercise}

\begin{exercise}[Maximum compression under the no-absorption contract]
\label{exer:max-compression}
Design a compression policy that satisfies the no-absorption contract
(\cref{def:no-absorption-contract}) while achieving the maximum
compression ratio.  Specifically, for an event block $B$ with
$\dtotal(B) = 20$ non-focal events and grammar prefix requirement
$k = 3$:
\begin{enumerate}[label=(\alph*)]
  \item What is the theoretical minimum number of non-focal events that
        must be retained?
  \item What is the maximum number of events that can be removed?
  \item Does the answer change if the block also contains focal events?
        Explain.
\end{enumerate}
\end{exercise}

\begin{exercise}[Why $\dtotal$ and not $\dpre$ in the contract]
\label{exer:why-dtotal}
Explain why the no-absorption contract
(\cref{def:no-absorption-contract}) is stated in terms of $\dtotal$
rather than $\dpre$.  Construct a concrete scenario in which a
compression map $\mu$ preserves $\dpre(\mu(B)) \ge k$ for the current
pivot but produces a compressed block with $\dtotal(\mu(B)) < k$.  Show
that a subsequent pivot shift (under $\opendo$ semantics) can then push
the compressed block into the absorbing set, yielding $\dpre < k$ for
the new pivot.
\end{exercise}
