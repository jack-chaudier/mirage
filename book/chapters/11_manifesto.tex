% ══════════════════════════════════════════════════════════════
%  Chapter 11 — Manifesto
% ══════════════════════════════════════════════════════════════
\chapter{Manifesto}\label{ch:manifesto}

This chapter distils the preceding theory, experiments, and case studies
into seven numbered principles.  Each principle is grounded in a specific
theorem or empirical result developed in the book; none are aspirational
slogans.  We follow the principles with a practitioner checklist---a
concrete gate that any long-context system with endogenous pivot
selection should pass before deployment---and close with an honest
accounting of what this book does \emph{not} claim.

% ══════════════════════════════════════════════════════════════
\section{Seven Principles}\label{sec:principles}
% ══════════════════════════════════════════════════════════════

% ──────────────────────────────────────────────────────────────
\subsection{Principle 1: Validity Is Not Semantics}
\label{sec:p1}
% ──────────────────────────────────────────────────────────────

Raw validity---does the output satisfy the grammar, the schema, the
constraint set?---is necessary but insufficient.  When the output's
meaning depends on an \emph{endogenous pivot} selected by $\argmax$
over the output itself, a valid output with a substituted pivot is
syntactically well-formed yet semantically different from the intended
output.  In our experiments, raw validity stays at $1.0$ while pivot
preservation drops to ${\sim}0.35$---a ${\approx}\,65\%$ silent substitution rate
(\cref{ch:mirage}).  Any system that reports only raw validity is hiding
potential semantic drift behind a perfect score.

\medskip\noindent\textbf{Ground truth.}\quad
The validity mirage experiments of \cref{ch:mirage} measure raw validity,
pivot preservation, and semantic regret on the same corpora.  The gap
between the first metric and the second two is the empirical basis for
this principle.

% ──────────────────────────────────────────────────────────────
\subsection{Principle 2: Endogenous Pivots Demand Pivot-Consistent
Metrics}
\label{sec:p2}
% ──────────────────────────────────────────────────────────────

If your system selects a distinguished element from within the solution---a
turning point, root cause, reference activity, schema anchor---you must
measure whether that element is preserved under compression, truncation,
and streaming.  Standard metrics (accuracy, validity rate, BLEU, ROUGE)
are oblivious to pivot identity; they evaluate the surface form without
checking which element was chosen as the structural linchpin.  The three
diagnostics introduced in \cref{ch:mirage}---\emph{pivot preservation},
\emph{fixed-pivot feasibility}, and \emph{semantic regret}---are the
minimum viable measurement framework for any system with endogenous
coupling.

% ──────────────────────────────────────────────────────────────
\subsection{Principle 3: Greedy Has No Guarantees Under Endogenous
Constraints}
\label{sec:p3}
% ──────────────────────────────────────────────────────────────

The feasible family under endogenous pivot selection violates both the
hereditary axiom and the exchange axiom
(\cref{prop:non-matroid}).  This structural violation
means that greedy algorithms---the default workhorse for constrained
selection in combinatorial optimisation---have \emph{zero} approximation
guarantees for endogenous pivot problems.

The absorbing-state theorem (\cref{ch:absorbing-states}) quantifies the
worst case: when $\jdev < k$, greedy produces zero valid sequences.
This is not a soft degradation; it is a hard structural impossibility.
The event graph enters an absorbing state from which no continuation
can reach acceptance, and the impossibility is detectable from the
prefix alone.

% ──────────────────────────────────────────────────────────────
\subsection{Principle 4: Compression Must Be Algebra-Preserving}
\label{sec:p4}
% ──────────────────────────────────────────────────────────────

Compression is the unique closure-breaking operation in the context
algebra: $13.3\%$ of naive compressions violate monoid closure, while
composition, pivot update, and split-at-point all have $0\%$ violation
rates (\cref{ch:absorbing-ideal}).  Contract-guarded compression---preserving $\dtotal \ge \min(\dtotal, k)$---reduces silent mirages
from $21.95\%$ to $0\%$ on real incident data (\cref{ch:absorbing-ideal}).

Treating compression as a free operation---``just drop some
tokens''---is a safety failure.  Every compression map must satisfy the
no-absorption contract, or it risks pushing an algebraically healthy
context element across the absorbing boundary into the ideal from which
no continuation can produce a faithful output.

% ──────────────────────────────────────────────────────────────
\subsection{Principle 5: Commitment Timing Is a First-Class Design
Dimension}
\label{sec:p5}
% ──────────────────────────────────────────────────────────────

Under commit-now streaming, $54.9\%$ of organic sequences fall into
oscillation traps (\cref{ch:streaming}).  Deferring commitment to
$f \approx 0.25$ recovers $80.1\%$ validity at $85\%$ of offline
quality.  The choice of \emph{when} to commit to a pivot is not an
implementation detail---it is a fundamental design parameter with a
$2\times$ impact on system reliability.

Every streaming system with endogenous pivot selection should explicitly
specify and justify its commitment policy.  The commitment fraction~$f$
interacts with the pivot arrival distribution; the experiments in
\cref{ch:streaming} show that the optimal $f$ varies by agent archetype,
and that committing too early is strictly worse than committing too late.

% ──────────────────────────────────────────────────────────────
\subsection{Principle 6: Constraints Regularise; Relaxing Them Enables
Exploitation}
\label{sec:p6}
% ──────────────────────────────────────────────────────────────

Strict grammar constraints achieve $88\%$ validity; relaxed (strictly
more permissive) constraints achieve $32\%$ (\cref{ch:narrative}).  The
constraints act as regularisers that prevent the optimisation metric from
exploiting degenerate solutions.  Removing constraints in the name of
``flexibility'' or ``generality'' opens the door to Goodhart collapse~\citep{goodhart1984problems}:
the metric is optimised, but the output is meaningless.

The single necessary constraint---minimum development beats $\ge 1$---accounts for $100\%$ of observed failures in the narrative extraction
experiments of \cref{ch:narrative}.  This is a concrete instance of a
general pattern: a small number of structural constraints do
disproportionate regularisation work, and their removal has catastrophic
(not graceful) consequences.

% ──────────────────────────────────────────────────────────────
\subsection{Principle 7: Measure and Report Mirage Gaps}
\label{sec:p7}
% ──────────────────────────────────────────────────────────────

For every system that performs constrained extraction, compression, or
streaming with endogenous pivots, report the \emph{mirage gap}:
\[
  \text{mirage gap}
  \;=\;
  \text{raw validity} - \text{pivot preservation}.
\]
A mirage gap of~$0$ means the system is semantically faithful: every
valid output preserves the intended pivot.  A mirage gap $> 0$ means
some fraction of ``valid'' outputs have silently substituted pivots.
This number should appear in every evaluation table alongside accuracy
and raw validity (\cref{ch:mirage}).

The mirage gap is a single scalar that summarises the severity of the
endogenous coupling problem for a given system configuration.  It is
cheap to compute (it requires only a pivot-identity check on outputs
already produced) and impossible to game without actually preserving
pivots.

% ══════════════════════════════════════════════════════════════
\section{Practitioner Checklist}\label{sec:checklist}
% ══════════════════════════════════════════════════════════════

Before deploying a long-context system with endogenous pivot selection,
verify that every item below is satisfied.  Each item maps to one or
more of the seven principles and to a specific chapter of this book.

\begin{enumerate}[label=\textbf{\arabic*.},itemsep=6pt]

  \item \textbf{Identify all endogenous pivots.}\quad
    List every element in your system that is selected by $\argmax$ (or
    any selection operator) over the output itself.  If no such element
    exists, the endogenous coupling problem does not apply.  If one or
    more exist, every subsequent item is mandatory.
    (\cref{ch:formal-problem})

  \item \textbf{Measure pivot preservation.}\quad
    Under your compression, truncation, and streaming policy, what
    fraction of outputs preserve the pivot that would have been selected
    under the offline, uncompressed pipeline?  Report this number.
    (\cref{ch:mirage})

  \item \textbf{Measure fixed-pivot feasibility.}\quad
    When forced to use the original (offline) pivot, can the system still
    produce a valid output?  A low fixed-pivot feasibility rate means the
    system cannot even \emph{express} the correct answer under its
    current constraints.
    (\cref{ch:mirage})

  \item \textbf{Measure semantic regret.}\quad
    What is the quality loss attributable to pivot substitution?  Semantic
    regret isolates the cost of choosing a different pivot from the cost
    of other compression artefacts.
    (\cref{ch:mirage})

  \item \textbf{Verify the no-absorption contract.}\quad
    Your compression policy must satisfy $\dtotal$-preservation:
    $\dtotal$ after compression must remain at least $\min(\dtotal, k)$.
    A single violation is sufficient to push a context element into the
    absorbing ideal.
    (\cref{ch:absorbing-ideal})

  \item \textbf{Specify an explicit commitment point.}\quad
    Your streaming policy must name a commitment fraction~$f$, justified
    by the pivot arrival distribution in your domain.  ``Commit
    immediately'' ($f = 0$) is a valid choice only if you can demonstrate
    that the pivot arrival distribution is concentrated at the start of
    the sequence.
    (\cref{ch:streaming})

  \item \textbf{Test constraints under relaxation.}\quad
    Relax each grammar or schema constraint independently and measure the
    change in validity and pivot preservation.  Constraints that cause a
    large validity drop when relaxed are acting as regularisers, not
    merely as validators.  Do not remove them.
    (\cref{ch:narrative})

  \item \textbf{Report the mirage gap.}\quad
    Every evaluation table should include the mirage gap (raw validity
    minus pivot preservation) alongside raw validity.  A mirage gap of
    zero is the target; a nonzero mirage gap is a quantified warning.
    (\cref{ch:mirage})

  \item \textbf{Produce a deterministic witness.}\quad
    Construct or identify at least one concrete instance---a specific
    input, a specific compression policy, a specific pivot---that
    demonstrates the mirage effect on your system.  Understand its
    mechanism: which operation broke pivot preservation, and why.  A
    system without a known witness has not been tested; it has merely
    been lucky.
    (\cref{ch:absorbing-states}, \cref{ch:mirage})

\end{enumerate}

% ══════════════════════════════════════════════════════════════
\section{What This Book Does Not Claim}\label{sec:limitations}
% ══════════════════════════════════════════════════════════════

Honesty about scope is not a weakness; it is a prerequisite for the
claims that remain.  We record the following limitations explicitly.

\begin{itemize}[itemsep=6pt]

  \item \textbf{The tropical semiring is not claimed to be unique or
    optimal.}\quad
    We do not claim that the $(\max, +)$ tropical semiring
    (\cref{ch:tropical-lift}) is the only or the best algebraic framework
    for endogenous pivot problems.  It is the one that naturally arises
    from the weight-comparison structure of $\argmax$-based pivot
    selection.  Other coupling structures (e.g., attention-based or
    learned pivots) may demand different semirings, and we regard the
    identification of such alternatives as open.

  \item \textbf{The commitment fraction $f = 0.25$ is not universally
    optimal.}\quad
    The value $f \approx 0.25$ is the empirical optimum for the
    narrative-extraction domain studied in \cref{ch:streaming}.  It
    depends on the pivot arrival distribution, which varies by domain,
    agent archetype, and event-graph topology.  We do not claim that
    $0.25$ transfers to other settings without re-estimation.

  \item \textbf{Contract-guarded compression does not eliminate all
    failure modes.}\quad
    The no-absorption contract (\cref{ch:absorbing-ideal}) eliminates the
    specific failure mode of silent pivot substitution caused by
    $\dtotal$-deficiency after compression.  It does not address failure
    modes arising from weight perturbation, causal-graph corruption, or
    adversarial input construction.  These are distinct problems that
    require distinct guarantees.

  \item \textbf{We have not conducted human evaluation studies.}\quad
    All quality metrics reported in this book are automated: pivot
    preservation, fixed-pivot feasibility, semantic regret, and raw
    validity are computed programmatically from system outputs and
    ground-truth pivot identities.  The gap between automated metrics and
    human judgement of narrative quality, causal correctness, or
    explanatory adequacy is an open empirical question.

  \item \textbf{The theory is developed on temporal DAGs with
    weight-based pivot selection.}\quad
    Every theorem and experiment in this book operates on directed
    acyclic event graphs where the pivot is selected by $\argmax$ over a
    scalar weight function.  Extension to other endogenous coupling
    structures---attention-based selection, learned pivot functions,
    multi-pivot systems, cyclic dependency graphs---is conjectured to
    exhibit analogous absorbing-state phenomena, but this conjecture is
    unproved.  We flag it as the most important direction for future work
    (\cref{ch:discussion}).

\end{itemize}
