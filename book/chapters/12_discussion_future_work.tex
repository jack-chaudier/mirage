% ══════════════════════════════════════════════════════════════
%  Chapter 12 — Discussion and Future Work
% ══════════════════════════════════════════════════════════════
\chapter{Discussion and Future Work}\label{ch:discussion}

This book began with a single collapsed arc---Diana's---and traced the
failure through four layers of analysis: empirical observation, algebraic
formalisation, streaming extension, and unified theory.  This chapter
steps back to survey the landscape.  We first assemble the unified
picture (\cref{sec:unified-picture}), showing how the four constituent
papers tell a single coherent story.  We then catalogue research
directions (\cref{sec:research-directions}), grading each by its current
evidence level.  \Cref{sec:broader-connections} draws connections to
systems beyond narrative generation.  \Cref{sec:closing-remarks} offers
brief closing reflections.

% ══════════════════════════════════════════════════════════════
\section{The Unified Picture}\label{sec:unified-picture}
% ══════════════════════════════════════════════════════════════

The four papers that make up this book's empirical and theoretical spine
are not four independent contributions; they are four acts of a single
argument.  Each act addresses a different facet of the same structural
phenomenon---the validity mirage---and each builds on the conclusions of
its predecessors.

\paragraph{Paper~00: Narrative / Lorien~\citep{gaffney2026narrative} (Chapters~1--2).}
The endogenous pivot problem was discovered empirically through Diana's
phase collapse.  Across 50~seeds of a dinner-party simulation, Diana's
arc failed in 9~seeds with a uniform signature: zero development beats,
an extremely early turning point (median normalised position $0.13$),
and full candidate pools.  The failure was not caused by data scarcity
but by a structural interaction between protagonist-event injection and
endogenous pivot selection.  Grammar relaxation experiments confirmed
that the standard grammar could not be satisfied when the pool's
highest-weight events arrived too early.  This empirical observation
motivated every formal development that followed.

\paragraph{Paper~01: Absorbing States~\citep{gaffney2026absorbing} (Chapters~3--4).}
The absorbing-state analysis formalised the impossibility.  When $\jdev
< k$---that is, when the number of development-eligible events before
the argmax-selected pivot falls below the prefix requirement---greedy
search produces zero valid sequences.  The failure is not probabilistic;
it is structural.  \Cref{ch:taxonomy} classified failures into four
classes: Class~A (absorbing state), Class~B (pipeline coupling),
Class~C (commitment timing), and Class~D (assembly compression) and constructed a
hierarchy of solvers: greedy, enumerative, and TP-conditioned.  Each
level in the hierarchy trades computational cost for strictly greater
coverage of the failure space.

\paragraph{Paper~02: Streaming~\citep{gaffney2026streaming} (Chapter~8).}
The streaming analysis extended the theory to online settings, where
events arrive sequentially and labels must be emitted before the full
sequence is known.  The central result is that commit-now policies fall
into oscillation traps---sequences in which the running-max pivot
overtakes the committed pivot, leaving the policy in an absorbing state
from which no continuation is grammar-valid.  Empirically, such traps
affect 54.9\% of organically generated sequences.  Two solutions
emerged: deferred commitment (with patience parameter $f$) and tropical
streaming, which maintains a weight vector over all possible pre-pivot
development counts and defers role assignment until the vector stabilises.

\paragraph{Paper~03: Context Algebra~\citep{gaffney2026mirage} (Chapters~5, 6, 7, 9).}
The context algebra unified everything into a single algebraic
framework.  The context monoid $(\mathcal{C}, \opendo)$ compresses any
contiguous block of events into a triple $(\wstar, \dtotal, \dpre)$
that composes associatively.  The extended monoid $(\bar{\mathcal{C}},
\opcommit)$ adds a commitment flag $\kappa$ and reveals that
prefix-deficient elements form a left ideal---the absorbing ideal.
Compression is the unique closure-breaking operation: it is the only
elementary edit that can force a valid context into the absorbing ideal.
The \emph{validity mirage} is the quantitative gap between raw validity
(which remains at $1.0$ under compression) and pivot preservation (which
drops to $0.35$), confirming that standard metrics are structurally
blind to the failure.

\medskip
\noindent
\textbf{The unified thesis.}\;
Endogenous constraints---where a distinguished element is selected by
argmax over the solution---create structural failure modes that standard
validity metrics miss.  The context algebra provides the formal
framework to analyse, predict, and prevent these failures.  The absorbing
ideal explains \emph{why} failures are irrecoverable; the tropical lift
explains \emph{when} they become detectable; the streaming analysis
explains \emph{how often} they occur in practice; and the compression
contract explains \emph{what operations} must be guarded to avoid them.

% ══════════════════════════════════════════════════════════════
\section{Research Directions}\label{sec:research-directions}
% ══════════════════════════════════════════════════════════════

The following directions are ordered roughly by distance from the
current results.  Each is labelled with its evidence status.

% ──────────────────────────────────────────────────────────────
\subsection{Set-Valued Algebra / Context Semiring}
\label{sec:future-semiring}
\textsc{Status: Conjectured.}

\medskip
\noindent
The context monoid tracks a single best weight per slot.  When pivot
identity is uncertain---for example, when two focal events have nearly
equal weight---a single-pivot summary discards information that may be
decision-relevant.  A natural extension is to replace the scalar
$\wstar$ with a \emph{set} of candidate pivots, each annotated with its
own feasibility profile.  The resulting structure would be a context
\emph{semiring} rather than a monoid: addition corresponds to
maintaining the non-dominated frontier of pivot candidates, and
multiplication corresponds to block composition.

The tropical context (\cref{ch:tropical-lift}) already moves in this
direction.  The weight vector $W[0..k]$ records the best pivot weight
achievable with exactly $j$ pre-pivot development events, for each $j$.
A full set-valued semiring would generalise this to maintain the
non-dominated Pareto frontier over an arbitrary set of quality
dimensions, not just the development count.

The conjecture is that such a semiring admits an efficient parallel
reduction analogous to the $O(\log n)$ holographic tree, with the
set-valued composition reducing to a tropical matrix product.  No proof
or implementation exists at present.

% ──────────────────────────────────────────────────────────────
\subsection{Agent-Relative Tension Metrics}
\label{sec:future-tension}
\textsc{Status: Proposed.}

\medskip
\noindent
Diana's failures occurred partly because the composite quality metric
$Q$ valued kinetic events---catastrophes, confrontations,
revelations---over epistemic events---observations, realisations,
belief updates.  The $Q$-metric measures global tension: how much the
event changes the state of the world.  An agent-relative metric would
instead measure how much the event changes the \emph{focal agent's
beliefs}, regardless of the event's global salience.

Concretely, define the \emph{agent-relative tension} of event $e$ for
focal actor $a^{\star}$ as the KL divergence between $a^{\star}$'s
belief state before and after $e$.  Turning points would then be
selected by argmax over agent-relative tension rather than over global
kinetic weight.  This would naturally favour events that are
informationally significant to the focal agent, even if they are globally
unremarkable.

No implementation or evaluation exists.  The proposal is motivated by
the observation that Diana's highest-weight events (by the global
$Q$-metric) were protagonist-driven catastrophes that Diana merely
witnessed, while the events most relevant to Diana's own arc were
quieter observations that scored low on global tension.

% ──────────────────────────────────────────────────────────────
\subsection{Quality-Diversity Search via MAP-Elites}
\label{sec:future-map-elites}
\textsc{Status: Proposed.}

\medskip
\noindent
The current solver hierarchy explores the space of narrative arcs by
varying the turning-point candidate (enumerative solver) or by
conditioning on turning-point position (TP-conditioned solver).
Neither method systematically explores the full frontier of
turning-point position versus quality.

MAP-Elites~\citep{mouret2015illuminating} (multi-dimensional archive of phenotypic elites) offers a
principled alternative, drawing on the insight from novelty
search~\citep{lehman2011abandoning} that exploring diverse solutions
can outperform purely objective-driven optimisation.  The behaviour space is discretised into bins
indexed by normalised turning-point position.  Each bin stores the
highest-$Q$ arc whose turning point falls in that bin.  The archive is
populated by mutation and crossover over event-selection vectors.

The expected benefit is diagnostic: the MAP-Elites archive would reveal
whether metric misalignment (the tendency of $Q$ to favour early turning
points) is a \emph{local} phenomenon---affecting only a narrow band of
turning-point positions---or a \emph{global} one that distorts the
entire quality landscape.  If the archive shows high-$Q$ arcs at late
turning-point positions, the problem is local and can be addressed by
reweighting.  If the archive is empty at late positions, the problem is
structural and requires a metric redesign.

% ──────────────────────────────────────────────────────────────
\subsection{Adaptive Temporal Injection Thresholds}
\label{sec:future-adaptive-injection}
\textsc{Status: Partially Validated.}

\medskip
\noindent
The temporal injection filter---which excludes protagonist events
arriving before a normalised position threshold of $0.20$---recovered
7~out of 9 of Diana's failed arcs in the original experiments.  The
threshold $0.20$ was hand-tuned.

A more principled approach would derive the threshold from the pivot
arrival CDF developed in \cref{ch:streaming}.  Specifically, let
$F_{\tp}(t)$ denote the cumulative distribution of the running-max
pivot's arrival position.  The adaptive threshold is then
\[
  t^{\star} \;=\; \inf\bigl\{\,t : F_{\tp}(t) \ge 1 - \epsilon\,\bigr\},
\]
where $\epsilon$ is a tolerance parameter controlling how much early
pivot mass is acceptable.  This ties the injection filter directly to
the record-process analysis: the threshold adapts to the pool's
statistical properties rather than being fixed across all agents.

Partial validation comes from the observation that 7/9 recoveries at
$0.20$ is consistent with the record-process prediction that early
pivots concentrate in the first quintile.  A full validation would
require sweeping $\epsilon$ over a range of agent archetypes and
measuring the recovery rate as a function of the adaptive threshold.

% ──────────────────────────────────────────────────────────────
\subsection{Human Evaluation Gap}
\label{sec:future-human-eval}
\textsc{Status: Acknowledged Limitation.}

\medskip
\noindent
All quality metrics in this work are automated.  The composite
$Q$-score aggregates tension variance, peak tension, trajectory shape,
irony, significance, thematic coherence, and protagonist coverage.
Each component is computed from the event graph without human input.
No human evaluation studies have been conducted.

The correlation between automated $Q$-score and human narrative quality
judgement is unknown.  It is entirely possible that arcs judged
high-quality by the $Q$-metric are perceived as contrived or
incoherent by human readers, or conversely that arcs with low
$Q$-scores are compelling.  This is the single largest open question
for practical deployment of the theory.

A human evaluation study would require: (i)~a corpus of generated arcs
spanning the full range of $Q$-scores and turning-point positions;
(ii)~human raters scoring each arc on narrative coherence, emotional
engagement, and structural satisfaction; and (iii)~a correlation
analysis between human scores and automated metrics, broken down by
pivot preservation status.  The last point is critical: if human raters
reliably distinguish pivot-preserved arcs from mirage arcs, the
validity mirage has perceptual consequences.  If they do not, the
mirage is a formal property with no practical impact.

% ──────────────────────────────────────────────────────────────
\subsection{Extension to LLM Context Management}
\label{sec:future-llm-context}
\textsc{Status: Conjectured.}

\medskip
\noindent
The context algebra's results on compression and commitment timing may
apply directly to several problems in large language model (LLM)
infrastructure.

\paragraph{KV-cache eviction.}
When a transformer's key--value cache exceeds its budget, the system
must evict entries.  The retention sweep experiments in
\cref{ch:mirage} showed that naive eviction (removing low-attention
entries) can silently shift the semantic pivot of the context window.
The compression contract (\cref{ch:absorbing-ideal}) predicts that any
eviction policy lacking a pivot-preservation guard will eventually
produce mirage states.  The connection to attention sink methods (e.g.,
StreamingLLM's~\citep{xiao2023streamingllm} initial-token retention) is suggestive: the ``sink''
tokens may function as pivot anchors, and their retention may be a
special case of the no-absorption contract.

\paragraph{RAG chunk selection.}
Retrieval-augmented generation~\citep{lewis2020rag} selects chunks from a document store to
form the context for a query.  The selected chunks collectively
determine the answer's semantic pivot---the dominant theme or fact
around which the answer is organised.  Chunk selection is thus an
endogenous pivot problem: the ``best'' chunk depends on the answer,
which depends on which chunks are selected.  The context algebra
predicts that greedy chunk selection (top-$k$ by embedding similarity)
will produce mirage states when the highest-similarity chunk anchors
the answer to a subtopic that lacks sufficient supporting detail.

\paragraph{Multi-document summarisation.}
Summarising multiple documents under a token budget requires compressing
context.  The dominant theme of the summary---its pivot---is determined
by the compressed representation.  The algebra predicts that naive
compression (uniform truncation) will shift the pivot toward whichever
document's key sentences survive truncation, regardless of whether that
document is globally dominant.  Contract-guarded compression would
check, after each truncation step, whether the pivot has survived.

These connections are plausible but unproved beyond the KV-cache
experiments reported in \cref{ch:mirage}.

% ══════════════════════════════════════════════════════════════
\section{Connections to Broader Systems}\label{sec:broader-connections}
% ══════════════════════════════════════════════════════════════

The endogenous pivot problem is not specific to narrative generation.
Any system that selects a distinguished element by optimising over its
own output---including generative agent
architectures~\citep{park2023generative}---faces the same structural
coupling.  This section sketches
five domains where the theory applies and states what the algebra
predicts.

% ──────────────────────────────────────────────────────────────
\subsection{LLM Context Management}
\label{sec:conn-llm}

Modern large language models operate under strict context-window
budgets.  When the budget is exceeded, the system must compress:
KV-cache entries are evicted, attention windows are chunked, or
prefixes are summarised.  Each of these operations is a lossy
context reduction.

\paragraph{Endogenous pivot analogue.}
The semantic pivot of the context window is the token (or span of
tokens) that most strongly determines the model's next-token
distribution---the ``attention sink'' in the terminology of
StreamingLLM.  This pivot is endogenous: it depends on the full
context, yet compression modifies the context.

\paragraph{Theory predicts.}
The compression contract (\cref{ch:absorbing-ideal}) predicts that any
eviction policy that does not explicitly check pivot preservation will
produce mirage states: the model's output will remain fluent and
locally coherent (high raw validity) while the semantic anchor of the
generation has silently shifted (low pivot preservation).  Attention
sink methods that retain initial tokens may be understood as an
implicit pivot-preservation heuristic---retaining the tokens most
likely to serve as the global attention anchor.

% ──────────────────────────────────────────────────────────────
\subsection{Incident Triage and Root Cause Analysis}
\label{sec:conn-incident}

Incident investigation---whether in aviation (NTSB reports), software
(post-incident reviews), or industrial safety---requires identifying a
\emph{root cause} from a sequence of events.  The root cause is the
event that, if removed, would have prevented the incident.

\paragraph{Endogenous pivot analogue.}
The root cause is selected by argmax over causal contribution, which
depends on which events are included in the analysis.  When the event
log is truncated (e.g., limited to the final 60~minutes before the
incident), the root cause may shift to a proximate trigger that
happened to fall within the window, even though the true root cause
occurred hours earlier.

\paragraph{Theory predicts.}
The absorbing-state theorem predicts that truncation-induced pivot
shifts are irrecoverable: once the true root cause is outside the
analysis window, no amount of reasoning over the truncated log can
recover it.  The tropical context predicts that maintaining a ranked
list of candidate root causes (not just the top one) provides a natural
defence: if the second-best candidate is qualitatively different from
the best, the analysis is fragile and the window should be extended.

% ──────────────────────────────────────────────────────────────
\subsection{Process Mining}
\label{sec:conn-process-mining}

Process mining~\citep{vanderaalst2016process} discovers workflow models from event logs.  A central
task is \emph{reference activity selection}: choosing the activity that
anchors the alignment of all traces.

\paragraph{Endogenous pivot analogue.}
The reference activity is the process-mining analogue of the turning
point.  It is typically selected as the most frequent or most
informative activity---an argmax over the discovered model.  But the
model depends on how traces are aligned, which depends on the reference
activity.

\paragraph{Theory predicts.}
When event logs are sampled or filtered (e.g., retaining only the most
recent $n$ cases), the reference activity may shift to a different
activity that happens to dominate the filtered sample.  The context
algebra predicts that this shift is absorbing under committed
semantics: once downstream analyses are conditioned on the wrong
reference activity, no subsequent correction can recover the original
alignment without re-processing the full log.

% ──────────────────────────────────────────────────────────────
\subsection{Grammar-Constrained Decoding}
\label{sec:conn-grammar-decoding}

Grammar-constrained decoding systems---such as PICARD~\citep{scholak2021picard} (for SQL),
Outlines~\citep{willard2023outlines} (for structured JSON), and constrained beam search~\citep{gange2020argmax}---guide
language model generation to satisfy a formal grammar.

\paragraph{Endogenous pivot analogue.}
In many structured outputs, a single token or clause serves as the
semantic pivot: the \texttt{WHERE} clause in SQL, the root key in JSON,
the topic sentence in a paragraph.  This pivot is endogenous: it
emerges from the generation process and determines the meaning of
surrounding tokens.

\paragraph{Theory predicts.}
Grammar constraints ensure syntactic validity but cannot enforce
semantic coherence with respect to the pivot.  A grammar-valid SQL
query may satisfy all syntactic rules while selecting the wrong
\texttt{WHERE} predicate---a pivot shift that changes the query's
meaning without triggering a grammar violation.  The theory predicts
that grammar constraints are necessary but not sufficient: they must be
augmented with pivot-aware semantic checks.  This is precisely the
``grammar as regulariser, not validator'' principle from
\cref{ch:manifesto}.

% ──────────────────────────────────────────────────────────────
\subsection{Constrained Beam Search}
\label{sec:conn-beam-search}

Beam search with constraints maintains a set of $B$ partial hypotheses,
pruning at each step to the top-$B$ candidates.  When the output must
satisfy a structural constraint (e.g., balanced parentheses, valid XML),
the beam is further filtered by constraint satisfaction.

\paragraph{Endogenous pivot analogue.}
The semantic pivot of a partial hypothesis is the token or span that
most strongly determines the hypothesis's eventual meaning.  As the
beam evolves, the pivot of the top hypothesis may oscillate---exactly
the streaming oscillation trap of \cref{ch:streaming}.  When the beam
commits to a hypothesis (by pruning all alternatives), the pivot is
locked.

\paragraph{Theory predicts.}
The streaming analysis predicts that beam width $B$ controls the
effective patience parameter $f$: wider beams defer commitment longer
and therefore have lower trap rates.  The tropical context provides a
natural beam-scoring function: instead of scoring hypotheses by
log-likelihood alone, score them by their tropical feasibility
vector, preferring hypotheses that maintain feasibility at multiple
development thresholds.  This is a concrete instantiation of deferred
commitment in the beam-search setting.

% ══════════════════════════════════════════════════════════════
\section{Closing Remarks}\label{sec:closing-remarks}
% ══════════════════════════════════════════════════════════════

The validity mirage is not a bug in a specific system.  It is a
structural property of any system where interpretation depends on
endogenous selection---where the meaning of the output is determined by
an element that is itself selected from the output.  Such systems are
ubiquitous: narrative generators select turning points, incident
analysers select root causes, process miners select reference
activities, retrieval systems select anchor documents, and language
models select attention sinks.  In every case, standard validity
metrics ask ``does the output satisfy the constraints?''\ and miss the
deeper question: ``does the output mean what it was supposed to mean?''

The algebra developed in this book provides tools to see the mirage
(the pivot-preservation metric), measure it (the mirage gap between raw
validity and fixed-pivot feasibility), and prevent it (the compression
contract, the deferred commitment policy, the tropical streaming
algorithm).  The manifesto of \cref{ch:manifesto} distils these tools
into a practitioner's checklist.  The research directions catalogued in
\cref{sec:research-directions} give researchers a roadmap.

The central message is simple.  Validity is necessary but not
sufficient.  Compression is dangerous but manageable.  Commitment is
irreversible but deferrable.  And the algebra is there to tell you
which is which.
