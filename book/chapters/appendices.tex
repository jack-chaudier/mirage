% ══════════════════════════════════════════════════════════════
%  Appendices
% ══════════════════════════════════════════════════════════════

% ──────────────────────────────────────────────────────────────
\chapter{Notation}\label{app:notation}
% ──────────────────────────────────────────────────────────────

\Cref{tab:notation} collects every symbol used in this book, together
with a brief gloss and the chapter in which it is introduced.

\noindent
\begin{tabular}{@{} l p{8.2cm} l @{}}
\toprule
\textbf{Symbol} & \textbf{Meaning} & \textbf{Introduced} \\
\midrule

$G = (V, E, t, w, a)$
  & Event graph: vertices $V$, edges $E$, timestamp function $t$,
    weight function $w$, actor function $a$
  & \cref{ch:formal-problem} \\

$a^{\star}$
  & Focal actor: the agent whose narrative arc is being extracted
  & \cref{ch:formal-problem} \\

$P$
  & Candidate pool: the set of events involving the focal actor
  & \cref{ch:formal-problem} \\

$\tp(S)$
  & Endogenous turning point of event set $S$: the focal event with
    maximum weight in the selected subsequence
  & \cref{ch:formal-problem} \\

$A = (Q, \Sigma, \delta, q_0, F)$
  & Phase grammar DFA: states $Q$, alphabet $\Sigma$ (phase labels),
    transition function $\delta$, initial state $q_0$, accepting
    states~$F$
  & \cref{ch:formal-problem} \\

$k$
  & Prefix requirement: the minimum number of \textsc{development}
    beats before the turning point
  & \cref{ch:formal-problem} \\

$M$
  & Solver budget: the number of pivot candidates explored by the
    enumerative or TP-conditioned solver
  & \cref{ch:formal-problem} \\

$\jdev$
  & Development-eligible event count: the number of non-focal events
    whose timestamp precedes the argmax-selected pivot
  & \cref{ch:absorbing-states} \\

$C = (\wstar, \dtotal, \dpre)$
  & Context element: a triple summarising the structural state of a
    contiguous block of events
  & \cref{ch:context-algebra} \\

$\opendo$
  & Endogenous composition: the associative binary operation on
    context elements
  & \cref{ch:context-algebra} \\

$e = (-\infty, 0, 0)$
  & Identity element of the context monoid
  & \cref{ch:context-algebra} \\

$T = (W, \dtotal)$
  & Tropical context: a weight vector $W$ paired with a total
    development count
  & \cref{ch:tropical-lift} \\

$W[j]$
  & Best pivot weight achievable with exactly $j$ pre-pivot
    development events
  & \cref{ch:tropical-lift} \\

$\bar{C} = (\wstar, \dtotal, \dpre, \kappa)$
  & Extended context element: a context element augmented with a
    commitment flag
  & \cref{ch:absorbing-ideal} \\

$\opcommit$
  & Committed composition: the composition operation on extended
    context elements that respects commitment
  & \cref{ch:absorbing-ideal} \\

$\kappa$
  & Commitment flag: $\kappa = 0$ (uncommitted, pivot is provisional)
    or $\kappa = 1$ (committed, pivot is locked)
  & \cref{ch:absorbing-ideal} \\

$\absorb$
  & Absorbing predicate: $\dpre < k$, indicating that the context is
    in the absorbing ideal
  & \cref{ch:absorbing-ideal} \\

$\mu$
  & Compression map: a lossy reduction of the context that may shift
    the pivot
  & \cref{ch:absorbing-ideal} \\

$\tp_t$
  & Running-max pivot at step $t$: the highest-weight focal event
    seen in the first $t$ events of the stream
  & \cref{ch:streaming} \\

$\peff$
  & Effective pre-pivot capacity: the number of development-eligible
    events before the running-max pivot in a streaming setting
  & \cref{ch:streaming} \\

$f$
  & Patience parameter: the fraction of the sequence that the deferred
    commitment policy waits before locking the pivot
  & \cref{ch:streaming} \\

$\mingap$
  & Minimum inter-record gap: the smallest number of events between
    consecutive running-max pivot updates
  & \cref{ch:streaming} \\

PPR
  & Pivot preservation rate: the fraction of sequences in which the
    post-compression pivot matches the pre-compression pivot
  & \cref{ch:mirage} \\

FPF
  & Fixed-pivot feasibility: the fraction of sequences that remain
    grammar-valid when evaluated against the original (pre-compression)
    pivot
  & \cref{ch:mirage} \\

SR
  & Semantic regret: the quality loss attributable to pivot shift,
    measured as $Q_{\text{original}} - Q_{\text{shifted}}$
  & \cref{ch:mirage} \\

$Q$
  & Composite quality score: a weighted aggregate of tension variance,
    peak tension, trajectory shape, irony, significance, thematic
    coherence, and protagonist coverage
  & \cref{ch:narrative} \\

VA
  & Validity-adjusted score: a metric that interpolates between raw
    validity and pivot-preserving validity
  & \cref{ch:narrative} \\

$\alpha$
  & Interpolation parameter: controls the weight of pivot preservation
    in the validity-adjusted score
  & \cref{ch:narrative} \\
\bottomrule
\end{tabular}
\label{tab:notation}

% ──────────────────────────────────────────────────────────────
\chapter{Glossary}\label{app:glossary}
% ──────────────────────────────────────────────────────────────

\begin{description}[style=nextline, leftmargin=2.5cm, labelwidth=2.3cm, itemsep=6pt]

\item[Absorbing state]
A state of the product automaton from which no accepting state is
reachable.  Once entered, no continuation can produce a grammar-valid
sequence.

\item[Blast radius]
The number of downstream labels or decisions invalidated by a single
pivot shift.

\item[Candidate pool]
The set of events involving the focal actor that are available for
inclusion in the narrative arc.

\item[Commit-now policy]
A streaming policy that irrevocably assigns phase labels to each event
as it arrives, based on the current running-max pivot.

\item[Committed semantics]
The interpretation regime under which the pivot identity is locked and
cannot be revised by future events.

\item[Compression contract]
A pre/post condition pair that a compression map must satisfy to
guarantee that the pivot is preserved and the context does not enter the
absorbing ideal.

\item[Context element]
A triple $(\wstar, \dtotal, \dpre)$ that summarises the structural
state of a contiguous block of events: the best pivot weight, the total
development capacity, and the pre-pivot development capacity.

\item[Deferred commitment]
A streaming policy that withholds label assignment for the first
fraction~$f$ of the sequence, committing only after the pivot has
stabilised.

\item[Development-eligible]
An event is development-eligible if it is a non-focal event whose
timestamp precedes the current pivot.  Such events can serve as
\textsc{development} beats in the phase grammar.

\item[Endogenous pivot]
A distinguished element of the output that is selected by argmax over
the output itself, creating a circular dependency between selection and
interpretation.

\item[Enumerative solver]
A solver that iterates over the top-$M$ pivot candidates by weight,
attempting grammar-valid extraction for each, and returns the first
success.

\item[Event graph]
The directed graph $G = (V, E, t, w, a)$ encoding all events in the
simulation, with timestamps, weights, and actor assignments.

\item[Feasibility]
A sequence is feasible if it satisfies the phase grammar with at least
$k$ development beats before the turning point selected by the
endogenous pivot function.

\item[Fixed-pivot feasibility (FPF)]
The fraction of compressed sequences that remain grammar-valid when
evaluated against the pre-compression pivot.

\item[Focal actor]
The agent $a^{\star}$ whose narrative arc is being extracted from the
event graph.

\item[Grammar-aware classifier]
A classifier that uses the phase grammar's state to prune or reweight
events during arc extraction, treating the grammar as a regulariser
rather than a post-hoc validator.

\item[Mirage gap]
The quantitative difference between raw validity (which may be $1.0$)
and pivot-preserving metrics such as FPF or PPR.  A large mirage gap
indicates that standard metrics are hiding semantic failures.

\item[Monotonic DFA]
A DFA whose state transitions are monotonically ordered: once a phase
is exited, it cannot be re-entered.

\item[Oscillation trap]
A streaming state in which the running-max pivot has overtaken the
committed pivot, leaving the policy in an absorbing state from which
no grammar-valid continuation exists.

\item[Patience parameter]
The fraction $f \in [0, 1]$ of the sequence that the deferred
commitment policy processes before locking the pivot.

\item[Phase grammar]
A deterministic finite automaton that specifies the legal ordering of
narrative phases: \textsc{setup}, \textsc{development},
\textsc{turning\_point}, \textsc{resolution}.

\item[Pivot preservation (PPR)]
The fraction of sequences in which the post-operation pivot (after
compression, streaming, or other transformation) matches the
pre-operation pivot.

\item[Prefix requirement]
The parameter $k$ specifying the minimum number of \textsc{development}
beats that must precede the turning point for a sequence to be
grammar-valid.

\item[Record process]
The stochastic process formed by the running maximum of pivot weights
in a stream.  Each new record corresponds to a pivot update.

\item[Semantic regret]
The quality loss caused by a pivot shift: the difference in composite
$Q$-score between the arc built around the original pivot and the arc
built around the shifted pivot.

\item[Tropical context]
A pair $(W, \dtotal)$ where $W$ is a weight vector indexed by pre-pivot
development count.  The tropical composition rule is a shift-and-max
operation derived from the tropical semiring.

\item[Turning point]
The event selected by the endogenous pivot function---the single
highest-weight focal event in the chosen subsequence.

\item[Validity mirage]
The phenomenon in which standard validity metrics (grammar satisfaction,
constraint checks) report perfect scores while the semantic content of
the output has been silently corrupted by a pivot shift.

\item[Weight vector]
The vector $W[0..k]$ in a tropical context, where $W[j]$ is the best
pivot weight achievable with exactly $j$ pre-pivot development events.

\end{description}

% ──────────────────────────────────────────────────────────────
\chapter{Determinism Contract}\label{app:determinism}
% ──────────────────────────────────────────────────────────────

Every experimental result in this book is deterministically
reproducible.  This appendix documents the mechanisms that guarantee
reproducibility.

\section{Fixed Random Seeds}\label{sec:det-seeds}

Every random generator in the codebase takes an explicit seed parameter.
No experiment relies on a global random state or on the default seed.
Each test function creates its own \texttt{numpy.random.Generator}
instance initialised with a fixed seed, ensuring that the random state
of one test cannot contaminate another.

\begin{verbatim}
rng = numpy.random.default_rng(seed=42)
\end{verbatim}

\noindent
The seed value is recorded in the test file and in the output CSV.
Re-running the test with the same seed on the same platform produces
bit-identical results.

\section{Deterministic Tie-Breaking}\label{sec:det-tiebreak}

When two events have equal weight, the system breaks ties using a
three-key tuple:
\[
  (\,w,\;\; {-t},\;\; \texttt{event\_id}\,),
\]
where $w$ is the event weight, $t$ is the timestamp (negated so that
earlier events win ties), and \texttt{event\_id} is a unique integer
identifier assigned at event creation.  This ensures a strict total
order on events, eliminating any dependence on hash randomisation or
memory layout.

\section{Stable Sorting}\label{sec:det-sorting}

All sorting operations use a stable sort with an explicit key function.
No code path relies on the default comparison of complex objects or on
the stability guarantees of a particular sorting algorithm beyond the
Python specification that \texttt{list.sort()} and \texttt{sorted()} are
stable.  Where numpy sorting is used, \texttt{kind='stable'} is
specified explicitly.

\section{Isolated Random State}\label{sec:det-isolation}

Each test creates its own \texttt{numpy.random.Generator} with a fixed
seed.  Tests do not share random state, and the order in which tests
are executed does not affect their outputs.  This isolation is
enforced by convention: no test reads from or writes to a global
random state.

\section{Platform Independence}\label{sec:det-platform}

All results reported in this book were produced and verified on the
following platform:
\begin{itemize}[itemsep=2pt]
  \item Python~3.14
  \item NumPy (current stable release)
  \item macOS (Darwin)
\end{itemize}
No platform-specific numerical libraries (e.g., MKL, CUDA) are used
in the core experimental pipeline.  Floating-point operations follow
IEEE~754 double-precision semantics.  No experiment depends on
nondeterministic GPU operations.

% ──────────────────────────────────────────────────────────────
\chapter{Experiment-to-Artifact Map}\label{app:artifact-map}
% ──────────────────────────────────────────────────────────────

\Cref{tab:artifact-map} links every major experimental claim in this
book to the test file that produces it, the CSV artifact that records
the raw data, and the figure (if any) that visualises the result.  All
paths are relative to the repository root.

\noindent
\begin{small}
\begin{tabular}{@{} p{3.0cm} p{1.6cm} p{4.6cm} p{3.4cm} @{}}
\toprule
\textbf{Claim} & \textbf{Test} & \textbf{Artifact CSV} & \textbf{Figure} \\
\midrule

Tropical exactness (0~violations)
  & \texttt{test\_01}
  & \texttt{results/raw/test\_01\_exactness.csv}
  & \texttt{test\_01\_exactness\_heatmap.png} \\

Associativity (0~violations)
  & \texttt{test\_02}
  & \texttt{results/raw/test\_02\_associativity.csv}
  & \texttt{test\_02\_associativity\_heatmap.png} \\

Monoid subsumption
  & \texttt{test\_03}
  & \texttt{results/raw/test\_03\_monoid\_subsumption.csv}
  & \texttt{test\_03\_monoid\_richness.png} \\

Absorbing ideal (0~violations)
  & \texttt{test\_04}
  & \texttt{results/raw/test\_04\_absorbing\_ideal.csv}
  & \texttt{test\_04\_absorption\_escape\_rates.png} \\

Holographic exactness
  & \texttt{test\_05}
  & \texttt{results/raw/test\_05\_holographic\_exactness.csv}
  & --- \\

Incremental consistency
  & \texttt{test\_06}
  & \texttt{results/raw/test\_06\_incremental\_consistency.csv}
  & --- \\

Scaling $O(n \log n)$
  & \texttt{test\_07}
  & \texttt{results/raw/test\_07\_scaling.csv}
  & \texttt{test\_07\_scaling\_loglog.png} \\

Divergence (super-linear)
  & \texttt{test\_08}
  & \texttt{results/raw/test\_08\_divergence\_raw.csv}
  & \texttt{test\_08\_divergence\_loglog.png} \\

Record process
  & \texttt{test\_09}
  & \texttt{results/raw/test\_09\_record\_process\_raw.csv}
  & \texttt{test\_09\_records\_and\_min\_gap.png} \\

Tropical shield
  & \texttt{test\_10}
  & \texttt{results/raw/test\_10\_tropical\_shield\_raw.csv}
  & \texttt{test\_10\_tropical\_shield\_policy\_compare.png} \\

Compression mirage
  & \texttt{test\_11}
  & \texttt{results/raw/test\_11\_mirage\_raw.csv}
  & \texttt{test\_11\_validity\_mirage.png} \\

Deterministic witness
  & \texttt{test\_12}
  & \texttt{results/raw/test\_12\_deterministic\_witness.csv}
  & \texttt{test\_12\_deterministic\_witness.png} \\

Contract compression
  & \texttt{test\_13}
  & \texttt{results/raw/test\_13\_contract\_compression\_raw.csv}
  & \texttt{test\_13\_contract\_vs\_naive.png} \\

Margin correlation
  & \texttt{test\_14}
  & \texttt{results/raw/test\_14\_margin\_correlation\_raw.csv}
  & \texttt{test\_14\_margin\_quartiles.png} \\

Adaptive compression
  & \texttt{test\_15}
  & \texttt{results/raw/test\_15\_adaptive\_compression\_raw.csv}
  & \texttt{test\_15\_adaptive\_vs\_uniform.png} \\

Organic traps (54.9\%)
  & \texttt{test\_16}
  & \texttt{results/raw/test\_16\_organic\_traps.csv}
  & \texttt{test\_16\_trap\_rate\_heatmaps.png} \\

Tropical streaming
  & \texttt{test\_17}
  & \texttt{results/raw/test\_17\_tropical\_streaming.csv}
  & \texttt{test\_17\_policy\_validity\_bar.png} \\

Transition depth
  & \texttt{test\_18}
  & \texttt{results/raw/test\_18\_transition\_vector\_raw.csv}
  & \texttt{test\_18\_transition\_vector\_depth.png} \\

5-model blackbox
  & ---
  & \texttt{release/results/blackbox\_bf16\_5model/*.csv}
  & --- \\

KV-cache eviction
  & ---
  & \texttt{release/results/kv\_cache\_eviction\_llama31\_8b/*.csv}
  & --- \\

NTSB incidents
  & ---
  & \texttt{results/ntsb/*.csv}
  & --- \\
\bottomrule
\end{tabular}
\end{small}
\label{tab:artifact-map}
