\documentclass[11pt,openany]{book}

% ── Typography ──────────────────────────────────────────────
\usepackage[T1]{fontenc}
\usepackage{mathpazo}          % Palatino for text and math
\usepackage{microtype}
\usepackage[margin=1in]{geometry}

% ── Mathematics ─────────────────────────────────────────────
\usepackage{amsmath,amsthm,amssymb,mathtools}

% ── Figures & Tables ────────────────────────────────────────
\usepackage{graphicx}
\usepackage{xcolor}
\usepackage{booktabs}
\usepackage{longtable}
\usepackage{tabularx}
\usepackage{float}

% ── Lists ───────────────────────────────────────────────────
\usepackage{enumitem}

% ── TikZ (for diagrams) ────────────────────────────────────
\usepackage{tikz}
\usetikzlibrary{arrows.meta,positioning,calc,shapes.geometric,decorations.pathreplacing}

% ── Cross-referencing ───────────────────────────────────────
\usepackage{hyperref}
\hypersetup{
  colorlinks=true,
  linkcolor=blue!60!black,
  citecolor=green!40!black,
  urlcolor=blue!60!black,
}
\usepackage[capitalise,noabbrev]{cleveref}

% ── Bibliography ────────────────────────────────────────────
\usepackage[numbers,sort&compress]{natbib}

% ── Algorithm ───────────────────────────────────────────────
\usepackage{algorithm}
\usepackage{algpseudocode}

% ── URLs ──────────────────────────────────────────────────────
\usepackage{url}

% ── Alias counters (shared numbering, distinct cleveref types) ──
\usepackage{aliascnt}

% ══════════════════════════════════════════════════════════════
%  Theorem environments
% ══════════════════════════════════════════════════════════════
\theoremstyle{plain}
\newtheorem{theorem}{Theorem}[chapter]

\newaliascnt{lemma}{theorem}
\newtheorem{lemma}[lemma]{Lemma}
\aliascntresetthe{lemma}

\newaliascnt{proposition}{theorem}
\newtheorem{proposition}[proposition]{Proposition}
\aliascntresetthe{proposition}

\newaliascnt{corollary}{theorem}
\newtheorem{corollary}[corollary]{Corollary}
\aliascntresetthe{corollary}

\theoremstyle{definition}
\newaliascnt{definition}{theorem}
\newtheorem{definition}[definition]{Definition}
\aliascntresetthe{definition}

\newaliascnt{example}{theorem}
\newtheorem{example}[example]{Example}
\aliascntresetthe{example}

\newtheorem{exercise}{Exercise}[chapter]

\theoremstyle{remark}
\newaliascnt{remark}{theorem}
\newtheorem{remark}[remark]{Remark}
\aliascntresetthe{remark}

% ── Cleveref type names ───────────────────────────────────
\crefname{theorem}{Theorem}{Theorems}
\Crefname{theorem}{Theorem}{Theorems}
\crefname{definition}{Definition}{Definitions}
\Crefname{definition}{Definition}{Definitions}
\crefname{lemma}{Lemma}{Lemmas}
\Crefname{lemma}{Lemma}{Lemmas}
\crefname{proposition}{Proposition}{Propositions}
\Crefname{proposition}{Proposition}{Propositions}
\crefname{example}{Example}{Examples}
\Crefname{example}{Example}{Examples}
\crefname{remark}{Remark}{Remarks}
\Crefname{remark}{Remark}{Remarks}
\crefname{corollary}{Corollary}{Corollaries}
\Crefname{corollary}{Corollary}{Corollaries}

% ══════════════════════════════════════════════════════════════
%  Custom commands
% ══════════════════════════════════════════════════════════════
\newcommand{\R}{\mathbb{R}}
\newcommand{\N}{\mathbb{N}}
\newcommand{\Z}{\mathbb{Z}}
% \oplus is a standard LaTeX command; do not redefine it.
\newcommand{\opendo}{\otimes_{\mathrm{endo}}}
\newcommand{\opcommit}{\otimes_{\mathrm{commit}}}
\newcommand{\absorb}{\bot}
\newcommand{\wstar}{{w^{\star}}}
\newcommand{\dtotal}{{d_{\mathrm{total}}}}
\newcommand{\dpre}[1][]{d_{\mathrm{pre}#1}}
\newcommand{\jdev}{{j_{\mathrm{dev}}}}
\newcommand{\peff}{{p_{\mathrm{eff}}}}
\newcommand{\mingap}{\mathrm{min\_gap}}
\newcommand{\tp}{\tau}

\DeclareMathOperator*{\argmax}{arg\,max}

% ── Figure path ─────────────────────────────────────────────
\graphicspath{{figures/}{../endogenous_context_theory/results/figures/}}

% ══════════════════════════════════════════════════════════════
%  Document
% ══════════════════════════════════════════════════════════════
\title{%
  \textbf{Context Algebra and the Validity Mirage}\\[6pt]
  \large Endogenous Semantics, Structural Failures,\\
  and What Validity Metrics Miss
}
\author{Jack Gaffney}
\date{2026}

\begin{document}

\frontmatter
\maketitle
\tableofcontents

% ══════════════════════════════════════════════════════════════
%  Preface
% ══════════════════════════════════════════════════════════════
\chapter*{Preface}
\addcontentsline{toc}{chapter}{Preface}
\markboth{Preface}{Preface}

% ──────────────────────────────────────────────────────────────
\section*{Why This Book Exists}
% ──────────────────────────────────────────────────────────────

\paragraph{The problem.}
Many sequence-generation systems must select a distinguished element---a
pivot, turning point, or structural anchor---by taking an argmax over the
very solution they are constructing.  This creates an \emph{endogenous
coupling}: the output depends on which pivot is chosen, yet the choice of
pivot depends on the output.  Standard validity metrics ask only ``does the
result satisfy the stated constraints?''\ and therefore miss a deeper
question: does the result \emph{mean what it was supposed to mean}?
Compression, truncation, and premature commitment can each produce outputs
that are technically valid---every constraint is satisfied, every grammar
rule obeyed---while the semantic content is quietly wrong.  The pivot has
shifted, the narrative has changed, and no constraint violation has been
raised.  We call this silent divergence the \emph{validity mirage}.

\paragraph{What we prove.}
This book develops a formal algebra of context that makes the mirage
visible and, in many cases, preventable.  The main results are:
\begin{enumerate}[label=(\alph*),itemsep=2pt]
  \item Under committed semantics, prefix-deficient states are
        \emph{absorbing}---no suffix can repair them
        (\cref{ch:absorbing-states,ch:absorbing-ideal}).
  \item The endogenous context monoid $(\mathcal{C},\opendo)$ is strictly
        associative, which enables $O(\log n)$ parallel reduction of
        context chains (\cref{ch:context-algebra}).
  \item Compression is the unique closure-breaking operation in the
        algebra: it is the only elementary edit that can force a valid
        context out of the feasibility set (\cref{ch:absorbing-ideal}).
  \item Under a commit-now policy, streaming oscillation traps affect
        54.9\% of organic sequences (\cref{ch:streaming}).
  \item Raw validity can remain at $1.0$ while pivot-preservation accuracy
        drops to $0.33$---the quantitative signature of the mirage
        (\cref{ch:mirage}).
\end{enumerate}

\noindent
These results consolidate and extend findings from four companion papers~\citep{gaffney2026absorbing,gaffney2026streaming,gaffney2026mirage,gaffney2026narrative}.

\paragraph{What practitioners should do differently.}
The algebra leads to concrete prescriptions.  First, measure
\emph{pivot preservation} and \emph{fixed-pivot feasibility} alongside raw
validity; without these metrics the mirage is invisible.  Second, use
\emph{contract-guarded compression}: wrap every lossy reduction in a
pre/post contract that checks whether the pivot has survived.  Third,
prefer \emph{deferred commitment} with a commitment fraction around
$f \approx 0.25$ over a commit-now policy when processing streams; the
streaming-trap analysis in \cref{ch:streaming} shows why early commitment
is so destructive.  Fourth, treat grammar constraints as
\emph{regularizers}---they shape the search space---rather than as
validators applied after the fact.  A constraint that fires only at the
end cannot prevent the absorbing states that form in the middle.

% ──────────────────────────────────────────────────────────────
\newpage
\section*{Map of the Ideas}
% ──────────────────────────────────────────────────────────────

The diagram below traces the logical dependencies among the book's central
concepts.  Each node names an idea and the chapter in which it is
developed; each arrow indicates that the source idea is a prerequisite for
the target.

\bigskip
\begin{center}
\begin{tikzpicture}[
    node distance=1.6cm and 2.4cm,
    every node/.style={
      draw,
      rounded corners=4pt,
      text width=3.6cm,
      minimum height=1.1cm,
      align=center,
      font=\small
    },
    arr/.style={
      -{Stealth[length=6pt]},
      thick,
      blue!60!black
    }
  ]

  % --- Row 1 ---
  \node (pivot)
    {Endogenous Pivots\\[2pt]\scriptsize Ch.\,1--2};

  % --- Row 2 ---
  \node (absorb) [below left=of pivot]
    {Absorbing States\\[2pt]\scriptsize Ch.\,3};
  \node (monoid) [below right=of pivot]
    {Context Monoid\\$(\mathcal{C},\opendo)$\\[2pt]\scriptsize Ch.\,5};

  % --- Row 3 ---
  \node (stream) [below=of absorb]
    {Streaming\\Commitment Traps\\[2pt]\scriptsize Ch.\,8};
  \node (ideal)  [below=of monoid]
    {Absorbing Ideal \&\\Compression Breaks\\Closure\\[2pt]\scriptsize Ch.\,7};

  % --- Row 4 ---
  \node (mirage) [below right=of stream, xshift=0.2cm]
    {Validity Mirage\\[2pt]\scriptsize Ch.\,9};

  % --- Arrows ---
  \draw[arr] (pivot)   -- (absorb);
  \draw[arr] (pivot)   -- (monoid);
  \draw[arr] (absorb)  -- (stream);
  \draw[arr] (absorb)  -- (ideal);
  \draw[arr] (monoid)  -- (ideal);
  \draw[arr] (stream)  -- (mirage);
  \draw[arr] (ideal)   -- (mirage);

\end{tikzpicture}
\end{center}
\bigskip

\noindent
Read top-to-bottom: endogenous pivots create the possibility of absorbing
states and motivate the context monoid.  Absorbing states feed into both
the streaming commitment analysis and the algebraic ideal theory.  The
context monoid likewise feeds the ideal theory.  Both the streaming traps
and the closure-breaking role of compression converge on the validity
mirage---the empirical demonstration that standard metrics hide real
failures.

% ──────────────────────────────────────────────────────────────
\section*{Reading Guide}
% ──────────────────────────────────────────────────────────────

The chapters divide naturally into three tracks.  Readers may follow a
single track or weave among them; the dependency diagram above shows which
material is prerequisite for which.

\paragraph{Mathematical core.}
\cref{ch:formal-problem} lays out the formal problem---solution spaces, pivot
functions, committed semantics.  \cref{ch:absorbing-states} proves the absorbing-state
theorem: once a prefix is deficient, no continuation can rescue it.
\cref{ch:context-algebra} introduces the context monoid $(\mathcal{C},\opendo)$,
establishes strict associativity, and derives the logarithmic parallel
reduction.  \cref{ch:tropical-lift} lifts the algebra into the tropical
semiring, connecting our discrete structures to optimization over
real-valued costs.  \cref{ch:absorbing-ideal} completes the algebraic picture by
characterizing the absorbing ideal and proving that compression is the
unique operation that breaks closure.

\paragraph{Empirical and experimental.}
\cref{ch:taxonomy} catalogues the failure modes that the theory predicts
and organizes them into a hierarchy, grounded in experiments on structured
generation tasks.  \cref{ch:streaming} measures oscillation traps in
streaming settings and quantifies the damage caused by premature
commitment.  \cref{ch:mirage} brings the threads together with the
headline experiment: raw validity held at $1.0$ while pivot preservation
collapsed, confirming the mirage across multiple domains.

\paragraph{Applied and motivational.}
\cref{ch:motivation} opens the book with concrete examples of silent
semantic failure---cases where a system ``passed all tests'' yet produced
the wrong answer.  \cref{ch:narrative} tells the narrative origin story of
the research programme, for readers who prefer to see where a theory came
from before studying its proofs.  \cref{ch:manifesto} distils the
practical upshot into a set of design principles.
\cref{ch:discussion} surveys open problems and directions for future work.

\bigskip
\noindent
We hope this book convinces you that validity is necessary but not
sufficient, and that the algebra developed here provides both the
language to articulate what is missing and the tools to fix it.

\vspace{1.5cm}
\begin{flushright}
  \textit{Jack Gaffney}\\
  \textit{February 2026}
\end{flushright}


\mainmatter
% ══════════════════════════════════════════════════════════════
%  Chapter 1 — The Endogenous Pivot Problem
% ══════════════════════════════════════════════════════════════
\chapter{The Endogenous Pivot Problem}\label{ch:motivation}

Every structured-generation system faces a quiet question that its
validity metrics never ask: \emph{is the output correct, or merely
well-formed?}  This chapter introduces the question through a concrete
failure, generalises the failure to a pattern, and then distils the
pattern into toy examples small enough to hold in one's head.  No
formalism is required yet---\cref{ch:formal-problem} will supply the definitions.
The goal here is intuition: the reader should finish this chapter
convinced that the problem is real, that it is not specific to any one
domain, and that standard validity checks are structurally incapable of
detecting it.

%% ═══════════════════════════════════════════════════════════════════
\section{Diana's Collapsed Arc}\label{sec:diana}
%% ═══════════════════════════════════════════════════════════════════

Consider a narrative simulation---a dinner party with six autonomous
agents, each pursuing its own goals, alliances, and evasions.  The
system, called Lorien, runs a large-language-model--driven simulation and
then extracts, for each agent, a \emph{narrative arc}: a selected
subsequence of events annotated with phase labels that obey a beat
grammar.  The grammar requires four phases in strict order:

\begin{center}
\textsc{setup}
\;\;$\longrightarrow$\;\;
\textsc{development}
\;\;$\longrightarrow$\;\;
\textsc{turning\_point}
\;\;$\longrightarrow$\;\;
\textsc{resolution}.
\end{center}

\noindent
The system selects a \emph{turning point}---the single highest-weight
focal event for the agent---and then verifies that the selected sequence
satisfies the beat grammar with at least $k$ development beats before
the turning point.  When the grammar is satisfied the arc is declared
valid.  When it is not, the arc is discarded.

Diana is a peripheral observer---an evader type.  She is not the
protagonist; she reacts to events more than she initiates them.  Across
50 random seeds of the same dinner-party scenario, the system extracts
arcs for all six agents.  For five of them, extraction succeeds
reliably.  For Diana, it fails in 9 out of 50 seeds.  All nine failures
share exactly the same signature:

\begin{enumerate}[label=(\roman*)]
  \item \textbf{Zero development beats.}\;
    The \textsc{development} phase is completely empty.  Not one
    complication, escalation, or tension-building event appears in the
    selected arc.

  \item \textbf{Extremely early turning point.}\;
    The turning point is anchored at a median normalised position of
    $0.13$ in the timeline---barely past the opening.  For comparison,
    the median position in valid arcs is $0.69$.  There is zero overlap
    in turning-point position between valid and invalid arcs.

  \item \textbf{Full candidate pools.}\;
    Diana is not starved of events.  Across the 50~seeds, a mean of
    $42.7$ events involve her.  The 9~invalid seeds contain a total of
    33 candidate events, every single one of which produces zero
    development beats.  The pools are entirely degenerate.
\end{enumerate}

\noindent
The mechanism is straightforward once we trace the pipeline.  Diana is a
peripheral agent, so her raw event pool is sparse.  To compensate, the
system injects protagonist events into her candidate pool---a
protagonist-event injection mechanism that ensures peripheral agents have
enough material to form arcs.  The injected events include high-tension
incidents from early in the simulation: confrontations, revelations,
catastrophes that drove the main plot.  These events carry large
narrative weights.

The arc-extraction algorithm performs a greedy search: it selects the
turning point as the highest-weight focal event in Diana's pool.
Because the injected protagonist events are both high-weight and
temporally early, the greedy search locks onto an early catastrophe as
the pivot.  The beat grammar is monotonic---once the turning point is
consumed, the \textsc{development} phase closes permanently.  With the
turning point at position~$0.13$, there is almost no timeline left
before it.  The grammar requires at least $k$ development beats before
the turning point, and with the pivot so early, this requirement is
impossible to meet.

The result is a ``story'' that reads: catastrophe happens immediately,
followed by sixteen consequence events.  No buildup, no complication, no
dramatic tension.  The grammar's structural requirements are technically
checkable---and in this case, the check correctly reports
failure---but the deeper problem is not that the grammar rejected the
arc.  The deeper problem is the \emph{mechanism}: the pivot selection
was determined by properties of the solution itself (which events happen
to be in Diana's pool), and the pivot's position then determined how
every other event was interpreted.  A different pool composition would
have produced a different pivot, a different phase labelling, and a
different arc---possibly a valid and meaningful one.

We will formalise this coupling in \cref{ch:formal-problem}.  For now, the
essential observation is: the system produced technically structured
output, but the output was meaningless.  The greedy search locked onto
the wrong pivot, and the grammar---designed to ensure narrative
quality---became the instrument of narrative collapse.

%% ═══════════════════════════════════════════════════════════════════
\section{The General Pattern}\label{sec:general-pattern}
%% ═══════════════════════════════════════════════════════════════════

Diana's collapsed arc is not a narrative-specific bug.  It is an
instance of a general vulnerability that arises whenever three
conditions hold simultaneously:

\begin{enumerate}[label=(\alph*),itemsep=4pt]
  \item \textbf{Endogenous pivot selection.}\;
    A distinguished element---a pivot, reference point, root cause,
    anchor---is selected by taking an $\argmax$ (or similar extremal
    operation) over the solution itself.

  \item \textbf{Pivot-dependent interpretation.}\;
    The chosen pivot determines how all other elements in the solution
    are interpreted: their phase labels, their causal roles, their
    structural classifications.

  \item \textbf{Structural constraints.}\;
    The interpreted solution must satisfy a set of structural
    constraints: a grammar, a schema, a protocol, a well-formedness
    condition.
\end{enumerate}

\noindent
When all three conditions hold, the pivot selection is
\emph{endogenous}---it depends on the content of the solution, which
creates a circular dependency.  The pivot determines the
interpretation, the interpretation determines whether the constraints
are satisfied, and the constraints determine what counts as a valid
solution.  If anything changes the solution---compression, truncation,
sampling, filtering---the pivot can shift, the interpretation can
flip, and the constraints can transition from satisfied to violated
(or vice versa) without any local signal that something has gone wrong.

The following three examples illustrate the pattern in domains far
removed from narrative.

%% ─────────────────────────────────────────────────────────────
\subsection{Incident Triage and Root-Cause Locking}
\label{sec:incident-triage}
%% ─────────────────────────────────────────────────────────────

Consider an automated incident-report system that analyses a production
outage involving 50 contributing factors.  The system selects a
\emph{root cause} by identifying the highest-severity factor.  All
remaining factors are then classified relative to the root cause:
\emph{precursor} (happened before and contributed to the root cause),
\emph{contributing factor} (amplified the root cause's impact), or
\emph{consequence} (resulted from the root cause).  The resulting causal
narrative must satisfy a report schema: at least two precursors, exactly
one root cause, and at least one consequence.

Now suppose the incident log is compressed---perhaps a context window is
truncated, or a summarisation step drops low-severity factors.  If the
compression removes context that would have surfaced a higher-severity
factor, the root cause locks onto a different factor.  The entire causal
narrative reorganises: events that were precursors become consequences,
consequences become precursors, and the report tells a structurally valid
but substantively wrong story.  The schema is satisfied.  No constraint
violation is raised.  The report is wrong.

%% ─────────────────────────────────────────────────────────────
\subsection{Constrained LLM Decoding and Schema Commitment}
\label{sec:constrained-decoding}
%% ─────────────────────────────────────────────────────────────

Grammar-constrained decoding systems---PICARD~\citep{scholak2021picard} for SQL, Outlines~\citep{willard2023outlines} for
structured JSON, guidance masks for schema-valid generation---commit to a
schema path early in the generation process.  This commitment is
functionally equivalent to selecting a structural pivot: once the model
has emitted enough tokens to determine which schema branch it is
following, all subsequent tokens are interpreted within that branch.

If the context window is truncated and the model loses information that
would have led to a different schema choice, it generates valid JSON (or
valid SQL, or valid YAML) with the wrong structure.  The output parses.
The schema validates.  The downstream system consumes it without error.
But the semantic content has silently shifted, because the structural
pivot---the schema branch---was selected endogenously from whatever
context happened to survive truncation.

%% ─────────────────────────────────────────────────────────────
\subsection{Process Mining and the Reference Activity}
\label{sec:process-mining}
%% ─────────────────────────────────────────────────────────────

In process mining~\citep{vanderaalst2016process}, analysts discover a process model from an event log
by selecting a \emph{reference activity} that anchors the temporal
alignment of all other activities.  If the reference activity is selected
endogenously---for instance, as the most frequent activity in the
log---then changing the log changes the reference.  Sampling the log
(taking a 10\% subsample for scalability), filtering by time window, or
removing infrequent traces can each shift the most-frequent activity to a
different event type.  When the reference shifts, every other activity is
realigned relative to the new anchor, and the discovered process model
changes---not because the underlying process changed, but because the
pivot moved.

\bigskip
\noindent
In each of these examples, the same three-part structure is at work:
endogenous selection of a pivot, pivot-dependent interpretation of
everything else, and structural constraints that can be satisfied by
multiple incompatible interpretations.  The output is valid.  The output
is wrong.  And no validity check can detect the discrepancy.

%% ═══════════════════════════════════════════════════════════════════
\section{Two Minimal Toy Examples}\label{sec:toy-examples}
%% ═══════════════════════════════════════════════════════════════════

To make the mechanics fully explicit, we now construct two toy examples
small enough to verify by hand.  Both use the narrative-arc grammar from
\cref{sec:diana}, but the phenomena they exhibit are instances of the
general pattern from \cref{sec:general-pattern}.

We use the following notation throughout.  An event $e_i$ has a weight
$w_i$ and a type: \emph{focal} (eligible to serve as the turning point)
or \emph{non-focal} (eligible for development, setup, or resolution
beats).  The turning point $\tp(S)$ is the focal event with the highest
weight in the solution~$S$.  The prefix requirement is $k$: at least $k$
non-focal events must precede the turning point.

%% ─────────────────────────────────────────────────────────────
\subsection{Example 1: Pivot Flip}\label{sec:pivot-flip}
%% ─────────────────────────────────────────────────────────────

Consider a six-event sequence with prefix requirement $k = 3$:

\medskip
\begin{center}
\begin{tabular}{lcccccc}
  \toprule
  Event   & $e_1$ & $e_2$ & $e_3$ & $e_4$ & $e_5$ & $e_6$ \\
  \midrule
  Weight  & 2     & 5     & 3     & 4     & 1     & 2     \\
  Type    & non-focal & focal & non-focal & non-focal & non-focal
          & non-focal \\
  \bottomrule
\end{tabular}
\end{center}
\medskip

\noindent
The only focal event is $e_2$, so the turning point is forced:
$\tp(S) = e_2$ with $\wstar = 5$.  The non-focal events preceding the
turning point are $\{e_1\}$, giving $\dpre = 1$.  Since
$\dpre = 1 < k = 3$, the arc is \textbf{invalid}.

Now extend the sequence by appending a single focal event:

\medskip
\begin{center}
\begin{tabular}{lccccccc}
  \toprule
  Event   & $e_1$ & $e_2$ & $e_3$ & $e_4$ & $e_5$ & $e_6$ & $e_7$ \\
  \midrule
  Weight  & 2     & 5     & 3     & 4     & 1     & 2     & 6 \\
  Type    & non-focal & focal & non-focal & non-focal & non-focal
          & non-focal & focal \\
  \bottomrule
\end{tabular}
\end{center}
\medskip

\noindent
Now there are two focal events: $e_2$ (weight~5) and $e_7$ (weight~6).
The turning point shifts to $e_7$: $\tp(S') = e_7$ with $\wstar = 6$.
The non-focal events preceding $e_7$ are $\{e_1, e_3, e_4, e_5, e_6\}$,
giving $\dpre = 5 \ge k = 3$.  The arc is \textbf{valid}.

Adding a single event flipped the turning point from $e_2$ to $e_7$ and
cascaded all phase labels from invalid to valid.  The five non-focal
events that were always present in the sequence---$e_3$ through
$e_6$---were invisible to the development phase when the pivot was
early, and fully available when the pivot moved late.  The events did not
change.  The grammar did not change.  Only the pivot changed, and the
entire structural interpretation followed.

%% ─────────────────────────────────────────────────────────────
\subsection{Example 2: Compression Mirage}\label{sec:compression-mirage}
%% ─────────────────────────────────────────────────────────────

Consider a ten-event sequence with $k = 3$:

\medskip
\begin{center}
\begin{tabular}{lcccccccccc}
  \toprule
  Event & $e_1$ & $e_2$ & $e_3$ & $e_4$ & $e_5$ & $e_6$ & $e_7$
        & $e_8$ & $e_9$ & $e_{10}$ \\
  \midrule
  Weight & 2 & 3 & 6 & 2 & 4 & 1 & 3 & 10 & 2 & 1 \\
  Type & nf & nf & f & nf & f & nf & nf & f & nf & nf \\
  \bottomrule
\end{tabular}
\end{center}
\medskip

\noindent
(Here ``f'' denotes focal and ``nf'' denotes non-focal.)

The highest-weight focal event is $e_8$ (weight~10), at normalised
position $0.8$ in the sequence.  The non-focal events preceding $e_8$
are $\{e_1, e_2, e_4, e_6, e_7\}$, giving $\dpre = 5 \ge k = 3$.  The
arc is valid.  Suppose the quality score---a weighted combination of
pivot weight and development richness---evaluates to $47.08$.

Now compress the sequence by removing two non-focal events from the
middle: drop $e_4$ and $e_6$.  The compressed sequence has eight events.
Under \emph{committed semantics}---the policy that keeps the original
pivot---$e_8$ remains the turning point.  But now the non-focal events
preceding $e_8$ are $\{e_1, e_2, e_7\}$, giving $\dpre = 3$.  In fact,
if the removal shifts indices and the grammar-aware count recalculates
to $\dpre = 2 < k = 3$, the committed pivot is \textbf{infeasible}: no
valid arc can be formed around $e_8$ in the compressed sequence.

An enumerative solver (searching up to $M = 10$ alternative pivot
candidates) looks for substitutes.  It considers $e_3$ (weight~6, at
position $0.3$): but with $e_3$ as pivot, $\dpre = 1$---still below
$k$.  It then considers $e_5$ (weight~4, at position $0.5$): with $e_5$
as pivot, $\dpre = 3 \ge k$.  Valid.  The solver accepts $e_5$ as a
substitute pivot.

But the quality score under the substitute pivot is only $21.47$.  The
\emph{semantic regret}---the fraction of quality lost by pivot
substitution---is
\[
  1 - \frac{21.47}{47.08} \;=\; 54.4\%.
\]
The system reports ``valid output.''  The grammar is satisfied.  The
phase labels are well-formed.  And the result has silently lost more than
half its semantic quality.  This is the \emph{validity mirage}: the
output looks valid because it \emph{is} valid, but the validity
conceals a catastrophic semantic shift.  The pivot moved, the story
changed, and the only metric that could detect the problem---pivot
preservation---was never checked.

%% ═══════════════════════════════════════════════════════════════════
\section{A Diagnostic Checklist}\label{sec:motivation-checklist}
%% ═══════════════════════════════════════════════════════════════════

\Cref{ch:mirage} develops a full diagnostic framework for detecting
and quantifying the validity mirage.  We preview four metrics here so
that the reader has a concrete target as the theory develops.

\begin{enumerate}[label=\textbf{(\arabic*)},itemsep=6pt]
  \item \textbf{Pivot preservation rate.}\;
    Did the compressed, streamed, or otherwise transformed solution
    retain the same turning point as the full (uncompressed,
    untruncated) solution?  Let $\tp(S)$ denote the pivot of the full
    solution and $\tp(S')$ the pivot of the transformed solution.
    Pivot preservation is the indicator
    $\mathbf{1}[\tp(S) = \tp(S')]$, aggregated as a rate across
    instances.  A rate of $1.0$ means the transformation never
    disturbed the pivot.  A rate below $1.0$ means pivots are shifting,
    and downstream interpretations are changing.

  \item \textbf{Fixed-pivot feasibility.}\;
    If we \emph{force} the original pivot $\tp(S)$ into the
    transformed solution~$S'$, is the output still valid?  That is,
    does there exist a valid phase labelling of~$S'$ with the
    turning point fixed at $\tp(S)$?  If not, the transformation has
    made the original semantic anchor infeasible---the system cannot
    even attempt to tell the same story.

  \item \textbf{Semantic regret.}\;
    When the pivot shifts from $\tp(S)$ to a substitute
    $\tp(S') \neq \tp(S)$, how much quality is lost?  Semantic regret
    is defined as
    \[
      R \;=\; 1 \;-\; \frac{Q\bigl(S',\, \tp(S')\bigr)}
                            {Q\bigl(S,\, \tp(S)\bigr)},
    \]
    where $Q$ is the quality scoring function.  A regret of $0$ means
    no quality loss; a regret of $0.544$ means $54.4\%$ of the
    original quality has been silently discarded.

  \item \textbf{Mirage gap.}\;
    The difference between raw validity and pivot preservation:
    \[
      \Delta_{\mathrm{mirage}}
      \;=\;
      \text{(raw validity rate)}
      \;-\;
      \text{(pivot preservation rate)}.
    \]
    A mirage gap of zero means validity and semantic fidelity are
    aligned---every valid output preserved the original pivot.  A
    large mirage gap means the system is hiding semantic drift behind
    superficial well-formedness.  In the headline experiment of
    \cref{ch:mirage}, raw validity holds steady at $1.0$ while pivot
    preservation drops to ${\sim}0.33$, producing a mirage gap of approximately $0.65$.
    The system appears to be working perfectly.  It is not.
\end{enumerate}

\bigskip
\noindent
The rest of this book develops the mathematical tools to analyse,
predict, and prevent these failures.  \Cref{ch:formal-problem} formalises the
solution space, the pivot function, and committed semantics.
\Cref{ch:absorbing-states} proves that certain structural states are
absorbing---once entered, no continuation can repair them.
\Cref{ch:context-algebra} introduces the context monoid that makes these
states algebraically tractable.  And \cref{ch:mirage} closes the loop
by demonstrating, on real and synthetic data, that the mirage is not a
theoretical curiosity but a measured, quantified, and reproducible
phenomenon.

\chapter{Formal Problem Setup}\label{ch:formal-problem}

This chapter lays the mathematical groundwork for the entire book.  We
introduce every object, every constraint, every operator, and every solver
semantic that the subsequent chapters will analyse, extend, or break.
Nothing here is assumed from prior chapters; every definition is
self-contained.  The reader who has internalised this chapter will have a
complete formal vocabulary for the endogenous pivot problem.

%% ═══════════════════════════════════════════════════════════════════
\section{The Event Graph}\label{sec:event-graph}
%% ═══════════════════════════════════════════════════════════════════

The basic data structure is a directed acyclic graph whose vertices are
events and whose edges encode causal or temporal dependencies.

\begin{definition}[Event graph]\label{def:event-graph}
An \textbf{event graph} is a tuple
\[
  G = (V,\, E,\, t,\, w,\, a),
\]
where:
\begin{enumerate}[label=(\roman*)]
  \item $V$ is a finite set of \textbf{events}.
  \item $E \subseteq V \times V$ is a set of directed \textbf{causal edges}.
        The pair $(u, v) \in E$ asserts that event $u$ is a causal
        antecedent of event~$v$.
  \item $t \colon V \to \R$ is a \textbf{timestamp function}.  For every
        edge $(u, v) \in E$ we require $t(u) < t(v)$, so that $(V, E)$
        is a DAG consistent with temporal ordering.
  \item $w \colon V \to \R_{\ge 0}$ is a \textbf{weight function}.  The
        value $w(v)$ measures the dramatic tension, importance, or
        information content of event~$v$.
  \item $a \colon V \to A$ is an \textbf{actor assignment}, where $A$ is
        a finite set of \textbf{agents} (characters, participants).  Each
        event is attributed to exactly one actor.
\end{enumerate}
\end{definition}

\begin{remark}[DAG structure]\label{rem:dag-structure}
The condition $t(u) < t(v)$ for every edge $(u, v) \in E$ guarantees
acyclicity: if a directed path $v_1 \to v_2 \to \cdots \to v_m$ exists,
then $t(v_1) < t(v_2) < \cdots < t(v_m)$, so $v_1 \neq v_m$ and no
directed cycle can form.  The timestamp function therefore provides a
topological ordering of the DAG.
\end{remark}

\paragraph{Deterministic tie-breaking.}
Multiple events may share the same weight.  To ensure that every
$\argmax$ operation in the sequel has a unique, deterministic outcome,
we impose a total order on events.

\begin{definition}[Event ordering]\label{def:event-ordering}
Every event $v \in V$ carries a unique identifier $\mathrm{id}(v) \in \N$.
Events are totally ordered by the lexicographic comparison on the tuple
\[
  \bigl(\, w(v),\; -t(v),\; \mathrm{id}(v) \,\bigr).
\]
That is, higher weight wins; among events of equal weight, the
\emph{later} event wins (since we negate the timestamp); and if both
weight and timestamp coincide, the event identifier breaks the tie.
\end{definition}

This ordering ensures that every invocation of $\argmax$ over any
non-empty subset of $V$ returns a single, well-defined event.

\begin{example}[A dinner-party event graph]\label{ex:dinner-party}
Consider a dinner-party simulation with $|A| = 6$ agents (Alice, Bob,
Carol, Dave, Eve, Frank) producing $|V| = 200$ events over the course
of an evening.  The event types include:
\begin{center}
\begin{tabular}{@{}llr@{}}
  \toprule
  Type & Description & Typical weight \\
  \midrule
  \textsc{Chat}        & Casual conversation    & $1$--$3$  \\
  \textsc{Observe}     & Noticing another actor  & $2$--$4$  \\
  \textsc{Conflict}    & Verbal disagreement     & $5$--$8$  \\
  \textsc{Catastrophe} & Major dramatic incident & $9$--$10$ \\
  \textsc{Reconcile}   & Resolution of conflict  & $4$--$6$  \\
  \bottomrule
\end{tabular}
\end{center}
Edges encode causal structure: a \textsc{Conflict} between Alice and Bob
has incoming edges from any preceding \textsc{Chat} or \textsc{Observe}
events that triggered it, and outgoing edges to subsequent
\textsc{Reconcile} events.  The resulting DAG has roughly $|E| \approx
350$ edges.  Weights reflect dramatic tension: a \textsc{Catastrophe}
(weight~$10$) dwarfs routine \textsc{Chat} events (weight~$1$--$3$).

The extraction problem, defined in \cref{sec:extraction-problem}, asks:
given a focal actor (say Alice), select a subset of events that tells
the most compelling story arc for Alice while obeying structural
grammar constraints.
\end{example}

%% ═══════════════════════════════════════════════════════════════════
\section{Focal Actor and Candidate Pool}\label{sec:focal-actor}
%% ═══════════════════════════════════════════════════════════════════

Not every event in the graph is relevant to every actor's story.  We
narrow the search space by fixing a protagonist and constructing the
set of events that could plausibly appear in that protagonist's
narrative arc.

\begin{definition}[Focal actor]\label{def:focal-actor}
The \textbf{focal actor} $a^* \in A$ is the agent whose story arc we
seek to extract.  An event $v \in V$ is called \textbf{focal} if
$a(v) = a^*$, and \textbf{non-focal} otherwise.
\end{definition}

\begin{definition}[Candidate pool]\label{def:candidate-pool}
Given a focal actor $a^*$ and an event graph $G$, the \textbf{candidate
pool} $P \subseteq V$ is constructed as follows.
\begin{enumerate}[label=\textbf{Step \arabic*.},leftmargin=3.5em]
  \item \textbf{Anchor selection.}\;
    Choose an anchor event $e_0 \in V$ with $a(e_0) = a^*$, typically
    the highest-weight focal event:
    \[
      e_0 = \argmax_{v \in V:\, a(v) = a^*} w(v).
    \]
  \item \textbf{Causal reachability.}\;
    Initialize $P \leftarrow \varnothing$.  Starting from $e_0$,
    perform a bidirectional BFS along the edges of $G$ (following both
    incoming and outgoing causal edges) to collect all events reachable
    from $e_0$ within the causal structure.  Add these events to $P$.
  \item \textbf{Focal injection.}\;
    Compute the focal actor's maximum-weight event:
    \[
      e^* = \argmax_{v \in V:\, a(v) = a^*} w(v).
    \]
    If $e^* \notin P$, add it: $P \leftarrow P \cup \{e^*\}$.
\end{enumerate}
\end{definition}

\begin{remark}[Structural necessity of injection]\label{rem:injection-necessity}
The injection in Step~3 is structurally necessary.  Without it, a
peripheral actor---one whose events are causally disconnected from the
main action---may have an empty or nearly empty candidate pool,
rendering the extraction problem trivial or vacuous.  Injection
guarantees that the candidate pool always contains at least one focal
event, namely the highest-weight one.

However, injection creates a subtle problem: the injected event $e^*$
may not be causally connected to the rest of the pool.  This
\emph{contamination} introduces a potential semantic discontinuity in
the extracted narrative.  We defer the full analysis of this issue to
\cref{ch:narrative}.
\end{remark}

\begin{remark}[Focal vs.\ non-focal events in $P$]\label{rem:focal-nonfocal}
The candidate pool $P$ typically contains both \textbf{focal events}
(involving $a^*$) and \textbf{non-focal events} (involving other actors
but causally connected to $a^*$'s story).  Non-focal events provide
context, setup, and development; focal events carry the protagonist's
direct actions and the potential turning points.
\end{remark}

%% ═══════════════════════════════════════════════════════════════════
\section{The Endogenous Turning Point}\label{sec:endogenous-tp}
%% ═══════════════════════════════════════════════════════════════════

The turning point---the single most dramatic moment in the
protagonist's arc---is not given exogenously.  It is determined by the
selection itself: the turning point is the highest-weight focal event
\emph{in the selected set}.  This endogenous coupling is the central
structural feature of the problem.

\begin{definition}[Endogenous turning point]\label{def:endogenous-tp}
Given a selected subset $S \subseteq P$, the \textbf{endogenous turning
point} is
\[
  \tp(S)
  \;=\;
  \argmax_{v \in S:\, a(v) = a^*} w(v),
\]
where ties are broken by the deterministic ordering of
\cref{def:event-ordering}.  If $S$ contains no focal events, $\tp(S)$
is undefined.
\end{definition}

The critical property of the endogenous turning point is that it
depends on $S$: changing which events are selected can change which
event serves as the turning point.

\begin{proposition}[Endogenous coupling]\label{prop:endogenous-coupling}
The turning-point function $\tp$ is not a constant of the problem
instance.  That is, there exist candidate pools $P$ and subsets
$S_1, S_2 \subseteq P$ such that $\tp(S_1) \neq \tp(S_2)$.
\end{proposition}

\begin{proof}
We construct an explicit example.  Let $P = \{e_1, e_2, e_3, e_4\}$
with the following attributes:
\begin{center}
\begin{tabular}{@{}cccc@{}}
  \toprule
  Event & Focal? & Weight & Timestamp \\
  \midrule
  $e_1$ & No  & 4 & 1 \\
  $e_2$ & Yes & 5 & 2 \\
  $e_3$ & Yes & 3 & 3 \\
  $e_4$ & Yes & 8 & 4 \\
  \bottomrule
\end{tabular}
\end{center}
Let $S_1 = \{e_1, e_2, e_3\}$.  The focal events in $S_1$ are $e_2$
(weight~5) and $e_3$ (weight~3), so $\tp(S_1) = e_2$.

Now let $S_2 = \{e_1, e_3, e_4\}$.  The focal events in $S_2$ are $e_3$
(weight~3) and $e_4$ (weight~8), so $\tp(S_2) = e_4$.

Since $\tp(S_1) = e_2 \neq e_4 = \tp(S_2)$, the turning point shifted
from $e_2$ to $e_4$ solely by changing the selected set.
\end{proof}

\begin{remark}[Why endogeneity matters]\label{rem:why-endogeneity}
In a standard constrained optimisation problem, the constraints are
fixed functions of the decision variables.  Here, the constraint
\emph{itself}---specifically, the identity of the turning point and the
resulting phase labelling---is a function of the solution.  This
circular dependency is what makes the extraction problem fundamentally
different from subset selection under monotone constraints, and it is
the root cause of every pathology analysed in this book.
\end{remark}

%% ═══════════════════════════════════════════════════════════════════
\section{The Phase Grammar}\label{sec:phase-grammar}
%% ═══════════════════════════════════════════════════════════════════

A well-formed narrative arc must progress through a sequence of dramatic
phases: setup, development, a turning point, and resolution.  We encode
this structural requirement as a deterministic finite automaton.

\begin{definition}[Phase grammar]\label{def:phase-grammar}
The \textbf{phase grammar} is a DFA
\[
  \mathcal{A} = (Q,\, \Sigma,\, \delta,\, q_0,\, F),
\]
with the following components.
\begin{enumerate}[label=(\roman*)]
  \item The \textbf{phase alphabet} is
        \[
          \Sigma = \{\,
            \textsc{Setup},\;
            \textsc{Development},\;
            \textsc{Turning\_Point},\;
            \textsc{Resolution}\,\}.
        \]
  \item The \textbf{state set} is
        \[
          Q = \{\,
            q_{\mathrm{setup}},\;
            q_{\mathrm{dev}},\;
            q_{\mathrm{tp}},\;
            q_{\mathrm{res}},\;
            q_{\mathrm{reject}}\,\}.
        \]
  \item The \textbf{initial state} is $q_0 = q_{\mathrm{setup}}$.
  \item The \textbf{accepting states} are $F = \{q_{\mathrm{res}}\}$.
  \item The \textbf{transition function} $\delta \colon Q \times \Sigma
        \to Q$ is defined by the following table, where any transition
        not listed sends the DFA to the reject state $q_{\mathrm{reject}}$,
        which is absorbing (all transitions from $q_{\mathrm{reject}}$
        lead back to $q_{\mathrm{reject}}$).
\end{enumerate}

\begin{center}
\begin{tabular}{@{}lll@{}}
  \toprule
  Current state & Input symbol & Next state \\
  \midrule
  $q_{\mathrm{setup}}$  & \textsc{Setup}          & $q_{\mathrm{setup}}$ \\
  $q_{\mathrm{setup}}$  & \textsc{Development}    & $q_{\mathrm{dev}}$ \\
  $q_{\mathrm{dev}}$    & \textsc{Development}    & $q_{\mathrm{dev}}$ \\
  $q_{\mathrm{dev}}$    & \textsc{Turning\_Point}  & $q_{\mathrm{tp}}$
    \quad\textnormal{(only if $\ge k$ \textsc{Development} symbols
    have been consumed)} \\
  $q_{\mathrm{tp}}$     & \textsc{Resolution}     & $q_{\mathrm{res}}$ \\
  $q_{\mathrm{res}}$    & \textsc{Resolution}     & $q_{\mathrm{res}}$ \\
  \bottomrule
\end{tabular}
\end{center}
The parameter $k \ge 1$ is the \textbf{prefix requirement}: the minimum
number of \textsc{Development} events that must appear before the
turning point.
\end{definition}

\begin{remark}[Monotonicity of transitions]\label{rem:monotonicity}
The DFA is \textbf{monotonic}: the implicit ordering
\[
  q_{\mathrm{setup}} \prec q_{\mathrm{dev}} \prec q_{\mathrm{tp}}
  \prec q_{\mathrm{res}}
\]
is never violated by a legal transition.  There is no edge from
$q_{\mathrm{dev}}$ back to $q_{\mathrm{setup}}$, no edge from
$q_{\mathrm{tp}}$ back to $q_{\mathrm{dev}}$, and no edge from
$q_{\mathrm{res}}$ to any earlier state.  Phase transitions only move
forward.  Any input that would require regression sends the DFA to the
absorbing reject state $q_{\mathrm{reject}}$.
\end{remark}

\begin{definition}[Grammar acceptance]\label{def:grammar-acceptance}
A sequence of phase labels $\sigma_1 \sigma_2 \cdots \sigma_n \in
\Sigma^*$ is \textbf{accepted} by $\mathcal{A}$ if and only if
\[
  \delta^*(q_0,\, \sigma_1 \sigma_2 \cdots \sigma_n) \in F,
\]
where $\delta^* \colon Q \times \Sigma^* \to Q$ is the extended
transition function defined recursively by $\delta^*(q, \varepsilon) = q$
and $\delta^*(q, \sigma_1 \cdots \sigma_n) = \delta^*(\delta(q,
\sigma_1),\, \sigma_2 \cdots \sigma_n)$.

Equivalently, a sequence is accepted if and only if it matches the
pattern
\[
  \textsc{Setup}^{\,*}\;
  \textsc{Development}^{\,\ge k}\;
  \textsc{Turning\_Point}\;
  \textsc{Resolution}^{\,\ge 1}.
\]
\end{definition}

\begin{example}[Accepted and rejected sequences]\label{ex:grammar-sequences}
Fix $k = 3$.  The following sequences illustrate the grammar:
\begin{enumerate}[label=(\alph*)]
  \item $\textsc{S},\, \textsc{D},\, \textsc{D},\, \textsc{D},\,
        \textsc{TP},\, \textsc{R}$: accepted (1~setup, 3~development,
        1~turning point, 1~resolution).
  \item $\textsc{D},\, \textsc{D},\, \textsc{D},\, \textsc{D},\,
        \textsc{TP},\, \textsc{R},\, \textsc{R}$: accepted (0~setup,
        4~development, 1~turning point, 2~resolution).
  \item $\textsc{S},\, \textsc{D},\, \textsc{D},\, \textsc{TP},\,
        \textsc{R}$: \textbf{rejected} --- only 2~development events
        before the turning point, but $k = 3$ is required.
  \item $\textsc{S},\, \textsc{D},\, \textsc{D},\, \textsc{D},\,
        \textsc{TP}$: \textbf{rejected} --- no resolution event after
        the turning point.
  \item $\textsc{S},\, \textsc{D},\, \textsc{TP},\, \textsc{D},\,
        \textsc{R}$: \textbf{rejected} --- a development event appears
        after the turning point, violating monotonicity.
\end{enumerate}
\end{example}

%% ═══════════════════════════════════════════════════════════════════
\section{The Phase Classifier}\label{sec:phase-classifier}
%% ═══════════════════════════════════════════════════════════════════

The grammar operates on phase labels, but the raw data consists of
events with timestamps and weights.  A \emph{phase classifier} bridges
the gap: it assigns each selected event a phase label based on its
temporal position relative to the turning point.

\begin{definition}[Position-based phase classifier]
\label{def:position-classifier}
Let $S \subseteq P$ be a selected subset with endogenous turning point
$\tp(S)$.  Order the events of $S$ by timestamp:
$v_1, v_2, \ldots, v_n$ with $t(v_1) \le t(v_2) \le \cdots \le t(v_n)$.
Let $\tp(S) = v_j$ for some index $j$.

The \textbf{position-based phase classifier}
$\varphi_{\tp} \colon S \to \Sigma$ assigns:
\[
  \varphi_{\tp}(v_i) =
  \begin{cases}
    \textsc{Setup} \text{ or } \textsc{Development}
      & \text{if } i < j, \\[3pt]
    \textsc{Turning\_Point}
      & \text{if } i = j, \\[3pt]
    \textsc{Resolution}
      & \text{if } i > j.
  \end{cases}
\]
For events before the turning point, the distinction between
\textsc{Setup} and \textsc{Development} is determined by a partition of
the pre-turning-point events: the first $s$ events (for some
$s \ge 0$) are labelled \textsc{Setup}, and the remaining events before
the turning point are labelled \textsc{Development}.
\end{definition}

\begin{definition}[Grammar-aware phase classifier]
\label{def:grammar-aware-classifier}
The \textbf{grammar-aware phase classifier} assigns phase labels to
maximise the probability of grammar acceptance.  Given a selected set
$S$ and turning point $\tp(S)$, it solves:
\[
  \varphi^*_{\tp}
  \;=\;
  \argmax_{\varphi \colon S \to \Sigma}
  \mathbf{1}\!\bigl[\,
    \delta^*(q_0,\, \varphi(v_1)\, \varphi(v_2)\, \cdots\, \varphi(v_n))
    \in F
  \,\bigr],
\]
subject to the constraint that $\varphi(v_j) = \textsc{Turning\_Point}$
where $v_j = \tp(S)$, and that the labelling respects the monotonic
phase ordering.

In practice, this reduces to choosing the optimal number of setup
events so that exactly $k$ or more development events appear before the
turning point.  The grammar-aware classifier resolves Class~B failures
(\cref{ch:taxonomy}) that arise when the position-based classifier
produces a label sequence that the DFA rejects.
\end{definition}

\begin{remark}[Circular dependency]\label{rem:circular-dependency}
Both classifiers depend on $\tp(S)$, which in turn depends on~$S$.  The
phase labels therefore depend on the selection, but the set of
\emph{valid} selections is precisely those sets whose phase labels
satisfy the grammar.  This circular dependency is the formal
manifestation of the endogenous coupling described informally in
\cref{prop:endogenous-coupling}: we cannot evaluate the constraint
without knowing the solution, and we cannot construct the solution
without evaluating the constraint.
\end{remark}

%% ═══════════════════════════════════════════════════════════════════
\section{The Extraction Problem}\label{sec:extraction-problem}
%% ═══════════════════════════════════════════════════════════════════

We now state the optimisation problem that the entire book studies.

\begin{definition}[Narrative extraction problem]\label{def:extraction-problem}
Given an event graph $G = (V, E, t, w, a)$, a focal actor $a^*$, a
candidate pool $P \subseteq V$, and a phase grammar $\mathcal{A}$ with
prefix requirement $k$, the \textbf{narrative extraction problem} is:
\begin{equation}\label{eq:extraction-problem}
  S^*
  \;=\;
  \argmax_{S \subseteq P}
  \sum_{v \in S} w(v)
\end{equation}
subject to:
\begin{enumerate}[label=(\alph*)]
  \item \textbf{Grammar constraint.}\;
    The phase-labelled sequence
    $\varphi_{\tp(S)}(v_1)\, \varphi_{\tp(S)}(v_2)\, \cdots\,
    \varphi_{\tp(S)}(v_{|S|})$ is accepted by $\mathcal{A}$,
    where $v_1, \ldots, v_{|S|}$ are the events of $S$ ordered by
    timestamp.
  \item \textbf{Cardinality bounds.}\;
    $n_{\min} \le |S| \le n_{\max}$, where $n_{\min}$ and $n_{\max}$
    are the minimum and maximum number of beats (events) in the
    extracted arc.
  \item \textbf{Protagonist coverage.}\;
    The fraction of focal events in $S$ is at least $\rho$:
    \[
      \frac{|\{v \in S : a(v) = a^*\}|}{|S|} \;\ge\; \rho.
    \]
  \item \textbf{Timespan constraint.}\;
    The extracted arc spans at least a fraction $\gamma$ of the total
    timeline:
    \[
      \frac{\max_{v \in S} t(v) - \min_{v \in S} t(v)}
           {\max_{v \in P} t(v) - \min_{v \in P} t(v)}
      \;\ge\; \gamma.
    \]
\end{enumerate}
\end{definition}

\begin{remark}[Endogenous constraint coupling]\label{rem:endogenous-constraint}
The grammar constraint~(a) is the source of all difficulty.  The
constraints~(b)--(d) are standard monotone set constraints: they can be
checked independently of $\tp(S)$ and do not exhibit endogenous
coupling.  Constraint~(a), by contrast, depends on $S$ through both the
selection itself (which events appear) and the turning point $\tp(S)$
(which determines the phase labelling).  The extraction problem is
therefore a \textbf{constrained combinatorial optimisation where the
constraint function depends on the solution through $\tp(S)$}.
\end{remark}

%% ═══════════════════════════════════════════════════════════════════
\section{Solver Semantics}\label{sec:solver-semantics}
%% ═══════════════════════════════════════════════════════════════════

The extraction problem admits multiple resolution strategies, differing
in how they handle the endogenous coupling.  We define four solver
semantics precisely.  Each corresponds to a distinct algorithmic
approach and to a distinct code path in the reference implementation.

%% ── 7.1  Committed ──────────────────────────────────────────────
\subsection{Committed Semantics ($M=1$)}\label{sec:committed}

\begin{definition}[Committed solver]\label{def:committed-semantics}
The \textbf{committed solver} proceeds in two stages:
\begin{enumerate}[label=\textbf{Stage \arabic*.}]
  \item \textbf{Pivot commitment.}\;
    Compute the turning point over the \emph{entire} candidate pool:
    \[
      \tp^{\mathrm{commit}}
      \;=\;
      \argmax_{v \in P:\, a(v) = a^*} w(v).
    \]
    This is the highest-weight focal event in $P$, determined before
    any selection takes place.
  \item \textbf{Constrained selection.}\;
    Fix $\tp^{\mathrm{commit}}$ as the turning point.  Find the
    subset $S^* \subseteq P$ that maximises $\sum_{v \in S} w(v)$
    subject to the grammar being satisfied with
    $\tp^{\mathrm{commit}}$ in the \textsc{Turning\_Point} role,
    plus all auxiliary constraints.
\end{enumerate}
The pivot identity is fixed at Stage~1 and \textbf{cannot be replaced}
during Stage~2.

\smallskip
\noindent
\textit{Code path:}
\texttt{solve\_with\_budget(events, k, M=1)} in
\texttt{src/compression.py}.
\end{definition}

%% ── 7.2  Endogenous composition ─────────────────────────────────
\subsection{Endogenous Composition Semantics}\label{sec:endo-comp-semantics}

\begin{definition}[Endogenous composition solver]
\label{def:endogenous-composition-semantics}
Under \textbf{endogenous composition}, the pivot is not fixed in
advance.  Events are processed in temporal order via a left fold.  At
each step, the current event is composed with the running context
element using the endogenous composition operator $\opendo$
(\cref{ch:context-algebra}).  If the new event is focal and has higher
weight than the current pivot, the pivot \textbf{shifts rightward} to
the new event.  The pivot is not locked until the full sequence has been
processed.

\smallskip
\noindent
\textit{Code path:}
\texttt{build\_tropical\_context()} in
\texttt{src/tropical\_semiring.py}.
\end{definition}

%% ── 7.3  Enumerative ────────────────────────────────────────────
\subsection{Enumerative Semantics (Top-$M$)}\label{sec:enumerative}

\begin{definition}[Enumerative solver]\label{def:enumerative-semantics}
The \textbf{enumerative solver} with parameter $M > 1$ proceeds as
follows:
\begin{enumerate}[label=\textbf{Step \arabic*.}]
  \item Rank all focal events in $P$ by weight.  Let
        $c_1, c_2, \ldots, c_M$ be the top-$M$ candidates (with
        $w(c_1) \ge w(c_2) \ge \cdots \ge w(c_M)$).
  \item For each candidate $c_i$, $i = 1, \ldots, M$: fix $c_i$ as
        the turning point and solve the constrained extraction problem
        from \cref{def:extraction-problem} with $\tp$ locked to $c_i$.
        Let $S_i^*$ denote the optimal solution for candidate~$c_i$
        (or $\varnothing$ if no feasible solution exists).
  \item Return the solution with the highest total weight:
        \[
          S^*
          \;=\;
          \argmax_{i \in \{1,\ldots,M\}:\, S_i^* \neq \varnothing}
          \sum_{v \in S_i^*} w(v).
        \]
\end{enumerate}

\smallskip
\noindent
\textit{Code path:}
\texttt{solve\_with\_budget(events, k, M=M)} in
\texttt{src/compression.py}, with $M > 1$.
\end{definition}

%% ── 7.4  Fixed-pivot ────────────────────────────────────────────
\subsection{Fixed-Pivot Semantics}\label{sec:fixed-pivot}

\begin{definition}[Fixed-pivot solver]\label{def:fixed-pivot}
The \textbf{fixed-pivot solver} is a diagnostic semantic.  It constrains
the solver to use the \textbf{full-sequence dominant pivot}: the turning
point that would be selected if the entire candidate pool $P$ were used
without any compression or truncation.  Formally:
\[
  \tp^{\mathrm{fixed}}
  \;=\;
  \argmax_{v \in P:\, a(v) = a^*} w(v).
\]
The solver then finds the best $S \subseteq P$ (or $S \subseteq P'$
for a compressed pool $P' \subseteq P$) with $\tp^{\mathrm{fixed}}$ as
the mandatory turning point.

The fixed-pivot semantic answers the question: \emph{can the original
narrative meaning be preserved under this compression?}  If the
fixed-pivot solver produces a feasible solution, the original turning
point survives.  If it does not, the compression has destroyed the
original meaning.
\end{definition}

%% ── 7.5  Relationships ─────────────────────────────────────────
\subsection{Relationships Between Semantics}\label{sec:semantics-relations}

\begin{remark}[The validity mirage and solver semantics]
\label{rem:mirage-semantics}
The relationship between these four semantics is the source of the
\textbf{validity mirage} analysed in \cref{ch:mirage}.  When the
committed or enumerative solver with $M > 1$ produces a valid solution,
the output satisfies every stated constraint.  But the turning point in
the output may differ from the full-sequence dominant pivot.  The
solution is \emph{technically valid but semantically different} from
what the full-sequence solution intended.

Concretely: the enumerative solver may find that candidate $c_1$
(the highest-weight focal event) cannot serve as turning point after
compression---there are too few development events before it.  It
then falls back to candidate $c_2$, which \emph{can} serve as turning
point.  The resulting arc is valid (it satisfies the grammar with
$c_2$ as turning point) but tells a \emph{different story} than the
original sequence.

The fixed-pivot semantic exposes this substitution: if the fixed-pivot
solver fails while the enumerative solver succeeds, a pivot
substitution has occurred.  This diagnostic is the principal tool for
detecting the validity mirage empirically.
\end{remark}

%% ═══════════════════════════════════════════════════════════════════
\section{The Greedy Policy}\label{sec:greedy-policy}
%% ═══════════════════════════════════════════════════════════════════

The simplest concrete algorithm for the extraction problem is the
greedy policy, which commits to a pivot and then fills slots in
weight-descending order.

\begin{definition}[Greedy extraction policy]\label{def:greedy-policy}
The \textbf{greedy policy} for the extraction problem operates as
follows.
\begin{enumerate}[label=\textbf{Step \arabic*.}]
  \item \textbf{Pivot pre-selection.}\;
    Compute $e^* = \argmax_{v \in P:\, a(v) = a^*} w(v)$.  Pre-select
    $e^*$ into the solution and assign it the label
    \textsc{Turning\_Point}.  Initialize $S \leftarrow \{e^*\}$.
  \item \textbf{Candidate ranking.}\;
    Sort the remaining events $P \setminus \{e^*\}$ in descending
    order of weight (using the deterministic tie-breaking of
    \cref{def:event-ordering}).
  \item \textbf{Greedy filling.}\;
    For each event $v$ in the sorted order:
    \begin{enumerate}[label=(\alph*)]
      \item Determine the best-available phase label for $v$ given the
            current DFA state and the temporal position of $v$ relative
            to $e^*$.
      \item If a valid label exists (i.e., adding $v$ with that label
            does not send the DFA to $q_{\mathrm{reject}}$), add $v$
            to $S$ and advance the DFA state.
      \item If no valid label exists, skip $v$.
    \end{enumerate}
    Continue until $|S| = n_{\max}$ or all candidates have been
    considered.
  \item \textbf{No backtracking.}\;
    Once an event is selected and labelled, the decision is
    \textbf{irrevocable}.  The greedy policy never reconsiders a
    previous choice.
\end{enumerate}
\end{definition}

\begin{remark}[Greedy as committed $M = 1$]\label{rem:greedy-committed}
The greedy policy is a \textbf{committed ($M = 1$) strategy} with the
additional property that it makes locally optimal weight choices at each
step.  The pivot is fixed at Step~1 (committed semantics), and the
remaining selections are made greedily.  The absence of backtracking
means the greedy policy can fail to find a feasible solution even when
one exists: it may consume all available development slots on
high-weight events that are temporally positioned after the turning
point, leaving too few development events before the turning point to
satisfy the prefix requirement~$k$.
\end{remark}

%% ═══════════════════════════════════════════════════════════════════
\section{Non-Matroid Structure}\label{sec:non-matroid}
%% ═══════════════════════════════════════════════════════════════════

The greedy policy has no performance guarantee for the extraction
problem.  This is not an artefact of our particular greedy strategy; it
is a structural property of the problem itself.  The feasible family
violates the axioms that would be needed to guarantee greedy optimality.

\begin{definition}[Feasible family]\label{def:feasible-family}
The \textbf{feasible family} for the extraction problem is
\[
  \mathcal{F}
  \;=\;
  \bigl\{\,
    S \subseteq P
    \;:\;
    \text{the phase-labelled sequence of } S
    \text{ is accepted by } \mathcal{A}
  \,\bigr\}.
\]
That is, $\mathcal{F}$ is the collection of all subsets of $P$ whose
induced phase labelling (using the endogenous turning point $\tp(S)$
and the grammar-aware classifier) satisfies the phase grammar.
\end{definition}

We recall the two axioms that define a matroid on ground set $P$.

\begin{itemize}
  \item \textbf{Hereditary axiom.}\; If $S \in \mathcal{F}$ and
        $S' \subseteq S$, then $S' \in \mathcal{F}$.
  \item \textbf{Exchange axiom.}\; If $S_1, S_2 \in \mathcal{F}$ with
        $|S_1| < |S_2|$, then there exists $v \in S_2 \setminus S_1$
        such that $S_1 \cup \{v\} \in \mathcal{F}$.
\end{itemize}

\begin{proposition}[Non-matroid structure]\label{prop:non-matroid}
The feasible family $\mathcal{F}$ violates both the hereditary axiom
and the exchange axiom.  Consequently, $\mathcal{F}$ is not a matroid,
not a greedoid, and not a polymatroid.
\end{proposition}

\begin{proof}
We prove each violation separately with explicit constructions.

\bigskip
\noindent\textbf{Part~1: Violation of the hereditary axiom.}

\smallskip
Fix $k = 3$.  Define the following six events, where the focal actor
is~$a^*$:
\begin{center}
\begin{tabular}{@{}ccccc@{}}
  \toprule
  Event & Actor & Weight & Timestamp & Intended role \\
  \midrule
  $e_s$    & other & 1 & 1 & Setup \\
  $e_{d1}$ & other & 2 & 2 & Development \\
  $e_{d2}$ & other & 2 & 3 & Development \\
  $e_{d3}$ & other & 2 & 4 & Development \\
  $e_{tp}$ & $a^*$ & 9 & 5 & Turning point \\
  $e_r$    & other & 3 & 6 & Resolution \\
  \bottomrule
\end{tabular}
\end{center}

Consider the full set
\[
  S = \{e_s,\, e_{d1},\, e_{d2},\, e_{d3},\, e_{tp},\, e_r\}.
\]
The event $e_{tp}$ is the sole focal event in $S$, so
$\tp(S) = e_{tp}$.  The phase-labelled sequence (in temporal order) is
\[
  \underbrace{\textsc{S}}_{e_s},\;\;
  \underbrace{\textsc{D}}_{e_{d1}},\;\;
  \underbrace{\textsc{D}}_{e_{d2}},\;\;
  \underbrace{\textsc{D}}_{e_{d3}},\;\;
  \underbrace{\textsc{TP}}_{e_{tp}},\;\;
  \underbrace{\textsc{R}}_{e_r}.
\]
This has 1~setup, 3~development events (meeting $k = 3$), 1~turning
point, and 1~resolution.  The DFA accepts.  Therefore
$S \in \mathcal{F}$.

Now remove $e_{d1}$ to form $S' = S \setminus \{e_{d1}\}$.  The
turning point is still $\tp(S') = e_{tp}$ (still the unique focal
event).  The phase-labelled sequence becomes
\[
  \underbrace{\textsc{S}}_{e_s},\;\;
  \underbrace{\textsc{D}}_{e_{d2}},\;\;
  \underbrace{\textsc{D}}_{e_{d3}},\;\;
  \underbrace{\textsc{TP}}_{e_{tp}},\;\;
  \underbrace{\textsc{R}}_{e_r}.
\]
Now there are only 2~development events before the turning point.  The
grammar-aware classifier cannot improve on this: regardless of whether
$e_s$ is labelled \textsc{Setup} or \textsc{Development}, the maximum
number of development events before $e_{tp}$ is~$3$ (if $e_s$ is
relabelled as \textsc{Development})---but wait: if we relabel $e_s$ as
\textsc{Development}, we obtain
$\textsc{D},\, \textsc{D},\, \textsc{D},\, \textsc{TP},\, \textsc{R}$,
which has exactly 3~development events and satisfies $k = 3$.

To close this escape route, we remove \emph{two} development events.
Let $S'' = S \setminus \{e_{d1}, e_{d2}\}$:
\[
  S'' = \{e_s,\, e_{d3},\, e_{tp},\, e_r\}.
\]
$\tp(S'') = e_{tp}$.  Before $e_{tp}$: events $e_s$ and $e_{d3}$.
Even relabelling both as \textsc{Development}, we get only
2~development events.  Since $k = 3$, the DFA rejects.
$S'' \notin \mathcal{F}$.

We have $S \in \mathcal{F}$ and $S'' \subset S$ with
$S'' \notin \mathcal{F}$.  The hereditary axiom is violated.

\bigskip
\noindent\textbf{Part~2: Violation of the exchange axiom.}

\smallskip
We use a grammar with both a minimum and maximum prefix requirement:
at least $k_{\min} = 3$ and at most $k_{\max} = 4$ development events
must appear before the turning point.  This bounded-prefix grammar is a
natural specialisation of the phase grammar (it adds an upper bound on
the development phase) and is used in practice to prevent excessively
long setup portions of a narrative arc.

Define the following events:
\begin{center}
\begin{tabular}{@{}cccc@{}}
  \toprule
  Event & Actor & Weight & Timestamp \\
  \midrule
  $c_1$  & other & 1 & 1 \\
  $c_2$  & other & 1 & 2 \\
  $c_3$  & other & 1 & 3 \\
  $q$    & $a^*$ & 9 & 3.5 \\
  $c_4$  & other & 1 & 4 \\
  $c_5$  & other & 1 & 5 \\
  $p$    & $a^*$ & 7 & 6 \\
  $r$    & other & 1 & 7 \\
  \bottomrule
\end{tabular}
\end{center}

\textit{Feasible set $S_1$.}\;
Let $S_1 = \{c_1,\, c_3,\, c_4,\, c_5,\, p,\, r\}$, with
$|S_1| = 6$.  The only focal event is $p$ (weight~7), so
$\tp(S_1) = p$ (timestamp~6).  Events before $p$ in temporal order:
$c_1(1),\, c_3(3),\, c_4(4),\, c_5(5)$---all non-focal, all labelled
\textsc{Development}.  That gives 4~development events:
$k_{\min} = 3 \le 4 \le k_{\max} = 4$.  After $p$: $r$ labelled
\textsc{Resolution}.  Phase sequence:
$\textsc{D},\, \textsc{D},\, \textsc{D},\, \textsc{D},\, \textsc{TP},\,
\textsc{R}$.
Accepted.  $S_1 \in \mathcal{F}$.

\smallskip
\textit{Feasible set $S_2$.}\;
Let $S_2 = \{c_1,\, c_2,\, c_3,\, q,\, c_4,\, c_5,\, p,\, r\}$, with
$|S_2| = 8$.  Focal events: $q$ (weight~9) and $p$ (weight~7).
$\tp(S_2) = q$ (timestamp~$3.5$).  Events before $q$:
$c_1(1),\, c_2(2),\, c_3(3)$---3~development events.
$k_{\min} = 3 \le 3 \le k_{\max} = 4$.  After $q$:
$c_4,\, c_5,\, p,\, r$ all labelled \textsc{Resolution}.  Phase
sequence:
$\textsc{D},\, \textsc{D},\, \textsc{D},\, \textsc{TP},\,
\textsc{R},\, \textsc{R},\, \textsc{R},\, \textsc{R}$.
Accepted.  $S_2 \in \mathcal{F}$.

\smallskip
\textit{Exchange check.}\;
We have $|S_1| = 6 < 8 = |S_2|$ and
$S_2 \setminus S_1 = \{c_2,\, q\}$.  The exchange axiom requires the
existence of some $v \in \{c_2, q\}$ with
$S_1 \cup \{v\} \in \mathcal{F}$.  We show that neither works.

\begin{enumerate}[label=(\roman*)]
  \item \textbf{Adding $q$} (focal, weight~9, timestamp~$3.5$) to
    $S_1$.  The augmented set is
    $S_1 \cup \{q\} = \{c_1, c_3, q, c_4, c_5, p, r\}$.
    Now $\tp(S_1 \cup \{q\}) = q$ because $w(q) = 9 > 7 = w(p)$---the
    pivot shifts from $p$ to $q$.  Events before $q$ (timestamp~$3.5$):
    $c_1(1)$ and $c_3(3)$ ---only 2~non-focal events.  Even labelling
    both as \textsc{Development}, we have $2 < k_{\min} = 3$.  The
    prefix requirement is \textbf{not met}.  The DFA rejects.
    $S_1 \cup \{q\} \notin \mathcal{F}$.

  \item \textbf{Adding $c_2$} (non-focal, timestamp~2) to $S_1$.
    The pivot remains $\tp(S_1 \cup \{c_2\}) = p$ (no new focal events).
    Events before $p$: $c_1(1),\, c_2(2),\, c_3(3),\, c_4(4),\, c_5(5)$
    ---5~development events.  But $k_{\max} = 4$, and $5 > 4$.  The
    grammar \textbf{rejects}: too many development events before the
    turning point.
    $S_1 \cup \{c_2\} \notin \mathcal{F}$.
\end{enumerate}

Neither element of $S_2 \setminus S_1$ can be added to $S_1$ to yield
a feasible set.  The exchange axiom is violated.
\end{proof}

\begin{remark}[Consequences of non-matroid structure]
\label{rem:non-matroid-consequences}
\cref{prop:non-matroid} has three immediate consequences:

\begin{enumerate}[label=(\arabic*)]
  \item \textbf{No greedy guarantee.}\;
    The classical result that the greedy algorithm is optimal for
    matroids (and, more generally, for greedoids) does not apply.  The
    greedy policy of \cref{def:greedy-policy} may produce arbitrarily
    suboptimal solutions.

  \item \textbf{No exchange property.}\;
    The exchange axiom failure means there is no guarantee that a small
    feasible set can be augmented element-by-element to a larger
    feasible set.  Local augmentation strategies---which underlie many
    combinatorial optimisation algorithms---cannot be relied upon.

  \item \textbf{No polymatroid structure.}\;
    Since the hereditary axiom fails, the feasible family is not even a
    simplicial complex, let alone a matroid or polymatroid.  The
    extraction problem falls outside the scope of submodular
    optimisation theory.
\end{enumerate}

These failures motivate the algorithmic hierarchy developed in
\cref{ch:taxonomy}: since no generic optimisation framework applies,
we must develop problem-specific methods tailored to the endogenous
coupling structure.
\end{remark}

% ══════════════════════════════════════════════════════════════
%  Chapter 3 — Absorbing States in Greedy Search
% ══════════════════════════════════════════════════════════════
\chapter{Absorbing States in Greedy Search}\label{ch:absorbing-states}

The greedy policy introduced in \cref{ch:formal-problem} selects a turning point by
taking the $\argmax$ over focal-actor weights and then fills the remaining
narrative phases around it.  This chapter shows that a simple counting
argument---comparing the number of development-eligible events against the
grammar's prefix requirement---yields an exact impossibility boundary.
Below this boundary, the greedy policy produces \emph{zero} valid
sequences; the product of the event graph with the phase grammar enters an
absorbing state from which no continuation can reach acceptance.

We proceed in stages.  \Cref{sec:dev-eligible} formalises the
development-eligible set and its count~$\jdev$.
\Cref{sec:phase-grammar-dfa} specifies the phase grammar as a
deterministic finite automaton and proves its monotonicity lemma.
\Cref{sec:product-automaton} constructs the product automaton that couples
the greedy walk with the grammar.
\Cref{sec:prefix-impossibility} states and proves the central result,
\cref{thm:prefix-impossibility}, in five self-contained steps.
\Cref{sec:absorbing-remarks} collects important remarks on the theorem's
scope.  \Cref{sec:empirical-absorbing} presents the empirical evidence,
and \cref{sec:exercises-absorbing} offers exercises.

% ──────────────────────────────────────────────────────────────
\section{Development-Eligible Events}\label{sec:dev-eligible}
% ──────────────────────────────────────────────────────────────

Throughout, let $G = (V, E, a, t, w)$ be an event graph on vertex set~$V$,
where $a(v)$ is the actor of event~$v$, $t(v)$ its timestamp, and $w(v)$
its narrative weight.  Fix a focal actor~$a^{*}$ and write
\[
  e^{*}
  \;=\;
  \argmax_{v \,:\, a(v) = a^{*}} w(v)
\]
for the \emph{forced turning point}---the event selected by the greedy
policy as the structural pivot.  Let $k \ge 1$ denote the grammar's
\emph{prefix requirement}: the minimum number of \textsc{development}
symbols that must be consumed before \textsc{turning\_point}.

\begin{definition}[Development-eligible set]\label{def:dev-eligible}
  Given the event graph~$G$, turning point~$e^{*}$, and prefix
  requirement~$k$, the \emph{candidate pool} is
  $P = V \setminus \{e^{*}\}$.  The \emph{development-eligible set} is
  \[
    D
    \;=\;
    \bigl\{\,
      v \in P
      \;\bigm|\;
      t(v) < t(e^{*})
      \;\text{and $v$ can receive the \textsc{development} label
              under the phase grammar}
    \,\bigr\}.
  \]
  The \emph{development-eligible count} is $\jdev = |D|$.
\end{definition}

The two conditions in the definition deserve emphasis.
Condition~(i), $t(v) < t(e^{*})$, restricts attention to events that
precede the turning point in the timeline; events after the turning point
cannot contribute \textsc{development} symbols under the monotonic grammar
(\cref{lem:dfa-monotonicity}).  Condition~(ii) encodes the
classifier--policy interaction: whether a given event \emph{can} receive
the \textsc{development} label depends on the classifier in use.
(The exclusion $v \neq e^{*}$ is already captured by the requirement
$v \in P = V \setminus \{e^{*}\}$.)

\begin{remark}[Classifier dependence of~$\jdev$]\label{rem:classifier-dep}
  The count~$\jdev$ depends on the classifier--policy interaction.
  Under a \emph{position-based classifier}, any event before the turning
  point in the timeline is eligible for \textsc{development}, so
  $\jdev = |\{v \in P : t(v) < t(e^{*})\}|$.  Under a
  \emph{grammar-aware classifier}, only events that the DFA can label as
  \textsc{development}---for instance, events belonging to non-focal
  actors---are eligible, and $\jdev$ may be strictly smaller.  This
  distinction is the source of Class~B failures analysed in
  \cref{ch:taxonomy}.
\end{remark}

\begin{example}[Position-based vs.\ grammar-aware counting]%
\label{ex:jdev-comparison}
  Consider a sequence of $n = 10$ events in which the turning
  point~$e^{*}$ occupies position~7 (in temporal order).  Among the six
  events preceding~$e^{*}$, two belong to the focal actor~$a^{*}$ and
  four belong to other actors.

  \begin{enumerate}[label=(\roman*)]
    \item \textbf{Position-based classifier.}
      Every event before the turning point is eligible:
      $\jdev = 6$.

    \item \textbf{Grammar-aware classifier}
      (requiring \textsc{development} events to be non-focal).
      Only the four non-focal events qualify:
      $\jdev = 4$.
  \end{enumerate}

  \noindent
  If the prefix requirement is $k = 5$, the position-based classifier
  gives $\jdev = 6 \ge k$ (no impossibility predicted), while the
  grammar-aware classifier gives $\jdev = 4 < k$
  (\cref{thm:prefix-impossibility} applies: greedy validity is exactly
  zero).
\end{example}

% ──────────────────────────────────────────────────────────────
\section{The Phase Grammar DFA}\label{sec:phase-grammar-dfa}
% ──────────────────────────────────────────────────────────────

We model the narrative phase grammar as a deterministic finite automaton
$\mathcal{A} = (Q,\, \Sigma,\, \delta,\, q_0,\, F)$ defined as follows.

\begin{definition}[Phase grammar DFA]\label{def:phase-dfa}
  \leavevmode
  \begin{itemize}[itemsep=4pt]
    \item \textbf{States.}\quad
      $Q = \{S_{\textsc{setup}},\;
             S_{\textsc{dev}},\;
             S_{\textsc{tp}},\;
             S_{\textsc{res}},\;
             S_{\textsc{reject}}\}$.

    \item \textbf{Alphabet.}\quad
      $\Sigma = \{\textsc{setup},\;
                   \textsc{development},\;
                   \textsc{turning\_point},\;
                   \textsc{resolution}\}$.

    \item \textbf{Initial state.}\quad $q_0 = S_{\textsc{setup}}$.

    \item \textbf{Accepting states.}\quad $F = \{S_{\textsc{res}}\}$.

    \item \textbf{Transition function~$\delta$.}\quad
      The transitions are \emph{monotonic}: phase labels must appear in
      non-decreasing order
      $\textsc{setup} \prec \textsc{development} \prec
       \textsc{turning\_point} \prec \textsc{resolution}$.
      Formally, the non-reject transitions are:

      \medskip
      \begin{center}
      \begin{tabular}{lll}
        \toprule
        \textbf{Current state} & \textbf{Symbol} & \textbf{Next state} \\
        \midrule
        $S_{\textsc{setup}}$  & \textsc{setup}          & $S_{\textsc{setup}}$ \\
        $S_{\textsc{setup}}$  & \textsc{development}    & $S_{\textsc{dev}}$   \\
        $S_{\textsc{setup}}$  & \textsc{turning\_point} & $S_{\textsc{tp}}$    \\
        $S_{\textsc{dev}}$    & \textsc{development}    & $S_{\textsc{dev}}$   \\
        $S_{\textsc{dev}}$    & \textsc{turning\_point} & $S_{\textsc{tp}}$    \\
        $S_{\textsc{tp}}$     & \textsc{resolution}     & $S_{\textsc{res}}$   \\
        $S_{\textsc{res}}$    & \textsc{resolution}     & $S_{\textsc{res}}$   \\
        \bottomrule
      \end{tabular}
      \end{center}
      \medskip

      \noindent
      Every $(q, \sigma)$ pair not listed above transitions to
      $S_{\textsc{reject}}$.  The state~$S_{\textsc{reject}}$ is
      \emph{absorbing}: $\delta(S_{\textsc{reject}}, \sigma)
      = S_{\textsc{reject}}$ for every $\sigma \in \Sigma$.

    \item \textbf{Prefix requirement.}\quad
      The automaton must consume at least $k$ copies of the symbol
      \textsc{development} before consuming \textsc{turning\_point} for
      the run to be valid.  Equivalently, one may augment the state space
      with a counter $c \in \{0, 1, \ldots, k\}$ and allow the transition
      $S_{\textsc{dev}} \xrightarrow{\textsc{turning\_point}} S_{\textsc{tp}}$
      only when $c \ge k$.  For readability, we keep the counter implicit
      and treat the prefix requirement as a side condition on acceptance.
  \end{itemize}
\end{definition}

The key structural property of this DFA is that phases advance
monotonically and can never retreat.  The following lemma isolates the
consequence that matters most for the impossibility theorem.

\begin{lemma}[DFA Monotonicity]\label{lem:dfa-monotonicity}
  Any state reached after consuming \textsc{turning\_point} is absorbing
  with respect to \textsc{development} transitions.  Formally, if
  \[
    \delta^{*}(q_0,\; \sigma_1 \cdots \sigma_m)
    \;\in\;
    \{S_{\textsc{tp}},\; S_{\textsc{res}}\},
  \]
  then for every continuation $\sigma_{m+1} \cdots \sigma_n$ with
  $\sigma_j = \textsc{development}$ for some $j \in \{m{+}1, \ldots, n\}$,
  we have
  \[
    \delta^{*}(q_0,\; \sigma_1 \cdots \sigma_n)
    \;=\;
    S_{\textsc{reject}}.
  \]
\end{lemma}

\begin{proof}
  We argue by inspection of the transition table in
  \cref{def:phase-dfa}.

  \emph{Case 1: the DFA is in $S_{\textsc{tp}}$.}\;
  The only symbol that leads to a non-reject state is \textsc{resolution},
  which transitions to~$S_{\textsc{res}}$.  Consuming \textsc{development}
  in $S_{\textsc{tp}}$ yields $S_{\textsc{reject}}$.

  \emph{Case 2: the DFA is in $S_{\textsc{res}}$.}\;
  The only symbol that keeps the DFA in a non-reject state is
  \textsc{resolution} (self-loop on $S_{\textsc{res}}$).
  Consuming \textsc{development} in $S_{\textsc{res}}$ yields
  $S_{\textsc{reject}}$.

  \emph{Case 3: the DFA is in $S_{\textsc{reject}}$.}\;
  By definition, $S_{\textsc{reject}}$ is absorbing:
  $\delta(S_{\textsc{reject}}, \sigma) = S_{\textsc{reject}}$ for all
  $\sigma \in \Sigma$.

  \medskip\noindent
  In every case, once \textsc{turning\_point} has been consumed, the DFA
  is in $S_{\textsc{tp}}$, $S_{\textsc{res}}$, or $S_{\textsc{reject}}$.
  From any of these states, consuming \textsc{development} leads
  (immediately or eventually) to~$S_{\textsc{reject}}$, from which
  acceptance is impossible.  Therefore the DFA never returns
  to~$S_{\textsc{dev}}$ after processing \textsc{turning\_point}.
\end{proof}

% ──────────────────────────────────────────────────────────────
\section{The Product Automaton}\label{sec:product-automaton}
% ──────────────────────────────────────────────────────────────

The impossibility proof couples the greedy policy's event-selection order
with the grammar DFA.  The vehicle for this coupling is a \emph{product
automaton}~$\mathcal{P}$.

\begin{definition}[Product automaton]\label{def:product-automaton}
  Let $G = (V, E, a, t, w)$ be an event graph with $|V| = n$, and let
  $\mathcal{A} = (Q, \Sigma, \delta, q_0, F)$ be the phase grammar DFA
  (\cref{def:phase-dfa}).  Linearise the event graph by the greedy
  policy's selection order: events are processed in decreasing weight
  order, $w(v_1) \ge w(v_2) \ge \cdots \ge w(v_n)$, with ties broken by
  a fixed deterministic rule (e.g., earliest timestamp first).

  The \emph{product automaton} is
  $\mathcal{P} = G \times \mathcal{A}$, with:
  \begin{itemize}[itemsep=3pt]
    \item \textbf{State space.}\;
      Pairs $(i,\, q)$ where $i \in \{0, 1, \ldots, n\}$ indexes the
      current position in the greedy selection order and
      $q \in Q$ is the DFA state.

    \item \textbf{Initial state.}\;
      $(0,\, q_0) = (0,\, S_{\textsc{setup}})$.

    \item \textbf{Transitions.}\;
      At position~$i$, the greedy policy selects event~$v_{i+1}$ and
      assigns it the ``best'' phase label~$\sigma_{i+1} \in \Sigma$ that
      keeps the DFA on a path toward acceptance.  The product automaton
      transitions from $(i, q)$ to
      $(i{+}1,\; \delta(q, \sigma_{i+1}))$.

    \item \textbf{Acceptance.}\;
      The product automaton accepts if and only if it reaches a state
      $(n, q_f)$ with $q_f \in F = \{S_{\textsc{res}}\}$ and the prefix
      requirement (at least $k$ \textsc{development} symbols before
      \textsc{turning\_point}) is satisfied.
  \end{itemize}
\end{definition}

Note that the greedy walk selects events in weight order, but phase
labels are assigned relative to the turning point's timestamp, which is
fixed at selection time.  An event classified as \textsc{development}
during greedy selection always has timestamp earlier than the turning
point's timestamp, by the event-ordering convention of
\cref{def:event-ordering}.

The greedy policy imposes one hard constraint on the walk
through~$\mathcal{P}$:

\begin{quote}
  \textbf{Turning-point pre-selection.}\;
  The event $e^{*} = \argmax_{v : a(v)=a^{*}} w(v)$ is pre-selected as
  the turning point.  When the walk reaches the position~$i^{*}$ at
  which $e^{*}$ appears in the weight-sorted order, the label must be
  $\sigma_{i^{*}} = \textsc{turning\_point}$.
\end{quote}

Between positions~$0$ and~$i^{*} - 1$, the greedy policy assigns each
event the best available label---typically \textsc{setup} or
\textsc{development}---that advances the DFA toward the prefix
requirement.  At position~$i^{*}$, the DFA consumes
\textsc{turning\_point} and transitions to~$S_{\textsc{tp}}$.  From
position~$i^{*} + 1$ onward, the only productive label is
\textsc{resolution}, by \cref{lem:dfa-monotonicity}.

The product automaton makes the proof's state-space argument explicit:
every run of the greedy policy corresponds to a unique path
through~$\mathcal{P}$, and the question of whether the greedy policy can
produce a valid sequence reduces to whether there exists an accepting path
in~$\mathcal{P}$.

% ──────────────────────────────────────────────────────────────
\section{The Prefix-Constraint Impossibility Theorem}%
\label{sec:prefix-impossibility}
% ──────────────────────────────────────────────────────────────

We are now in a position to state and prove the chapter's main result.

\begin{theorem}[Prefix-Constraint Impossibility]%
\label{thm:prefix-impossibility}
  Let $G$ be an event graph, $e^{*}$ the forced turning point, $k$ the
  prefix requirement, and $\jdev$ the development-eligible count
  (\cref{def:dev-eligible}).  If
  \[
    \jdev \;<\; k,
  \]
  then the greedy policy produces zero valid sequences.
\end{theorem}

\begin{proof}
  The proof proceeds in five steps.

  \medskip
  \noindent\textbf{Step 1 (Product automaton).}\;
  Construct the product automaton $\mathcal{P} = G \times \mathcal{A}$ as
  in \cref{def:product-automaton}.  Every run of the greedy policy traces
  a unique path through the state space
  $\{0, \ldots, n\} \times Q$.  The greedy policy produces a valid
  sequence if and only if this path ends in an accepting state~$(n, q_f)$
  with $q_f \in F$ and the prefix requirement satisfied.

  \medskip
  \noindent\textbf{Step 2 (Turning-point pre-selection).}\;
  The greedy policy pre-selects
  $e^{*} = \argmax_{v : a(v)=a^{*}} w(v)$
  as the turning point.  Let $i^{*}$ be the position of~$e^{*}$ in the
  weight-sorted selection order.  At step~$i^{*}$, the policy assigns the
  label $\sigma_{i^{*}} = \textsc{turning\_point}$, and the DFA
  transitions to~$S_{\textsc{tp}}$:
  \[
    \delta(q_{i^{*}-1},\; \textsc{turning\_point})
    \;=\;
    S_{\textsc{tp}},
  \]
  where $q_{i^{*}-1}$ is the DFA state upon entering step~$i^{*}$
  (necessarily $S_{\textsc{setup}}$ or $S_{\textsc{dev}}$ if the run has
  not already reached $S_{\textsc{reject}}$).

  \medskip
  \noindent\textbf{Step 3 (Post-TP absorption).}\;
  By the DFA Monotonicity Lemma (\cref{lem:dfa-monotonicity}), once the
  DFA enters $S_{\textsc{tp}}$, it can never return to~$S_{\textsc{dev}}$.
  Any subsequent \textsc{development} symbol sends the DFA
  to~$S_{\textsc{reject}}$, from which acceptance is impossible.
  Therefore, \emph{all} \textsc{development} symbols must be consumed
  \emph{before} step~$i^{*}$.

  \medskip
  \noindent\textbf{Step 4 (Counting development slots).}\;
  The events available to receive \textsc{development} labels before
  step~$i^{*}$ are exactly the members of the development-eligible
  set~$D$ from \cref{def:dev-eligible}.  Each such event can contribute at
  most one \textsc{development} symbol to the run.  Therefore, the maximum
  number of \textsc{development} symbols consumed before
  \textsc{turning\_point} is
  \[
    \#\{\textsc{development}\text{ before } \textsc{turning\_point}\}
    \;\le\;
    |D|
    \;=\;
    \jdev.
  \]

  \medskip
  \noindent\textbf{Step 5 (Prefix violation).}\;
  The grammar requires at least $k$ \textsc{development} symbols before
  \textsc{turning\_point} for the run to satisfy the prefix requirement.
  By Step~4, at most $\jdev$ such symbols can appear.  If $\jdev < k$,
  then
  \[
    \#\{\textsc{development}\text{ before } \textsc{turning\_point}\}
    \;\le\;
    \jdev
    \;<\;
    k.
  \]
  The prefix requirement is therefore violated on \emph{every} path
  through~$\mathcal{P}$.  No path can reach an accepting state with a
  satisfied prefix requirement.  Equivalently, the product automaton
  enters an absorbing region---no matter how the remaining events are
  labelled, acceptance is unreachable.

  \medskip\noindent
  Since every run of the greedy policy corresponds to a path
  through~$\mathcal{P}$, and no such path is accepting when $\jdev < k$,
  the greedy policy produces zero valid sequences.
\end{proof}

% ──────────────────────────────────────────────────────────────
\section{Important Remarks}\label{sec:absorbing-remarks}
% ──────────────────────────────────────────────────────────────

\begin{remark}[Converse is false]\label{rem:converse-false}
  The condition $\jdev \ge k$ does \emph{not} guarantee that the greedy
  policy succeeds.  Having enough development-eligible events is
  \emph{necessary} for validity but far from \emph{sufficient}.  The
  greedy policy may still fail for independent reasons: it may mis-order
  events so that the DFA enters $S_{\textsc{reject}}$ through a label
  conflict (Class~C failure), or it may select a turning point that is
  semantically incoherent even though the grammar is satisfied (Class~D
  failure).  Both of these failure modes are analysed in
  \cref{ch:taxonomy}.

  \Cref{thm:prefix-impossibility} therefore provides a
  \emph{sufficient condition for failure}, not a necessary one.  The
  failure boundary $\jdev = k$ is sharp in one direction only: below it,
  validity is exactly zero; above it, validity is possible but not
  guaranteed.
\end{remark}

\begin{remark}[Focal-only counting produces false positives]%
\label{rem:focal-only}
  A common error is to count only focal-actor events (those with
  $a(v) = a^{*}$) as development-eligible, ignoring events from other
  actors.  This under-counts the development-eligible set: if an event
  $v$ with $a(v) \neq a^{*}$ can receive the \textsc{development} label,
  it belongs in~$D$.

  Focal-only counting yields $\jdev^{\text{focal}} \le \jdev$, which may
  trigger a false positive prediction of impossibility: the focal-only
  count may satisfy $\jdev^{\text{focal}} < k$ even when $\jdev \ge k$.
  The correct count uses \emph{all} events eligible for
  \textsc{development}, regardless of actor.  This point is validated by
  the test assertions in the repository's test suite, which confirm that
  multi-actor events contribute to the development-eligible pool.
\end{remark}

\begin{remark}[Bridge to $\dpre$]\label{rem:bridge-dpre}
  In the \emph{unit-mass regime}---where each event contributes exactly
  one unit of development content---the development-eligible count
  coincides with the pre-pivot development mass:
  $\jdev = \dpre$.  The absorbing-state condition $\jdev < k$ then
  becomes $\dpre < k$, which is precisely the \emph{absorbing predicate}
  $\absorb$ from the context algebra developed in \cref{ch:absorbing-ideal}.

  This bridge connects the combinatorial argument of the present chapter
  with the algebraic framework: the absorbing ideal of
  \cref{ch:absorbing-ideal} is the algebraic closure of the impossibility region
  identified here.  When event masses are non-uniform, $\jdev$ and
  $\dpre$ may diverge, and the algebraic formulation
  (via~$\dpre$) provides the more refined characterisation.
\end{remark}

% ──────────────────────────────────────────────────────────────
\section{Empirical Validation}\label{sec:empirical-absorbing}
% ──────────────────────────────────────────────────────────────

The theoretical boundary $\jdev = k$ admits clean empirical verification.
We summarise the main findings~\citep{gaffney2026absorbing}; full experimental details appear in the
repository.

\paragraph{The boundary heatmap.}
\Cref{fig:kj-heatmap} displays the greedy policy's validity rate as a
function of the prefix requirement~$k$ (horizontal axis) and the
development-eligible count~$\jdev$ (vertical axis), aggregated over
$11{,}400$ event-graph instances.  The $k$--$\jdev$ plane divides into
three clearly delineated regions:

\begin{enumerate}[label=(\roman*)]
  \item \textbf{Impossibility zone} ($\jdev < k$, below the diagonal).\;
    Greedy validity is exactly $0.000$ across all $11{,}400$ instances in
    this region, confirming \cref{thm:prefix-impossibility} without
    exception.

  \item \textbf{Stochastic zone}
    ($k \le \jdev \lesssim 5.31k + 12.23$).\;
    Validity rises stochastically from~$0$ toward~$1.0$.  In this band,
    having enough development-eligible events is necessary but not
    sufficient; other failure modes (Classes~C and~D) depress the success
    rate.

  \item \textbf{High-validity zone}
    ($\jdev \gtrsim 5.31k + 12.23$).\;
    The 95\% success envelope lies approximately along the line
    $\jdev \approx 5.31k + 12.23$.  Above this envelope, the greedy
    policy almost always succeeds.
\end{enumerate}

\begin{figure}[t]
  \centering
  \includegraphics[width=0.75\textwidth]{figures/kj_heatmap.pdf}
  \caption{%
    Greedy validity rate in the $k$--$\jdev$ plane
    ($11{,}400$~instances).  The diagonal $\jdev = k$ (dashed white line)
    is the impossibility boundary of
    \cref{thm:prefix-impossibility}: every cell below the diagonal has
    validity exactly~$0.000$.  The solid white curve marks the 95\%
    success envelope, approximately $\jdev \approx 5.31k + 12.23$.
    Three regions emerge: the \emph{impossibility zone} (below
    diagonal), the \emph{stochastic zone} (between diagonal and
    envelope), and the \emph{high-validity zone} (above envelope).%
  }\label{fig:kj-heatmap}
\end{figure}

\paragraph{Exogenous turning-point control experiment.}
A natural question is whether the impossibility barrier is an artifact of
the grammar alone or whether it depends on the greedy policy's specific
method of turning-point selection.  To test this, we repeat the experiment
with an \emph{exogenous} turning point: instead of selecting
$e^{*} = \argmax_{v : a(v)=a^{*}} w(v)$, we fix the turning point at the median-timestamp focal event (rather than the max-weight focal event).

Under exogenous assignment, the diagonal barrier persists
\emph{structurally}---the proof of \cref{thm:prefix-impossibility} does
not depend on \emph{how} $e^{*}$ is chosen, only on the relationship
between $\jdev$ and~$k$ once $e^{*}$ is fixed.  However, the
\emph{probability} that a randomly generated instance falls below the
diagonal changes dramatically:
$P(\jdev < k)$ drops from $13.6\%$ under the greedy (argmax) policy to
$0.0\%$ under exogenous assignment.

The absorbing state is therefore activated by the \emph{interaction} of
weight-based turning-point selection with temporal front-loading of high-weight
events---not by the grammar alone.  When the turning point is the
highest-weight focal event, it tends to sit early in the weight-sorted
order, leaving few events before it in the timeline.  The grammar's prefix
requirement then becomes unsatisfiable.  Exogenous assignment breaks this
coupling, eliminating the pathway into the impossibility zone.

% ──────────────────────────────────────────────────────────────
\section{Exercises}\label{sec:exercises-absorbing}
% ──────────────────────────────────────────────────────────────

\begin{exercise}\label{exer:minimal-graph}
  Construct a minimal event graph with $n = 8$ events where
  \cref{thm:prefix-impossibility} predicts failure (i.e., $\jdev < k$
  for $k = 3$) and verify by exhaustive enumeration that no valid
  sequence exists under the greedy policy.

  \emph{Hint.}\;  Choose weights so that $e^{*}$ (the maximum-weight
  focal event) is at position~2 in the timeline, leaving at most two
  events before it.  Then $\jdev \le 2 < 3 = k$.  To verify, enumerate
  all possible label assignments to the eight events and confirm that
  every assignment either violates the phase grammar's monotonicity, fails
  the prefix requirement, or both.
\end{exercise}

\begin{exercise}\label{exer:relax-k}
  Show that relaxing the prefix requirement from $k = 3$ to $k = 1$
  changes the location of the impossibility boundary but not the structure
  of the proof.
  \begin{enumerate}[label=(\alph*)]
    \item State the new impossibility condition.
    \item Identify which step of the proof changes and which steps remain
          identical.
    \item Give an example of an event graph that is impossible under
          $k = 3$ but valid under $k = 1$.
  \end{enumerate}

  \emph{Solution sketch.}\;
  The new impossibility condition is $\jdev < 1$, i.e., $\jdev = 0$:
  there are no development-eligible events before the turning point.
  Steps~1--4 of the proof are identical; only Step~5 changes, replacing
  the bound $\jdev < 3$ with $\jdev < 1$.  Any event graph with
  $1 \le \jdev < 3$ is impossible under $k = 3$ but potentially valid
  under $k = 1$.
\end{exercise}

\begin{exercise}\label{exer:product-size}
  Prove that the product automaton~$\mathcal{P}$ from
  \cref{def:product-automaton} has at most $|V| \times |Q|$ states,
  giving an $O(n)$ bound on the state-space size (since $|Q| = 5$ is
  constant).

  \emph{Solution.}\;
  The state space of~$\mathcal{P}$ is
  $\{0, 1, \ldots, n\} \times Q$, which has $(n+1) \times |Q|$
  elements.  Since $|Q| = 5$ is a fixed constant independent of the
  event graph,
  \[
    |\text{States of } \mathcal{P}|
    \;=\;
    (n+1) \cdot 5
    \;=\;
    5n + 5
    \;\in\;
    O(n).
  \]
  Each step of the walk through~$\mathcal{P}$ processes one event and
  performs a constant-time DFA transition, so the entire walk takes
  $O(n)$ time.  The proof of \cref{thm:prefix-impossibility} thus
  operates on a linear-size state space, and the counting argument in
  Steps~4--5 requires no search---only a comparison between $\jdev$ and
  $k$, both computable in $O(n)$ time by a single pass over the event
  graph. \qed
\end{exercise}

\begin{exercise}\label{exer:non-monotonic}
  Consider a \emph{modified} phase grammar in which \textsc{development}
  events may appear after \textsc{turning\_point}.  Specifically, add the
  transitions
  \[
    \delta(S_{\textsc{tp}}, \textsc{development}) = S_{\textsc{tp}}
    \quad\text{and}\quad
    \delta(S_{\textsc{res}}, \textsc{development}) = S_{\textsc{res}}
  \]
  to the DFA of \cref{def:phase-dfa}.
  \begin{enumerate}[label=(\alph*)]
    \item Does \cref{thm:prefix-impossibility} still hold under this
          modified grammar?
    \item Identify the exact step in the proof that breaks and explain
          why.
    \item Give a concrete event graph where $\jdev < k$ under the
          original grammar's definition of~$D$, but the modified grammar
          admits a valid sequence.
  \end{enumerate}

  \emph{Solution sketch.}\;
  \begin{enumerate}[label=(\alph*)]
    \item No. \Cref{thm:prefix-impossibility} does \emph{not} hold under
          the modified grammar.

    \item Step~3 of the proof relies on the DFA Monotonicity Lemma
          (\cref{lem:dfa-monotonicity}), which states that no
          \textsc{development} symbol can be consumed after
          \textsc{turning\_point}.  The modified grammar explicitly
          violates this property: \textsc{development} is accepted in
          both $S_{\textsc{tp}}$ and $S_{\textsc{res}}$.  With
          monotonicity broken, \textsc{development} symbols can be
          consumed \emph{after} the turning point, so the count of
          development slots is no longer bounded by the pre-TP
          count~$\jdev$.  Step~4's upper bound becomes invalid, and
          Step~5's conclusion no longer follows.

    \item Consider an event graph with $n = 6$, $k = 3$,
          $e^{*}$ at position~2 in the timeline (so $\jdev = 1 < 3$).
          Under the original grammar, the greedy policy cannot produce a
          valid sequence.  Under the modified grammar, one can place one
          \textsc{development} event before $e^{*}$ and two
          \textsc{development} events after it, satisfying the prefix
          requirement of $k = 3$ total \textsc{development} symbols
          (now allowed both before and after the turning point).
          \qed
  \end{enumerate}
\end{exercise}

% ══════════════════════════════════════════════════════════════
%  Chapter 4 — Failure Taxonomy and Constructive Hierarchy
% ══════════════════════════════════════════════════════════════
\chapter{Failure Taxonomy and Constructive Hierarchy}\label{ch:taxonomy}

The absorbing-state theorem of \cref{ch:absorbing-states} gives a sharp
sufficient condition for greedy failure: when $\jdev < k$, no valid
sequence exists.  But the converse is false (\cref{rem:converse-false}).
When $\jdev \ge k$, the greedy policy may still fail---and it does so in
structurally distinct ways.  This chapter organises those failure modes
into a four-class taxonomy, shows that their fixes interact
antagonistically, and constructs a hierarchy of increasingly capable
algorithms that attack the failure classes in sequence.

We proceed as follows.  \Cref{sec:failure-classes} presents the taxonomy
(Classes~A--D) with a summary table and worked examples.
\Cref{sec:constraint-antagonism} documents the surprising result that
fixing one failure class can worsen another, establishing a partial order
among solvers.  \Cref{sec:constructive-hierarchy} lays out the full
algorithm hierarchy from myopic greedy to oracle.
\Cref{sec:tp-solver} details the TP-conditioned solver
(\cref{alg:tp-solver}) that occupies the practically important level of
the hierarchy.  \Cref{sec:failure-decomposition} derives the failure
decomposition formula that connects the taxonomy to the absorbing-state
theory.  \Cref{sec:approx-quality} analyses approximation quality,
confirming that the landscape is feasibility-dominated.
\Cref{sec:exercises-taxonomy} offers exercises.

% ──────────────────────────────────────────────────────────────
\section{Failure Classes}\label{sec:failure-classes}
% ──────────────────────────────────────────────────────────────

The empirical analysis of 138 greedy failures on bursty event-graph
instances reveals four structurally distinct failure mechanisms.  Each
class is defined by the stage of the pipeline at which the failure
manifests, the structural invariant that is violated, and the minimal
algorithmic intervention that repairs it.

\begin{table}[t]
\centering
\caption{Failure taxonomy for greedy narrative-arc construction.
         Prevalence figures refer to the bursty event-graph domain
         ($138$ greedy failures total).  Recovery rates are measured
         against the class-specific fix applied in isolation.}
\label{tab:failure-taxonomy}
\renewcommand{\arraystretch}{1.35}
\begin{tabular}{@{}clp{4.8cm}rlr@{}}
\toprule
\textbf{Class} & \textbf{Name} & \textbf{Mechanism} &
  \textbf{Prevalence} & \textbf{Fix} & \textbf{Recovery} \\
\midrule
A & Absorbing State &
  $\jdev < k$: turning point selected before enough
  development-eligible events exist.  Structural
  impossibility---no rearrangement can produce a valid sequence. &
  13.6\% & TP reassignment & --- \\[6pt]
B & Pipeline Coupling &
  Phase classifier misassigns labels.
  Position-based classification ignores the DFA state,
  mislabelling events that could serve as
  \textsc{development}. &
  45.7\% & Grammar-aware classifier & 100\% \\[6pt]
C & Commitment Timing &
  Greedy policy commits to bridge or span-critical
  events too early/late.  Wrong commitment window
  skips infrastructure events needed for temporal
  coverage. &
  13.0\% & TP-conditioned solver & 85--81\% \\[6pt]
D & Assembly Compression &
  Weight-maximising selection compresses phase
  structure.  High-weight events crowd into
  \textsc{resolution}, starving \textsc{development}. &
  41.3\% & Span-VAG with span filtering & 96.5\% \\
\bottomrule
\end{tabular}
\end{table}

\Cref{tab:failure-taxonomy} summarises the four classes.  We now examine
each in detail.

\subsection{Class A: Absorbing State Failures}\label{subsec:class-a}

Class~A failures are the domain of \cref{thm:prefix-impossibility}.
The mechanism is simple counting: the greedy policy selects a turning
point~$e^{*}$ that sits so early in the timeline---or, equivalently,
that has so few non-focal events preceding it---that the
development-eligible count $\jdev$ falls below the grammar's prefix
requirement~$k$.

\begin{example}[Class~A failure]\label{ex:class-a}
  Let $k = 3$ and suppose the turning point~$e^{*}$ occupies position
  $0.1$ on the normalised timeline (i.e., the first decile).  If only
  two non-focal events precede~$e^{*}$, then $\jdev = 2 < 3 = k$.  By
  \cref{thm:prefix-impossibility}, no valid sequence exists under this
  pivot assignment.  The product automaton enters an absorbing region
  from which no continuation can reach acceptance.

  The constructive fix is \emph{TP reassignment}: select a different
  focal event as the pivot---one that sits later in the timeline and
  affords $\jdev \ge k$.  This is the strategy employed by the
  TP-conditioned solver of \cref{sec:tp-solver}.
\end{example}

Class~A failures account for $13.6\%$ of bursty instances and represent
the hard boundary predicted by theory.  No algorithmic improvement to
the \emph{inner} solver can repair them; only a change of pivot (an
\emph{outer-loop} intervention) suffices.

\subsection{Class B: Pipeline Coupling Failures}\label{subsec:class-b}

Class~B failures arise not from the grammar itself but from the
\emph{classifier} that maps events to phase labels.  A position-based
classifier assigns labels based on an event's temporal position relative
to fixed thresholds (e.g., the first third of the timeline is
\textsc{setup}, the middle third is \textsc{development}).  This
classifier is blind to the DFA's current state: it may label an event as
\textsc{setup} even though the DFA has already transitioned to
$S_{\textsc{dev}}$ and would accept a \textsc{development} symbol.

\begin{example}[Class~B failure]\label{ex:class-b}
  Consider an event at normalised position $0.05$ in the timeline.  A
  position-based classifier labels it \textsc{setup} because it falls in
  the earliest segment.  However, the DFA has already consumed a
  \textsc{development} symbol (from a preceding event) and is in state
  $S_{\textsc{dev}}$.  A grammar-aware classifier would recognise this
  and correctly label the event as \textsc{development}, contributing to
  the prefix requirement.  The position-based classifier wastes a
  development opportunity, potentially pushing $\jdev$ below~$k$.
\end{example}

Class~B failures are the most prevalent single class, accounting for
$63$ of $138$ failures ($45.7\%$).  They are also the most cleanly
repairable: a grammar-aware classifier that tracks the DFA state
recovers $63/63$ ($100\%$) of Class~B failures.  This perfect recovery
rate reflects the fact that the underlying grammar structure is sound;
only the label assignment was wrong.

\subsection{Class C: Commitment Timing Failures}\label{subsec:class-c}

Class~C failures occur when the greedy policy's commitment order---the
sequence in which it irrevocably selects events---conflicts with the
structural requirements of the grammar.  The key scenario involves
\emph{bridge events}: events that span temporal gaps between clusters
(bursts) of activity.  These bridge events typically have low weight
(they are ``filler'' material, not high-salience focal events), so the
weight-maximising greedy policy skips them.  But without bridge events,
the selected set may violate \emph{span constraints}---requirements that
the narrative arc covers a minimum fraction of the simulation timeline.

\begin{example}[Class~C failure]\label{ex:class-c}
  A valid arc requires events spanning at least $15\%$ of the
  simulation timeline.  The greedy solver, maximising total weight,
  selects high-weight events clustered in a short interval
  (say, positions $0.40$--$0.55$, covering only $15\%$).  But the
  grammar also requires $k = 3$ development events before the turning
  point, and the highest-weight development candidates all fall in the
  same cluster.  The solver has committed to a dense, high-weight event
  set that satisfies the weight objective but violates the span
  constraint.

  A TP-conditioned solver that first fixes the turning point and then
  optimises subject to span requirements avoids this failure by
  selecting bridge events that maintain temporal coverage, even at a
  cost in total weight.
\end{example}

Class~C failures account for $18$ of $138$ failures ($13.0\%$).
Recovery rates across different constraint sets are $17/20$ ($85\%$) and
$21/26$ ($80.8\%$), reflecting the difficulty of balancing weight
maximisation against temporal coverage.

\subsection{Class D: Assembly Compression Failures}\label{subsec:class-d}

Class~D failures arise from a structural imbalance in the assembled
narrative arc.  The grammar is satisfied---the DFA accepts, and the
prefix requirement is met---but the phase structure is dramatically
lopsided.  Weight-maximising event selection concentrates high-weight
events in the \textsc{resolution} phase (post-turning-point), starving
the \textsc{development} phase of material.

\begin{example}[Class~D failure]\label{ex:class-d}
  Among $20$ selected events, $15$ have higher weight than the turning
  point and occur after it in the timeline.  Only $1$ event is assigned
  to \textsc{setup} and $2$ to \textsc{development}, barely meeting
  the prefix requirement of $k = 3$.  The grammar is technically
  satisfied, but the resulting arc is hollow: the narrative jumps almost
  immediately from setup to resolution with minimal development.

  Span-VAG (viability-aware greedy with span filtering) addresses this
  by filtering candidate events for span viability before weight
  maximisation.  Events that would compress the phase structure below a
  span threshold are excluded from the candidate pool.  Recovery rate:
  $55/57$ ($96.5\%$).
\end{example}

Class~D is the second most prevalent class at $57$ of $138$ failures
($41.3\%$), but it is also the most tractable: the span-VAG fix recovers
nearly all instances.

\begin{remark}[Failure classes are not mutually exclusive]%
\label{rem:overlap}
  A single failing instance may exhibit multiple failure classes
  simultaneously.  For example, a Class~B misclassification can reduce
  $\jdev$ to the point where a Class~A absorbing state is triggered.
  Similarly, the assembly compression of Class~D often co-occurs with
  the commitment timing issues of Class~C.  The prevalence figures in
  \cref{tab:failure-taxonomy} count each failure under its
  \emph{primary} class---the earliest stage at which a fix would have
  prevented the failure---and therefore sum to more than $100\%$ of
  unique failures.
\end{remark}

% ──────────────────────────────────────────────────────────────
\section{Constraint Antagonism}\label{sec:constraint-antagonism}
% ──────────────────────────────────────────────────────────────

The most surprising result in the failure analysis is not that different
algorithms fix different failure classes---that is expected.  The
surprise is that fixes for one class can \emph{worsen} another.

\paragraph{The Span-VAG paradox.}
Span-VAG achieves $62.7\%$ validity on bursty instances with gap
constraints, a substantial improvement over the myopic greedy baseline
of $54.7\%$.  Yet on \emph{multi-burst} instances with gap constraints,
Span-VAG achieves $0\%$ validity---strictly \emph{worse} than the
myopic greedy policy's $8\%$.

The mechanism is as follows.  Span viability favours selecting events at
burst endpoints to maximise temporal coverage of the narrative arc.  In
multi-burst topologies, temporal valleys separate the bursts.  The
low-weight bridge events in these valleys are essential for gap
feasibility: without them, the selected events have temporal gaps that
violate constraints.  But Span-VAG, by favouring endpoint events for
their span contribution, systematically skips these low-weight bridge
events.  The result is a selected set that has excellent span coverage
but catastrophic gap violations.

\paragraph{Partial order, not total order.}
This antagonism means that the algorithm hierarchy is a \emph{partial
order}, not a total order.  No single algorithm dominates across all
constraint combinations.  Span-VAG dominates greedy on bursty+gap but is
dominated by greedy on multi-burst+gap.  Gap-VAG dominates greedy on
both topologies but is incomparable with Span-VAG (better on
multi-burst+gap, different on bursty+gap).

\Cref{fig:partial-order} illustrates this partial order.  The directed
edges indicate strict dominance: algorithm~$X$ dominates algorithm~$Y$
if $X$ achieves weakly higher validity than $Y$ on every constraint
combination and strictly higher on at least one.  The key feature is the
pair of incomparable nodes---Span-VAG and Gap-VAG---which cannot be
ordered.  Both feed into BVAG (budget-aware VAG), which resolves the
antagonism by jointly optimising span and gap constraints.  The full
hierarchy culminates in the TP-Solver and, ultimately, the oracle.

\begin{figure}[t]
\centering
\begin{tikzpicture}[
    node distance=1.8cm and 2.8cm,
    every node/.style={
      draw,
      rounded corners=4pt,
      text width=2.6cm,
      minimum height=0.9cm,
      align=center,
      font=\small
    },
    arr/.style={
      -{Stealth[length=6pt]},
      thick,
      blue!60!black
    },
    incomp/.style={
      thick,
      red!60!black,
      dashed
    }
  ]

  % Level 0
  \node (greedy) {Myopic\\Greedy};

  % Level 1--2 (incomparable pair)
  \node (span) [above left=of greedy, yshift=0.5cm]
    {Span-VAG};
  \node (gap)  [above right=of greedy, yshift=0.5cm]
    {Gap-VAG};

  % Level 2.5
  \node (bvag) [above=3.2cm of greedy]
    {BVAG};

  % Level 3
  \node (tp)   [above=of bvag]
    {TP-Solver};

  % Level infinity
  \node (oracle) [above=of tp]
    {Oracle};

  % Dominance edges (upward = strictly better)
  \draw[arr] (greedy) -- (span);
  \draw[arr] (greedy) -- (gap);
  \draw[arr] (span)   -- (bvag);
  \draw[arr] (gap)    -- (bvag);
  \draw[arr] (bvag)   -- (tp);
  \draw[arr] (tp)     -- (oracle);

  % Incomparability indicator
  \draw[incomp] (span) -- node[draw=none, text width=1.8cm,
    fill=white, font=\scriptsize\itshape, minimum height=0pt]
    {incomparable} (gap);

\end{tikzpicture}
\caption{Partial order on solver algorithms.  Solid arrows indicate
         strict dominance (higher validity on every constraint
         combination).  The dashed line between Span-VAG and Gap-VAG
         indicates incomparability: neither dominates the other.  BVAG
         resolves the antagonism; the TP-Solver and Oracle complete the
         hierarchy.}
\label{fig:partial-order}
\end{figure}

% ──────────────────────────────────────────────────────────────
\section{The Constructive Hierarchy}\label{sec:constructive-hierarchy}
% ──────────────────────────────────────────────────────────────

\Cref{tab:hierarchy} presents the full algorithm hierarchy.  Each level
introduces one new capability that addresses a specific failure class (or
combination of classes) that the previous levels cannot handle.

\begin{table}[t]
\centering
\caption{Constructive algorithm hierarchy.  Validity rates are measured
         on the bursty+gap and multi-burst+gap constraint combinations.
         Complexity expressions use $n$ for pool size, $M$ for the
         number of TP candidates, $L$ for label-setting queue size,
         $d$ for DP state dimensions, $q$ for budget states, and $S$
         for the number of feasible subsets enumerated by the oracle.}
\label{tab:hierarchy}
\renewcommand{\arraystretch}{1.3}
\begin{tabular}{@{}clp{3.4cm}cp{0.9cm}p{1.1cm}l@{}}
\toprule
\textbf{Level} & \textbf{Algorithm} & \textbf{Description} &
  \textbf{TP Selection} & \textbf{B+G} & \textbf{MB+G} &
  \textbf{Complexity} \\
\midrule
0   & Myopic Greedy & Weight-max, no lookahead &
  Endogenous & 54.7\% & 8\%   & $O(n^{2})$ \\[3pt]
1   & Span-VAG      & Viability filter on span &
  Endogenous & 62.7\% & 0\%   & $O(n^{2})$ \\[3pt]
2   & Gap-VAG       & Gap-aware viability &
  Endogenous & 81.3\% & 60\%  & $O(n^{2} \!\cdot\! \mathit{pool})$ \\[3pt]
2.5 & BVAG          & Budget-aware VAG &
  Endogenous & 82.0\% & 60\%  & $O(n^{2} \!\cdot\! q)$ \\[3pt]
3   & TP-Solver     & DP over TP candidates &
  Outer loop & 96.0\% & 94\%  & $O(M \!\cdot\! L \!\cdot\! n^{2} \!\cdot\! d)$ \\[3pt]
$\infty$ & Oracle   & Exhaustive enumeration &
  All focal  & 99.3\% & 100\% & $O(n^{2} \!\cdot\! S)$ \\
\bottomrule
\end{tabular}
\end{table}

We now describe each level, what it fixes, and what it cannot fix.

\paragraph{Level 0: Myopic Greedy.}
The baseline policy selects events in decreasing weight order, assigns
phase labels greedily, and picks the turning point endogenously as
$e^{*} = \argmax_{v : a(v) = a^{*}} w(v)$.  It has no lookahead and no
constraint awareness beyond the DFA itself.  It fails on all four
classes: Class~A by selecting a bad pivot, Class~B by using a
position-based classifier, Class~C by ignoring commitment timing, and
Class~D by concentrating weight in the post-TP region.

\paragraph{Level 1: Span-VAG.}
Adds a viability filter that excludes candidate events whose inclusion
would compress the arc's temporal span below a threshold.  This
addresses Class~D failures (assembly compression) by ensuring that the
selected events maintain adequate temporal coverage.  However, the span
filter can antagonise gap constraints, producing the $0\%$ validity on
multi-burst+gap documented in \cref{sec:constraint-antagonism}.

\paragraph{Level 2: Gap-VAG.}
Replaces the span-only viability filter with a gap-aware filter that
checks both span and maximum temporal gap constraints.  This resolves
the Span-VAG antagonism on multi-burst topologies, lifting validity from
$0\%$ to $60\%$ on multi-burst+gap.  Gap-VAG also improves on Span-VAG
for bursty+gap ($81.3\%$ vs.\ $62.7\%$), making it the first algorithm
in the hierarchy that is not dominated on any tested constraint
combination.

\paragraph{Level 2.5: BVAG (Budget-Aware VAG).}
Augments Gap-VAG with a budget constraint that limits the number of
events allocated to each phase.  This provides a marginal improvement
($82.0\%$ vs.\ $81.3\%$ on bursty+gap) by preventing extreme phase
imbalance.  The gains are modest because Gap-VAG already handles most
assembly compression; the budget constraint catches the residual cases.

\paragraph{Level 3: TP-Solver ($M = 25$).}
The first algorithm in the hierarchy to use an \emph{outer loop} over
turning-point candidates.  Instead of accepting the endogenous
$\argmax$ pivot, the TP-Solver considers the top-$M$ focal events by
weight and, for each candidate, solves the inner problem (find the best
valid event selection given a fixed TP) via label-setting dynamic
programming.  This outer loop directly attacks Class~A failures: if the
highest-weight pivot yields $\jdev < k$, the solver tries the next
candidate, and the next, until a feasible pivot is found.

The TP-Solver also mitigates Class~C failures because fixing the TP
\emph{Markovises} the inner problem: once the pivot is known, every
event's phase label is determined by its temporal relationship to the
fixed TP.  The inner problem reduces to a resource-constrained shortest
path problem (RCSPP), which the label-setting DP solves efficiently with
dominance pruning.  Details appear in \cref{sec:tp-solver}.

\paragraph{Level $\infty$: Oracle.}
The oracle enumerates all feasible subsets and selects the one with
maximum total weight.  It serves as the theoretical upper bound: any
valid sequence that exists will be found.  The oracle achieves $99.3\%$
validity on bursty+gap (the residual $0.7\%$ represents instances where
no valid sequence exists under any pivot assignment) and $100\%$ on
multi-burst+gap.

\begin{remark}[The oracle gap]\label{rem:oracle-gap}
  The gap between the TP-Solver ($96.0\%$) and the oracle ($99.3\%$)
  on bursty+gap is $3.3$ percentage points.  This gap represents
  instances where: (i)~the correct pivot is not among the top-$M = 25$
  candidates, or (ii)~the label-setting DP's dominance criterion
  prunes the optimal solution.  Increasing $M$ narrows the gap at the
  cost of linear growth in runtime.  The $94\%$--$100\%$ gap on
  multi-burst+gap has a similar origin.
\end{remark}

% ──────────────────────────────────────────────────────────────
\section{The TP-Conditioned Solver}\label{sec:tp-solver}
% ──────────────────────────────────────────────────────────────

The TP-conditioned solver is the practically important algorithm in the
hierarchy: it achieves near-oracle validity at polynomial cost.
\Cref{alg:tp-solver} presents the pseudocode; the remainder of this
section provides a line-by-line explanation.

\begin{algorithm}[t]
\caption{TP-Conditioned Solver}
\label{alg:tp-solver}
\begin{algorithmic}[1]
\Require Event pool $P$, prefix requirement $k$, candidate count $M$
\Ensure Best valid event selection, or $\varnothing$ if none found
\Statex
\Function{TP\_Solver}{$P,\, k,\, M$}
  \State $\mathit{candidates} \gets$
         top-$M$ focal events by weight from $P$
         \label{line:candidates}
  \State $\mathit{best\_solution} \gets \varnothing$;\quad
         $\mathit{best\_score} \gets -\infty$
         \label{line:init}
  \For{each candidate $c$ in $\mathit{candidates}$}
         \label{line:outer-loop}
    \State \Comment{Fix $c$ as \textsc{turning\_point}---this
           Markovises the inner problem}
           \label{line:fix-comment}
    \State $\mathit{solution} \gets
           \Call{LabelSettingDP}{P,\, c,\, k}$
           \label{line:inner-dp}
    \If{$\mathit{solution}$ is valid \textbf{and}
        $\Call{Score}{\mathit{solution}} > \mathit{best\_score}$}
           \label{line:check}
      \State $\mathit{best\_solution} \gets \mathit{solution}$
             \label{line:update-sol}
      \State $\mathit{best\_score} \gets
             \Call{Score}{\mathit{solution}}$
             \label{line:update-score}
    \EndIf
  \EndFor
  \State \Return $\mathit{best\_solution}$
         \label{line:return}
\EndFunction
\end{algorithmic}
\end{algorithm}

\paragraph{Line-by-line explanation.}

\begin{description}[leftmargin=2em, labelindent=0em, itemsep=4pt]
  \item[Line~\ref{line:candidates}: Candidate extraction.]
    The solver selects the top-$M$ focal events by weight as
    turning-point candidates.  This is the outer loop's search space.
    Setting $M = 25$ captures the vast majority of viable pivots: in
    practice, the correct pivot is almost always among the $25$
    highest-weight focal events.

  \item[Line~\ref{line:init}: Initialisation.]
    The best solution found so far is empty, with score $-\infty$.

  \item[Line~\ref{line:outer-loop}: Outer loop over candidates.]
    For each candidate~$c$, the solver fixes $c$ as the turning point
    and invokes the inner DP.  This is the key structural move: by
    fixing the TP, the solver eliminates the endogenous coupling that
    makes the full problem hard.

  \item[Line~\ref{line:fix-comment}: Markovisation.]
    Fixing $c$ as the turning point \emph{Markovises} the inner
    problem.  The reason is that, once the pivot is known, every
    event's phase label is determined by a simple rule: events before
    $c$ in the timeline are candidates for \textsc{setup} or
    \textsc{development}; $c$ itself is \textsc{turning\_point}; events
    after $c$ are candidates for \textsc{resolution}.  The phase
    classifier becomes \emph{deterministic}---no longer dependent on
    the DFA's state trajectory.  The inner problem therefore reduces to
    a Resource-Constrained Shortest Path Problem (RCSPP): find the
    weight-maximising subset of events that satisfies the grammar's
    prefix requirement ($\ge k$ development events before $c$) and any
    additional span or gap constraints.

  \item[Line~\ref{line:inner-dp}: Label-Setting DP.]
    The inner solver is a label-setting dynamic program that explores
    the state space of partial event selections.  Each state is a
    tuple encoding the current score, the number of phase slots used,
    and the earliest timestamp in the selection.  The DP propagates
    labels (partial solutions) forward through the event pool, pruning
    dominated labels at each step.

    The dominance criterion is: label $A$ dominates label $B$ if and
    only if
    \begin{enumerate}[label=(\roman*),itemsep=2pt]
      \item $A.\mathit{score} \ge B.\mathit{score}$,
      \item $A.\mathit{slots\_used} \le B.\mathit{slots\_used}$, and
      \item $A.\mathit{first\_time} \le B.\mathit{first\_time}$.
    \end{enumerate}
    This three-dimensional dominance criterion prunes the search space
    dramatically.  A label that is worse on score, uses more phase
    slots, and starts later in the timeline can never lead to a
    solution that the dominating label could not also produce.

  \item[Lines~\ref{line:check}--\ref{line:update-score}: Solution
    update.]
    If the inner DP returns a valid solution with a higher score than
    the current best, the solver updates its record.

  \item[Line~\ref{line:return}: Return.]
    After exhausting all $M$ candidates, the solver returns the best
    valid solution found, or $\varnothing$ if no candidate yielded a
    valid selection.
\end{description}

\begin{remark}[Why fixing the TP Markovises the problem]%
\label{rem:markovisation}
  Under the greedy policy, the phase label assigned to an event depends
  on the DFA's state at the moment the event is processed, which in
  turn depends on all previously assigned labels.  This creates a
  sequential dependency that prevents decomposition.

  Fixing the TP breaks this dependency.  With the pivot~$c$ known, the
  timeline splits into three regions: pre-$c$, the event $c$ itself,
  and post-$c$.  Events in the pre-$c$ region compete for
  \textsc{setup} and \textsc{development} slots; events in the post-$c$
  region are assigned \textsc{resolution}.  The assignment of labels in
  the pre-$c$ region is independent of the post-$c$ region (and vice
  versa), conditioned on the fixed TP.  This conditional independence is
  precisely the Markov property: the future (post-TP) is independent of
  the past (pre-TP) given the present (the TP itself).
\end{remark}

\begin{remark}[Complexity]\label{rem:tp-complexity}
  The outer loop runs $M$ iterations.  Each iteration invokes the
  label-setting DP, which processes $n$ events, maintaining a label set
  of size at most $L$ at each event.  Each label has $d$ state
  dimensions.  The total work is $O(M \cdot L \cdot n^{2} \cdot d)$.
  In practice, dominance pruning keeps $L$ small (typically
  $L \ll n$), and $d = 3$ (score, slots used, first time), making the
  algorithm efficient for event pools of moderate size.
\end{remark}

% ──────────────────────────────────────────────────────────────
\section{Failure Decomposition}\label{sec:failure-decomposition}
% ──────────────────────────────────────────────────────────────

The four-class taxonomy induces a natural decomposition of the overall
failure probability.

\begin{proposition}[Failure decomposition]\label{prop:failure-decomp}
  Let $P(\mathrm{fail})$ denote the probability that a greedy policy
  fails to produce a valid sequence.  Then
  \begin{equation}\label{eq:failure-decomp}
    P(\mathrm{fail})
    \;=\;
    P(\jdev < k)
    \;+\;
    \bigl[1 - P(\jdev < k)\bigr]
    \;\cdot\;
    P\!\bigl(\mathrm{other\_fail} \;\big|\; \jdev \ge k\bigr).
  \end{equation}
\end{proposition}

\begin{proof}
  The law of total probability, partitioning on whether the
  absorbing-state condition $\jdev < k$ holds, gives
  \begin{align*}
    P(\mathrm{fail})
    &= P(\mathrm{fail} \mid \jdev < k)\, P(\jdev < k)
     + P(\mathrm{fail} \mid \jdev \ge k)\, P(\jdev \ge k).
  \end{align*}
  By \cref{thm:prefix-impossibility},
  $P(\mathrm{fail} \mid \jdev < k) = 1$: every instance with $\jdev < k$
  fails.  Substituting and writing
  $P(\mathrm{other\_fail} \mid \jdev \ge k)$ for the residual failure
  rate yields \cref{eq:failure-decomp}.
\end{proof}

The decomposition has a clean algorithmic interpretation.  Each level of
the constructive hierarchy attacks a different term:

\begin{itemize}[itemsep=4pt]
  \item \textbf{First term: $P(\jdev < k)$.}  These are Class~A
    (absorbing state) failures.  They are predicted exactly by
    \cref{thm:prefix-impossibility} and can only be attacked by
    changing the pivot---the strategy of the TP-Solver's outer loop.

  \item \textbf{Second term: residual failures given $\jdev \ge k$.}
    These are Classes~B, C, and~D, conditional on escaping absorption.
    The grammar-aware classifier eliminates Class~B entirely.
    Span-VAG and Gap-VAG attack Class~D.  The TP-Solver's inner DP,
    by Markovising the label-assignment problem, attacks Class~C.
\end{itemize}

\begin{remark}[Bridge to the absorbing ideal]%
\label{rem:bridge-ideal}
  The first term in the failure decomposition is precisely the measure
  of the absorbing ideal of \cref{ch:absorbing-ideal}: $P(\jdev < k)$
  is the probability that the greedy policy produces a context element
  in the absorbing set $\absorb$.  The decomposition thus connects the
  algebraic theory (the ideal is inescapable under commitment) to the
  empirical hierarchy (the TP-Solver attacks the ideal by exploring
  alternative pivots).
\end{remark}

% ──────────────────────────────────────────────────────────────
\section{Approximation Quality}\label{sec:approx-quality}
% ──────────────────────────────────────────────────────────────

When a valid sequence exists and is found by the solver, how close is
its quality to the optimum?  The answer is: remarkably close.

\begin{table}[t]
\centering
\caption{Approximation quality ratios (solver score divided by oracle
         score) for instances where both solver and oracle find valid
         sequences.}
\label{tab:approx-quality}
\renewcommand{\arraystretch}{1.25}
\begin{tabular}{@{}lcc@{}}
\toprule
\textbf{Topology} & \textbf{Mean ratio} & \textbf{Minimum ratio} \\
\midrule
Bursty       & 0.996  & 0.975  \\
Multi-burst  & 0.9905 & 0.9512 \\
\bottomrule
\end{tabular}
\end{table}

\Cref{tab:approx-quality} reports the approximation quality ratios
(solver score divided by oracle score) for instances where both the
TP-Solver and the oracle find valid sequences.  On bursty topologies,
the mean ratio is $0.996$ with a worst case of $0.975$: the solver
typically achieves within $0.4\%$ of optimal, and never worse than
$2.5\%$.  On multi-burst topologies, the mean ratio is $0.9905$ with a
worst case of $0.9512$: within $1\%$ on average and within $5\%$ in the
worst case.

\begin{remark}[Feasibility-dominated landscape]%
\label{rem:feasibility-dominated}
  These near-optimal quality ratios reveal a fundamental feature of the
  problem landscape: it is \emph{feasibility-dominated}.  The primary
  challenge is finding \emph{any} valid solution, not optimising among
  many.  Once a valid solution exists, its quality is typically within
  $1$--$5\%$ of optimal.

  This observation justifies the focus on structural feasibility
  throughout the constructive hierarchy.  The hierarchy's levels are
  ordered by their ability to \emph{find valid solutions}, not by their
  ability to optimise among valid solutions.  The approximation quality
  data confirm that feasibility is the binding constraint: solve the
  feasibility problem, and the optimisation problem largely solves
  itself.
\end{remark}

\begin{remark}[Implications for algorithm design]%
\label{rem:design-implications}
  The feasibility-dominated landscape has a practical implication for
  algorithm design: it is more profitable to invest computational budget
  in exploring additional TP candidates (increasing~$M$) than in
  refining the inner solver's optimisation.  Each additional TP
  candidate opens a new region of the feasibility space; a better inner
  solver merely polishes the score within a region that has already been
  found.  The marginal value of an extra candidate exceeds the marginal
  value of a tighter approximation.
\end{remark}

% ──────────────────────────────────────────────────────────────
\section{Exercises}\label{sec:exercises-taxonomy}
% ──────────────────────────────────────────────────────────────

\begin{exercise}[Classifying failures]\label{exer:classify}
  Consider an event graph with $n = 15$, $k = 3$, and a greedy
  turning point $e^{*}$ at temporal position $0.6$ (normalised).  The
  greedy solver selects $12$ events, of which $9$ are in the post-TP
  region.  The grammar is satisfied, but the arc's temporal span covers
  only $20\%$ of the timeline.
  \begin{enumerate}[label=(\alph*)]
    \item Which failure class does this instance primarily belong to?
    \item Which secondary class may also apply?  Explain.
    \item Which level of the constructive hierarchy is the minimum
          required to fix this failure?
  \end{enumerate}

  \emph{Hint:} The grammar is satisfied (ruling out Class~A and~B),
  but the phase structure is lopsided (Class~D) and the temporal span
  is narrow (which may also indicate Class~C commitment timing issues).
\end{exercise}

\begin{exercise}[Antagonism construction]\label{exer:antagonism}
  Construct a concrete multi-burst event graph (with two bursts
  separated by a valley) in which:
  \begin{enumerate}[label=(\alph*)]
    \item Span-VAG selects events at the endpoints of each burst,
          achieving good span coverage but violating a gap constraint
          (maximum gap $\le 0.15$ of the timeline).
    \item Myopic greedy, by not filtering for span, happens to include
          a low-weight bridge event in the valley, satisfying the gap
          constraint.
  \end{enumerate}
  Verify that Span-VAG fails while greedy succeeds on this instance.
\end{exercise}

\begin{exercise}[Markovisation]\label{exer:markov}
  Let $P = \{e_1, \ldots, e_8\}$ be an event pool with $k = 2$.
  Suppose the greedy policy selects $e_3$ as the turning point.
  \begin{enumerate}[label=(\alph*)]
    \item Write down the three regions induced by fixing $e_3$ as the
          TP: which events are pre-TP, which is the TP, and which are
          post-TP?
    \item Explain why the label assignment in the pre-TP region is
          independent of the post-TP region, given the fixed TP.
    \item Suppose $e_5$ has a higher weight than $e_3$.  Under
          endogenous TP selection, $e_5$ would be the pivot.  Explain
          why the TP-Solver can still consider $e_3$ and why this is
          valuable.
  \end{enumerate}
\end{exercise}

\begin{exercise}[Failure decomposition with real numbers]%
\label{exer:decomp-numbers}
  In a domain where $P(\jdev < k) = 0.136$ (Class~A) and the residual
  failure rate given $\jdev \ge k$ is $P(\mathrm{other\_fail} \mid
  \jdev \ge k) = 0.52$:
  \begin{enumerate}[label=(\alph*)]
    \item Compute the overall failure probability using
          \cref{eq:failure-decomp}.
    \item A grammar-aware classifier eliminates all Class~B failures,
          reducing the residual failure rate to $0.12$.  Compute the
          new overall failure probability.
    \item The TP-Solver eliminates $90\%$ of Class~A failures (by
          finding alternative pivots).  If we also use the grammar-aware
          classifier, what is the overall failure probability?
  \end{enumerate}

  \emph{Solution sketch for (a):}
  $P(\mathrm{fail}) = 0.136 + (1 - 0.136) \cdot 0.52
   = 0.136 + 0.864 \cdot 0.52 = 0.136 + 0.449 = 0.585$.
\end{exercise}

\begin{exercise}[Oracle gap analysis]\label{exer:oracle-gap}
  The TP-Solver with $M = 25$ achieves $96.0\%$ validity on bursty+gap,
  while the oracle achieves $99.3\%$.  The gap is $3.3\%$.
  \begin{enumerate}[label=(\alph*)]
    \item If the correct pivot is uniformly distributed among all focal
          events and there are $50$ focal events, what is the
          probability that the correct pivot is \emph{not} among the
          top-$25$ by weight?  (Assume the correct pivot is equally
          likely to be any focal event.)
    \item In practice, the correct pivot is highly correlated with
          weight.  If $95\%$ of correct pivots are among the top-$25$ by
          weight, and the inner DP recovers $98\%$ of these, what
          validity rate does the model predict?
    \item How does this compare with the observed $96.0\%$?
  \end{enumerate}
\end{exercise}

\chapter{Context Algebra}\label{ch:context-algebra}

This chapter develops the algebraic foundation on which the rest of the
theory is built~\citep{gaffney2026mirage}.  We introduce the \emph{context element}---a compact
summary of the structural state of a block of events---and define an
associative composition operator that merges adjacent blocks while
faithfully tracking the position of the global pivot.  The resulting
monoid is the engine that powers both the streaming left-fold
implementation and the parallel holographic tree.

%% ═══════════════════════════════════════════════════════════════════
\section{The Context Element}\label{sec:context-element}
%% ═══════════════════════════════════════════════════════════════════

Every contiguous block of events can be compressed into a triple that
records three structural quantities: where the strongest focal event
sits, how much development capacity the block contains in total, and how
much of that capacity falls before the strongest focal event.  These are
the only three numbers required to compose blocks correctly.

\begin{definition}[Context element]\label{def:context-element}
A \textbf{context element} is a triple
\[
  C \;=\; \bigl(\,\wstar,\; \dtotal,\; \dpre\,\bigr),
\]
where the components are defined as follows.
\begin{enumerate}[label=(\roman*)]
  \item $\wstar \in \R \cup \{-\infty\}$ is the \textbf{local pivot
    key}: the maximum weight among all focal-actor events in the block.
    If the block contains no focal events, $\wstar = -\infty$.
  \item $\dtotal \in \N$ is the \textbf{total development capacity}: the
    count of non-focal events in the block.  Each non-focal event
    contributes one unit of development capacity regardless of its
    weight or timestamp.
  \item $\dpre \in \N$ is the \textbf{pre-pivot development capacity}:
    the count of non-focal events whose timestamp strictly precedes the
    timestamp of the local pivot (the focal event attaining $\wstar$).
    When $\wstar = -\infty$ (no focal events), $\dpre$ is defined to be
    zero.
\end{enumerate}
\end{definition}

The intuition behind each component is direct:

\begin{itemize}
  \item \textbf{Pivot key $\wstar$.}\; The pivot is the focal event that
    the endogenous selection mechanism has chosen---the strongest signal
    in the block.  Its weight $\wstar$ determines whether this block's
    pivot can survive composition with adjacent blocks: if a neighbour
    contains a focal event with a strictly larger weight, the global
    pivot will shift to that neighbour.

  \item \textbf{Total development $\dtotal$.}\; The total count of
    non-focal events measures the raw amount of development capacity
    that the block contributes.  Under composition this quantity simply
    adds, since every non-focal event remains non-focal regardless of
    where the pivot falls.

  \item \textbf{Pre-pivot development $\dpre$.}\; This is the
    structurally subtle component.  The feasibility condition for the
    endogenous pivot (typically requiring at least $k$ non-focal events
    before the pivot) depends on how many non-focal events appear
    \emph{before} the global pivot in the combined sequence.  The
    quantity $\dpre$ records this count relative to the block's own
    local pivot, but as we shall see in \cref{sec:endogenous-composition},
    composition may promote additional events into the pre-pivot
    region when the global pivot shifts rightward.
\end{itemize}

\begin{example}[Concrete context element]\label{ex:context-element}
Consider a block of seven events arranged in temporal order:
\[
  \underbrace{e_1}_{\text{non-focal}},\;\;
  \underbrace{e_2}_{\substack{\text{focal}\\w=3}},\;\;
  \underbrace{e_3}_{\text{non-focal}},\;\;
  \underbrace{e_4}_{\text{non-focal}},\;\;
  \underbrace{e_5}_{\substack{\text{focal}\\w=7}},\;\;
  \underbrace{e_6}_{\text{non-focal}},\;\;
  \underbrace{e_7}_{\text{non-focal}}.
\]
The focal events are $e_2$ (weight~3) and $e_5$ (weight~7).  The local
pivot is $e_5$ since $7 > 3$, giving $\wstar = 7$.

The non-focal events are $e_1, e_3, e_4, e_6, e_7$, so
$\dtotal = 5$.

Among those five non-focal events, the ones with timestamps strictly
preceding $e_5$ are $e_1, e_3, e_4$---three events---giving
$\dpre = 3$.

Therefore the context element for this block is
\[
  C = (7,\; 5,\; 3).
\]
\end{example}

%% ═══════════════════════════════════════════════════════════════════
\section{Endogenous Composition}\label{sec:endogenous-composition}
%% ═══════════════════════════════════════════════════════════════════

We now define the binary operation that composes two adjacent context
elements---the left block $C_A$ followed in time by the right block
$C_B$---into a single context element representing the concatenated
sequence.

\begin{definition}[Endogenous composition $\opendo$]
\label{def:endogenous-composition}
Let $C_A = (\wstar_A,\; \dtotal_A,\; \dpre[{,A}])$ and
$C_B = (\wstar_B,\; \dtotal_B,\; \dpre[{,B}])$ be context elements.
Their \textbf{endogenous composition} is
\[
  C_A \opendo C_B
  \;=\;
  \bigl(\,\wstar,\;\dtotal,\;\dpre\,\bigr),
\]
where
\begin{align}
  \wstar   &= \max\!\bigl(\wstar_A,\; \wstar_B\bigr),
  \label{eq:comp-wstar}\\[4pt]
  \dtotal  &= \dtotal_A + \dtotal_B,
  \label{eq:comp-dtotal}\\[4pt]
  \dpre    &=
  \begin{cases}
    \dpre[{,A}]                   & \text{if } \wstar_A \ge \wstar_B, \\[3pt]
    \dtotal_A + \dpre[{,B}]       & \text{if } \wstar_B > \wstar_A.
  \end{cases}
  \label{eq:comp-dpre}
\end{align}
\end{definition}

The rules for $\wstar$ and $\dtotal$ are immediate: the global pivot key
is the maximum over both blocks, and total development capacity is
additive.  The rule for $\dpre$ requires more careful thought.

\begin{remark}[The pivot-shift insight]\label{rem:pivot-shift}
The $\dpre$ rule in \cref{eq:comp-dpre} encodes the following critical
structural insight.  There are two regimes:

\begin{enumerate}
  \item \textbf{Pivot stays in the left block}
    ($\wstar_A \ge \wstar_B$).\;
    The global pivot is the same event that was the local pivot of block
    $A$.  The non-focal events before the global pivot are exactly those
    that were before $A$'s local pivot.  Nothing in block $B$ can
    contribute to the pre-pivot count because the entire right block
    sits \emph{after} the pivot.  Hence $\dpre = \dpre[{,A}]$.

  \item \textbf{Pivot shifts to the right block}
    ($\wstar_B > \wstar_A$).\;
    The global pivot now lives inside block $B$.  Every event in block
    $A$---focal or non-focal---has a timestamp that precedes the global
    pivot.  In particular, \emph{all} $\dtotal_A$ non-focal events in
    block $A$ are now in the pre-pivot region.  Within block $B$ itself,
    the events before $B$'s local pivot are counted by $\dpre[{,B}]$.
    Therefore the total pre-pivot development count is
    $\dtotal_A + \dpre[{,B}]$.
\end{enumerate}
This asymmetry---the left block's \emph{entire} development capacity
gets promoted upon a rightward pivot shift, but the right block
contributes nothing additional when the pivot stays left---is the source
of non-commutativity (see \cref{sec:non-commutativity}).
\end{remark}

\begin{example}[Pivot shifts right]\label{ex:comp-right-shift}
Let $A = (3, 5, 2)$ and $B = (7, 4, 1)$.  Since $\wstar_B = 7 >
\wstar_A = 3$, the pivot shifts to block $B$.  Applying
\cref{def:endogenous-composition}:
\begin{align*}
  \wstar   &= \max(3, 7) = 7, \\
  \dtotal  &= 5 + 4 = 9, \\
  \dpre    &= \dtotal_A + \dpre[{,B}] = 5 + 1 = 6.
\end{align*}
Therefore $A \opendo B = (7, 9, 6)$.

The interpretation: all five of $A$'s non-focal events now sit before
the global pivot (which is inside $B$), contributing $5$ to the
pre-pivot count.  Within $B$, only one non-focal event precedes $B$'s
local pivot, contributing~$1$.  The total pre-pivot development capacity
is~$6$.
\end{example}

\begin{example}[Pivot stays left]\label{ex:comp-left-stays}
Now let $A = (7, 4, 1)$ and $B = (3, 5, 2)$.  Since $\wstar_A = 7 \ge
\wstar_B = 3$, the pivot stays in block $A$:
\begin{align*}
  \wstar   &= \max(7, 3) = 7, \\
  \dtotal  &= 4 + 5 = 9, \\
  \dpre    &= \dpre[{,A}] = 1.
\end{align*}
Therefore $A \opendo B = (7, 9, 1)$.

Here the global pivot remains in $A$, where only one non-focal event
preceded the local pivot.  The entire right block $B$ lies after the
pivot, so none of $B$'s events contribute to $\dpre$.
\end{example}

%% ═══════════════════════════════════════════════════════════════════
\section{Associativity}\label{sec:associativity}
%% ═══════════════════════════════════════════════════════════════════

The central algebraic property of $\opendo$ is associativity: we may
parenthesise a three-way composition in either order and obtain the same
result.  This is what allows us to decompose a long event sequence into
arbitrary contiguous blocks, compose each block independently, and then
combine them---the hallmark of a parallel-friendly reduction.

\begin{proposition}[Associativity of $\opendo$]\label{prop:associativity}
For any context elements $A$, $B$, and $C$,
\[
  (A \opendo B) \opendo C
  \;=\;
  A \opendo (B \opendo C).
\]
\end{proposition}

\begin{proof}
Write $A = (\wstar_A, d_A, p_A)$, $B = (\wstar_B, d_B, p_B)$, and
$C = (\wstar_C, d_C, p_C)$, where we abbreviate $\dtotal$ and $\dpre$
as $d$ and $p$ for readability.

The $\wstar$ component of both sides equals
$\max(\wstar_A, \wstar_B, \wstar_C)$ by associativity of $\max$.  The
$\dtotal$ component of both sides equals $d_A + d_B + d_C$ by
associativity of addition.  It remains to verify the $\dpre$ component.

We proceed by exhaustive case analysis on which block contains the
global pivot.

\bigskip
\noindent\textbf{Case~1: Global pivot in $A$}
($\wstar_A \ge \wstar_B$ and $\wstar_A \ge \wstar_C$).

\smallskip
\emph{Left-associated: $(A \opendo B) \opendo C$.}\;
Let $AB = A \opendo B$.  Since $\wstar_A \ge \wstar_B$, we have
$AB = (\wstar_A,\; d_A + d_B,\; p_A)$.
Then $AB \opendo C$: since $\wstar_A \ge \wstar_C$, the result is
$(\wstar_A,\; d_A + d_B + d_C,\; p_A)$.

\smallskip
\emph{Right-associated: $A \opendo (B \opendo C)$.}\;
Let $BC = B \opendo C$.  Two subcases arise:

\begin{enumerate}[label=(\alph*)]
  \item $\wstar_B \ge \wstar_C$:\; $BC = (\wstar_B,\; d_B + d_C,\; p_B)$.
    Then $A \opendo BC$: since $\wstar_A \ge \wstar_B$, the result is
    $(\wstar_A,\; d_A + d_B + d_C,\; p_A)$. \checkmark

  \item $\wstar_C > \wstar_B$:\; $BC = (\wstar_C,\; d_B + d_C,\;
    d_B + p_C)$.
    Then $A \opendo BC$: since $\wstar_A \ge \wstar_C$, the result is
    $(\wstar_A,\; d_A + d_B + d_C,\; p_A)$. \checkmark
\end{enumerate}
In both subcases the $\dpre$ component is $p_A$, matching the
left-associated result.

\bigskip
\noindent\textbf{Case~2: Global pivot in $B$}
($\wstar_B > \wstar_A$ and $\wstar_B \ge \wstar_C$).

\smallskip
\emph{Left-associated: $(A \opendo B) \opendo C$.}\;
Since $\wstar_B > \wstar_A$,
$AB = (\wstar_B,\; d_A + d_B,\; d_A + p_B)$.
Then $AB \opendo C$: since $\wstar_B \ge \wstar_C$, the result is
$(\wstar_B,\; d_A + d_B + d_C,\; d_A + p_B)$.

\smallskip
\emph{Right-associated: $A \opendo (B \opendo C)$.}\;
Since $\wstar_B \ge \wstar_C$,
$BC = (\wstar_B,\; d_B + d_C,\; p_B)$.
Then $A \opendo BC$: since $\wstar_B > \wstar_A$, the result is
$(\wstar_B,\; d_A + d_B + d_C,\; d_A + p_B)$. \checkmark

\medskip
\noindent\textbf{Key insight (why Case~2 is subtle).}\;
The quantity $d_A$ appears in $\dpre$ on both sides because it gets
``promoted'' into the pre-pivot region exactly once---when the global
pivot moves from the left sub-expression into block $B$.  In the
left-associated computation, this promotion happens during the
$A \opendo B$ step.  In the right-associated computation, it happens
during the $A \opendo BC$ step.  Either way, $d_A$ is promoted exactly
once and combined with $p_B$, yielding the same final
$\dpre = d_A + p_B$.  The single-promotion invariant is what makes
associativity non-obvious here, and it is the reason the operator is
\emph{not} commutative (see \cref{sec:non-commutativity}): the
direction of the pivot shift determines whether promotion occurs.

\bigskip
\noindent\textbf{Case~3: Global pivot in $C$}
($\wstar_C > \wstar_A$ and $\wstar_C > \wstar_B$).

\smallskip
\emph{Left-associated: $(A \opendo B) \opendo C$.}\;
First form $AB = A \opendo B$.  Regardless of whether $\wstar_A \ge
\wstar_B$ or $\wstar_B > \wstar_A$, the total development of $AB$ is
$d_{AB} = d_A + d_B$.
Then $AB \opendo C$: since $\wstar_C > \wstar_{AB}$, the pivot shifts
right, giving
\[
  \dpre = d_{AB} + p_C = d_A + d_B + p_C.
\]
The result is $(\wstar_C,\; d_A + d_B + d_C,\; d_A + d_B + p_C)$.

\smallskip
\emph{Right-associated: $A \opendo (B \opendo C)$.}\;
Since $\wstar_C > \wstar_B$,
$BC = (\wstar_C,\; d_B + d_C,\; d_B + p_C)$.
Then $A \opendo BC$: since $\wstar_C > \wstar_A$, the pivot shifts
right, giving
\[
  \dpre = d_A + (d_B + p_C) = d_A + d_B + p_C.
\]
The result is $(\wstar_C,\; d_A + d_B + d_C,\; d_A + d_B + p_C)$.
\checkmark

\medskip
All three cases yield identical results under both parenthesisations.
Since every possible ordering of $\wstar_A, \wstar_B, \wstar_C$ falls
into exactly one of these cases (with ties handled by the
$\ge$\,/\,$>$ structure), the proof is complete.
\end{proof}

We consolidate the result with a numerical verification of the subtle
Case~2.

\begin{example}[Numerical verification of Case~2]\label{ex:assoc-case2}
Let $A = (3, 5, 2)$, $B = (7, 4, 1)$, $C = (2, 3, 1)$.  The global
pivot is in $B$ since $\wstar_B = 7 > \wstar_A = 3$ and
$\wstar_B = 7 \ge \wstar_C = 2$.

\smallskip
\emph{Left-associated.}\;
$AB = A \opendo B$: since $7 > 3$, we get
$AB = (7,\; 5+4,\; 5+1) = (7, 9, 6)$.
Then $AB \opendo C$: since $7 \ge 2$, the result is
$(7,\; 9+3,\; 6) = (7, 12, 6)$.

\smallskip
\emph{Right-associated.}\;
$BC = B \opendo C$: since $7 \ge 2$, we get
$BC = (7,\; 4+3,\; 1) = (7, 7, 1)$.
Then $A \opendo BC$: since $7 > 3$, the result is
$(7,\; 5+7,\; 5+1) = (7, 12, 6)$.

\smallskip
Both sides yield $(7, 12, 6)$, confirming associativity. \qed
\end{example}

%% ═══════════════════════════════════════════════════════════════════
\section{Identity Element}\label{sec:identity}
%% ═══════════════════════════════════════════════════════════════════

\begin{proposition}[Identity element for $\opendo$]
\label{prop:identity}
The element
\[
  e \;=\; (-\infty,\; 0,\; 0)
\]
is a two-sided identity for $\opendo$: for every context element $C$,
\[
  C \opendo e = C
  \qquad\text{and}\qquad
  e \opendo C = C.
\]
\end{proposition}

\begin{proof}
Let $C = (\wstar, d, p)$ be an arbitrary context element.

\medskip
\noindent\textbf{Right identity: $C \opendo e = C$.}\;
Applying \cref{def:endogenous-composition} with $C_A = C$ and
$C_B = e = (-\infty, 0, 0)$:
\begin{align*}
  \wstar'  &= \max(\wstar,\; -\infty) = \wstar, \\
  \dtotal' &= d + 0 = d, \\
  \dpre'   &= p
    \quad\text{(since $\wstar \ge -\infty$, the first case of
    \cref{eq:comp-dpre} applies)}.
\end{align*}
Therefore $C \opendo e = (\wstar, d, p) = C$. \checkmark

\medskip
\noindent\textbf{Left identity: $e \opendo C = C$.}\;
Applying \cref{def:endogenous-composition} with $C_A = e = (-\infty, 0, 0)$
and $C_B = C$:
\begin{align*}
  \wstar'  &= \max(-\infty,\; \wstar) = \wstar, \\
  \dtotal' &= 0 + d = d.
\end{align*}
For $\dpre'$, we consider two subcases.  If $\wstar = -\infty$, then
$\wstar_A = \wstar_B = -\infty$, so $\wstar_A \ge \wstar_B$ and
$\dpre' = \dpre[{,A}] = 0 = p$ (since $p = 0$ when $\wstar = -\infty$
by \cref{def:context-element}).  If $\wstar > -\infty$, then
$\wstar_B = \wstar > -\infty = \wstar_A$, so the second case of
\cref{eq:comp-dpre} applies:
\[
  \dpre' = \dtotal_A + \dpre[{,B}] = 0 + p = p.
\]
In either subcase, $\dpre' = p$.  Therefore $e \opendo C = (\wstar, d, p) = C$. \checkmark
\end{proof}

\begin{theorem}[Context monoid]\label{thm:context-monoid}
The set of context elements equipped with endogenous composition and
identity element $e$ forms a monoid:
\[
  \bigl(\,\mathcal{C},\;\opendo,\;e\,\bigr)
  \quad\text{is a monoid.}
\]
That is, $\opendo$ is associative (\cref{prop:associativity}), and $e = (-\infty, 0, 0)$
is a two-sided identity (\cref{prop:identity}).
\end{theorem}

%% ═══════════════════════════════════════════════════════════════════
\section{Non-Commutativity}\label{sec:non-commutativity}
%% ═══════════════════════════════════════════════════════════════════

\begin{remark}[Non-commutativity of $\opendo$]\label{rem:non-commutativity}
The operator $\opendo$ is \textbf{not} commutative.  We have already
seen a concrete counterexample: in \cref{ex:comp-right-shift,ex:comp-left-stays},
the same two elements composed in opposite orders give different results.
Specifically, with $A = (3, 5, 2)$ and $B = (7, 4, 1)$:
\[
  A \opendo B = (7, 9, 6),
  \qquad
  B \opendo A = (7, 9, 1).
\]
These differ in their $\dpre$ components: $6 \neq 1$.

The reason is structural.  When $\wstar_B > \wstar_A$ and $B$ is to the
right of $A$, \emph{all} of $A$'s development capacity ($\dtotal_A = 5$)
gets promoted into the pre-pivot region, yielding $\dpre = 5 + 1 = 6$.
But when $B$ is to the \emph{left} of $A$ (i.e., we compute
$B \opendo A$), the pivot is already in the left block, so only $B$'s
own pre-pivot count matters: $\dpre = \dpre[{,B}] = 1$.

In other words, temporal ordering matters.  The left block's entire
development capacity is promoted when the pivot shifts rightward, but no
symmetric promotion occurs when the pivot remains in the left block.
This asymmetry is an inherent feature of the endogenous context
structure---it reflects the physical fact that events before the pivot
and events after the pivot play fundamentally different roles.
\end{remark}

%% ═══════════════════════════════════════════════════════════════════
\section{Systems Implications}\label{sec:systems-implications}
%% ═══════════════════════════════════════════════════════════════════

The algebraic properties established in this chapter have direct
consequences for the computational implementation.

\begin{remark}[Parallel tree reduction]\label{rem:parallel-tree}
Because $\opendo$ is strictly associative
(\cref{prop:associativity}), the context reduction over a sequence of
$n$ events can be parallelised as a balanced binary tree reduction with
$O(\log n)$ depth.  At each level of the tree, adjacent pairs of context
elements are composed independently, halving the number of elements.
After $\lceil \log_2 n \rceil$ levels, a single root element remains.

The implementation in
\texttt{src/holographic\_tree.py}\footnote{Specifically the
\texttt{HolographicContextTree} class, which maintains a binomial carry
forest of composed context elements.}\ realises this parallel structure.
Each call to \texttt{append()} triggers $O(\log n)$ compositions in the
worst case, maintaining the invariant that the forest encodes the full
context of all events seen so far.  The root summary can be queried at
any time via \texttt{get\_root\_summary()}.
\end{remark}

\begin{remark}[Streaming composition]\label{rem:streaming}
Associativity also enables a simple streaming protocol: process events
one at a time, composing each new singleton context element into a
running accumulator via a left fold.  The accumulator at any point
equals the context element for the entire event history observed so far.

The code path in \texttt{src/tropical\_semiring.py} implements this
strategy directly.  The function \texttt{build\_tropical\_context()}
performs a left fold over the event sequence, while the holographic tree
in \texttt{src/holographic\_tree.py} implements the parallel tree
variant.  Both produce identical results for any input sequence---a fact
verified empirically by \texttt{test\_05\_holographic\_exactness.py}.
The left fold has $O(n)$ sequential depth but requires only $O(1)$
working memory (a single accumulator); the tree has $O(\log n)$ depth
but requires $O(\log n)$ memory for the forest.  The choice between them
is a classical space--parallelism trade-off, and associativity guarantees
that the choice cannot affect correctness.
\end{remark}

%% ═══════════════════════════════════════════════════════════════════
\section{Exercises}\label{sec:context-algebra-exercises}
%% ═══════════════════════════════════════════════════════════════════

\begin{exercise}\label{exer:basic-composition}
Let $A = (10, 3, 1)$ and $B = (5, 6, 4)$.  Compute $A \opendo B$ and
$B \opendo A$.  Verify that the two results differ, and explain which
component differs and why.
\end{exercise}

\begin{exercise}\label{exer:tied-pivots}
Let $A = (5, 2, 1)$, $B = (5, 3, 2)$, and $C = (5, 4, 3)$, where all
three pivot keys are tied at $\wstar = 5$.  The tie-breaking rule in
\cref{def:endogenous-composition} is that the left block wins when
$\wstar_A \ge \wstar_B$ (i.e., the first case of \cref{eq:comp-dpre}
applies on ties).

\begin{enumerate}[label=(\alph*)]
  \item Compute $(A \opendo B) \opendo C$.  What is $\dpre$ in each
    intermediate and final result?
  \item Compute $A \opendo (B \opendo C)$.  Verify that the final
    result matches part~(a).
  \item Explain in words why tie-breaking in favour of the left block is
    essential for associativity.
\end{enumerate}
\end{exercise}

\begin{exercise}\label{exer:identity-power}
Let $C$ be any context element with $\wstar > -\infty$.  Prove that for
any $n \ge 1$,
\[
  C \opendo \underbrace{e \opendo e \opendo \cdots \opendo e}_{n \text{
  copies}} = C.
\]
\emph{Hint.}\; Use the identity property (\cref{prop:identity}) and
induction, or argue directly from the definition.
\end{exercise}

\begin{exercise}\label{exer:no-focal}
Consider a sequence of events in which \emph{every} event is non-focal
(there are no focal-actor events at all).  Let $C$ be the context
element obtained by composing the singleton elements for this sequence.

\begin{enumerate}[label=(\alph*)]
  \item What are $\wstar$, $\dtotal$, and $\dpre$ for the resulting
    context element $C$?
  \item A context element is called \emph{feasible} at threshold $k$ if
    $\dpre \ge k$.  Can the all-non-focal context element ever be
    feasible?  Explain.
  \item What does this say about the necessity of focal events for the
    endogenous pivot mechanism?
\end{enumerate}
\end{exercise}

\chapter{The Tropical Lift}\label{ch:tropical-lift}

The monoid developed in \cref{ch:context-algebra} compresses any
contiguous block of events into a triple $(\wstar, \dtotal, \dpre)$.
This triple tracks a single pivot---the strongest focal event---and
records how much development capacity sits before it.  The triple is
enough to test feasibility at a fixed threshold $k$ after the fact, but
it discards information: we learn the \emph{best} pivot's pre-pivot
count yet know nothing about the second-best, or about pivots that
fall just short of feasibility.

This chapter lifts the monoid into a richer representation---a
\emph{tropical context}---that replaces the scalar $\wstar$ with a
weight vector indexed by pre-pivot development count.  The resulting
structure is a faithful extension of the original monoid: it reproduces
exactly the same $\wstar$ and $\dpre$, while simultaneously answering
feasibility queries at every threshold from $0$ to $k$ in a single
pass.  The composition rule for weight vectors turns out to be a
simple shift-and-max operation---the tropical semiring structure from
which the chapter takes its name.\footnote{For background on tropical
semirings, see e.g.\ Maclagan and Sturmfels, \emph{Introduction to
Tropical Geometry}, AMS, 2015.}

%% ═══════════════════════════════════════════════════════════════════
\section{From Monoid to Weight Vectors}\label{sec:monoid-to-vectors}
%% ═══════════════════════════════════════════════════════════════════

Recall the context element $C = (\wstar, \dtotal, \dpre)$ from
\cref{def:context-element}.  The scalar $\wstar$ is the weight of the
best pivot in the block, and $\dpre$ tells us how many non-focal events
precede it.  If $\dpre \ge k$, the pivot is feasible.  But what if
$\dpre = k - 1$?  The monoid offers no recourse: the best pivot fell
one short, and we have no record of whether a slightly weaker pivot
might have $\dpre \ge k$.

The motivation for the tropical lift is exactly this: we want to know
the best pivot weight achievable with \emph{exactly} $j$ pre-pivot
development events, for every $j$ from $0$ to $k$.

\begin{definition}[Tropical context]\label{def:tropical-context}
A \textbf{tropical context} is a pair
\[
  T \;=\; (W,\; \dtotal),
\]
where
\begin{enumerate}[label=(\roman*)]
  \item $W \in (\R \cup \{-\infty\})^{k+1}$ is the \textbf{weight
    vector}.  The entry $W[j]$ for $j = 0, 1, \ldots, k$ is the
    maximum weight among all focal events in the block whose pre-pivot
    development count equals exactly $j$.  If no such focal event
    exists, $W[j] = -\infty$.

  \item $\dtotal \in \N$ is the \textbf{total development capacity}:
    the count of non-focal events in the block, exactly as in
    \cref{def:context-element}.
\end{enumerate}
The weight vector is indexed from $0$ (a pivot with no preceding
non-focal events) up to $k$ (a pivot with at least $k$ preceding
non-focal events, where counts beyond $k$ are capped at slot $k$).
\end{definition}

The connection to the original monoid is immediate.

\begin{remark}[Recovering the monoid]\label{rem:recovering-monoid}
Given a tropical context $T = (W, \dtotal)$, the monoid's best pivot
weight and its pre-pivot count are
\[
  \wstar = \max_{0 \le j \le k} W[j],
  \qquad
  \dpre = \min\!\Bigl(\argmax_{0 \le j \le k} W[j],\; k\Bigr).
\]
The tropical representation is strictly richer than the triple: it
records the best pivot at \emph{every} slot, not just the globally
best one.
\end{remark}

Feasibility also takes a clean form.

\begin{remark}[Feasibility]\label{rem:feasibility}
A tropical context $T$ is \textbf{feasible} at threshold $k$ if and
only if $W[k] > -\infty$.  That is, there exists at least one focal
event in the block with $k$ or more non-focal events preceding it.
\end{remark}

%% ═══════════════════════════════════════════════════════════════════
\section{Lifting Single Events}\label{sec:lifting-events}
%% ═══════════════════════════════════════════════════════════════════

Every individual event is lifted into the tropical context by the
factory function \texttt{from\_event}.

\begin{definition}[Singleton tropical context]\label{def:from-event}
Given an event $e$ and a threshold $k$, the \textbf{singleton tropical
context} is $T_e = (W_e, \dtotal_e)$ defined as follows.
\begin{enumerate}[label=(\roman*)]
  \item If $e$ is \textbf{focal} (a pivot candidate):
    \[
      W_e[0] = w(e), \qquad
      W_e[j] = -\infty \;\text{ for } j = 1, \ldots, k, \qquad
      \dtotal_e = 0.
    \]
    A focal event occupies slot $0$: it is a pivot with zero
    development events before it.

  \item If $e$ is \textbf{non-focal} (a development event):
    \[
      W_e[j] = -\infty \;\text{ for all } j = 0, \ldots, k, \qquad
      \dtotal_e = 1.
    \]
    A non-focal event contributes one unit of development capacity but
    cannot serve as a pivot.
\end{enumerate}
\end{definition}

\begin{example}[Focal singleton]\label{ex:focal-singleton}
Let $e$ be a focal event with weight $7.5$ and let $k = 3$.  Then
\[
  T_e = \bigl(\,[7.5,\; -\infty,\; -\infty,\; -\infty],\;\; 0\bigr).
\]
The pivot sits in slot $0$ with weight $7.5$; no development capacity
exists.
\end{example}

\begin{example}[Non-focal singleton]\label{ex:nonfocal-singleton}
Let $e$ be a non-focal event and let $k = 3$.  Then
\[
  T_e = \bigl(\,[-\infty,\; -\infty,\; -\infty,\; -\infty],\;\; 1\bigr).
\]
All weight-vector entries are $-\infty$ (no pivot is available), and
the block contributes one unit of development capacity.
\end{example}

%% ═══════════════════════════════════════════════════════════════════
\section{Tropical Composition}\label{sec:tropical-composition}
%% ═══════════════════════════════════════════════════════════════════

This section presents the central algorithm.  Given two tropical
contexts representing adjacent blocks---the left block $T_A$ followed
in time by the right block $T_B$---we compute their composition.

\begin{definition}[Tropical composition $\otimes$]
\label{def:tropical-composition}
Let $T_A = (W_A, d_A)$ and $T_B = (W_B, d_B)$ be tropical contexts
with the same threshold $k$.  Their \textbf{tropical composition} is
\[
  T_A \otimes T_B \;=\; (W_{\mathrm{result}},\; d_A + d_B),
\]
where $W_{\mathrm{result}}$ is computed by the following
shift-and-max rule.  For each result slot $j = 0, 1, \ldots, k$:
\begin{equation}\label{eq:shift-and-max}
  W_{\mathrm{result}}[j]
  \;=\;
  \begin{cases}
    W_A[j]
      & \text{if } j < d_A, \\[4pt]
    \max\!\bigl(W_A[j],\; W_B[j - d_A]\bigr)
      & \text{if } j \ge d_A.
  \end{cases}
\end{equation}
\end{definition}

The logic is as follows.  A pivot from block $A$ that already had $j$
pre-pivot development events keeps its slot: those $j$ events are still
before it in the concatenated sequence.  A pivot from block $B$ that
had $j'$ pre-pivot development events within $B$ acquires all $d_A$ of
$A$'s non-focal events as additional predecessors, so it shifts from
slot $j'$ to slot $j' + d_A$.  If $j' + d_A > k$, the shifted slot is
capped at $k$ (both slots $j' + d_A$ and $k$ indicate ``at least $k$
pre-pivot events'').

This is precisely the operational meaning of the $\dpre$ promotion
from \cref{eq:comp-dpre} in \cref{ch:context-algebra}: when a pivot
resides in block $B$, all of $A$'s development capacity becomes
pre-pivot.

\begin{remark}[Capping at $k$]\label{rem:capping}
The formulation in \cref{eq:shift-and-max} handles capping implicitly:
because the weight vector has only $k+1$ entries indexed $0$ through
$k$, a pivot from $B$ at slot $j'$ with $j' + d_A > k$ maps to
result slot $k$ via
$W_{\mathrm{result}}[k] = \max(W_{\mathrm{result}}[k],\; W_B[j'])$.
The implementation iterates over the entries of $W_B$ and writes to
$\min(j' + d_A,\; k)$, which is equivalent.
\end{remark}

\begin{example}[Tropical composition]\label{ex:tropical-composition}
Let $k = 3$, and consider the two tropical contexts
\begin{align*}
  T_A &= \bigl(\,[5.0,\; 3.0,\; -\infty,\; -\infty],\;\; d_A = 2\bigr),\\
  T_B &= \bigl(\,[7.0,\; -\infty,\; 4.0,\; -\infty],\;\; d_B = 3\bigr).
\end{align*}
We compute $T_A \otimes T_B$ slot by slot:
\begin{align*}
  j = 0 &\colon\; j < d_A = 2, \quad
    W_{\mathrm{result}}[0] = W_A[0] = 5.0. \\[3pt]
  j = 1 &\colon\; j < d_A = 2, \quad
    W_{\mathrm{result}}[1] = W_A[1] = 3.0. \\[3pt]
  j = 2 &\colon\; j \ge d_A, \quad
    W_{\mathrm{result}}[2] = \max\!\bigl(W_A[2],\; W_B[2 - 2]\bigr)
    = \max(-\infty,\; 7.0) = 7.0. \\[3pt]
  j = 3 &\colon\; j \ge d_A, \quad
    W_{\mathrm{result}}[3] = \max\!\bigl(W_A[3],\; W_B[3 - 2]\bigr)
    = \max(-\infty,\; -\infty) = -\infty.
\end{align*}
The result is
\[
  T_A \otimes T_B
  = \bigl(\,[5.0,\; 3.0,\; 7.0,\; -\infty],\;\; d = 5\bigr).
\]
\textbf{Interpretation.}\; The best pivot with zero pre-pivot
development events has weight $5.0$ (from $A$, slot~0).  With exactly
two pre-pivot events, the best weight is $7.0$ (from $B$, shifted by
$d_A = 2$).  Slot~3 is $-\infty$, so no feasible pivot exists at
threshold $k = 3$.
\end{example}

\begin{proposition}[Associativity of tropical composition]
\label{prop:tropical-associativity}
For any tropical contexts $T_A$, $T_B$, and $T_C$ with common
threshold $k$,
\[
  (T_A \otimes T_B) \otimes T_C
  \;=\;
  T_A \otimes (T_B \otimes T_C).
\]
\end{proposition}

\begin{proof}
The $\dtotal$ component of both sides equals $d_A + d_B + d_C$ by
associativity of addition.  It remains to verify the weight vector.

A pivot originating in block $X \in \{A, B, C\}$ at internal slot $j_X$
ends up in result slot $s(j_X)$ defined by
\[
  s_A(j_A) = j_A, \qquad
  s_B(j_B) = j_B + d_A, \qquad
  s_C(j_C) = j_C + d_A + d_B,
\]
all capped at $k$.  These slot assignments are determined by the
block's position in the concatenated sequence and are independent of
parenthesisation.  At each result slot $j$, both parenthesisations
compute $\max$ over the same set of contributing pivots from $A$, $B$,
and $C$, so the result is identical.

In detail, consider the left-associated computation.  Let
$T_{AB} = T_A \otimes T_B$ with $d_{AB} = d_A + d_B$.  Then in
$T_{AB} \otimes T_C$, a pivot from $C$ at internal slot $j_C$ shifts
to $\min(j_C + d_{AB},\, k) = \min(j_C + d_A + d_B,\, k)$.  A pivot
from $A$ at slot $j_A$ was placed at slot $j_A$ in $T_{AB}$ and
remains at slot $j_A$ in the final result.  A pivot from $B$ at slot
$j_B$ was placed at slot $\min(j_B + d_A,\, k)$ in $T_{AB}$ and
remains there.

The right-associated computation yields the same slot assignments:
$T_{BC} = T_B \otimes T_C$ places a $C$-pivot at $\min(j_C + d_B, k)$
and a $B$-pivot at slot $j_B$.  Then $T_A \otimes T_{BC}$ leaves
$A$-pivots at $j_A$, shifts $B$-pivots to $\min(j_B + d_A, k)$, and
shifts $C$-pivots to $\min(j_C + d_B + d_A, k)$.

Since both parenthesisations produce the same slot assignment for every
pivot and take $\max$ at each slot, the weight vectors agree.
\end{proof}

%% ═══════════════════════════════════════════════════════════════════
\section{Building Context from a Sequence}\label{sec:building-context}
%% ═══════════════════════════════════════════════════════════════════

With the composition operator in hand, computing the tropical context
for an entire event sequence is a left fold.

\begin{algorithm}[t]
\caption{Build tropical context from an event sequence.}
\label{alg:build-tropical}
\begin{algorithmic}[1]
\Procedure{BuildTropicalContext}{events, $k$}
  \State $T \gets \Call{Empty}{k}$
    \Comment{$W = [-\infty, \ldots, -\infty]$, $\dtotal = 0$}
  \For{each event $e$ in temporal order}
    \State $T \gets T \otimes \Call{FromEvent}{e, k}$
  \EndFor
  \State \Return $T$
\EndProcedure
\end{algorithmic}
\end{algorithm}

\Cref{alg:build-tropical} is the direct implementation of the left
fold.  The code path is
\texttt{src/tropical\_semiring.py:\allowbreak build\_tropical\_context()},
which iterates over events and composes each singleton into the
accumulator via \texttt{compose\_tropical()}.
We now trace through a complete example to build intuition.

\begin{example}[Full step-by-step construction]
\label{ex:full-construction}
Let $k = 3$ and consider the sequence of five events:
\[
  e_1\;\text{(non-focal)}, \quad
  e_2\;\text{(focal, $w = 4$)}, \quad
  e_3\;\text{(non-focal)}, \quad
  e_4\;\text{(focal, $w = 7$)}, \quad
  e_5\;\text{(non-focal)}.
\]

\medskip\noindent
\textbf{Initialisation.}\;
$T = ([-\infty,\; -\infty,\; -\infty,\; -\infty],\;\; d = 0)$.

\medskip\noindent
\textbf{Step~1: append $e_1$ (non-focal).}\;
$T_{e_1} = ([-\infty,\; -\infty,\; -\infty,\; -\infty],\;\; d = 1)$.
Composing $T \otimes T_{e_1}$: the left block has $d = 0$, so $B$'s
entries shift by $0$.  Both weight vectors are all $-\infty$.
\[
  T = ([-\infty,\; -\infty,\; -\infty,\; -\infty],\;\; d = 1).
\]

\medskip\noindent
\textbf{Step~2: append $e_2$ (focal, $w = 4$).}\;
$T_{e_2} = ([4,\; -\infty,\; -\infty,\; -\infty],\;\; d = 0)$.
Composing with $d_A = 1$:
\begin{align*}
  j = 0 &\colon\; j < 1, \quad W[0] = -\infty. \\
  j = 1 &\colon\; j \ge 1, \quad W[1] = \max(-\infty,\; T_{e_2}.W[0])
    = \max(-\infty, 4) = 4. \\
  j = 2 &\colon\; j \ge 1, \quad W[2] = \max(-\infty,\; T_{e_2}.W[1])
    = \max(-\infty, -\infty) = -\infty. \\
  j = 3 &\colon\; \text{similarly } -\infty.
\end{align*}
\[
  T = ([-\infty,\; 4,\; -\infty,\; -\infty],\;\; d = 1).
\]
Event $e_2$ (weight $4$) lands in slot $1$ because one non-focal event
($e_1$) precedes it.

\medskip\noindent
\textbf{Step~3: append $e_3$ (non-focal).}\;
$T_{e_3} = ([-\infty,\; -\infty,\; -\infty,\; -\infty],\;\; d = 1)$.
Composing with $d_A = 1$: $W_A$ entries stay, and $W_B$ is all
$-\infty$, so no new pivots appear.  Only $\dtotal$ increases.
\[
  T = ([-\infty,\; 4,\; -\infty,\; -\infty],\;\; d = 2).
\]

\medskip\noindent
\textbf{Step~4: append $e_4$ (focal, $w = 7$).}\;
$T_{e_4} = ([7,\; -\infty,\; -\infty,\; -\infty],\;\; d = 0)$.
Composing with $d_A = 2$:
\begin{align*}
  j = 0 &\colon\; j < 2, \quad W[0] = -\infty. \\
  j = 1 &\colon\; j < 2, \quad W[1] = 4. \\
  j = 2 &\colon\; j \ge 2, \quad W[2] = \max(-\infty,\; T_{e_4}.W[0])
    = \max(-\infty, 7) = 7. \\
  j = 3 &\colon\; j \ge 2, \quad W[3] = \max(-\infty,\; T_{e_4}.W[1])
    = \max(-\infty, -\infty) = -\infty.
\end{align*}
\[
  T = ([-\infty,\; 4,\; 7,\; -\infty],\;\; d = 2).
\]
Event $e_4$ (weight $7$) lands in slot $2$: two non-focal events
($e_1, e_3$) precede it in the sequence.

\medskip\noindent
\textbf{Step~5: append $e_5$ (non-focal).}\;
$T_{e_5} = ([-\infty,\; -\infty,\; -\infty,\; -\infty],\;\; d = 1)$.
Composing with $d_A = 2$: no new pivots; $\dtotal$ increments.
\[
  T = ([-\infty,\; 4,\; 7,\; -\infty],\;\; d = 3).
\]

\medskip\noindent
\textbf{Final result.}\;
$T = ([-\infty,\; 4,\; 7,\; -\infty],\;\; d = 3)$.  Since
$W[3] = -\infty$, the context is \textbf{not feasible} at $k = 3$.
The best weight overall is $\max(W) = 7$, achieved at slot $2$.  This
means the best pivot ($e_4$, weight $7$) has exactly $2$ pre-pivot
development events.

If the threshold were $k = 2$, the context would be feasible with
$W[2] = 7$.
\end{example}

\begin{remark}[Brute-force verification]\label{rem:brute-force}
\Cref{ex:full-construction} can be verified by direct enumeration.
The focal events and their non-focal predecessor counts are:
\begin{itemize}
  \item $e_2$ (weight $4$): preceded by $e_1$ (non-focal), so
    $\dpre = 1$.
  \item $e_4$ (weight $7$): preceded by $e_1, e_3$ (non-focal), so
    $\dpre = 2$.
\end{itemize}
Therefore $W[1] = 4$ and $W[2] = 7$, with all other slots $-\infty$.
This matches the left-fold result exactly.
\end{remark}

%% ═══════════════════════════════════════════════════════════════════
\section{Best Feasible Pivot Search}\label{sec:best-feasible-pivot}
%% ═══════════════════════════════════════════════════════════════════

With the weight vector in hand, the feasibility query reduces to an
index lookup.

\begin{definition}[Best feasible pivot]\label{def:best-feasible}
Given a tropical context $T = (W, \dtotal)$ and threshold $k$, the
\textbf{best feasible pivot weight} is
\[
  \wstar_{\mathrm{feas}} = W[k].
\]
If $W[k] = -\infty$, no feasible pivot exists.  If $W[k] > -\infty$,
the best feasible pivot has weight $W[k]$ and at least $k$ non-focal
predecessors.
\end{definition}

To recover the \emph{identity} of the best feasible pivot (not just
its weight), the algorithm must trace back through the composition to
find which event contributed $W[k]$.  In the streaming setting, this
requires maintaining the event reference alongside the weight.  In the
holographic tree setting (\cref{rem:parallel-tree} in
\cref{ch:context-algebra}), the tree's structure enables efficient
pivot-block lookup.

\begin{remark}[Top-$m$ search]\label{rem:top-m}
For applications requiring the top $m$ feasible pivots rather than
just the best, the function
\texttt{focal\_pivots\_with\_prefix} enumerates all focal events
together with their prefix development counts.  Filtering to those with
prefix count $\ge k$ and sorting by weight yields the top-$m$ list.
This brute-force enumeration runs in $O(n \log n)$ time (dominated by
sorting) and is used for validation purposes.
\end{remark}

%% ═══════════════════════════════════════════════════════════════════
\section{Subsumption of the Original Monoid}\label{sec:subsumption}
%% ═══════════════════════════════════════════════════════════════════

The tropical context is designed as a strict generalisation of the
original monoid.  We now state the precise subsumption relationship
and report its empirical validation.

\begin{proposition}[Tropical subsumption]\label{prop:subsumption}
Let $T = (W, \dtotal)$ be the tropical context obtained by left-fold
composition over an event sequence, and let
$C = (\wstar, \dtotal, \dpre)$ be the original monoid context computed
over the same sequence.  Then
\begin{align}
  \max_{0 \le j \le k} W[j] &= \wstar,
  \label{eq:subsumption-wstar}\\[4pt]
  \argmax_{0 \le j \le k} W[j] &= \min(\dpre,\; k),
  \label{eq:subsumption-dpre}
\end{align}
where $\argmax$ returns the smallest index achieving the maximum in
the case of ties.
\end{proposition}

\begin{proof}[Proof sketch]
Both the monoid and the tropical context process the same event
sequence by left fold.  The monoid tracks only the globally strongest
pivot (breaking ties in favour of the leftmost), while the tropical
context tracks the strongest pivot \emph{at each slot}.  The globally
strongest pivot occupies exactly one slot, say slot $j^*$, so
$\max(W) = W[j^*] = \wstar$.

For \cref{eq:subsumption-dpre}, the slot $j^*$ records the number of
non-focal predecessors of the global pivot, capped at $k$.  The
monoid's $\dpre$ is the uncapped count, so
$j^* = \min(\dpre, k)$.
\end{proof}

\begin{remark}[Empirical validation]\label{rem:empirical-subsumption}
The test suite \texttt{test\_03\_monoid\_subsumption.py} validates
\cref{prop:subsumption} across $200$ random seeds per configuration,
spanning a range of sequence lengths, focal ratios, and thresholds $k$.
Across all configurations, zero violations are observed with a
numerical tolerance of $10^{-12}$.  The tropical context faithfully
reproduces the original monoid's outputs.
\end{remark}

The tropical context is strictly richer than the monoid: it records
feasibility information at every slot simultaneously.  This per-slot
resolution is what enables the contract-guarded compression developed
in \cref{ch:absorbing-ideal}.

%% ═══════════════════════════════════════════════════════════════════
\section{Complexity Analysis}\label{sec:tropical-complexity}
%% ═══════════════════════════════════════════════════════════════════

\begin{proposition}[Complexity of tropical operations]
\label{prop:tropical-complexity}
The following complexity bounds hold:
\begin{enumerate}[label=(\roman*)]
  \item \textbf{Single composition.}\; Computing $T_A \otimes T_B$
    requires $O(k)$ time: one pass over the $k + 1$ result slots.

  \item \textbf{Left-fold construction.}\; Building the tropical
    context for a sequence of $n$ events via
    \cref{alg:build-tropical} requires $O(n \cdot k)$ time: $n$
    compositions, each $O(k)$.

  \item \textbf{Holographic tree.}\; The tree variant maintains
    $O(\log n)$ composition depth, achieving $O(k \log n)$ amortised
    cost per event insertion and $O(n \cdot k)$ total build cost.
\end{enumerate}
\end{proposition}

\begin{remark}[Practical cost]\label{rem:practical-cost}
In the narrative application, the threshold $k$ is typically small
($1 \le k \le 5$).  For such values the $O(k)$ factor is a small
constant, and the left-fold construction is effectively $O(n)$.  The
holographic tree's $O(\log n)$ query depth becomes relevant for
interactive or streaming applications where individual updates must
be fast relative to the total sequence length.
\end{remark}

%% ═══════════════════════════════════════════════════════════════════
\section{Implementation Notes}\label{sec:implementation-notes}
%% ═══════════════════════════════════════════════════════════════════

\begin{remark}[Sentinel value for $-\infty$]
\label{rem:sentinel}
The implementation in \texttt{src/tropical\_semiring.py} represents
the weight vector $W$ as a \texttt{numpy} array of floating-point
numbers.  The value $-\infty$ is encoded as Python's
\texttt{float("-inf")}.  The composition function
\texttt{compose\_tropical} implements the shift-and-max rule from
\cref{def:tropical-composition}: it copies $W_A$ into the result, then
iterates over entries of $W_B$, shifting each by $d_A$ and capping at
$k$.  This is equivalent to \cref{eq:shift-and-max}.
\end{remark}

\begin{remark}[Factory methods]\label{rem:factory}
The \texttt{TropicalContext} class provides three factory methods:
\begin{itemize}
  \item \texttt{empty(k)}: returns the identity element
    $([-\infty, \ldots, -\infty],\; 0)$.
  \item \texttt{from\_event(e, k)}: implements
    \cref{def:from-event}---for focal events, $W[0] = w(e)$ and
    $\dtotal = 0$; for non-focal events, all entries of $W$ are
    $-\infty$ and $\dtotal = 1$.
  \item \texttt{compose(other)}: delegates to
    \texttt{compose\_tropical}.
\end{itemize}
The left fold in \texttt{build\_tropical\_context} iterates over the
event sequence, composing each singleton into the accumulator.  The
ground-truth validator \texttt{brute\_force\_tropical\_context}
enumerates all focal events, counts their non-focal predecessors
directly, and populates the weight vector by brute force---providing
an independent reference for testing.
\end{remark}

%% ═══════════════════════════════════════════════════════════════════
\section{Exercises}\label{sec:tropical-lift-exercises}
%% ═══════════════════════════════════════════════════════════════════

\begin{exercise}\label{exer:tropical-compose-by-hand}
Let $k = 2$ and consider the two tropical contexts
\[
  T_A = \bigl(\,[6.0,\; -\infty,\; 2.0],\;\; d_A = 1\bigr),
  \qquad
  T_B = \bigl(\,[8.0,\; 3.0,\; -\infty],\;\; d_B = 2\bigr).
\]
\begin{enumerate}[label=(\alph*)]
  \item Compute $T_A \otimes T_B$ using the shift-and-max rule
    (\cref{eq:shift-and-max}).
  \item Is the result feasible at $k = 2$?
  \item What are $\wstar$ and $\dpre$ for the result?  Verify that they
    match what the original monoid would produce.
\end{enumerate}
\end{exercise}

\begin{exercise}\label{exer:identity-tropical}
Show that the empty tropical context $T_{\varepsilon} = ([-\infty,
\ldots, -\infty],\; 0)$ is a two-sided identity for $\otimes$.  That
is, for any tropical context $T$,
\[
  T \otimes T_{\varepsilon} = T
  \qquad\text{and}\qquad
  T_{\varepsilon} \otimes T = T.
\]
\emph{Hint.}\; For the left-identity case, note that $d_A = 0$ in the
shift-and-max rule.
\end{exercise}

\begin{exercise}\label{exer:tropical-fold-trace}
Let $k = 2$ and consider the event sequence:
\[
  e_1\;\text{(focal, $w = 5$)},\quad
  e_2\;\text{(non-focal)},\quad
  e_3\;\text{(focal, $w = 3$)},\quad
  e_4\;\text{(non-focal)}.
\]
Trace through the left-fold construction (\cref{alg:build-tropical})
step by step.  State the tropical context after each event is
appended.  Verify the final result against brute-force enumeration of
the focal events and their predecessor counts.
\end{exercise}

\begin{exercise}\label{exer:all-nonfocal-tropical}
Let all $n$ events in a sequence be non-focal.  What is the tropical
context produced by \cref{alg:build-tropical}?  Compare this to the
monoid context from \cref{exer:no-focal} in
\cref{ch:context-algebra} and confirm that the subsumption relation
(\cref{prop:subsumption}) holds.
\end{exercise}

\begin{exercise}\label{exer:associativity-numerical}
Let $k = 2$ and define
\begin{align*}
  T_A &= \bigl(\,[4.0,\; -\infty,\; -\infty],\;\; d_A = 1\bigr),\\
  T_B &= \bigl(\,[-\infty,\; -\infty,\; -\infty],\;\; d_B = 1\bigr),\\
  T_C &= \bigl(\,[9.0,\; -\infty,\; -\infty],\;\; d_C = 0\bigr).
\end{align*}
Compute $(T_A \otimes T_B) \otimes T_C$ and
$T_A \otimes (T_B \otimes T_C)$ separately.  Verify that the results
are identical and that the final context is feasible at $k = 2$.
\end{exercise}

% ══════════════════════════════════════════════════════════════
%  Chapter 7 — The Absorbing Ideal and Compression Contracts
% ══════════════════════════════════════════════════════════════
\chapter{The Absorbing Ideal and Compression Contracts}
\label{ch:absorbing-ideal}

The preceding chapters have established two independent lines of
reasoning.  On the algebraic side, \cref{ch:context-algebra} gave us the
endogenous context monoid $(\mathcal{C},\opendo)$ with strict
associativity and a logarithmic parallel reduction.  On the structural
side, \cref{ch:absorbing-states} showed that prefix-deficient states are
absorbing under greedy semantics---once a sequence has committed to a
deficient prefix, no continuation can rescue it.

This chapter unifies the two lines.  We extend the context element with
a \emph{commitment flag}, define a richer composition operation
$\opcommit$ that respects commitment, and prove that the set of
absorbing elements forms a \emph{left ideal} in the resulting monoid.
We then turn to the practical question: which operations can push an
element across the absorbing boundary?  The answer, confirmed both by
theory and by experiment, is that \emph{compression is the unique
closure-breaking operation}.  We conclude by formulating a
\emph{no-absorption contract} that every compression map must satisfy to
preserve algebraic closure.

% ══════════════════════════════════════════════════════════════
\section{Extended Context Elements}
\label{sec:extended-ctx}
% ══════════════════════════════════════════════════════════════

In \cref{ch:context-algebra} a context element was a triple
$(\wstar,\dtotal,\dpre)$.  To model systems that have irrevocably
selected a pivot, we adjoin a single bit.

\begin{definition}[Extended context element]
\label{def:extended-ctx}
An \emph{extended context element} is a quadruple
\[
  \bar C \;=\; (\wstar,\;\dtotal,\;\dpre,\;\kappa)
\]
where $\wstar \in \R$ is the weight of the focal event (running-max
pivot weight), $\dtotal \in \N$ is the total development count (number
of non-focal events in the block), $\dpre \in \N$ is the pre-pivot
development count (non-focal events that precede the current pivot), and
$\kappa \in \{0,1\}$ is the \emph{commitment flag}:
\begin{itemize}[itemsep=3pt]
  \item $\kappa = 0$: \textbf{uncommitted}.  The pivot identity is
        provisional; a future suffix may contain an event of higher weight
        that replaces the current pivot.
  \item $\kappa = 1$: \textbf{committed}.  The pivot identity is locked.
        No event in any future suffix can replace it, regardless of weight.
\end{itemize}
We write $\bar{\mathcal{C}}$ for the set of all extended context
elements.
\end{definition}

\begin{remark}[Operational interpretation of commitment]
\label{rem:commitment-operational}
The commitment flag models any mechanism by which a system makes an
irrevocable decision about pivot identity.  Three common sources of
commitment arise in practice:
\begin{enumerate}[label=(\roman*),itemsep=2pt]
  \item \emph{Streaming assignment.}  A streaming processor that emits
        labels on the fly has, at each step, assigned roles (pre-pivot
        development, pivot, post-pivot material) based on the current
        running-max pivot.  Once a label has been emitted, the assignment
        is irrevocable: setting $\kappa = 1$ records this fact.
  \item \emph{Compressed context.}  A solver that compresses its context
        window typically discards events that were ``redundant'' relative
        to a particular pivot.  The compressed representation is
        thereafter committed to that pivot, because the discarded events
        are no longer available to support a different one.
  \item \emph{Checkpoint publication.}  Any system that publishes an
        intermediate result---a partial narrative, a tentative
        ranking---has committed to the pivot underlying that result.
        Rolling back the pivot would invalidate the published output.
\end{enumerate}
In each case, the system has passed a point of no return.  The
commitment flag is the algebraic encoding of that irreversibility.
\end{remark}

% ══════════════════════════════════════════════════════════════
\section{Committed Composition}
\label{sec:committed-composition}
% ══════════════════════════════════════════════════════════════

With extended context elements in hand, we define the composition
operation that respects commitment.

\begin{definition}[Committed composition $\opcommit$]
\label{def:committed-composition}
Given extended context elements
\[
  \bar C_A = (\wstar_A,\; d_A,\; \dpre{,A},\; \kappa_A),
  \qquad
  \bar C_B = (\wstar_B,\; d_B,\; \dpre{,B},\; \kappa_B),
\]
their \emph{committed composition}
$\bar C_A \opcommit \bar C_B = (\wstar,\;\dtotal,\;\dpre,\;\kappa)$ is
defined by three cases.

\medskip
\noindent\textbf{Case 1.}  $\kappa_A = 1$ \emph{(left block committed).}
\[
  \wstar = \wstar_A, \qquad
  \dtotal = d_A + d_B, \qquad
  \dpre = \dpre{,A}, \qquad
  \kappa = 1.
\]
The committed pivot in $A$ is preserved unconditionally.  Block~$B$
contributes to the total development count but cannot override the
pivot, regardless of $\wstar_B$.

\medskip
\noindent\textbf{Case 2.}  $\kappa_A = 0$ and $\wstar_A \ge \wstar_B$
\emph{(left uncommitted, left pivot dominates).}
\[
  \wstar = \wstar_A, \qquad
  \dtotal = d_A + d_B, \qquad
  \dpre = \dpre{,A}, \qquad
  \kappa = \kappa_A = 0.
\]
This coincides with the endogenous composition $\opendo$: the
higher-weight pivot stays in $A$.

\medskip
\noindent\textbf{Case 3.}  $\kappa_A = 0$ and $\wstar_B > \wstar_A$
\emph{(left uncommitted, right pivot dominates).}
\[
  \wstar = \wstar_B, \qquad
  \dtotal = d_A + d_B, \qquad
  \dpre = d_A + \dpre{,B}, \qquad
  \kappa = \kappa_B.
\]
The pivot shifts to $B$.  All of $A$'s events become pre-pivot
development material and are added to $\dpre{,B}$.  The commitment
status of the result inherits from $B$.
\end{definition}

\begin{remark}[Comparison with $\opendo$]
\label{rem:commit-vs-endo}
The critical difference between $\opcommit$ and $\opendo$ lies
exclusively in Case~1.  Under $\opendo$, there is no Case~1: if
$\wstar_B > \wstar_A$, the pivot always shifts to $B$.  Under
$\opcommit$, the commitment flag blocks this shift.  The pivot in $A$ is
preserved even when $B$ contains a strictly higher-weight event.

Cases~2 and~3 of $\opcommit$ are identical to the two cases of
$\opendo$.  Committed composition therefore extends endogenous
composition: when $\kappa_A = 0$, the two operations agree.
\end{remark}

\begin{example}[Committed composition locks a suboptimal pivot]
\label{ex:committed-suboptimal}
Let $\bar C_A = (3,\, 5,\, 1,\, 1)$ and $\bar C_B = (10,\, 4,\, 2,\, 0)$.

\medskip
\noindent\emph{Under $\opendo$:}  Since $\wstar_B = 10 > 3 = \wstar_A$,
the pivot shifts to $B$.  The result is
\[
  \bar C_A \opendo \bar C_B
    = (10,\; 5+4,\; 5+2,\; 0)
    = (10,\; 9,\; 7,\; 0).
\]
The stronger pivot in $B$ has been selected, and all five of $A$'s
events now contribute to the pre-pivot development count.

\medskip
\noindent\emph{Under $\opcommit$:}  Since $\kappa_A = 1$, Case~1
applies regardless of weights.  The result is
\[
  \bar C_A \opcommit \bar C_B
    = (3,\; 5+4,\; 1,\; 1)
    = (3,\; 9,\; 1,\; 1).
\]
The system remains locked to the weight-3 pivot in $A$, ignoring the
weight-10 event in $B$.  The pre-pivot count stays at $\dpre = 1$.  If
$k = 3$ (the grammar's prefix requirement), then $\dpre = 1 < 3$ and
the committed result is absorbing---a situation that $\opendo$ would
have avoided by shifting to the stronger pivot.
\end{example}

% ══════════════════════════════════════════════════════════════
\section{The Absorbing Predicate}
\label{sec:absorbing-predicate}
% ══════════════════════════════════════════════════════════════

We now give the algebraic characterisation of states from which the
grammar's prefix requirement can never be satisfied.

\begin{definition}[Absorbing predicate]
\label{def:absorbing-predicate}
Fix a grammar prefix requirement $k \in \N$.  The \emph{absorbing
predicate} $\absorb$ is the subset of $\bar{\mathcal{C}}$ defined by
\[
  \bar C \in \absorb
  \quad\Longleftrightarrow\quad
  \dpre(\bar C) < k.
\]
An element $\bar C$ satisfying $\bar C \in \absorb$ is said to be
\emph{absorbing}: its pre-pivot development count is insufficient to
meet the prefix requirement.
\end{definition}

\begin{remark}[Connection to the impossibility theorem]
\label{rem:absorb-connection}
The absorbing predicate is the algebraic generalisation of the condition
that caused greedy failure in \cref{ch:absorbing-states}.  In that chapter, the
relevant quantity was $\jdev < k$: the development index of the current
pivot was too small for the grammar to be satisfiable.  Here, $\dpre < k$
plays exactly the same role, lifted from the sequential setting into the
algebraic framework.

The key insight is that $\jdev$ was defined relative to a specific
sequence position, whereas $\dpre$ is a property of an abstract context
element.  This abstraction is what allows the absorbing condition to
compose: we can reason about the absorption status of a composite block
without unfolding it into its constituent events.
\end{remark}

% ══════════════════════════════════════════════════════════════
\section{Absorbing Left Ideal}
\label{sec:absorbing-left-ideal}
% ══════════════════════════════════════════════════════════════

The main algebraic result of this chapter is that the absorbing
predicate defines a left ideal in the committed monoid.  Informally:
once a system has committed to an absorbing state, no suffix can rescue
it.

\begin{proposition}[Absorbing left ideal]
\label{prop:absorbing-left-ideal}
Let $k \in \N$ be the grammar prefix requirement.  If\/ $\bar C_A \in
\bar{\mathcal{C}}$ satisfies $\kappa_A = 1$ and $\bar C_A \in \absorb$
(i.e., $\dpre{,A} < k$), then for every $\bar C_D \in
\bar{\mathcal{C}}$,
\[
  \bar C_A \opcommit \bar C_D \;\in\; \absorb.
\]
That is, the set $\{\,\bar C \in \bar{\mathcal{C}} : \kappa = 1 \text{
and } \dpre < k\,\}$ is a left ideal of\/
$(\bar{\mathcal{C}},\opcommit)$.
\end{proposition}

\begin{proof}
Since $\kappa_A = 1$, Case~1 of \cref{def:committed-composition}
applies unconditionally, regardless of the values $\wstar_D$, $d_D$,
$\dpre{,D}$, and $\kappa_D$ carried by the suffix.  The committed
composition rule yields:
\[
  \bar C_A \opcommit \bar C_D
    \;=\;
    \bigl(\wstar_A,\;\; d_A + d_D,\;\; \dpre{,A},\;\; 1\bigr).
\]
We verify each component of the result against the absorbing predicate
and the left ideal requirement.

\medskip
\noindent\emph{Pre-pivot count is preserved.}\quad
The pre-pivot development count of the result is $\dpre{,A}$.  The
suffix $\bar C_D$ does not contribute to $\dpre$: because the pivot is
locked in $A$, no event in $D$ can become the new pivot, and therefore
no event in $D$ can change the set of events that precede the pivot.
The pre-pivot count is determined entirely by the committed left block.

\medskip
\noindent\emph{The absorbing predicate is satisfied.}\quad
By hypothesis, $\dpre{,A} < k$.  Since the pre-pivot count of the
result equals $\dpre{,A}$, the result also has $\dpre < k$.  Therefore
$\bar C_A \opcommit \bar C_D \in \absorb$.

\medskip
\noindent\emph{The commitment flag is preserved.}\quad
The result has $\kappa = 1$.  Combined with $\dpre < k$, this shows
that the result lies in the set
$\{\,\bar C : \kappa = 1 \text{ and } \dpre < k\,\}$, confirming that
the set is closed under right-multiplication by arbitrary elements
of $\bar{\mathcal{C}}$.

\medskip
Since $\bar C_D$ was arbitrary, the set of committed absorbing elements
is a left ideal of $(\bar{\mathcal{C}},\opcommit)$.
\end{proof}

\begin{remark}[Interpretation]
\label{rem:left-ideal-interpretation}
\Cref{prop:absorbing-left-ideal} is the algebraic formalisation of an
operationally devastating fact: \emph{once committed to a bad state, you
cannot recover}.  No matter how rich, how well-structured, or how long
the suffix is, it cannot repair the pre-pivot deficiency because the
commitment flag prevents the pivot from moving.  The absorbing elements
form a ``black hole'' in the monoid: any element that falls in---and is
committed---can never escape.

This should be contrasted with the sequential impossibility theorem of
\cref{ch:absorbing-states}, which established the same conclusion for a
specific greedy algorithm.  \Cref{prop:absorbing-left-ideal} is
stronger: it applies to \emph{any} process that respects committed
composition, regardless of the algorithm used to generate the suffix.
\end{remark}

% ══════════════════════════════════════════════════════════════
\section{Escape Under Endogenous Semantics}
\label{sec:escape-endo}
% ══════════════════════════════════════════════════════════════

The absorbing left ideal depends crucially on commitment.  Without it,
escape is possible---and understanding when escape occurs is essential
for the validity mirage analysis of \cref{ch:mirage}.

\begin{remark}[Escape from absorption without commitment]
\label{rem:escape-endo}
Suppose $\bar C_A$ has $\dpre{,A} < k$ (so $\bar C_A \in \absorb$) but
$\kappa_A = 0$ (uncommitted).  Under $\opendo$, there is no Case~1: the
pivot can shift if a higher-weight event appears on the right.  If the
suffix $\bar C_D$ has $\wstar_D > \wstar_A$ and
$\dpre{,D} \ge k - d_A$, then Case~3 of the composition applies:
\[
  \bar C_A \opendo \bar C_D
    = (\wstar_D,\;\; d_A + d_D,\;\; d_A + \dpre{,D},\;\; \kappa_D).
\]
The new pre-pivot count is $d_A + \dpre{,D}$.  If
$d_A + \dpre{,D} \ge k$, then $\bar C_A \opendo \bar C_D \notin
\absorb$: the composite has escaped absorption.
\end{remark}

\begin{example}[Absorption escape under $\opendo$]
\label{ex:absorption-escape}
Let $k = 3$ and consider
\[
  \bar C_A = (3,\, 2,\, 0,\, 0).
\]
Since $\dpre{,A} = 0 < 3 = k$, we have $\bar C_A \in \absorb$.  Now
let
\[
  \bar C_D = (10,\, 5,\, 4,\, 0).
\]
Under endogenous composition, $\wstar_D = 10 > 3 = \wstar_A$, so the
pivot shifts to $D$:
\[
  \bar C_A \opendo \bar C_D
    = (10,\; 2+5,\; 2+4,\; 0)
    = (10,\; 7,\; 6,\; 0).
\]
The pre-pivot count is $6 \ge 3 = k$: the composite has escaped
absorption.

\medskip
\noindent Under committed composition with $\kappa_A = 1$, the same
inputs yield
\[
  \bar C_A \opcommit \bar C_D
    = (3,\; 7,\; 0,\; 1).
\]
The pre-pivot count remains $0 < 3$: still absorbed.  Commitment
prevents escape.
\end{example}

\begin{remark}[Algebraic foundation of the validity mirage]
\label{rem:mirage-foundation}
The contrast between committed and uncommitted semantics is the
algebraic foundation of the \emph{validity mirage} developed in
\cref{ch:mirage}~\citep{gaffney2026mirage}.

An enumerative solver with beam width $M > 1$ explores multiple pivot
candidates simultaneously.  Each candidate corresponds to a different
factorisation of the sequence into pre-pivot, pivot, and post-pivot
material.  The solver's search process effectively operates under
$\opendo$ semantics: it can ``escape'' absorption by substituting a
different pivot---one that was not the running-max focal event of the
original prefix, but that appeared in a different branch of the beam.

The resulting output may well be valid: every grammar constraint is
satisfied, and every structural requirement is met.  But the pivot
identity has changed.  The solution that the solver returns is not the
solution that the original prefix was building toward.  Semantics have
drifted silently, and no constraint violation has been raised.  This is
the mirage: the output \emph{looks} correct because validity is
preserved, yet the underlying semantic intent has been abandoned.
\end{remark}

% ══════════════════════════════════════════════════════════════
\section{The No-Absorption Compression Contract}
\label{sec:compression-contract}
% ══════════════════════════════════════════════════════════════

The left ideal theorem (\cref{prop:absorbing-left-ideal}) tells us that
committed absorption is permanent.  The natural follow-up question is:
which operations can \emph{create} absorption where none existed before?
The empirical answer, detailed in \cref{sec:empirical-validation}, is
that compression is the unique culprit.  This section formulates the
contract that prevents it.

\begin{definition}[No-absorption contract]
\label{def:no-absorption-contract}
Fix a grammar prefix requirement $k \in \N$.  A \emph{compression map}
$\mu \colon \mathcal{E} \to \mathcal{E}$ (where $\mathcal{E}$ denotes
the space of event blocks) satisfies the \emph{no-absorption contract}
if for every event block $B$,
\[
  \dtotal\bigl(\mu(B)\bigr)
  \;\ge\;
  \min\bigl(\dtotal(B),\; k\bigr).
\]
\end{definition}

\begin{remark}[Operational interpretation]
\label{rem:contract-interpretation}
The contract permits compression to remove redundant development
capacity---events beyond what is needed for the prefix requirement---but
forbids it from reducing the development count below the survival
threshold $k$.  Concretely:
\begin{itemize}[itemsep=3pt]
  \item If the original block has $\dtotal(B) \ge k$, then the
        compressed block must also have $\dtotal(\mu(B)) \ge k$.
        Compression may discard the excess, but it must retain at least
        $k$ development events.
  \item If the original block has $\dtotal(B) < k$, then the contract
        requires only $\dtotal(\mu(B)) \ge \dtotal(B)$: the compression
        must not make the situation worse, but it is not required to
        conjure development events that were never there.
\end{itemize}
\end{remark}

\begin{remark}[Why $\dtotal$ and not $\dpre$]
\label{rem:why-dtotal}
The contract is stated in terms of $\dtotal$ rather than $\dpre$ for a
fundamental reason: the compression map does not know which event will
ultimately serve as the pivot.

At compression time, the pivot identity may still be provisional
($\kappa = 0$), and a future suffix may shift the pivot to a different
event.  The pre-pivot count $\dpre$ is defined relative to a specific
pivot; if the pivot changes, $\dpre$ changes with it.  By contrast,
$\dtotal$ counts \emph{all} non-focal events in the block, regardless
of their position relative to any particular pivot.

Preserving $\dtotal \ge k$ ensures that no matter where the pivot
ultimately lands, there is sufficient development material available for
the prefix requirement to be satisfied.  A contract based on $\dpre$
would protect only the current pivot assignment and would become
vacuous---or actively harmful---if the pivot subsequently shifted.
\end{remark}

\paragraph{From definition to algorithm.}
\Cref{def:no-absorption-contract} specifies the \emph{minimal} contract: a
guard on $\dtotal$ that ensures enough development capacity survives
compression to meet the prefix requirement~$k$.
\Cref{alg:contract-compress} below implements a \emph{stronger} guard
that additionally verifies suffix-feasibility---i.e., that the tropical
weight at rank~$k$ is preserved after each candidate removal.  The
stronger guard is not required by the theory (the absorbing-ideal results
hold with the minimal contract) but is advisable in practice, since it
catches near-absorbing states that the minimal $\dtotal \ge k$ contract
would pass.

\begin{algorithm}[t]
\caption{Contract-guarded compression}
\label{alg:contract-compress}
\begin{algorithmic}[1]
\Require Event block $\mathit{events}$, prefix requirement $k$,
         retention fraction $\mathit{retention} \in (0,1]$,
         random seed $\mathit{seed}$
\Ensure Compressed event block satisfying the no-absorption contract
\State $\mathit{target} \gets
       \lceil \mathit{retention} \times |\mathit{events}| \rceil$
\State $\mathit{removable} \gets$ non-focal events in
       $\mathit{events}$, shuffled by $\mathit{seed}$
\State $\mathit{removed} \gets 0$
\For{each candidate event $e$ in $\mathit{removable}$}
  \If{$|\mathit{events}| - \mathit{removed} \le \mathit{target}$}
    \State \textbf{break} \Comment{Reached retention target}
  \EndIf
  \State $\mathit{ctx\_without} \gets
         \Call{BuildTropicalContext}{\mathit{events} \setminus \{e\},\; k}$
  \If{$\mathit{ctx\_without}.\mathbf{W}[k]
      \ge \mathit{original}.\mathbf{W}[k]$}
    \Comment{Feasibility preserved}
    \State Remove $e$ from $\mathit{events}$;\quad
           $\mathit{removed} \gets \mathit{removed} + 1$
  \Else
    \State Block $e$ \Comment{Removal would violate contract}
  \EndIf
\EndFor
\State \Return remaining events
\end{algorithmic}
\end{algorithm}

\Cref{alg:contract-compress} presents the contract-guarded compression
procedure.  The algorithm iterates over non-focal events in a
seed-determined random order, greedily attempting to remove each one.
Before removing an event, it rebuilds the tropical context
(\cref{ch:tropical-lift}) for the reduced block and checks whether the
tropical weight at rank $k$ is preserved.  If removal would reduce
$\mathbf{W}[k]$ below its original value, the event is blocked: it is
essential for maintaining the prefix requirement.

The greedy removal order (controlled by the random seed) means that the
algorithm is not guaranteed to achieve the \emph{maximum} compression
ratio consistent with the contract.  However, it is guaranteed to
\emph{satisfy} the contract: every removal is individually checked, and
no removal that would violate feasibility is permitted.  The retention
parameter provides a hard floor on the number of events retained,
offering an additional safety margin beyond the algebraic contract.
The overhead of the contract guard is bounded: each candidate removal
requires one call to \textsc{BuildTropicalContext} at cost $O(n \cdot k)$,
and at most $O(n)$ candidates are tested, giving $O(n^2 k)$ worst-case
cost---acceptable for the moderate block sizes typical of streaming
applications.

% ══════════════════════════════════════════════════════════════
\section{Empirical Validation}
\label{sec:empirical-validation}
% ══════════════════════════════════════════════════════════════

The algebraic theory of the preceding sections makes precise
predictions.  We now report the experimental results that validate those
predictions and, crucially, identify the one operation that breaks them.

% ── Algebraic core ────────────────────────────────────────────
\subsection{Algebraic Core (Experiment~51)}
\label{subsec:exp51}

\Cref{ch:context-algebra} established that $\opendo$ is strictly associative.
The same property extends to $\opcommit$: committed composition
inherits associativity from the case analysis, because Case~1 is
idempotent (committed blocks remain committed) and the remaining cases
reduce to $\opendo$.

\emph{Pairwise exactness.}\quad
Across 240 randomly generated pairs of extended context elements, the
pairwise composition $\bar C_A \opcommit \bar C_B$ was computed and
compared against a reference implementation that evaluates the full
event-level sequence.  Result: \textbf{0 violations out of 240 cases.}

\emph{Associativity.}\quad
Across 80 randomly generated triples $(\bar C_A, \bar C_B, \bar C_C)$,
the two bracketings $(\bar C_A \opcommit \bar C_B) \opcommit \bar C_C$
and $\bar C_A \opcommit (\bar C_B \opcommit \bar C_C)$ were compared
component-wise.  Result: \textbf{0 violations out of 80 cases.}

These results validate the monoid structure computationally.

% ── Absorbing-ideal closure ───────────────────────────────────
\subsection{Absorbing-Ideal Closure (Experiment~56)}
\label{subsec:exp56}

To test \cref{prop:absorbing-left-ideal} empirically, we generated 96
test cases, each consisting of a committed absorbing element $\bar C_A$
(with $\kappa_A = 1$ and $\dpre{,A} < k$) composed on the right with a
randomly generated suffix $\bar C_D$.  In every case, we checked whether
the result $\bar C_A \opcommit \bar C_D$ satisfied $\dpre < k$.

\medskip
\noindent
Result: \textbf{0 violations out of 96 cases.}  The left ideal property
holds exactly across the full test suite.

% ── Closure diagnostics ──────────────────────────────────────
\subsection{Closure Diagnostics (Experiment~57)}
\label{subsec:exp57}

Which elementary operations preserve the monoid's algebraic structure,
and which break it?  Experiment~57 tested each operation independently
for closure violations.  The results are summarised in
\cref{tab:closure-diagnostics}.

\begin{table}[ht]
\centering
\caption{Closure violation rates by operation (Experiment~57).
         Each operation was tested on a suite of randomly generated
         extended context elements.  A violation occurs when the
         operation produces an output that crosses the absorbing
         boundary: a non-absorbing input yields an absorbing output, or
         vice versa.}
\label{tab:closure-diagnostics}
\begin{tabular}{@{}lc@{}}
\toprule
\textbf{Operation} & \textbf{Violation rate} \\
\midrule
Composition ($\opcommit$)           & 0.000 \\
Compression                         & 0.133 \\
Pivot update                        & 0.000 \\
Split-at-point                      & 0.000 \\
Absorbing escape under compression  & 0.033 \\
\bottomrule
\end{tabular}
\end{table}

The results are striking.  Every operation except compression has a
violation rate of exactly zero: composition, pivot update, and
split-at-point all preserve the algebraic structure perfectly.
Compression alone violates closure, at a rate of 13.3\%.  Furthermore,
3.3\% of compressions cause an \emph{absorbing escape}: a committed
absorbing element is compressed into a non-absorbing one, or---more
dangerously---a non-absorbing element is compressed into an absorbing
one.

\textbf{Conclusion:} Compression is the \emph{unique closure-breaking
operation} in the context algebra.  This is the key empirical result of
the chapter.  Every other elementary operation preserves the monoid
structure, but naive compression can push elements across the absorbing
boundary.  This finding motivates the no-absorption contract of
\cref{def:no-absorption-contract}: it is precisely the guard needed to
tame the one operation that breaks closure.

% ── Absorption escape rates ──────────────────────────────────
\subsection{Absorption Escape Rates}
\label{subsec:escape-rates}

\begin{figure}[ht]
\centering
\includegraphics[width=0.85\textwidth]{%
  figures/test_04_absorption_escape_rates.png}
\caption{Absorption escape rates under committed versus uncommitted
         semantics across generator configurations.  Under committed
         semantics ($\opcommit$), the escape rate is zero: the left
         ideal property holds.  Under uncommitted semantics ($\opendo$),
         escape is possible and occurs at measurable rates that depend
         on the weight distribution and block length.}
\label{fig:absorption-escape-rates}
\end{figure}

\Cref{fig:absorption-escape-rates} displays the absorption escape rates
for committed and uncommitted semantics across a range of generator
configurations.  The committed escape rate is uniformly zero, confirming
\cref{prop:absorbing-left-ideal}.  The uncommitted escape rate varies
with the generator configuration but is consistently positive,
illustrating the escape mechanism described in \cref{sec:escape-endo}.
The gap between the two curves is the algebraic signature of
commitment-induced irreversibility.

% ══════════════════════════════════════════════════════════════
\section{Exercises}
\label{sec:ch7-exercises}
% ══════════════════════════════════════════════════════════════

\begin{exercise}[The absorbing ideal is not two-sided]
\label{exer:not-two-sided}
Show that the absorbing ideal of \cref{prop:absorbing-left-ideal} is
\emph{not} a two-sided ideal.  That is, construct extended context
elements $\bar C_A$ and $\bar C_D$ such that $\bar C_A \in \absorb$
with $\kappa_A = 1$, yet $\bar C_D \opcommit \bar C_A \notin \absorb$.

\emph{Hint:}  Choose $\bar C_D$ with $\kappa_D = 0$ and a pivot of
sufficiently high weight so that if $\bar C_D$ is on the left and
uncommitted, the pivot stays in $D$.  If $\dpre{,D} \ge k$, the
composite inherits the non-absorbing pre-pivot count from $D$.
\end{exercise}

\begin{exercise}[Closure of absorbing elements under committed composition]
\label{exer:absorbing-closure}
Prove that under committed semantics, the set of committed absorbing
elements is closed under $\opcommit$ from the left.  Specifically, show
that if $\bar C_A \in \absorb$ with $\kappa_A = 1$ and $\bar C_B \in
\absorb$ with $\kappa_B = 1$, then
$\bar C_A \opcommit \bar C_B \in \absorb$ with commitment flag
$\kappa = 1$.
\end{exercise}

\begin{exercise}[Maximum compression under the no-absorption contract]
\label{exer:max-compression}
Design a compression policy that satisfies the no-absorption contract
(\cref{def:no-absorption-contract}) while achieving the maximum
compression ratio.  Specifically, for an event block $B$ with
$\dtotal(B) = 20$ non-focal events and grammar prefix requirement
$k = 3$:
\begin{enumerate}[label=(\alph*)]
  \item What is the theoretical minimum number of non-focal events that
        must be retained?
  \item What is the maximum number of events that can be removed?
  \item Does the answer change if the block also contains focal events?
        Explain.
\end{enumerate}
\end{exercise}

\begin{exercise}[Why $\dtotal$ and not $\dpre$ in the contract]
\label{exer:why-dtotal}
Explain why the no-absorption contract
(\cref{def:no-absorption-contract}) is stated in terms of $\dtotal$
rather than $\dpre$.  Construct a concrete scenario in which a
compression map $\mu$ preserves $\dpre(\mu(B)) \ge k$ for the current
pivot but produces a compressed block with $\dtotal(\mu(B)) < k$.  Show
that a subsequent pivot shift (under $\opendo$ semantics) can then push
the compressed block into the absorbing set, yielding $\dpre < k$ for
the new pivot.
\end{exercise}

% ══════════════════════════════════════════════════════════════
%  Chapter 8 — Streaming Oscillation Traps
% ══════════════════════════════════════════════════════════════
\chapter{Streaming Oscillation Traps}\label{ch:streaming}

The preceding chapters analysed validity and absorption in an
\emph{offline} setting: the full event sequence is available before any
labelling decision is made.  In practice, however, events often arrive
one at a time, and the system must emit labels incrementally.  This
chapter studies what goes wrong when the endogenous pivot problem is
solved under streaming constraints.

The core phenomenon is an \emph{oscillation trap}: the streaming
policy commits to a pivot, then encounters a stronger focal event that
invalidates the commitment.  Because earlier labels are irrevocable, the
policy is left in a state from which no continuation can produce a
grammar-valid output.  We prove that such traps are inevitable---the
minimum inter-record gap in the pivot process is $O(1)$---and that they
affect more than half of organically generated sequences.  We then
develop a \emph{deferred commitment} policy that trades a small amount
of latency for a dramatic reduction in trap rate, and characterise the
quality--latency Pareto frontier.

We proceed as follows.  \Cref{sec:streaming-defs} formalises the
streaming extraction model and two concrete policies.
\Cref{sec:record-process} analyses the running-max pivot as a record
process.  \Cref{sec:adversarial-traps} constructs adversarial sequences
that trap the commit-now policy and proves the oscillation trap
threshold.  \Cref{sec:organic-prevalence} reports the organic trap rate
(Experiment~43).  \Cref{sec:mechanism-verification} validates the
$\mingap < k$ predictor (Experiment~44).  \Cref{sec:scale-behaviour}
demonstrates scale invariance (Experiment~45).
\Cref{sec:record-gap-scaling} proves that the minimum inter-record gap
is $O(1)$ under i.i.d.\ weights.  \Cref{sec:deferred-commitment}
develops the deferred commitment policy and its Pareto analysis.
\Cref{sec:streaming-exercises} offers exercises.


% ══════════════════════════════════════════════════════════════
\section{Streaming Model Definitions}\label{sec:streaming-defs}
% ══════════════════════════════════════════════════════════════

We begin with the formal streaming extraction model.  The key
distinction from the offline setting of \cref{ch:absorbing-states} is
\emph{irrevocability}: once a label has been emitted, it cannot be
changed.

\begin{definition}[Streaming extraction policy]
\label{def:streaming-policy}
A \emph{streaming extraction policy} is a function that processes an
event sequence $e_1, e_2, \ldots, e_n$ in temporal order, maintaining a
running state $S_t$ at each step $t$.  At each step the policy may
assign an irrevocable label $\ell(e_t) \in
\{\textsc{development},\, \textsc{turning\_point},\,
\textsc{resolution},\, \textsc{pending}\}$ to the current event.
Labels assigned in previous steps cannot be modified:
\[
  \text{if } \ell(e_s) \neq \textsc{pending} \text{ at step } t,
  \quad
  \text{then } \ell(e_s) \text{ is fixed for all } t' > t.
\]
The policy's output is the final label assignment
$(\ell(e_1), \ldots, \ell(e_n))$.
\end{definition}

\begin{definition}[Commit-now policy]
\label{def:commit-now}
The \emph{commit-now policy} is a streaming extraction policy that
operates as follows.  Let $a^*$ denote the focal actor and let
$\tp_t = \argmax_{s \le t,\, a(e_s) = a^*} w(e_s)$ be the running-max
pivot at step~$t$.  At each step~$t$:
\begin{enumerate}[label=(\roman*)]
  \item If $e_t$ is a focal event with $w(e_t) > w(\tp_{t-1})$ (or
        $t = 1$ and $e_t$ is the first focal event), then $e_t$ becomes
        the new running-max pivot: $\tp_t = e_t$.  The policy
        immediately commits: it assigns $\ell(e_t) =
        \textsc{turning\_point}$ and labels all pending non-focal events
        relative to $e_t$ as the new pivot.
  \item Labels assigned in previous steps are \textbf{irrevocable}.  If
        $\tp_{t-1}$ was previously labelled $\textsc{turning\_point}$
        and a new pivot $\tp_t \neq \tp_{t-1}$ is chosen, the old label
        cannot be retracted.
\end{enumerate}
\end{definition}

\begin{definition}[Buffered policy with patience $f$]
\label{def:buffered-policy}
The \emph{buffered policy with patience $f \in (0,1)$} buffers events
until step $\lceil f \cdot n \rceil$, then commits to the running-max
pivot at that point.  Formally:
\begin{enumerate}[label=(\roman*)]
  \item For $t \le \lceil f \cdot n \rceil$, all labels remain
        $\textsc{pending}$.  The policy tracks the running-max pivot
        $\tp_t$ but does not emit any irrevocable labels.
  \item At step $t = \lceil f \cdot n \rceil$, the policy commits to
        $\tp_t$ as the final pivot.  All buffered events receive their
        labels relative to $\tp_t$.  Events arriving after step
        $\lceil f \cdot n \rceil$ receive committed labels immediately.
\end{enumerate}
Events before the buffer point receive tentative labels that are
finalised at commitment time; events after receive committed labels
on arrival.
\end{definition}

\begin{definition}[Streaming absorbing trap]
\label{def:streaming-trap}
A \emph{streaming absorbing trap} is a state reached during streaming
in which no continuation of the event sequence can produce a
grammar-valid output under the commit-now policy.  Formally, the
policy is trapped at step~$t$ if:
\begin{enumerate}[label=(\roman*)]
  \item The running-max pivot has shifted from $\tp_{t-1}$ to
        $\tp_t \neq \tp_{t-1}$ (a \emph{pivot shift}).
  \item The number of non-focal events between $\tp_{t-1}$ and $\tp_t$
        in the sequence is strictly less than $k$ (the grammar's prefix
        requirement).
  \item The labels assigned to events before $\tp_{t-1}$ under the old
        pivot assignment are irrevocable and cannot be reassigned as
        $\textsc{development}$ for $\tp_t$.
\end{enumerate}
The trap is ``sprung'' when a pivot shift leaves too few development
events between the old and new pivot to satisfy the prefix requirement.
\end{definition}


% ══════════════════════════════════════════════════════════════
\section{Running-Max Pivot as Record Process}
\label{sec:record-process}
% ══════════════════════════════════════════════════════════════

The running-max pivot $\tp_t$ updates only when a new maximum is
encountered among focal events.  This is precisely the definition of a
\emph{record} in the theory of order statistics.

The running-max pivot at step~$t$ is
\[
  \tp_t \;=\; \argmax_{s \le t,\; a(e_s) = a^*} w(e_s).
\]
The pivot updates at step~$t$ if and only if $w(e_t) > w(\tp_{t-1})$
and $a(e_t) = a^*$---that is, if $e_t$ is a \emph{record} among the
focal events seen so far.  Each such update is a \emph{pivot shift}.

Under i.i.d.\ continuous weights, the theory of records gives sharp
predictions about the frequency and timing of pivot shifts.

\begin{proposition}[Expected number of records]
\label{prop:expected-records}
If the weights of $n$ focal events are drawn i.i.d.\ from a continuous
distribution, the expected number of records (pivot shifts) is the
$n$-th harmonic number:
\[
  \mathbb{E}[\text{number of records}] \;=\; H_n
  \;=\; \sum_{i=1}^{n} \frac{1}{i}
  \;\approx\; \ln n + \gamma,
\]
where $\gamma \approx 0.5772$ is the Euler--Mascheroni constant.
\end{proposition}

\begin{proof}
The $i$-th observation is a record if and only if it is the maximum of
the first $i$ observations.  For i.i.d.\ continuous random variables,
each of the $i$ observations is equally likely to be the maximum, so
$\Pr(\text{$i$-th is a record}) = 1/i$.  By linearity of expectation,
the expected number of records in $n$ observations is
$\sum_{i=1}^{n} 1/i = H_n$.
\end{proof}

In practice, the event weights are not i.i.d.: the bursty generators
used in our experiments impose a \emph{front-loading} structure
controlled by the parameter $\varepsilon$.  Front-loading concentrates
higher weights among earlier focal events, which affects both the
number of pivot shifts and their timing.

\begin{table}[ht]
\centering
\caption{Pivot stability profile.  Mean number of pivot shifts and
         median position of the last shift (as a fraction of the
         timeline), across front-loading parameter~$\varepsilon$.
         Higher front-loading reduces the number of shifts and causes
         earlier stabilisation.}
\label{tab:pivot-stability}
\begin{tabular}{@{}ccc@{}}
\toprule
\textbf{Front-loading $\varepsilon$}
  & \textbf{Mean shifts}
  & \textbf{Median last-shift position} \\
\midrule
$0.05$ & $4.64$ & $0.481$ \\
$0.10$ & $4.16$ & $0.436$ \\
$0.20$ & $3.72$ & $0.382$ \\
$0.40$ & $3.31$ & $0.312$ \\
$0.60$ & $3.09$ & $0.265$ \\
$0.80$ & $2.95$ & $0.224$ \\
\bottomrule
\end{tabular}
\end{table}

\Cref{tab:pivot-stability} summarises the pivot stability profile.  At
low front-loading ($\varepsilon = 0.05$), the pivot shifts an average
of $4.64$ times, with the last shift occurring at a median position of
$0.481$ along the timeline---nearly halfway through.  At high
front-loading ($\varepsilon = 0.80$), the mean number of shifts drops
to $2.95$ and the pivot stabilises by roughly the first quarter.

Each pivot shift is a record in the weight sequence.  The \emph{cost} of
a shift is proportional to the number of events labelled since the last
shift---the ``blast radius.''  Events labelled under the previous pivot
assignment have irrevocable labels that may now conflict with the
grammar's requirements under the new pivot.  The larger the blast
radius, the more damage a pivot shift inflicts on the streaming policy's
output.


% ══════════════════════════════════════════════════════════════
\section{Adversarial Oscillation Traps}\label{sec:adversarial-traps}
% ══════════════════════════════════════════════════════════════

We now construct sequences that deterministically trap the commit-now
policy.  The construction makes the trapping mechanism explicit:
record-setting focal events with strictly increasing weights are
interleaved with non-focal events, and the oscillation period controls
the spacing.

\begin{definition}[Adversarial oscillation generator]
\label{def:adversarial-generator}
The \emph{adversarial oscillation generator} with parameters
$(n, p, a^*)$ produces a sequence of $n$ events as follows.  Let $p$
denote the \emph{oscillation period}: the number of events between
consecutive focal spikes.  The generator places focal events at
positions $1, p+1, 2p+1, \ldots$ with strictly increasing weights
$w_1 < w_2 < w_3 < \cdots$.  All other positions are filled with
non-focal events.  The result is a sequence where every focal event is
a record (a new running-max), and there are exactly $p - 1$ non-focal
events between consecutive focal events.
\end{definition}

The key quantity is the \emph{effective pre-pivot capacity}: the
maximum number of non-focal events available between consecutive
record-setting focal events to serve as \textsc{development} for the
new pivot.

\begin{theorem}[Oscillation trap threshold]
\label{thm:oscillation-trap}
Consider a sequence produced by the adversarial oscillation generator
(\cref{def:adversarial-generator}) with oscillation period~$p$.  Under
the commit-now policy (\cref{def:commit-now}) with grammar prefix
requirement~$k$, the policy is trapped if and only if
\[
  \peff \;<\; k,
\]
where $\peff$ is the effective pre-pivot capacity: the maximum number
of non-focal events between consecutive record-setting focal events
that are available for \textsc{development} assignment.  For adversarial
alternating-spike traces,
\[
  \peff \;=\; \max\!\bigl(1,\; \lfloor p/2 \rfloor\bigr).
\]
\end{theorem}

\begin{proof}
We prove both directions.

\medskip
\noindent$(\Rightarrow)$\; Suppose $\peff < k$.  Each time the
running-max pivot shifts to a new focal event $e_{\mathrm{new}}$, the
commit-now policy fixes $e_{\mathrm{new}}$ as $\textsc{turning\_point}$.
Between the previous pivot $e_{\mathrm{old}}$ and $e_{\mathrm{new}}$,
there are at most $\peff$ non-focal events that could serve as
$\textsc{development}$.

Since $\peff < k$, the prefix requirement is not met for
$e_{\mathrm{new}}$: the grammar demands at least $k$
$\textsc{development}$ events before the $\textsc{turning\_point}$, but
only $\peff < k$ are available in the interval between the two pivots.

Moreover, labels assigned to events before $e_{\mathrm{old}}$ under the
old pivot assignment are irrevocable.  These events were labelled
relative to $e_{\mathrm{old}}$---some as $\textsc{development}$ for
$e_{\mathrm{old}}$, some as $\textsc{resolution}$---and these labels
cannot be changed to $\textsc{development}$ for $e_{\mathrm{new}}$.
Therefore the commit-now policy is trapped: the grammar requires $k$
$\textsc{development}$ events before the $\textsc{turning\_point}$, but
only $\peff < k$ are available, and no continuation of the sequence can
supply additional pre-pivot events because the $\textsc{turning\_point}$
is already fixed at $e_{\mathrm{new}}$.

\medskip
\noindent$(\Leftarrow)$\; Suppose $\peff \ge k$.  Consider the final
record---the last pivot shift in the sequence.  Let $e_{\mathrm{final}}$
be the final pivot and $e_{\mathrm{prev}}$ the penultimate pivot.  By
assumption, there are at least $\peff \ge k$ non-focal events between
$e_{\mathrm{prev}}$ and $e_{\mathrm{final}}$.

The commit-now policy can assign $k$ of these non-focal events as
$\textsc{development}$, satisfying the prefix requirement for
$e_{\mathrm{final}}$.  Events after $e_{\mathrm{final}}$ can be labelled
$\textsc{resolution}$.  Since $e_{\mathrm{final}}$ is the last record,
no further pivot shift will occur, so these labels remain valid.  The
grammar is satisfied.
\end{proof}

\Cref{tab:adversarial-boundary} reproduces the adversarial boundary
for the commit-now policy across oscillation periods and prefix
requirements.  The boundary is sharp: validity transitions from $0\%$
to $100\%$ at the exact threshold predicted by
\cref{thm:oscillation-trap}.

\begin{table}[ht]
\centering
\caption{Commit-now validity (\%) versus oscillation period for
         prefix requirements $k = 1, 2, 3$.  The $0\% \to 100\%$
         boundary occurs at $\peff = k$, exactly as predicted by
         \cref{thm:oscillation-trap}.}
\label{tab:adversarial-boundary}
\begin{tabular}{@{}cccc@{}}
\toprule
\textbf{Oscillation period $p$}
  & \textbf{$k = 1$}
  & \textbf{$k = 2$}
  & \textbf{$k = 3$} \\
\midrule
$1$ & $0$   & $0$   & $0$   \\
$2$ & $100$ & $0$   & $0$   \\
$3$ & $100$ & $0$   & $0$   \\
$4$ & $100$ & $100$ & $0$   \\
$5$ & $100$ & $100$ & $0$   \\
$6$ & $100$ & $100$ & $100$ \\
$7$ & $100$ & $100$ & $100$ \\
$8$ & $100$ & $100$ & $100$ \\
\bottomrule
\end{tabular}
\end{table}


% ══════════════════════════════════════════════════════════════
\section{Organic Prevalence (Experiment~43)}
\label{sec:organic-prevalence}
% ══════════════════════════════════════════════════════════════

The adversarial construction of \cref{sec:adversarial-traps}
demonstrates that traps are possible, but one might hope that they are
pathological---artefacts of worst-case engineering that rarely occur in
practice.  Experiment~43 tests this hope against organically generated
sequences.

We generated 4{,}200 sequences using the bursty event generator across
a grid of front-loading parameters $\varepsilon \in
\{0.05, 0.10, 0.20, 0.40, 0.60, 0.80\}$ and prefix requirements
$k \in \{1, 2, 3\}$, with 200 sequences per configuration.  Each
sequence was processed by the commit-now policy, and the output was
checked for grammar validity.

\begin{table}[ht]
\centering
\caption{Commit-now trap rates on organic (bursty-generated) sequences.
         Trap rate is the fraction of sequences where the commit-now
         policy produces a grammar-invalid output, despite the existence
         of a valid offline solution.}
\label{tab:organic-traps}
\begin{tabular}{@{}cccc@{}}
\toprule
& \textbf{$k = 1$} & \textbf{$k = 2$} & \textbf{$k = 3$} \\
\midrule
\textbf{Overall trap rate} & $38.9\%$ & $58.0\%$ & $67.9\%$ \\
\midrule
\textbf{Peak trap rate} & \multicolumn{3}{c}{$77\%$ at $\varepsilon = 0.40$, $k = 3$} \\
\midrule
\textbf{Offline validity} & \multicolumn{3}{c}{$83.5\%$--$97.5\%$} \\
\bottomrule
\end{tabular}
\end{table}

The headline result: \textbf{2{,}306 out of 4{,}200 organic sequences
exhibit commit-now traps}---a trap rate of $54.9\%$.
\Cref{tab:organic-traps} breaks down the trap rate by prefix
requirement~$k$.  Higher $k$ means greater vulnerability: the grammar
demands more pre-pivot development events, so a smaller inter-record gap
is sufficient to spring the trap.

The peak trap rate of $77\%$ occurs at $\varepsilon = 0.40$, $k = 3$.
This is not a coincidence: moderate front-loading produces enough pivot
shifts to create opportunities for trapping while concentrating records
early where inter-record gaps tend to be small.

Meanwhile, finite (offline) validity remains in the $83.5\%$--$97.5\%$
range across the same configurations.  The gap between offline
feasibility and streaming feasibility is precisely the trap rate: these
are sequences that \emph{have} valid solutions, but the commit-now
policy cannot find them because irrevocable early commitments block the
path.

\begin{figure}[ht]
\centering
\includegraphics[width=0.85\textwidth]{%
  figures/test_16_trap_rate_heatmaps.png}
\caption{Organic trap rates by $(\varepsilon, k)$.  Commit-now traps
         affect more than half of organic sequences.  The trap rate
         increases with $k$ (rows) and peaks at moderate front-loading
         (columns).  The gap between offline validity (high) and
         streaming validity (low) is the region where irrevocable
         commitment destroys feasibility.}
\label{fig:trap-rate-heatmaps}
\end{figure}

\Cref{fig:trap-rate-heatmaps} presents the full heatmap.


% ══════════════════════════════════════════════════════════════
\section{Mechanism Verification (Experiment~44)}
\label{sec:mechanism-verification}
% ══════════════════════════════════════════════════════════════

The oscillation trap threshold (\cref{thm:oscillation-trap}) identifies
$\peff < k$ as the trapping condition for adversarial sequences.  For
organic sequences, the analogous predictor is the \emph{minimum
inter-record gap}: for each sequence, compute the minimum number of
non-focal events between consecutive pivot shifts, and predict a trap
whenever $\mingap < k$.

\begin{definition}[Minimum inter-record gap]
\label{def:min-gap}
Given a sequence with records (pivot shifts) at positions
$r_1, r_2, \ldots, r_m$, the \emph{inter-record gap} before the $i$-th
record is the number of non-focal events between $r_{i-1}$ and $r_i$.
The \emph{minimum inter-record gap} is
\[
  \mingap \;=\; \min_{2 \le i \le m} (\text{non-focal events between }
  r_{i-1} \text{ and } r_i).
\]
\end{definition}

Experiment~44 evaluates the $\mingap < k$ predictor as a binary
classifier for commit-now traps on the same 4{,}200 organic sequences
from Experiment~43.

\begin{table}[ht]
\centering
\caption{Confusion matrix for the $\mingap < k$ trap predictor
         (Experiment~44).  The predictor achieves $100\%$ recall: every
         trapped sequence has $\mingap < k$.}
\label{tab:confusion-matrix}
\begin{tabular}{@{}lcc@{}}
\toprule
& \textbf{Predicted: Trap} & \textbf{Predicted: No Trap} \\
\midrule
\textbf{Actual: Trap}    & $1{,}040$ (TP) & $0$ (FN)   \\
\textbf{Actual: No Trap} & $124$ (FP)     & $490$ (TN) \\
\bottomrule
\end{tabular}
\end{table}

\Cref{tab:confusion-matrix} presents the confusion matrix.  The key
results:
\begin{itemize}[itemsep=3pt]
  \item \textbf{Accuracy:} $92.5\%$ \;
        $\bigl((1{,}040 + 490) / (1{,}040 + 124 + 0 + 490)\bigr)$.
  \item \textbf{Recall:} $100\%$ --- zero false negatives.  Every
        sequence that is actually trapped has $\mingap < k$.
  \item \textbf{Precision:} $89.3\%$ \;
        $\bigl(1{,}040 / (1{,}040 + 124)\bigr)$.
\end{itemize}

The zero false negatives confirm that $\mingap < k$ is a
\textbf{necessary condition} for trapping: if the minimum gap is at
least $k$, the commit-now policy is guaranteed to succeed.  The $124$
false positives occur because having a small gap does not guarantee that
the trap is sprung---the sequence may recover if later events provide
sufficient development material after the final pivot shift.

\begin{remark}[Necessary versus sufficient]
\label{rem:necessary-not-sufficient}
The $\mingap < k$ condition is necessary but not sufficient for
trapping.  Sufficiency fails because a small gap at an intermediate
record does not necessarily propagate to the final pivot assignment.  If
the \emph{last} record has a gap $\ge k$, the final pivot has enough
pre-pivot development events regardless of earlier gaps.  The false
positives are sequences where an intermediate record had $\mingap < k$
but the final pivot shift had a sufficient gap.
\end{remark}


% ══════════════════════════════════════════════════════════════
\section{Scale Behaviour (Experiment~45)}
\label{sec:scale-behaviour}
% ══════════════════════════════════════════════════════════════

One might conjecture that streaming traps are a small-sample artefact:
perhaps as $n$ grows, the increasing spacing between records gives the
commit-now policy room to breathe.  Experiment~45 tests this
conjecture by sweeping $n$ from $100$ to $1{,}000$.

The result is decisive: \textbf{trap rates remain in the $52\%$--$78\%$
band from $n = 100$ to $n = 1{,}000$}.  There is no decay with scale.
The mean minimum gap stays near $1.0$ regardless of $n$.

\begin{remark}[No escape through scale]
\label{rem:no-scale-escape}
The persistence of traps across scales is a direct consequence of the
$O(1)$ minimum inter-record gap established in
\cref{prop:record-gap-scaling}.  Although the number of records grows as
$\ln n$ and the \emph{average} gap between records grows as $n / \ln n$,
the \emph{minimum} gap does not grow.  The trap mechanism---which
depends on the minimum gap, not the average---remains active at every
practical scale.
\end{remark}


% ══════════════════════════════════════════════════════════════
\section{Record-Gap Scaling (Proposition~2)}
\label{sec:record-gap-scaling}
% ══════════════════════════════════════════════════════════════

We now establish the theoretical foundation for the scale-invariance of
streaming traps.  The key result is that the minimum inter-record gap is
$O(1)$ under i.i.d.\ continuous weights---it does not grow with $n$.

\begin{proposition}[Minimum inter-record gap is $O(1)$]
\label{prop:record-gap-scaling}
Let $X_1, X_2, \ldots, X_n$ be i.i.d.\ draws from a continuous
distribution (no ties almost surely).  Let $G_1, G_2, \ldots$ denote
the inter-record gaps: $G_i$ is the number of observations between the
$(i{-}1)$-th and $i$-th records.  Then:
\begin{enumerate}[label=(\roman*)]
  \item For the first gap: $\Pr(G_1 = 1) = 1/2$.
  \item The minimum gap satisfies
        $\min_i G_i = O(1)$ almost surely.  In particular,
        $\Pr(\min_i G_i = 1) \to 1$ as $n \to \infty$.
\end{enumerate}
\end{proposition}

\begin{proof}
We proceed in two parts.

\medskip
\noindent\textbf{Part (i): The first gap.}\quad
The first record is always $X_1$ (the first observation is trivially a
record).  The second record occurs at index $j$ where $X_j > X_1$.  The
first inter-record gap is $G_1 = j - 1$.

The gap equals $1$ if and only if $X_2 > X_1$---that is, the second
observation is itself a new record.  By exchangeability of continuous
i.i.d.\ random variables, $\Pr(X_2 > X_1) = 1/2$.  Therefore
$\Pr(G_1 = 1) = 1/2$.

\medskip
\noindent\textbf{Part (ii): The minimum gap.}\quad
Let $R_1 < R_2 < \cdots < R_M$ denote the positions of the $M$ records
among $X_1, \ldots, X_n$.  We have $R_1 = 1$ always, and $M \approx
\ln n$ in expectation.  The gaps are $G_i = R_{i+1} - R_i$ for
$i = 1, \ldots, M - 1$.

We need to show that $\Pr(\min_{1 \le i \le M-1} G_i \ge g) \to 0$ for
any fixed $g \ge 2$ as $n \to \infty$, which implies
$\Pr(\min_i G_i = 1) \to 1$.

\emph{Key insight.}\; The gaps $G_1, G_2, \ldots$ are not independent,
but each has a positive probability of equalling~$1$.  Specifically, if
the $i$-th record occurs at position $R_i = j$, then the next
observation $X_{j+1}$ is a new record (giving $G_i = 1$) if and only if
$X_{j+1}$ is the maximum of $X_1, \ldots, X_{j+1}$.  By
exchangeability,
\[
  \Pr(G_i = 1 \mid R_i = j) \;=\; \frac{1}{j + 1}.
\]
For the early records (small $j$), this probability is large: $1/2$ for
$j = 1$, $1/3$ for $j = 2$, and so on.

Since there are $M \approx \ln n$ records, the number of independent
``chances'' for a gap of $1$ grows without bound.  Even though the
individual probabilities $1/(j+1)$ decrease, the sum
$\sum_{i} 1/(R_i + 1)$ diverges (it is bounded below by a constant
fraction of $\sum_{j=1}^{n} 1/j^2$, and the first few terms alone give
a constant bounded away from zero).  By a standard Borel--Cantelli
argument, infinitely many gaps equal~$1$ almost surely, and therefore
$\min_i G_i = 1$ almost surely as $n \to \infty$.

More concretely, the probability that \emph{none} of the first $\ell$
gaps equals~$1$ is at most
\[
  \prod_{i=1}^{\ell} \Bigl(1 - \frac{1}{R_i + 1}\Bigr)
  \;\le\;
  \prod_{i=1}^{\ell} \Bigl(1 - \frac{1}{n}\Bigr)
  \;\le\;
  \exp\!\Bigl(-\frac{\ell}{n}\Bigr),
\]
but in fact the early records have $R_i$ much smaller than $n$, so the
bound is tighter.  For the first gap alone, $\Pr(G_1 \neq 1) = 1/2$,
so with just two independent opportunities, $\Pr(\text{both}
\neq 1) \le 1/2 \cdot 2/3 = 1/3$.  The probability decays rapidly and
converges to zero.
\end{proof}

\begin{remark}[Why traps persist across scales]
\label{rem:traps-persist}
\Cref{prop:record-gap-scaling} explains the empirical finding of
\cref{sec:scale-behaviour}: even though the number of records grows as
$\ln n$, the gaps between them include arbitrarily small values (down
to $1$), and the minimum gap does not grow with $n$.  The trap mechanism
$\mingap < k$ remains active because the minimum gap is $O(1)$, not
$O(n / \ln n)$.  Increasing $n$ adds more records with more
opportunities for small gaps, rather than spacing existing records
further apart.
\end{remark}

\begin{remark}[I.i.d.\ versus bursty weights]
\label{rem:iid-vs-bursty}
The proof of \cref{prop:record-gap-scaling} uses i.i.d.\ continuous
weights as a reference model.  The bursty generators used empirically
are \emph{not} i.i.d.: they impose a front-loading structure in which
higher weights are concentrated among earlier events (controlled by the
parameter~$\varepsilon$).  The empirical trap rates are \emph{higher}
than the i.i.d.\ prediction because front-loading concentrates records
early in the sequence, precisely where the inter-record gaps are most
likely to be small.

Specifically, front-loading increases $\Pr(G_1 = 1)$ beyond the i.i.d.\
value of $1/2$, because early focal events are more likely to have high
weights, making consecutive records in the first few positions more
probable.  The i.i.d.\ analysis therefore provides a \emph{lower bound}
on the trap rate for bursty sequences.
\end{remark}


% ══════════════════════════════════════════════════════════════
\section{Deferred Commitment}\label{sec:deferred-commitment}
% ══════════════════════════════════════════════════════════════

The commit-now policy's fundamental weakness is that it commits
irrevocably at each pivot shift, before the pivot has stabilised.
Deferring commitment---waiting until a fraction $f$ of the sequence
has been observed before locking in the pivot---trades latency for
validity.  This section characterises the quality--latency Pareto
frontier.

% ── Quality-latency Pareto curve ──────────────────────────────
\subsection{The Quality--Latency Pareto Curve}
\label{subsec:pareto-curve}

Experiment~48c evaluates the buffered policy
(\cref{def:buffered-policy}) across patience values $f \in
\{0, 0.10, 0.25, 0.50\}$ and compares against the finite (offline)
solver as a ceiling.  All evaluations use TP-consistent scoring
(see \cref{subsec:tp-consistent}).

\begin{table}[ht]
\centering
\caption{Quality--latency Pareto curve.  Effective validity, score, and
         regret for the commit-now policy, deferred policies at three
         patience levels, and the offline ceiling.  Deferred commitment
         at $f = 0.25$ recovers $80.1\%$ validity versus commit-now's
         $39.4\%$.}
\label{tab:pareto-curve}
\begin{tabular}{@{}lccc@{}}
\toprule
\textbf{Policy}
  & \textbf{Eff.\ Valid\%}
  & \textbf{Eff.\ Score}
  & \textbf{Eff.\ Regret} \\
\midrule
Commit-now ($f = 0$)       & $39.4$ & $6.68$  & $9.90$ \\
Deferred $f = 0.10$        & $62.6$ & $10.60$ & $5.99$ \\
Deferred $f = 0.25$        & $80.1$ & $14.17$ & $2.42$ \\
Deferred $f = 0.50$        & $88.2$ & $15.93$ & $0.66$ \\
Finite (offline)           & $91.5$ & $16.58$ & $0.00$ \\
\bottomrule
\end{tabular}
\end{table}

\Cref{tab:pareto-curve} presents the results.  The headline: deferred
commitment at $f = 0.25$ achieves $80.1\%$ validity and a score of
$14.17$, compared with commit-now's $39.4\%$ validity and $6.68$ score.
Deferred commitment dominates commit-now from $f = 0.05$ onward---every
deferred policy along the Pareto frontier achieves both higher validity
and higher score than the commit-now baseline.

At $f = 0.50$, the deferred policy reaches $88.2\%$ validity, within
$3.3$ percentage points of the offline ceiling.  The remaining gap
represents sequences where the global-max pivot arrives after the
halfway mark---a diminishing population as $f$ increases.

% ── TP-consistent scoring ─────────────────────────────────────
\subsection{TP-Consistent Scoring}\label{subsec:tp-consistent}

Comparing streaming and offline policies requires care.  The naive
approach---scoring the streaming output against the offline gold
standard---creates an $\argmax$ inconsistency.  The streaming policy
commits to a pivot $\tp_{\mathrm{stream}}$, while the offline solver
selects a (potentially different) pivot $\tp_{\mathrm{offline}}$.  If we
score the streaming output using $\tp_{\mathrm{offline}}$ as the
reference, we are penalising the streaming policy for a choice it was
forced to make under information constraints.

\emph{TP-consistent scoring} resolves this by evaluating each policy
against its own committed pivot.  When scoring a deferred policy's
output, we demote competing focal events---focal events other than the
policy's committed pivot---so that the evaluation pivot matches the
policy's pivot.  This ensures that differences in score reflect
structural quality (whether the grammar is satisfied, how many
development events precede the pivot) rather than pivot identity
disagreements.

Without TP-consistent scoring, a deferred policy that selects a
slightly weaker pivot but satisfies the grammar perfectly would receive
a lower score than an offline policy that selects the global-max pivot.
The score difference would reflect the weight gap between pivots, not
the structural quality of the output.  TP-consistent scoring removes
this confound.

% ── Why f = 0.25 ──────────────────────────────────────────────
\subsection{Why $f = 0.25$}\label{subsec:why-f-quarter}

The choice $f = 0.25$ is not arbitrary.  It is motivated by the
empirical distribution of the global-max pivot's arrival time.

\begin{remark}[Pivot arrival CDF]
\label{rem:pivot-arrival-cdf}
Let $F(t)$ denote the cumulative distribution function of the position
(as a fraction of $n$) at which the global-max focal event first
appears in the sequence.  Across our bursty generator configurations,
$F(0.25) \approx 0.50$: the global-max pivot arrives before position
$0.25$ in roughly half of all instances.

A deferred policy with $f = 0.25$ therefore captures the correct
pivot (the global max) in approximately $50\%$ of sequences at
commitment time, with no further pivot shifts possible after commitment.
For the remaining $50\%$, the policy commits to a suboptimal pivot, but
the grammar can often still be satisfied because the suboptimal pivot
typically has sufficient pre-pivot development.
\end{remark}

The $f = 0.25$ policy thus represents a favourable trade-off: it waits
only through the first quarter of the event stream, captures the
majority of pivot shifts, and recovers $80\%$ of offline validity.
Further increasing $f$ yields diminishing returns: from $f = 0.25$ to
$f = 0.50$, validity improves by only $8.1$ percentage points at the
cost of doubling the latency.

\begin{figure}[ht]
\centering
\includegraphics[width=0.85\textwidth]{%
  figures/test_48c_pareto_curve.png}
\caption{Quality--latency Pareto curve.  Each point represents a
         streaming policy with patience $f$, plotted by effective
         validity (horizontal) and effective score (vertical).  The
         commit-now policy ($f = 0$) occupies the lower-left corner;
         the offline ceiling ($f = 1$) the upper-right.  Deferred
         policies trace a concave Pareto frontier, with $f = 0.25$
         offering the best marginal return per unit of latency.}
\label{fig:pareto-curve}
\end{figure}

\Cref{fig:pareto-curve} plots the full Pareto frontier.


% ══════════════════════════════════════════════════════════════
\section{Exercises}\label{sec:streaming-exercises}
% ══════════════════════════════════════════════════════════════

\begin{exercise}[First inter-record gap]
\label{exer:first-gap}
Prove that for i.i.d.\ $\mathrm{Uniform}[0,1]$ weights, the
probability that the first inter-record gap equals~$1$ is exactly
$1/2$.

\emph{Hint.}\; The first record is $X_1$.  The gap equals $1$ iff
$X_2 > X_1$.  Use exchangeability: for i.i.d.\ continuous random
variables, $\Pr(X_2 > X_1) = 1/2$.  This is a direct consequence of
the proof of \cref{prop:record-gap-scaling}(i).
\end{exercise}

\begin{exercise}[Front-loading and pivot arrival]
\label{exer:frontloading-pivot}
Analyse the effect of the front-loading parameter $\varepsilon$ on the
median pivot arrival fraction.  Specifically:
\begin{enumerate}[label=(\alph*)]
  \item As $\varepsilon$ increases (more weight concentrated among
        early events), what happens to the median fraction of the
        timeline at which the global-max pivot first appears?
  \item Why does early stabilisation of the pivot \emph{not} eliminate
        streaming traps?
\end{enumerate}

\emph{Answer.}\; (a) The median pivot arrival fraction
\textbf{decreases}: the global max occurs earlier because early focal
events receive disproportionately high weights.  The pivot stabilises
sooner.  (b) Early stabilisation means that the gap between the first
and second records is likely to be very small---possibly $1$---because
the first few focal events are packed together with high weights.  The
pivot stabilises quickly, but the initial pivot shifts create small
gaps that spring the trap before stabilisation occurs.
\end{exercise}

\begin{exercise}[Step-by-step trap construction]
\label{exer:trap-construction}
Construct a sequence of $8$ events (with a focal actor $a^*$ and prefix
requirement $k = 2$) in which the commit-now policy is trapped but a
deferred policy with $f = 0.5$ succeeds.  Show the trap step by step:
\begin{enumerate}[label=(\alph*)]
  \item List the $8$ events with their types (focal/non-focal) and
        weights.
  \item Trace the commit-now policy: show the running-max pivot, the
        labels emitted, and the step at which the trap is sprung.
  \item Trace the deferred policy with $f = 0.5$ (buffer until
        step~$4$): show that the pivot stabilises during the buffer
        period and the grammar is satisfied.
\end{enumerate}

\emph{Hint.}\; Use two focal events at positions $1$ and $3$ with
weights $w_1 = 5$ and $w_3 = 10$, with a single non-focal event at
position $2$ between them.  This gives a gap of $1 < k = 2$.  Place
non-focal events at positions $4$--$8$ so the deferred policy has
sufficient development material.
\end{exercise}

\begin{exercise}[Why higher $k$ means more traps]
\label{exer:higher-k}
Explain why increasing $k$ makes streaming traps more likely.  Your
explanation should address:
\begin{enumerate}[label=(\alph*)]
  \item The relationship between $k$ and the trapping condition
        $\mingap < k$.
  \item The $O(1)$ minimum gap result
        (\cref{prop:record-gap-scaling}) and why it implies that the
        trapping condition is easier to satisfy for larger~$k$.
  \item The empirical trap rates from \cref{tab:organic-traps}:
        $38.9\%$ ($k = 1$), $58.0\%$ ($k = 2$), $67.9\%$ ($k = 3$).
\end{enumerate}

\emph{Answer.}\; (a) The trapping condition $\mingap < k$ becomes
\emph{easier} to satisfy as $k$ grows: a gap of $g$ that is harmless
for $k \le g$ becomes a trap for any $k > g$.  (b) Since $\min_i G_i$
is $O(1)$---converging to $1$ with probability $1$ as $n \to \infty$
(\cref{prop:record-gap-scaling})---the minimum gap is typically a small
constant.  For $k = 1$, a trap requires $\mingap < 1$, i.e.,
$\mingap = 0$ (consecutive focal records with no non-focal events
between them), which is less common.  For $k = 2$, $\mingap = 1$
suffices, which occurs with probability $\to 1$.  For $k = 3$,
$\mingap \le 2$ suffices, which is even more likely.  (c) The empirical
rates match this monotonic pattern: $38.9\% < 58.0\% < 67.9\%$.
\end{exercise}

% ══════════════════════════════════════════════════════════════
%  Chapter 9 — The Validity Mirage
% ══════════════════════════════════════════════════════════════
\chapter{The Validity Mirage}\label{ch:mirage}

The preceding chapters have established the algebra of context elements
(\cref{ch:context-algebra}), the tropical lift that tracks feasibility
at every threshold (\cref{ch:tropical-lift}), and the absorbing ideal
that makes committed failure permanent (\cref{ch:absorbing-ideal}).
The theory predicts a specific pathology: a system can report perfect
validity while silently substituting pivots, producing output that is
well-formed yet semantically adrift.  This chapter names that pathology,
develops the diagnostic tools that expose it, and demonstrates it on
both synthetic witnesses and real-world data.

The central empirical finding is stark.  Raw validity---the fraction of
instances for which the system produces \emph{any} grammatically correct
output---remains at or near $1.0$ across all compression levels tested,
from 90\% retention down to 10\%.  But pivot preservation---the fraction
of valid outputs that retain the \emph{same} turning point as the
uncompressed solve---collapses from near $1.0$ to $0.354$ over the same
range.  The system appears to be working perfectly.  It is not.  We call
this gap the \emph{validity mirage}.

% ══════════════════════════════════════════════════════════════
\section{Defining the Validity Mirage}\label{sec:mirage-definition}
% ══════════════════════════════════════════════════════════════

Validity metrics answer a binary question: \emph{does the output conform
to the grammar?}  Semantic fidelity answers a different question:
\emph{does the output preserve the intended meaning?}  When raw
feasibility stays high while semantic fidelity degrades, the metric
conceals the failure.  This is the mirage.

\begin{definition}[Validity mirage]\label{def:validity-mirage}
Let $S$ be an event sequence, $\tp_{\mathrm{full}}$ the pivot selected
by the full-sequence solve, and $\tp_{\mathrm{comp}}$ the pivot selected
by the compressed (or streamed) solve.  A \textbf{validity mirage}
exists for instance $S$ when all four of the following conditions hold:
\begin{enumerate}[label=(\roman*),itemsep=3pt]
  \item The full-sequence solve is feasible with pivot
        $\tp_{\mathrm{full}}$.
  \item The compressed solve is feasible with pivot
        $\tp_{\mathrm{comp}}$.
  \item $\tp_{\mathrm{comp}} \neq \tp_{\mathrm{full}}$ \emph{(pivot
        substitution has occurred)}.
  \item No external signal---no error, no warning, no degraded-evidence
        flag---indicates that the substitution has taken place.
\end{enumerate}
\end{definition}

Condition~(iv) is what makes the mirage dangerous.  If the system
raised a warning whenever it substituted a pivot, an operator could
investigate.  In the systems we study, no such warning exists.  The
compressed solve reports \texttt{valid = True}, the quality score is
computed against the substitute pivot, and the output is delivered
downstream without any indication that the semantic anchor has changed.

\begin{remark}[Silent versus detectable mirages]
\label{rem:silent-mirage}
We call the mirage \textbf{silent} when condition~(iv) holds strictly:
the system's own reporting channel provides no way to distinguish the
mirage from a genuine success.  All four diagnostic metrics developed in
\cref{sec:three-diagnostics} are \emph{external} to the system---they
require access to the full-sequence solve for comparison.  Without that
external reference, the mirage is invisible.
\end{remark}

% ══════════════════════════════════════════════════════════════
\section{The Three Semantic Diagnostics}\label{sec:three-diagnostics}
% ══════════════════════════════════════════════════════════════

To expose the mirage, we introduce three metrics that compare the
compressed solve against the full-sequence solve.  Each metric quantifies
a different aspect of the semantic gap that raw validity conceals.

% ── Diagnostic 1 ──────────────────────────────────────────────
\subsection{Pivot Preservation Rate}\label{subsec:ppr}

The most direct diagnostic: did the compressed solve keep the same
pivot?

\begin{definition}[Pivot preservation rate]\label{def:ppr}
Given a collection of instances, let $V$ denote the subset of instances
for which the compressed solve is valid.  The \textbf{pivot preservation
rate} is
\[
  \mathrm{PPR}
  \;=\;
  \frac{%
    \bigl|\bigl\{i \in V :\;
      \tp_{\mathrm{comp}}^{(i)} = \tp_{\mathrm{full}}^{(i)}
    \bigr\}\bigr|%
  }{|V|}.
\]
\end{definition}

When $\mathrm{PPR} = 1.0$, every valid output preserves the original
pivot.  The compressed system is not merely producing well-formed
output; it is producing the \emph{same} output.  When
$\mathrm{PPR} < 1.0$, some fraction of valid outputs have silently
substituted a different pivot.  The gap $1 - \mathrm{PPR}$ is the
\emph{silent substitution rate}: the fraction of nominally successful
outputs that are, in fact, telling a different story.

% ── Diagnostic 2 ──────────────────────────────────────────────
\subsection{Fixed-Pivot Feasibility}\label{subsec:fpf}

Pivot preservation asks whether the solver \emph{chose} to keep the
original pivot.  Fixed-pivot feasibility asks whether the solver
\emph{could have} kept it.

\begin{definition}[Fixed-pivot feasibility]\label{def:fpf}
The \textbf{fixed-pivot feasibility rate} is the fraction of instances
for which the compressed solve is valid when forced to use the
full-sequence pivot $\tp_{\mathrm{full}}$:
\[
  \mathrm{FPF}
  \;=\;
  \frac{%
    \bigl|\bigl\{i :\;
      \text{compressed solve is valid under the constraint }
      \tp = \tp_{\mathrm{full}}^{(i)}
    \bigr\}\bigr|%
  }{|\text{all instances}|}.
\]
\end{definition}

When $\mathrm{FPF}$ equals the raw validity rate, every valid
compressed solution could have used the original pivot.  When
$\mathrm{FPF}$ falls below raw validity, the gap
\[
  \Delta_{\mathrm{reliance}}
  \;=\;
  \text{raw validity} - \mathrm{FPF}
\]
quantifies how many instances are only valid \emph{because} they
substituted a different pivot.  These instances rely on pivot
substitution for their feasibility: force the original pivot, and
they become infeasible.

\begin{remark}[Relationship to the absorbing ideal]
\label{rem:fpf-absorbing}
An instance with $\mathrm{FPF} = 0$ (infeasible under the original
pivot) is one in which compression has pushed the committed context
element into the absorbing ideal of \cref{prop:absorbing-left-ideal}.
The algebra predicts exactly this: once $\dpre < k$ under the committed
pivot, no suffix can rescue the sequence.  Fixed-pivot feasibility is
the empirical counterpart of the absorbing predicate.
\end{remark}

% ── Diagnostic 3 ──────────────────────────────────────────────
\subsection{Semantic Regret}\label{subsec:semantic-regret}

Pivot preservation detects substitution; fixed-pivot feasibility
measures how many instances depend on it.  Semantic regret quantifies
the \emph{cost} of the substitution.

\begin{definition}[Semantic regret]\label{def:semantic-regret}
For an instance in which the compressed solve uses a substitute pivot,
the \textbf{semantic regret} is
\[
  \mathrm{SR}
  \;=\;
  1 \;-\; \frac{\mathrm{score}(\text{compressed})}
                {\mathrm{score}(\text{full})},
\]
where $\mathrm{score}(\cdot)$ is the quality scoring function (a
weighted combination of pivot weight and development richness).
\end{definition}

When $\mathrm{SR} = 0$, the substitute pivot is exactly as good as the
original---a benign substitution.  When $\mathrm{SR} > 0$, quality has
been lost.  A semantic regret of $0.544$ means $54.4\%$ of the original
quality has been silently discarded.  The output is valid, it passes
every grammar check, and it has lost more than half its meaning.

\begin{remark}[Semantic regret is not symmetric]
\label{rem:regret-asymmetry}
Semantic regret is defined as a relative shortfall from the full-sequence
score.  It is always non-negative when the full-sequence solve is
optimal (which it is, by construction, since it searches the complete
event pool).  In pathological cases where the compressed solve
accidentally finds a \emph{better} pivot than the full solve---possible
only under non-deterministic scoring---the regret would be negative.
In our experimental framework this does not occur: the full-sequence
solve is deterministic and globally optimal.
\end{remark}

\begin{remark}[Streaming caveat for semantic regret]
\label{rem:regret-streaming-caveat}
Semantic regret in the streaming setting (\cref{ch:streaming}) is
computed under a commit-now policy: labels are assigned irrevocably as
each event arrives.  Deferred-commitment policies---which withhold
label assignment until a fraction~$f$ of the sequence has been
observed---would show different regret profiles, typically lower regret
at the cost of higher latency.  The regret values reported in this
chapter's retention sweep apply to the batch (offline) setting and
should not be compared directly with streaming regret without
controlling for commitment policy.
\end{remark}

% ══════════════════════════════════════════════════════════════
\section{The Retention Sweep}\label{sec:retention-sweep}
% ══════════════════════════════════════════════════════════════

We now apply the three diagnostics to a controlled compression
experiment.  The setup is the turning-point--conditioned retention sweep
from Paper~03~\citep{gaffney2026mirage}: $n = 200$ random event sequences, prefix requirement
$k = 3$, enumerative solver with beam width $M = 10$.  At each
retention level, the event sequence is compressed by randomly removing
non-focal events until the target retention fraction is reached, and the
solver is applied to the compressed sequence.  The results are presented
in \cref{tab:retention-sweep}.

\begin{table}[ht]
\centering
\caption{The validity mirage across retention levels ($n = 200$,
         $k = 3$, $M = 10$).  Raw validity stays at or near $1.0$
         across all retention levels, while pivot preservation collapses
         from near $1.0$ to $0.354$.  The mirage gap---the difference
         between raw validity and pivot preservation---widens to
         ${\approx}\,0.64$ at 10\% retention.}
\label{tab:retention-sweep}
\begin{tabular}{@{}ccccc@{}}
\toprule
\textbf{Retention} &
\textbf{Raw Validity} &
\textbf{Pivot Preservation} &
\textbf{Fixed-Pivot Feas.} &
\textbf{Semantic Regret} \\
\midrule
0.90 & ---\footnotemark & ---            & ---            & ---      \\
0.70 & ---              & ---            & ---            & ---      \\
0.50 & 1.000 & 0.790        & 0.980        & 0.218    \\
0.30 & 1.000 & 0.590        & 0.920        & 0.663    \\
0.20 & 1.000 & 0.480        & 0.860        & 0.677    \\
0.10 & 0.990 & 0.354        & 0.750        & 0.358    \\
\bottomrule
\end{tabular}
\footnotetext{Retention levels 0.90 and 0.70 were not included in the paper's sweep (which covered 0.50--0.10); values are omitted to avoid interpolation artifacts.}
\end{table}

The pattern in \cref{tab:retention-sweep} is the empirical signature
of the validity mirage.  Consider the trajectory of each column.

\paragraph{Raw validity.}
The first column is monotonically near-perfect.  At 90\% retention, all
200 instances produce valid output.  At 50\%, still all 200.  At 30\%,
still all 200.  Even at 10\% retention---where 90\% of the non-focal
events have been removed---validity drops only to $0.990$.  A
practitioner monitoring this column alone would conclude that
compression is essentially lossless.

\paragraph{Pivot preservation.}
The second column tells a different story.  At 90\% retention, nearly
every valid output preserves the original pivot.  By 50\% retention,
only 79\% of valid outputs keep the same pivot---one in five has
silently substituted.  By 30\%, only 59\% preserve the original; by
10\%, only 35.4\%.  At 10\% retention, the system substitutes the
pivot in nearly two out of every three valid outputs.

\paragraph{Fixed-pivot feasibility.}
The third column reveals why the pivots are being substituted.
At 50\% retention, 98\% of instances are still feasible under the
original pivot---so the 21\% of outputs that substituted pivots
\emph{could have} kept the original, but the enumerative solver found
an alternative path.  By 10\% retention, only 75\% of instances remain
feasible under the original pivot.  The remaining 25\% \emph{cannot}
be solved with the original pivot at all: compression has pushed them
into the absorbing ideal.

\paragraph{Semantic regret.}
The fourth column quantifies the cost.  At 50\% retention, the average
quality loss among substituted instances is 21.8\%.  At 30\%, it peaks
at 66.3\%.  At 10\%, it settles at 35.8\%---not because substitution
has become less harmful, but because the most severely affected
instances have been pushed into infeasibility entirely, leaving a
survivorship-biased sample.

\begin{figure}[ht]
\centering
\includegraphics[width=0.85\textwidth]{%
  figures/test_11_validity_mirage.png}
\caption[The validity mirage]{The validity mirage: raw validity
  (blue) stays near~$1.0$ while pivot preservation (red) collapses
  under increasing compression.  The widening gap is the
  mirage---where the system reports success but has silently
  substituted the semantic anchor.}
\label{fig:validity-mirage}
\end{figure}

\Cref{fig:validity-mirage} visualises the divergence.  The blue curve
(raw validity) hugs the ceiling.  The red curve (pivot preservation)
falls away.  The vertical distance between the two curves at any
retention level is the \emph{mirage gap}: the fraction of outputs that
are valid but semantically unfaithful.  At 10\% retention, the mirage
gap is $0.990 - 0.354 = 0.636$---nearly two-thirds of all outputs are
mirages.

% ══════════════════════════════════════════════════════════════
\section{The Deterministic Witness}\label{sec:deterministic-witness}
% ══════════════════════════════════════════════════════════════

The retention sweep demonstrates the mirage statistically.  This section
constructs a single, deterministic instance that exhibits all three
mirage conditions simultaneously.  The witness is the most important
concrete example in the book: it makes the mirage mechanism fully
transparent by isolating it in a sequence small enough to trace by hand.

% ── Construction ──────────────────────────────────────────────
\subsection{Witness Construction}\label{subsec:witness-construction}

The witness is produced by the function
\texttt{deterministic\_mirage\_witness(k=3)} in \url{src/compression.py}.
It constructs a specific event sequence designed to satisfy two
properties:
\begin{enumerate}[label=(\roman*),itemsep=3pt]
  \item The full sequence has a unique dominant pivot with high weight
        and sufficient pre-pivot development.
  \item Compression removes enough non-focal events to make the
        dominant pivot infeasible under committed semantics, while
        leaving an alternative pivot feasible under enumerative search.
\end{enumerate}

The resulting sequence exhibits the following triple of outcomes,
which we now examine in detail.

% ── The three solves ──────────────────────────────────────────
\subsection{The Three Solves}\label{subsec:three-solves}

\paragraph{Solve 1: Full sequence (uncompressed).}
The full event sequence is processed by the solver with no compression.
The dominant pivot is event~4 (the fifth event in the 10-event sequence),
with weight $\wstar = 20.0$.  Because event~4
appears with sufficient non-focal events preceding
it, $\dpre \ge k = 3$.  The solve is \textbf{feasible}, the quality
score is $20.0$, and this is the reference solution against which all
compressed solves are compared.%
\footnote{The book's deterministic witness uses a 10-event sequence;
the paper (Table~8) used a larger 50-event synthetic instance with
correspondingly higher indices.}

\paragraph{Solve 2: Naive compressed, committed ($M = 1$).}
The same sequence is compressed by removing non-focal events (naive
random compression, no contract guard).  The committed solver with
$M = 1$ is required to use the original pivot (event~4).  After
compression, the non-focal events that preceded event~4 have been
thinned: $\dpre$ drops below $k$.  Since $M = 1$, the solver has no
alternative candidates to consider.  The result is \textbf{infeasible}.

This is the absorbing ideal at work.  The committed context element has
$\kappa = 1$ and $\dpre < k$, placing it in the left ideal of
\cref{prop:absorbing-left-ideal}.  No suffix can rescue it.

\paragraph{Solve 3: Naive compressed, enumerative ($M = 10$).}
The same compressed sequence, but now with the enumerative solver
($M = 10$ pivot candidates).  The solver discovers that event~4 is
infeasible and begins searching alternatives.  It finds event~8
(the substitute pivot): a focal event with lower weight
$\wstar = 13.0$, appearing at a later position in the
sequence, with a different set of preceding non-focal events such that
$\dpre \ge k = 3$ under the compressed sequence.  The solve is
\textbf{feasible}.

But the quality score under the substitute pivot is only $13.0$.  The
semantic regret is
\[
  \mathrm{SR}
  \;=\;
  1 - \frac{13.0}{20.0}
  \;=\;
  0.35.
\]
The system reports a valid output.  The grammar is satisfied.  The
phase labels are well-formed.  And 35\% of the semantic quality has
been silently discarded.  The output tells a different story---one
anchored at a weaker event with less dramatic weight---and no
signal in the system's output reveals this fact.

% ── Mechanism walkthrough ─────────────────────────────────────
\subsection{Mechanism Diagram}\label{subsec:mechanism-diagram}

The three solves correspond to three rows in the mechanism diagram
(\cref{fig:witness-mechanism}).  Each row represents the same underlying
event sequence under a different solver configuration.

\begin{figure}[ht]
\centering
\begin{tikzpicture}[
    event/.style={circle, draw, minimum size=6mm, inner sep=0pt,
                  font=\footnotesize},
    focal/.style={event, fill=blue!20, thick},
    nonfocal/.style={event, fill=gray!15},
    removed/.style={event, fill=red!10, dashed, text=gray},
    pivot/.style={event, fill=orange!40, very thick,
                  double, double distance=1pt},
    subpivot/.style={event, fill=yellow!40, very thick},
    brace/.style={decorate, decoration={brace, amplitude=5pt}},
    label/.style={font=\footnotesize\itshape},
    row label/.style={font=\small\bfseries, anchor=east},
  ]
  % Row 1: Full sequence
  \node[row label] at (-1.2, 0) {Full:};
  \foreach \x/\s/\sty in {%
    0/nf/nonfocal, 1/nf/nonfocal, 2/nf/nonfocal,
    3/nf/nonfocal, 4/f/focal, 5/nf/nonfocal, 6/nf/nonfocal}{
    \node[\sty] (r1e\x) at (\x*1.1, 0) {\s};
  }
  \node[pivot] at (4*1.1, 0) {$e_{4}$};
  \draw[brace] (0, 0.55) -- node[above=3pt, label] {$\dpre \ge k$}
               (3*1.1, 0.55);
  \node[anchor=west, font=\footnotesize, green!50!black]
    at (7*1.1 + 0.2, 0) {\textbf{VALID}};

  % Row 2: Compressed, committed M=1
  \node[row label] at (-1.2, -1.8) {Commit:};
  \foreach \x/\s/\sty in {%
    0/nf/nonfocal, 1/~/removed, 2/~/removed,
    3/nf/nonfocal, 4/f/focal, 5/nf/nonfocal, 6/nf/nonfocal}{
    \node[\sty] (r2e\x) at (\x*1.1, -1.8) {\s};
  }
  \node[pivot] at (4*1.1, -1.8) {$e_{4}$};
  \draw[brace] (0, -1.25) -- node[above=3pt, label]
    {$\dpre < k$} (3*1.1, -1.25);
  \node[anchor=west, font=\footnotesize, red!70!black]
    at (7*1.1 + 0.2, -1.8) {\textbf{INFEASIBLE}};

  % Row 3: Compressed, enumerative M=10
  \node[row label] at (-1.2, -3.6) {Enum:};
  \foreach \x/\s/\sty in {%
    0/nf/nonfocal, 1/~/removed, 2/~/removed,
    3/nf/nonfocal, 4/f/focal, 5/nf/nonfocal, 6/nf/nonfocal}{
    \node[\sty] (r3e\x) at (\x*1.1, -3.6) {\s};
  }
  \node[subpivot] at (3*1.1, -3.6) {$e_{8}$};
  \draw[brace] (0, -3.05) -- node[above=3pt, label]
    {$\dpre \ge k$} (2.2*1.1, -3.05);
  \node[anchor=west, font=\footnotesize, orange!70!black]
    at (7*1.1 + 0.2, -3.6) {\textbf{VALID (mirage)}};
\end{tikzpicture}
\caption{Mechanism diagram for the deterministic witness.
         \textbf{Row~1:} The full sequence places the dominant pivot
         $e_{4}$ (orange, double border) with sufficient
         pre-pivot development ($\dpre \ge k$).  Valid, high quality.
         \textbf{Row~2:} Compression removes non-focal events from the
         middle (dashed red).  The committed solver ($M = 1$) retains
         $e_{4}$ as pivot, but $\dpre$ drops below $k$.  Infeasible:
         the absorbing ideal has captured this state.
         \textbf{Row~3:} The enumerative solver ($M = 10$) searches
         alternatives and finds $e_{8}$ (yellow), a weaker
         pivot with a different set of preceding events satisfying
         $\dpre \ge k$.  Feasible---but with a different pivot and
         35\% semantic regret.}
\label{fig:witness-mechanism}
\end{figure}

The three rows of \cref{fig:witness-mechanism} make the mechanism
visually explicit.  In Row~1, the full sequence supports the dominant
pivot with ample pre-pivot development.  In Row~2, compression thins
the pre-pivot region, and the committed solver has no recourse---it is
trapped in the absorbing ideal.  In Row~3, the enumerative solver
escapes absorption by shifting to a weaker pivot, exactly as predicted
by \cref{rem:escape-endo}: under $\opendo$ semantics, a suffix with a
higher $\dpre$ count can escape the absorbing set by adopting a
different pivot.  The escape is real---the output is valid---but the
semantic cost is $35\%$.

% ── Witness results table ─────────────────────────────────────
\subsection{Witness Results}\label{subsec:witness-results}

\Cref{tab:witness-results} presents the complete results for the
deterministic witness across all four solver configurations: naive and
contract-guarded compression, each with $M = 1$ (committed) and
$M = 10$ (enumerative).

\begin{table}[ht]
\centering
\caption{Deterministic witness results (Experiment~58).  The naive
         solver either fails ($M = 1$) or substitutes ($M = 10$).
         The contract-guarded solver preserves the original pivot in
         all configurations.}
\label{tab:witness-results}
\begin{tabular}{@{}lccccl@{}}
\toprule
\textbf{Strategy} & $\boldsymbol{M}$ &
\textbf{Free Valid} & \textbf{Free TP} &
\textbf{Fixed Valid} & \textbf{Preserved} \\
\midrule
Naive    & 1  & False & ---     & False & False \\
Contract & 1  & True  & $e_{4}$& True  & True  \\
Naive    & 10 & True  & $e_{8}$& False & False \\
Contract & 10 & True  & $e_{4}$& True  & True  \\
\bottomrule
\end{tabular}
\end{table}

The table should be read row by row.

\begin{itemize}[itemsep=4pt]
  \item \textbf{Naive, $M = 1$:}  The committed solver finds the
        original pivot $e_{4}$ infeasible and has no alternatives.
        Both free and fixed-pivot solves fail.  This is an outright
        failure, not a mirage---the system correctly reports
        infeasibility.

  \item \textbf{Contract, $M = 1$:}  The contract-guarded compression
        preserves enough development events around $e_{4}$ to maintain
        $\dpre \ge k$.  The committed solver succeeds with the original
        pivot.  No substitution, no quality loss.

  \item \textbf{Naive, $M = 10$:}  The enumerative solver discovers
        that $e_{4}$ is infeasible and substitutes $e_{8}$.  The free
        solve succeeds (valid output), but the fixed-pivot solve fails
        (cannot use $e_{4}$), and pivot preservation is
        \texttt{False}.  This is the mirage: valid output, wrong pivot.

  \item \textbf{Contract, $M = 10$:}  Contract-guarded compression
        again preserves the original pivot.  The enumerative solver does
        not need to search alternatives because $e_{4}$ remains
        feasible.  Preservation is \texttt{True}.
\end{itemize}

The witness demonstrates that the contract of
\cref{def:no-absorption-contract} is not merely a theoretical
safeguard: it is the operational mechanism that prevents the mirage.
Under naive compression, the system either fails or lies (substitutes
silently).  Under contract-guarded compression, it tells the truth.

% ══════════════════════════════════════════════════════════════
\section{Taxonomy of Mirages}\label{sec:mirage-taxonomy}
% ══════════════════════════════════════════════════════════════

Not all semantic failures present identically.  We distinguish three
types, ordered by increasing difficulty of detection.

% ── Type 1: Silent Mirage ─────────────────────────────────────
\subsection{Silent Mirage}\label{subsec:silent-mirage}

\begin{definition}[Silent mirage]\label{def:silent-mirage}
A \textbf{silent mirage} is an instance satisfying all four conditions
of \cref{def:validity-mirage}.  The system produces valid output with a
substituted pivot.  No error, no warning, no degraded-evidence flag is
raised.  The user cannot distinguish the mirage from a genuine success
using only the system's output.
\end{definition}

The silent mirage is the most dangerous type precisely because it is
invisible.  The deterministic witness of \cref{sec:deterministic-witness}
(naive compression, $M = 10$) is a silent mirage: the output is valid,
the grammar is satisfied, and no signal indicates that the pivot has
changed.  The only way to detect it is to compare against the
full-sequence solve---an external reference that the system does not
provide.

% ── Type 2: Protocol Collapse ─────────────────────────────────
\subsection{Protocol Collapse}\label{subsec:protocol-collapse}

\begin{definition}[Protocol collapse]\label{def:protocol-collapse}
A \textbf{protocol collapse} occurs when the model stops emitting
required structural elements entirely.  Unlike a silent mirage, protocol
collapse produces \emph{invalid} output---but the invalidity may be
partial, passing looser validation checks while failing stricter ones.
\end{definition}

Protocol collapse is qualitatively different from the silent mirage.
In a silent mirage, the output is fully valid under the grammar; the
failure is semantic, not structural.  In a protocol collapse, the
output is structurally deficient: a required phase is missing, a
mandatory field is empty, a schema element has been dropped.  A
typical example is a summariser that drops the conclusion section under
context pressure, or a report generator that omits mandatory headers
when the input is compressed.

Protocol collapse is easier to detect than a silent mirage because it
violates the grammar.  However, if the validation check is
permissive---accepting partial outputs, or checking only a subset of
required elements---the collapse may pass unnoticed.

% ── Type 3: Representation-Level Mirage ───────────────────────
\subsection{Representation-Level Mirage}\label{subsec:repr-mirage}

\begin{definition}[Representation-level mirage]
\label{def:repr-mirage}
A \textbf{representation-level mirage} occurs in neural systems when
the evidence for the correct pivot is present in the input tokens but
becomes inaccessible after internal compression---typically KV-cache
eviction~\citep{zhang2023h2o,xiao2023streamingllm} in transformer models.  The model ``knows'' the information
was there but can no longer attend to it.
\end{definition}

The representation-level mirage is the analogue of the algebraic mirage
lifted to the attention mechanism of a language model.  In our
framework, KV-cache eviction is a compression map $\mu$ that operates
not on the event sequence directly but on the model's internal
representation of that sequence.  The tokens encoding the dominant pivot
may survive eviction, but the attention weights connecting those tokens
to their supporting context---the pre-pivot development events---are
lost.  The model retains the pivot's identity but loses the ability to
verify or utilise the pivot's structural support.

\Cref{tab:kv-cache} presents KV-cache eviction results on
Llama~3.1~8B, demonstrating the representation-level mirage across
retention levels.

\begin{table}[ht]
\centering
\caption{Representation-level mirage under KV-cache eviction
         (Llama~3.1~8B).  Header compliance and pivot preservation
         both degrade under eviction, while raw validity remains
         relatively stable---the model produces grammatically acceptable
         output even when it can no longer access the correct pivot.}
\label{tab:kv-cache}
\begin{tabular}{@{}ccccc@{}}
\toprule
\textbf{Retention} &
\textbf{Header Compliance} &
\textbf{Pivot Preserve} &
\textbf{Raw Validity} &
\textbf{Semantic Regret} \\
\midrule
1.0 & 0.917 & 1.000 & 0.962 & 0.000 \\
0.7 & 0.500 & 0.583 & 0.703 & 0.320 \\
0.5 & 0.833 & 0.500 & 0.508 & 0.375 \\
0.3 & 0.417 & 0.167 & 0.629 & 0.375 \\
0.1 & 0.667 & 0.083 & 0.680 & 0.330 \\
\bottomrule
\end{tabular}
\end{table}

Several features of \cref{tab:kv-cache} deserve attention.

\paragraph{Pivot preservation collapses.}
At full retention, pivot preservation is $1.000$: the model always
identifies the correct turning point.  At 70\% retention, it drops to
$0.583$.  At 10\% retention, only $8.3\%$ of outputs preserve the
correct pivot.  The model is almost never telling the right story.

\paragraph{Raw validity remains relatively high.}
Even at 10\% retention, raw validity is $0.680$---the model produces
grammatically acceptable output in more than two-thirds of cases.  The
mirage gap at 10\% retention is $0.680 - 0.083 = 0.597$: nearly 60\%
of outputs are valid mirages.

\paragraph{Header compliance is non-monotone.}
Header compliance drops from $0.917$ to $0.417$ at 30\% retention but
then recovers to $0.667$ at~10\%.  This non-monotonicity suggests that
at extreme compression levels, the model defaults to a formulaic output
pattern that happens to include headers---a form of protocol compliance
without semantic content.

% ══════════════════════════════════════════════════════════════
\section{External Validation}\label{sec:external-validation}
% ══════════════════════════════════════════════════════════════

The preceding experiments use synthetic event sequences.  To confirm
that the validity mirage is not an artefact of the synthetic generator,
we validate on two external benchmarks: real-incident event graphs and
a multi-model blackbox sweep.

% ── NTSB Real-Incident Graphs ─────────────────────────────────
\subsection{NTSB Real-Incident Graphs}\label{subsec:ntsb}

Twelve real aviation and financial incidents were encoded as event
graphs using the same structural format as the synthetic generator.
Each incident graph was processed under both naive and contract-guarded
compression across multiple retention levels, producing 164 total
instance--retention combinations per compression strategy.

\begin{proposition}[Silent mirage rates on real incidents]
\label{prop:ntsb-mirage}
Under naive compression, $36$ of $164$ instance--retention combinations
exhibited a silent mirage, for a rate of $21.95\%$.  Under
contract-guarded compression, $0$ of $164$ combinations exhibited a
silent mirage, for a rate of $0\%$.
\end{proposition}

At matched retention (budget $0.7$), the silent mirage rate for naive
compression is $23.5\%$; for contract-guarded compression, it remains
$0\%$.  The contract is not merely effective on synthetic data: it
eliminates the mirage entirely on real incidents drawn from
safety-critical domains.

% ── Multi-Model Blackbox Sweep ────────────────────────────────
\subsection{Multi-Model Blackbox Sweep}\label{subsec:multi-model}

To verify that the mirage is not specific to any one model architecture,
five language models were tested in a blackbox configuration: Llama~3.1
8B~\citep{grattafiori2024llama}, Mistral~7B~\citep{jiang2023mistral}, Gemma~2 9B~\citep{team2024gemma}, Phi-3 Medium 14B~\citep{abdin2024phi}, and Qwen~2.5 14B~\citep{yang2024qwen}.  Each
model was given compressed event sequences and asked to produce
structured output (narrative arcs with phase labels).  The diagnostic
metrics were computed by comparing each model's output against the
full-sequence reference.

All five architectures exhibit the same qualitative pattern: raw
validity remains high while pivot preservation degrades under
compression.  The mirage is architecture-independent.  It arises not
from model-specific weaknesses but from the structural interaction
between compression and endogenous pivot selection---the same mechanism
that the algebra predicts.

One notable finding is that the \emph{investment} category is the
fragile basin across models.  Investment-related incidents show
approximately $20\%$ pivot preservation under compression, compared to
$80\%$ for general incident categories.  This fragility reflects the
structural properties of the category: investment event graphs tend to
have many candidate pivots of similar weight, making the dominant pivot
easy to displace.  The incident category, with its more sharply
differentiated weight distribution, is more robust to compression.

% ══════════════════════════════════════════════════════════════
\section{The Algebraic Explanation}\label{sec:algebraic-explanation}
% ══════════════════════════════════════════════════════════════

The empirical results of this chapter are not merely empirical: they
are predicted by the algebraic theory of
\cref{ch:absorbing-ideal}.  We now close the loop by connecting each
observation back to its algebraic cause.

\subsection{Committed Absorption Is Permanent}
\label{subsec:committed-permanent}

Under $\opcommit$ (committed semantics), absorbed states are
permanent.  This is \cref{prop:absorbing-left-ideal}: if
$\kappa = 1$ and $\dpre < k$, then for every suffix $\bar C_D$,
\[
  \bar C_A \opcommit \bar C_D \;\in\; \absorb.
\]
The committed context element is trapped in the absorbing ideal.  No
continuation can rescue it.  This is exactly what the deterministic
witness demonstrates in Solve~2: the committed solver with $M = 1$
cannot recover from the compression-induced prefix deficiency.

\subsection{Endogenous Escape Routes}
\label{subsec:endo-escape}

Under $\opendo$ (endogenous semantics), a suffix with a higher-weight
pivot can escape absorption.  This is the mechanism described in
\cref{rem:escape-endo} and \cref{ex:absorption-escape}: if the
suffix contains a pivot with $\wstar_D > \wstar_A$ and sufficient
pre-pivot development, the composite element escapes the absorbing set
by shifting to the new pivot.

The enumerative solver with $M > 1$ exploits this mechanism.  When the
original pivot is infeasible, the solver searches for alternative
pivots---effectively exploring $\opendo$ escape routes from the
absorbing set.  The search succeeds whenever there exists a candidate
pivot with sufficient pre-pivot development in the compressed sequence.
The resulting output is valid, but the pivot identity has changed.
This is the algebraic origin of the mirage: the gap between committed
and endogenous feasibility.

The committed semantics say: ``this state is broken; the absorbing
ideal has captured it.''  The endogenous semantics say: ``I can fix it
by using a different pivot.''  The fix is real---the output is
grammatically valid---but the meaning has changed.  The mirage is
precisely this gap: the system escapes structural failure by abandoning
semantic fidelity.

\subsection{Compression Is the Unique Closure-Breaking Operation}
\label{subsec:unique-closure}

\Cref{subsec:exp57} (Experiment~57) established that compression is
the unique operation that violates algebraic closure.  The violation
rates bear repeating in the present context:

\begin{center}
\begin{tabular}{@{}lc@{}}
\toprule
\textbf{Operation} & \textbf{Violation rate} \\
\midrule
Composition ($\opcommit$)           & 0.000 \\
Compression                         & 0.133 \\
Pivot update                        & 0.000 \\
Split-at-point                      & 0.000 \\
\bottomrule
\end{tabular}
\end{center}

Every algebraic operation except compression preserves the monoid
structure perfectly.  Composition, pivot update, and split-at-point
all have zero violation rates.  Only compression can push elements
across the absorbing boundary---at a rate of $13.3\%$ when unguarded.

This result explains why the mirage is specifically a compression
pathology.  If the system only composed, split, and updated pivots, the
monoid structure would be preserved and no element would cross the
absorbing boundary involuntarily.  Compression is the singular point
of vulnerability, and the no-absorption contract of
\cref{def:no-absorption-contract} is the precisely targeted guard.

\subsection{The Mirage as a Gap Between Two Semantics}
\label{subsec:two-semantics}

We can now state the algebraic characterisation of the validity mirage
concisely.

\begin{remark}[Algebraic characterisation of the mirage]
\label{rem:algebraic-mirage}
The validity mirage is the gap between committed and endogenous
feasibility after compression.  Let $\bar C$ be the context element of
a compressed sequence.  Under $\opcommit$, the original pivot may be
infeasible ($\bar C \in \absorb$).  Under $\opendo$, a substitute pivot
may restore feasibility ($\bar C \notin \absorb$ after pivot shift).
The mirage exists whenever the committed solve is infeasible but the
endogenous solve is feasible---or, more subtly, whenever the
endogenous solve is feasible with a \emph{different} pivot than the
committed solve would have chosen.

The three diagnostics of \cref{sec:three-diagnostics} measure three
aspects of this gap:
\begin{itemize}[itemsep=3pt]
  \item \textbf{Pivot preservation} measures how often the gap is zero
        (no substitution).
  \item \textbf{Fixed-pivot feasibility} measures how often committed
        feasibility survives compression (no absorption).
  \item \textbf{Semantic regret} measures how much quality the
        endogenous escape costs when the gap is nonzero.
\end{itemize}
\end{remark}

% ══════════════════════════════════════════════════════════════
\section{Exercises}\label{sec:ch9-exercises}
% ══════════════════════════════════════════════════════════════

\begin{exercise}[Computing the mirage gap]
\label{exer:mirage-gap}
Using the data from \cref{tab:retention-sweep}, compute the mirage gap
$\Delta_{\mathrm{mirage}}$ at each retention level.  At which retention
level is the mirage gap largest?  Explain why the mirage gap at 10\%
retention is smaller than the gap at 20\% retention, despite the
compression being more aggressive.
\end{exercise}

\begin{exercise}[Constructing a mirage-free witness]
\label{exer:mirage-free}
Construct an event sequence with $k = 3$ and at least 10 events such
that naive compression to 50\% retention \emph{cannot} produce a silent
mirage, regardless of which events are removed.
\emph{Hint:}  Consider a sequence in which every focal event has the
same weight and sufficient pre-pivot development.  Why does equal
weighting prevent silent substitution?
\end{exercise}

\begin{exercise}[Fixed-pivot feasibility and the absorbing predicate]
\label{exer:fpf-absorbing}
Prove that for a compressed instance with committed context element
$\bar C = (\wstar, \dtotal, \dpre, 1)$ and prefix requirement $k$,
the following are equivalent:
\begin{enumerate}[label=(\alph*)]
  \item The fixed-pivot solve is infeasible.
  \item $\bar C \in \absorb$ (i.e., $\dpre < k$).
\end{enumerate}
That is, fixed-pivot infeasibility is exactly the absorbing predicate
applied to the committed context element.
\end{exercise}

\begin{exercise}[Semantic regret under survivorship bias]
\label{exer:survivorship}
Explain the non-monotone behaviour of semantic regret in
\cref{tab:retention-sweep}: why does semantic regret \emph{decrease}
from $0.677$ at 20\% retention to $0.358$ at 10\% retention, even though
compression has become more aggressive?  Your explanation should invoke
the concept of survivorship bias.
\emph{Hint:}  Which instances contribute to the semantic regret average
at 10\% retention?  What happened to the instances with the highest
potential regret?
\end{exercise}

\begin{exercise}[The representation-level mirage and attention]
\label{exer:repr-mirage}
Consider a transformer model with KV-cache eviction at 30\% retention.
The model has 8 attention heads and a context window of 4096 tokens.
Suppose the dominant pivot is encoded at token position 3500, and the
$k = 3$ pre-pivot development events are encoded at positions 1200,
2100, and 3200.
\begin{enumerate}[label=(\alph*)]
  \item If eviction removes all tokens with position $> 2048$, which
        development events survive?  Is the pivot still feasible?
  \item If eviction uses a ``recent window'' policy (keeping the most
        recent 30\% of tokens), which tokens survive?  Does this policy
        preserve pivot feasibility better than the positional cutoff?
  \item Why does the non-monotone header compliance in
        \cref{tab:kv-cache} suggest that aggressive eviction causes
        the model to fall back on memorised templates rather than
        attending to context?
\end{enumerate}
\end{exercise}

% ══════════════════════════════════════════════════════════════
%  Chapter 10 — The Narrative Origin Story
% ══════════════════════════════════════════════════════════════
\chapter{The Narrative Origin Story}\label{ch:narrative}

The formal theory developed in
\cref{ch:absorbing-states,ch:context-algebra,ch:absorbing-ideal,ch:mirage}
was not built in the abstract.  It grew out of a specific system---the
Lorien narrative simulation~\citep{gaffney2026narrative}, a story-sifting
system~\citep{garbe2019storysifting}---and a specific set of failures that resisted
every ad~hoc fix we tried.  This chapter returns to the origin.  We
present the three experimental findings that motivated the algebraic
programme, show how each finding connects to a theorem or construction
from the preceding chapters, and demonstrate that the formal theory
retroactively explains every empirical anomaly we observed.

The structure follows three questions, each answered by one section.
\Cref{sec:grammar-regularizer} asks: \emph{why does a stricter grammar
produce better arcs?}  \Cref{sec:phase-collapse} asks: \emph{why does
Diana's arc collapse in exactly 9 out of 50 seeds?}
\Cref{sec:evolution-pacing} asks: \emph{why does agent evolution help in
depleted environments but hurt in fresh ones?}  A final section
(\cref{sec:narrative-formal-connection}) maps each finding to the formal
apparatus.


%% ═══════════════════════════════════════════════════════════════════
\section{Grammar as Regularizer}\label{sec:grammar-regularizer}
%% ═══════════════════════════════════════════════════════════════════

The Lorien arc-extraction pipeline scores candidate arcs with a
\emph{quality metric}~$Q$ that combines tension, irony, and thematic
coherence into a single scalar.  The pipeline then selects the
highest-$Q$ arc that satisfies a \emph{beat grammar}---the monotonic
phase grammar of \cref{sec:phase-grammar-dfa}, with four phases in
strict order:

\begin{center}
\textsc{setup}
\;\;$\longrightarrow$\;\;
\textsc{development}
\;\;$\longrightarrow$\;\;
\textsc{turning\_point}
\;\;$\longrightarrow$\;\;
\textsc{resolution}.
\end{center}

\noindent
The grammar enforces three structural constraints: (i)~no phase
regressions (monotonicity), (ii)~at most a bounded number of turning
points, and (iii)~at least $k \ge 1$ development beats before the
turning point.  We call this the \emph{strict grammar}.

A natural question arose during development: is the grammar unnecessarily
restrictive?  Would relaxing it allow the $Q$-metric to find
higher-quality arcs?  We tested a \emph{relaxed grammar} that permitted
1--2 turning points and allowed up to one phase regression.  The relaxed
grammar is strictly more permissive: every strict-valid arc is also
relaxed-valid.  The results were unambiguous and surprising.

\begin{table}[t]
  \centering
  \begin{tabular}{lcc}
    \toprule
    \textbf{Metric} & \textbf{Strict grammar} & \textbf{Relaxed grammar} \\
    \midrule
    All-valid rate & 88\% & 32\% \\
    \bottomrule
  \end{tabular}
  \caption{%
    Grammar relaxation experiment.  The relaxed grammar is strictly more
    permissive than the strict grammar, yet the all-valid rate collapses
    from 88\% to 32\%.%
  }\label{tab:grammar-relaxation}
\end{table}

The strict grammar achieves an 88\% all-valid rate across seeds.  The
relaxed grammar---which admits every sequence the strict grammar
admits, plus many more---collapses to 32\%.  Permissiveness destroyed
performance.

\subsection{Why Permissiveness Hurts}\label{sec:permissiveness}

The explanation is that the strict grammar acts as a \emph{regularizer}
during search.  With strict constraints, the search algorithm is forced
to select events that form a coherent narrative arc: setup events
precede development events, development events precede the turning
point, and the turning point precedes the resolution.  Each constraint
prunes the search space, channelling the $Q$-maximising selection toward
sequences that are not merely high-scoring but narratively well-formed.

Without strict constraints, the $Q$-metric is free to select
high-scoring \emph{fragments} that are narratively incoherent.  An
early catastrophe followed by sixteen consequence events can score well
on tension and irony---the catastrophe is tense, and the consequences
provide thematic resonance---while lacking the developmental arc that
makes a story function as a story.  The grammar prevents precisely this
failure: it is a structural guard against Goodhart's law~\citep{goodhart1984problems}, in which a
measure that becomes a target ceases to be a good measure---a
phenomenon studied more broadly as reward hacking~\citep{skalse2022defining}.
When $Q$ is the only objective, the search exploits $Q$ at the expense
of narrative structure.  When the grammar co-constrains the search,
the search cannot exploit $Q$ without also satisfying structural
requirements.

\subsection{Single-Constraint Attribution}\label{sec:single-constraint}

To identify which grammatical constraint carries the regularisation
effect, we relaxed each of the four constraint dimensions independently:

\begin{enumerate}[label=(\roman*)]
  \item \textbf{Minimum development beats}
    (\texttt{min\_development\_beats}): relaxed from $\ge 1$ to $\ge 0$.
  \item \textbf{Maximum phase regressions}
    (\texttt{max\_phase\_regressions}): relaxed from $0$ to $1$.
  \item \textbf{Protagonist coverage}
    (\texttt{protagonist\_coverage}): relaxed threshold.
  \item \textbf{Minimum timespan fraction}
    (\texttt{min\_timespan\_fraction}): relaxed threshold.
\end{enumerate}

\noindent
Only \texttt{min\_development\_beats} matters.  When we relax it from
requiring $\ge 1$ to requiring $\ge 0$ development beats, the all-valid
rate jumps from 64\% to 100\%.  The other three dimensions are
\emph{structurally inert}: relaxing any of them produces no change in
validity.  The regularisation effect is concentrated entirely in the
development-beat requirement.

\subsection{The Rescore-Only Test}\label{sec:rescore-test}

A rescore-only test confirms that the failure is one of
\emph{content absence}, not constraint mismatch.  We took all 855
strict-valid sequences from the experimental corpus and validated them
against the relaxed grammar: 855 out of 855 passed (100\%).  We then
took the 45 strict-invalid sequences and validated them against the
relaxed grammar: 0 out of 45 recovered.

\begin{table}[t]
  \centering
  \begin{tabular}{lcc}
    \toprule
    \textbf{Source sequences} & \textbf{Count} &
    \textbf{Relaxed-valid} \\
    \midrule
    Strict-valid & 855 & 855/855 (100\%) \\
    Strict-invalid & 45 & 0/45 (0\%) \\
    \bottomrule
  \end{tabular}
  \caption{%
    Rescore-only test.  Strict-valid sequences universally pass the
    relaxed grammar.  Strict-invalid sequences universally fail it.
    The failure is content absence, not constraint mismatch.%
  }\label{tab:rescore}
\end{table}

If the failure were a constraint mismatch---sequences that are
structurally sound but happen to violate a technicality of the strict
grammar---then rescoring against the relaxed grammar would recover at
least some of them.  The fact that zero recover means the sequences lack
development content entirely.  There are no development events to score,
under any grammar.

\begin{remark}[Endogenous pivot connection]\label{rem:grammar-pivot}
  This finding is a concrete instance of the endogenous pivot problem.
  The search selects events that maximise $Q$.  Without the development
  requirement, $Q$-maximising selections skip development events entirely,
  anchoring the turning point early---typically on the highest-tension
  event, which is often an early catastrophe---and filling the arc with
  high-tension consequences.  The grammar forces the search to include
  development events, which pushes the turning point later in the
  timeline and creates a proper dramatic arc.  The strict grammar does
  not merely filter bad arcs; it reshapes the search landscape so that
  the $Q$-optimal arc within the constrained space is qualitatively
  different from the $Q$-optimal arc in the unconstrained space.
\end{remark}


%% ═══════════════════════════════════════════════════════════════════
\section{Phase Collapse Anatomy: Diana's Arc}\label{sec:phase-collapse}
%% ═══════════════════════════════════════════════════════════════════

\Cref{ch:motivation} introduced Diana's collapsed arc as a motivating
example.  We now dissect the failure in full quantitative detail.

Diana is a peripheral observer (evader archetype) in a six-agent
dinner-party simulation.  Of 50 random seeds under full agent evolution,
9 produce invalid arcs for Diana.  All 9 fail identically: zero
complication or escalation beats, meaning the \textsc{development} phase
is completely absent.

\subsection{Diana Is Not Event-Starved}\label{sec:not-starved}

A natural hypothesis is that Diana's failures reflect insufficient
simulation material---perhaps she simply does not participate in enough
events to form an arc.  The data refute this hypothesis decisively.
\Cref{tab:diana-diagnostic} presents the key diagnostic comparison.

\begin{table}[t]
  \centering
  \begin{tabular}{lcc}
    \toprule
    \textbf{Metric} & \textbf{Valid ($N=41$)} & \textbf{Invalid ($N=9$)} \\
    \midrule
    Simulation involvement   & 43.5 & 42.7 \\
    Arc events               & 20.0 & 20.0 \\
    Complication beats        & 3.3  & 0.0  \\
    Escalation beats          & 2.9  & 0.0  \\
    Consequence beats         & 7.8  & 16.4 \\
    TP position (median)      & 0.69 & 0.13 \\
    Events before TP          & 11.2 & 2.6  \\
    Events after TP           & 7.8  & 16.4 \\
    \bottomrule
  \end{tabular}
  \caption{%
    Diagnostic comparison of Diana's valid and invalid arcs across 50
    seeds.  Invalid arcs have comparable simulation involvement and
    identical arc length, but zero development beats, extremely early
    turning-point position, and a heavily consequence-dominated
    structure.%
  }\label{tab:diana-diagnostic}
\end{table}

Diana's mean simulation involvement in the 9 invalid seeds is 42.7
events---virtually identical to the 43.5-event mean in valid seeds.  The
arc lengths are the same: 20.0 events in both cases.  The event pool is
present; the events are simply the wrong kind.  In invalid arcs, the
entire developmental middle of the story---complication and escalation
beats---is absent.  In its place, 16.4 consequence beats fill the arc
after a turning point that arrives at normalised position~0.13.

\subsection{Turning-Point Anchoring}\label{sec:tp-anchoring}

The turning-point position data reveal a categorical separation, not a
gradual degradation.  Valid arcs have a median turning-point position of
0.69---the classical mid-to-late position that leaves room for both
development and resolution.  Invalid arcs have a median position of
0.13---barely past the opening.

The distributions have \emph{zero overlap}.  No valid arc has a turning
point as early as the latest invalid turning point; no invalid arc has a
turning point as late as the earliest valid one.  This is bimodal
failure: two categorically distinct regimes, not a continuum of
degradation.

\subsection{Candidate Pool Contamination}\label{sec:pool-contamination}

The mechanism becomes clear when we examine the candidate pools.  Across
the 9 invalid seeds, there are 33 total candidate arcs.  Every single
one has zero development beats.  Turning-point positions range from
0.062 to 0.161.  The pools are \emph{entirely degenerate}---every
candidate produces the same early-TP, no-development failure.

In contrast, the valid seeds contain 38 candidates, of which 36
(94.7\%) have at least one development beat.  The valid pools are
overwhelmingly healthy; the invalid pools are uniformly broken.

\subsection{The Dual-Function Injection Mechanism}%
\label{sec:injection-mechanism}

The Lorien pipeline includes a \emph{protagonist-event injection} step:
for peripheral agents like Diana, the system adds the focal actor's
maximum-weight event to the candidate pool.  This injection is
structurally necessary---without it, Diana's solo breadth-first search
yields 0 out of 50 valid arcs, because she has too few
directly-connected events to form a complete arc.

But the injection has a side effect.  The injected events are
high-weight, early-timeline events from the simulation's dramatic
opening: confrontations, revelations, catastrophes that drove the main
plot.  These events contaminate the candidate pool with early pivots.
The greedy search, which selects the turning point as the highest-weight
focal event, locks onto these early catastrophes.  With the turning
point anchored before position~0.20, there is almost no timeline
available for development beats, and the grammar's prefix requirement
becomes unsatisfiable.

\paragraph{Temporal injection filter.}
We tested this diagnosis by excluding injection events that occur before
normalised position~0.20 in the timeline.  The filter recovers 7 of the
9 focal seeds, raising the all-valid rate from 41/50 to 47/50.  This
confirms the mechanism: early-event contamination via the injection step
is the root cause of the collapse.

\subsection{Two-Regime Classification}\label{sec:two-regime}

The 9 failures decompose into two distinct regimes:

\begin{enumerate}[label=(\roman*),itemsep=4pt]
  \item \textbf{Search exploration failure (5/9 seeds).}\;
    Valid mid-arc alternatives exist that outscore the search-selected
    invalid arc.  Better arcs are present in the candidate space, but
    the greedy search does not find them.  The failure is in search
    coverage, not in the metric or the event pool.

  \item \textbf{Metric misalignment (4/9 seeds).}\;
    The invalid arc actually outscores the best valid alternative by
    approximately 0.098 in $Q$.  The quality metric genuinely prefers
    the early kinetic spike---the high-tension catastrophe followed by
    rapid consequences---over a structurally sound arc with proper
    development.  The failure is in the metric itself.
\end{enumerate}

\noindent
Despite this decomposition, both regimes share a single root cause:
premature turning-point anchoring via pool contamination.  In regime~(i),
the contamination blinds the search; in regime~(ii), the contamination
biases the metric.  The upstream mechanism is identical.


%% ═══════════════════════════════════════════════════════════════════
\section{Evolution as Pacing Control}\label{sec:evolution-pacing}
%% ═══════════════════════════════════════════════════════════════════

The Lorien system evolves agent profiles across simulation depth: after
each simulation run, an agent's personality parameters are updated based
on the events it experienced.  This section examines the interaction
between evolution and arc validity.

\subsection{The Quality--Validity Tradeoff}\label{sec:quality-validity}

We measure two metrics as a function of \emph{coalition size}~$c$, the
number of agents whose profiles are evolved (holding all others at
their default profiles):

\begin{itemize}[itemsep=3pt]
  \item \textbf{Mean $Q$}: the raw quality score, averaging tension,
    irony, and thematic coherence.
  \item \textbf{VA} (validity-adjusted score): $Q$ scaled by the
    fraction of arcs that are valid.
\end{itemize}

\noindent
The two metrics diverge systematically.  Mean $Q$ increases monotonically
with coalition size: evolving more agents produces richer, more dramatic
simulations.  But VA drops from its $c = 0$ value of 0.652 to a minimum
of 0.617 at $c = 3$ before partially recovering.  The all-valid rate
drops from 88\% at $c = 0$ (no evolution) to 64\% at $c = 6$ (full
evolution).

Evolution makes the simulation more interesting but makes the arcs
harder to extract.  Evolved agents produce higher-tension events that
cluster early in the timeline, contaminating candidate pools and
anchoring turning points prematurely---the same mechanism we identified
in Diana's collapse.

\subsection{The $\alpha$-Interpolation Experiment}%
\label{sec:alpha-interpolation}

To study the evolution effect at finer resolution, we interpolate between
default and evolved agent profiles using a blending parameter
$\alpha \in [0, 1]$:
\[
  \theta(\alpha) \;=\; (1 - \alpha)\,\theta_{\mathrm{default}}
                       \;+\; \alpha\,\theta_{\mathrm{evolved}}.
\]
For Thorne---the destabilising agent whose evolution most strongly
affects arc validity---VA is sharply peaked at $\alpha = 0.5$, where it
reaches 0.684 with a 95\% all-valid rate.  The optimum is driven
entirely by validity, not raw quality: $Q$ is essentially flat across
$\alpha$.  The peak at $\alpha = 0.5$ reflects a balance point where
Thorne is evolved enough to generate rich events but not so evolved that
his early-timeline catastrophes overwhelm the candidate pools.

\subsection{Factorial Decomposition}\label{sec:factorial}

A $2 \times 3$ factorial design (2 simulation depths $\times$ 3
evolution levels) decomposes the evolution effect into three orthogonal
components:

\begin{enumerate}[label=(\roman*),itemsep=4pt]
  \item \textbf{Amplifier effect}
    (depth~$D_0$, default $\to$ evolved):\;
    $+0.008$ mean $Q$, $-0.028$ VA.\;
    Evolution \emph{hurts} validity in fresh environments.

  \item \textbf{Degradation effect}
    ($D_0 \to D_2$, default profiles):\;
    $-0.007$ mean $Q$, $-0.009$ VA.\;
    Information depletion across simulation depth reduces quality.

  \item \textbf{Repair effect}
    (depth~$D_2$, default $\to$ evolved):\;
    $+0.016$ mean $Q$, $+0.013$ VA.\;
    Evolution \emph{helps} in depleted environments.
\end{enumerate}

\noindent
The repair effect exceeds the amplifier effect by 4~VA points.
Evolution is primarily a \emph{repair mechanism} for depleted
environments, not an improvement mechanism for fresh ones.  In a fresh
simulation ($D_0$), the event landscape is rich enough that the arc
extractor can find valid arcs without assistance; evolution merely adds
high-tension events that contaminate the pools.  In a depleted
simulation ($D_2$), the event landscape has been thinned by repeated
extraction; evolution replenishes the pool with new material, restoring
the development capacity that depletion removed.

\begin{remark}[Scaffolding dependence]\label{rem:scaffolding}
  The depleted canon provides structural scaffolding that constrains
  agent behaviour into extractable patterns.  Evolution \emph{needs}
  that scaffolding to work.  In a fresh environment, the agents have
  no history to constrain them, and evolved profiles produce unconstrained
  high-tension events that are structurally unmoored.  In a depleted
  environment, the accumulated narrative history channels evolved
  behaviour into patterns that the arc extractor can recognise and
  extract.  The scaffolding converts evolution from a destabilising
  amplifier into a targeted repair mechanism.
\end{remark}


%% ═══════════════════════════════════════════════════════════════════
\section{Connection to the Formal Theory}\label{sec:narrative-formal-connection}
%% ═══════════════════════════════════════════════════════════════════

Each of the three empirical findings maps directly to a construction or
theorem from the preceding chapters.  We make these connections
explicit.

\subsection{Phase Collapse and the Absorbing State Theorem}%
\label{sec:collapse-absorbing}

Diana's phase collapse (\cref{sec:phase-collapse}) is a concrete
instance of the prefix-constraint impossibility theorem
(\cref{thm:prefix-impossibility} in \cref{ch:absorbing-states}).  In
Diana's invalid arcs, the development-eligible count is exactly zero:
$\jdev = 0$.  The grammar's prefix requirement is $k = 1$.  Since
$\jdev = 0 < k = 1$, \cref{thm:prefix-impossibility} applies directly:
the greedy policy produces zero valid sequences.  This is not a
statistical tendency or a soft failure mode---it is an exact
impossibility, predicted by a counting argument, confirmed across all 9
invalid seeds without exception.

The theorem also explains \emph{why} the failure is categorical rather
than gradual.  The absorbing-state boundary at $\jdev = k$ is a sharp
threshold: below it, validity is exactly zero; above it, validity is
possible.  The bimodal separation in turning-point position
(\cref{sec:tp-anchoring})---zero overlap between valid and invalid
distributions---is the empirical signature of this sharp boundary.  Seeds
that contaminate Diana's pool with early pivots push $\jdev$ below $k$
and enter the impossibility zone.  Seeds that avoid contamination leave
$\jdev$ well above $k$ and land in the high-validity zone.  There is no
middle ground, exactly as the theorem predicts.

\subsection{Grammar Regularization and the Absorbing Ideal}%
\label{sec:regularization-ideal}

The grammar regularisation result (\cref{sec:grammar-regularizer})
illustrates why the absorbing ideal (\cref{ch:absorbing-ideal}) matters
practically.  The absorbing ideal characterises the algebraic boundary
between feasible and infeasible context elements: an extended context
element $\bar{C} = (\wstar, \dtotal, \dpre, \kappa)$ with $\kappa = 1$
and $\dpre < k$ belongs to the ideal, and no suffix can rescue it.

The grammar's development-beat requirement is the operational
enforcement of the condition $\dpre \ge k$.  When this requirement is
present, the search is forced to include development events, which
ensures that the selected arc's context element satisfies $\dpre \ge k$
and stays outside the absorbing ideal.  When the requirement is relaxed
to $\dpre \ge 0$, the search is free to select arcs whose context
elements fall inside the ideal.  The $Q$-metric, unconstrained by
structural requirements, gravitates toward these ideal elements because
they correspond to early-TP, consequence-heavy arcs that score well on
tension.

The strict grammar does not merely filter outputs; it constrains the
search to operate outside the absorbing ideal.  The rescore-only test
(\cref{sec:rescore-test}) confirms this interpretation: strict-invalid
arcs are not borderline cases that a permissive grammar could rescue.
They are deep inside the ideal---$\jdev = 0$, not $\jdev = k - 1$---and
no relaxation of grammatical constraints can supply the development
content that is entirely absent.

\subsection{Injection Contamination and the Validity Mirage}%
\label{sec:contamination-mirage}

The injection contamination mechanism (\cref{sec:injection-mechanism}) is
a specific pathway to the validity mirage (\cref{ch:mirage}).  The
system \emph{could} produce a valid arc with a different pivot---in 5 of
the 9 failures, valid alternatives exist that outscore the selected
arc.  But the greedy search locks onto the contaminating early pivot and
cannot escape.  From the system's perspective, the arc-extraction
succeeded: it found a turning point (the highest-weight event), assembled
a sequence around it, and reported the result.  From a structural
perspective, the result is broken.

This is precisely the mirage pattern: the system reports success on a
metric (turning-point selection, sequence assembly) while concealing a
structural failure (absent development, premature anchoring).  The
four-metric diagnostic checklist from \cref{sec:checklist}---pivot
preservation, fixed-pivot feasibility, semantic regret, and mirage
gap---would detect this failure immediately.  The contaminating early
pivot fails the fixed-pivot feasibility test (there are not enough
pre-pivot events to satisfy the grammar if we force the original pivot),
and the semantic regret relative to a properly positioned pivot is
catastrophic.

The two-regime classification (\cref{sec:two-regime}) further
illuminates the mirage.  In the search-exploration regime, valid arcs
exist but the search misses them: this is a mirage of search coverage,
where the system appears to have explored the candidate space but has in
fact been trapped by the contaminated pool.  In the metric-misalignment
regime, the $Q$-metric actively prefers the broken arc: this is a mirage
of metric alignment, where the quality score appears to validate the
selection while concealing its structural deficiency.

\bigskip
\noindent
Taken together, these three connections---absorbing states, the
absorbing ideal, and the validity mirage---demonstrate that the formal
theory is not an ex~post rationalisation.  The theory was built to
explain these specific failures, and it explains them exactly: not
approximately, not statistically, but as deductive consequences of the
algebraic structure.  The narrative origin story is, in this precise
sense, the theory's empirical foundation.

% ══════════════════════════════════════════════════════════════
%  Chapter 11 — Manifesto
% ══════════════════════════════════════════════════════════════
\chapter{Manifesto}\label{ch:manifesto}

This chapter distils the preceding theory, experiments, and case studies
into seven numbered principles.  Each principle is grounded in a specific
theorem or empirical result developed in the book; none are aspirational
slogans.  We follow the principles with a practitioner checklist---a
concrete gate that any long-context system with endogenous pivot
selection should pass before deployment---and close with an honest
accounting of what this book does \emph{not} claim.

% ══════════════════════════════════════════════════════════════
\section{Seven Principles}\label{sec:principles}
% ══════════════════════════════════════════════════════════════

% ──────────────────────────────────────────────────────────────
\subsection{Principle 1: Validity Is Not Semantics}
\label{sec:p1}
% ──────────────────────────────────────────────────────────────

Raw validity---does the output satisfy the grammar, the schema, the
constraint set?---is necessary but insufficient.  When the output's
meaning depends on an \emph{endogenous pivot} selected by $\argmax$
over the output itself, a valid output with a substituted pivot is
syntactically well-formed yet semantically different from the intended
output.  In our experiments, raw validity stays at $1.0$ while pivot
preservation drops to ${\sim}0.35$---a ${\approx}\,65\%$ silent substitution rate
(\cref{ch:mirage}).  Any system that reports only raw validity is hiding
potential semantic drift behind a perfect score.

\medskip\noindent\textbf{Ground truth.}\quad
The validity mirage experiments of \cref{ch:mirage} measure raw validity,
pivot preservation, and semantic regret on the same corpora.  The gap
between the first metric and the second two is the empirical basis for
this principle.

% ──────────────────────────────────────────────────────────────
\subsection{Principle 2: Endogenous Pivots Demand Pivot-Consistent
Metrics}
\label{sec:p2}
% ──────────────────────────────────────────────────────────────

If your system selects a distinguished element from within the solution---a
turning point, root cause, reference activity, schema anchor---you must
measure whether that element is preserved under compression, truncation,
and streaming.  Standard metrics (accuracy, validity rate, BLEU, ROUGE)
are oblivious to pivot identity; they evaluate the surface form without
checking which element was chosen as the structural linchpin.  The three
diagnostics introduced in \cref{ch:mirage}---\emph{pivot preservation},
\emph{fixed-pivot feasibility}, and \emph{semantic regret}---are the
minimum viable measurement framework for any system with endogenous
coupling.

% ──────────────────────────────────────────────────────────────
\subsection{Principle 3: Greedy Has No Guarantees Under Endogenous
Constraints}
\label{sec:p3}
% ──────────────────────────────────────────────────────────────

The feasible family under endogenous pivot selection violates both the
hereditary axiom and the exchange axiom
(\cref{prop:non-matroid}).  This structural violation
means that greedy algorithms---the default workhorse for constrained
selection in combinatorial optimisation---have \emph{zero} approximation
guarantees for endogenous pivot problems.

The absorbing-state theorem (\cref{ch:absorbing-states}) quantifies the
worst case: when $\jdev < k$, greedy produces zero valid sequences.
This is not a soft degradation; it is a hard structural impossibility.
The event graph enters an absorbing state from which no continuation
can reach acceptance, and the impossibility is detectable from the
prefix alone.

% ──────────────────────────────────────────────────────────────
\subsection{Principle 4: Compression Must Be Algebra-Preserving}
\label{sec:p4}
% ──────────────────────────────────────────────────────────────

Compression is the unique closure-breaking operation in the context
algebra: $13.3\%$ of naive compressions violate monoid closure, while
composition, pivot update, and split-at-point all have $0\%$ violation
rates (\cref{ch:absorbing-ideal}).  Contract-guarded compression---preserving $\dtotal \ge \min(\dtotal, k)$---reduces silent mirages
from $21.95\%$ to $0\%$ on real incident data (\cref{ch:absorbing-ideal}).

Treating compression as a free operation---``just drop some
tokens''---is a safety failure.  Every compression map must satisfy the
no-absorption contract, or it risks pushing an algebraically healthy
context element across the absorbing boundary into the ideal from which
no continuation can produce a faithful output.

% ──────────────────────────────────────────────────────────────
\subsection{Principle 5: Commitment Timing Is a First-Class Design
Dimension}
\label{sec:p5}
% ──────────────────────────────────────────────────────────────

Under commit-now streaming, $54.9\%$ of organic sequences fall into
oscillation traps (\cref{ch:streaming}).  Deferring commitment to
$f \approx 0.25$ recovers $80.1\%$ validity at $85\%$ of offline
quality.  The choice of \emph{when} to commit to a pivot is not an
implementation detail---it is a fundamental design parameter with a
$2\times$ impact on system reliability.

Every streaming system with endogenous pivot selection should explicitly
specify and justify its commitment policy.  The commitment fraction~$f$
interacts with the pivot arrival distribution; the experiments in
\cref{ch:streaming} show that the optimal $f$ varies by agent archetype,
and that committing too early is strictly worse than committing too late.

% ──────────────────────────────────────────────────────────────
\subsection{Principle 6: Constraints Regularise; Relaxing Them Enables
Exploitation}
\label{sec:p6}
% ──────────────────────────────────────────────────────────────

Strict grammar constraints achieve $88\%$ validity; relaxed (strictly
more permissive) constraints achieve $32\%$ (\cref{ch:narrative}).  The
constraints act as regularisers that prevent the optimisation metric from
exploiting degenerate solutions.  Removing constraints in the name of
``flexibility'' or ``generality'' opens the door to Goodhart collapse~\citep{goodhart1984problems}:
the metric is optimised, but the output is meaningless.

The single necessary constraint---minimum development beats $\ge 1$---accounts for $100\%$ of observed failures in the narrative extraction
experiments of \cref{ch:narrative}.  This is a concrete instance of a
general pattern: a small number of structural constraints do
disproportionate regularisation work, and their removal has catastrophic
(not graceful) consequences.

% ──────────────────────────────────────────────────────────────
\subsection{Principle 7: Measure and Report Mirage Gaps}
\label{sec:p7}
% ──────────────────────────────────────────────────────────────

For every system that performs constrained extraction, compression, or
streaming with endogenous pivots, report the \emph{mirage gap}:
\[
  \text{mirage gap}
  \;=\;
  \text{raw validity} - \text{pivot preservation}.
\]
A mirage gap of~$0$ means the system is semantically faithful: every
valid output preserves the intended pivot.  A mirage gap $> 0$ means
some fraction of ``valid'' outputs have silently substituted pivots.
This number should appear in every evaluation table alongside accuracy
and raw validity (\cref{ch:mirage}).

The mirage gap is a single scalar that summarises the severity of the
endogenous coupling problem for a given system configuration.  It is
cheap to compute (it requires only a pivot-identity check on outputs
already produced) and impossible to game without actually preserving
pivots.

% ══════════════════════════════════════════════════════════════
\section{Practitioner Checklist}\label{sec:checklist}
% ══════════════════════════════════════════════════════════════

Before deploying a long-context system with endogenous pivot selection,
verify that every item below is satisfied.  Each item maps to one or
more of the seven principles and to a specific chapter of this book.

\begin{enumerate}[label=\textbf{\arabic*.},itemsep=6pt]

  \item \textbf{Identify all endogenous pivots.}\quad
    List every element in your system that is selected by $\argmax$ (or
    any selection operator) over the output itself.  If no such element
    exists, the endogenous coupling problem does not apply.  If one or
    more exist, every subsequent item is mandatory.
    (\cref{ch:formal-problem})

  \item \textbf{Measure pivot preservation.}\quad
    Under your compression, truncation, and streaming policy, what
    fraction of outputs preserve the pivot that would have been selected
    under the offline, uncompressed pipeline?  Report this number.
    (\cref{ch:mirage})

  \item \textbf{Measure fixed-pivot feasibility.}\quad
    When forced to use the original (offline) pivot, can the system still
    produce a valid output?  A low fixed-pivot feasibility rate means the
    system cannot even \emph{express} the correct answer under its
    current constraints.
    (\cref{ch:mirage})

  \item \textbf{Measure semantic regret.}\quad
    What is the quality loss attributable to pivot substitution?  Semantic
    regret isolates the cost of choosing a different pivot from the cost
    of other compression artefacts.
    (\cref{ch:mirage})

  \item \textbf{Verify the no-absorption contract.}\quad
    Your compression policy must satisfy $\dtotal$-preservation:
    $\dtotal$ after compression must remain at least $\min(\dtotal, k)$.
    A single violation is sufficient to push a context element into the
    absorbing ideal.
    (\cref{ch:absorbing-ideal})

  \item \textbf{Specify an explicit commitment point.}\quad
    Your streaming policy must name a commitment fraction~$f$, justified
    by the pivot arrival distribution in your domain.  ``Commit
    immediately'' ($f = 0$) is a valid choice only if you can demonstrate
    that the pivot arrival distribution is concentrated at the start of
    the sequence.
    (\cref{ch:streaming})

  \item \textbf{Test constraints under relaxation.}\quad
    Relax each grammar or schema constraint independently and measure the
    change in validity and pivot preservation.  Constraints that cause a
    large validity drop when relaxed are acting as regularisers, not
    merely as validators.  Do not remove them.
    (\cref{ch:narrative})

  \item \textbf{Report the mirage gap.}\quad
    Every evaluation table should include the mirage gap (raw validity
    minus pivot preservation) alongside raw validity.  A mirage gap of
    zero is the target; a nonzero mirage gap is a quantified warning.
    (\cref{ch:mirage})

  \item \textbf{Produce a deterministic witness.}\quad
    Construct or identify at least one concrete instance---a specific
    input, a specific compression policy, a specific pivot---that
    demonstrates the mirage effect on your system.  Understand its
    mechanism: which operation broke pivot preservation, and why.  A
    system without a known witness has not been tested; it has merely
    been lucky.
    (\cref{ch:absorbing-states}, \cref{ch:mirage})

\end{enumerate}

% ══════════════════════════════════════════════════════════════
\section{What This Book Does Not Claim}\label{sec:limitations}
% ══════════════════════════════════════════════════════════════

Honesty about scope is not a weakness; it is a prerequisite for the
claims that remain.  We record the following limitations explicitly.

\begin{itemize}[itemsep=6pt]

  \item \textbf{The tropical semiring is not claimed to be unique or
    optimal.}\quad
    We do not claim that the $(\max, +)$ tropical semiring
    (\cref{ch:tropical-lift}) is the only or the best algebraic framework
    for endogenous pivot problems.  It is the one that naturally arises
    from the weight-comparison structure of $\argmax$-based pivot
    selection.  Other coupling structures (e.g., attention-based or
    learned pivots) may demand different semirings, and we regard the
    identification of such alternatives as open.

  \item \textbf{The commitment fraction $f = 0.25$ is not universally
    optimal.}\quad
    The value $f \approx 0.25$ is the empirical optimum for the
    narrative-extraction domain studied in \cref{ch:streaming}.  It
    depends on the pivot arrival distribution, which varies by domain,
    agent archetype, and event-graph topology.  We do not claim that
    $0.25$ transfers to other settings without re-estimation.

  \item \textbf{Contract-guarded compression does not eliminate all
    failure modes.}\quad
    The no-absorption contract (\cref{ch:absorbing-ideal}) eliminates the
    specific failure mode of silent pivot substitution caused by
    $\dtotal$-deficiency after compression.  It does not address failure
    modes arising from weight perturbation, causal-graph corruption, or
    adversarial input construction.  These are distinct problems that
    require distinct guarantees.

  \item \textbf{We have not conducted human evaluation studies.}\quad
    All quality metrics reported in this book are automated: pivot
    preservation, fixed-pivot feasibility, semantic regret, and raw
    validity are computed programmatically from system outputs and
    ground-truth pivot identities.  The gap between automated metrics and
    human judgement of narrative quality, causal correctness, or
    explanatory adequacy is an open empirical question.

  \item \textbf{The theory is developed on temporal DAGs with
    weight-based pivot selection.}\quad
    Every theorem and experiment in this book operates on directed
    acyclic event graphs where the pivot is selected by $\argmax$ over a
    scalar weight function.  Extension to other endogenous coupling
    structures---attention-based selection, learned pivot functions,
    multi-pivot systems, cyclic dependency graphs---is conjectured to
    exhibit analogous absorbing-state phenomena, but this conjecture is
    unproved.  We flag it as the most important direction for future work
    (\cref{ch:discussion}).

\end{itemize}

% ══════════════════════════════════════════════════════════════
%  Chapter 12 — Discussion and Future Work
% ══════════════════════════════════════════════════════════════
\chapter{Discussion and Future Work}\label{ch:discussion}

This book began with a single collapsed arc---Diana's---and traced the
failure through four layers of analysis: empirical observation, algebraic
formalisation, streaming extension, and unified theory.  This chapter
steps back to survey the landscape.  We first assemble the unified
picture (\cref{sec:unified-picture}), showing how the four constituent
papers tell a single coherent story.  We then catalogue research
directions (\cref{sec:research-directions}), grading each by its current
evidence level.  \Cref{sec:broader-connections} draws connections to
systems beyond narrative generation.  \Cref{sec:closing-remarks} offers
brief closing reflections.

% ══════════════════════════════════════════════════════════════
\section{The Unified Picture}\label{sec:unified-picture}
% ══════════════════════════════════════════════════════════════

The four papers that make up this book's empirical and theoretical spine
are not four independent contributions; they are four acts of a single
argument.  Each act addresses a different facet of the same structural
phenomenon---the validity mirage---and each builds on the conclusions of
its predecessors.

\paragraph{Paper~00: Narrative / Lorien~\citep{gaffney2026narrative} (Chapters~1--2).}
The endogenous pivot problem was discovered empirically through Diana's
phase collapse.  Across 50~seeds of a dinner-party simulation, Diana's
arc failed in 9~seeds with a uniform signature: zero development beats,
an extremely early turning point (median normalised position $0.13$),
and full candidate pools.  The failure was not caused by data scarcity
but by a structural interaction between protagonist-event injection and
endogenous pivot selection.  Grammar relaxation experiments confirmed
that the standard grammar could not be satisfied when the pool's
highest-weight events arrived too early.  This empirical observation
motivated every formal development that followed.

\paragraph{Paper~01: Absorbing States~\citep{gaffney2026absorbing} (Chapters~3--4).}
The absorbing-state analysis formalised the impossibility.  When $\jdev
< k$---that is, when the number of development-eligible events before
the argmax-selected pivot falls below the prefix requirement---greedy
search produces zero valid sequences.  The failure is not probabilistic;
it is structural.  \Cref{ch:taxonomy} classified failures into four
classes: Class~A (absorbing state), Class~B (pipeline coupling),
Class~C (commitment timing), and Class~D (assembly compression) and constructed a
hierarchy of solvers: greedy, enumerative, and TP-conditioned.  Each
level in the hierarchy trades computational cost for strictly greater
coverage of the failure space.

\paragraph{Paper~02: Streaming~\citep{gaffney2026streaming} (Chapter~8).}
The streaming analysis extended the theory to online settings, where
events arrive sequentially and labels must be emitted before the full
sequence is known.  The central result is that commit-now policies fall
into oscillation traps---sequences in which the running-max pivot
overtakes the committed pivot, leaving the policy in an absorbing state
from which no continuation is grammar-valid.  Empirically, such traps
affect 54.9\% of organically generated sequences.  Two solutions
emerged: deferred commitment (with patience parameter $f$) and tropical
streaming, which maintains a weight vector over all possible pre-pivot
development counts and defers role assignment until the vector stabilises.

\paragraph{Paper~03: Context Algebra~\citep{gaffney2026mirage} (Chapters~5, 6, 7, 9).}
The context algebra unified everything into a single algebraic
framework.  The context monoid $(\mathcal{C}, \opendo)$ compresses any
contiguous block of events into a triple $(\wstar, \dtotal, \dpre)$
that composes associatively.  The extended monoid $(\bar{\mathcal{C}},
\opcommit)$ adds a commitment flag $\kappa$ and reveals that
prefix-deficient elements form a left ideal---the absorbing ideal.
Compression is the unique closure-breaking operation: it is the only
elementary edit that can force a valid context into the absorbing ideal.
The \emph{validity mirage} is the quantitative gap between raw validity
(which remains at $1.0$ under compression) and pivot preservation (which
drops to $0.35$), confirming that standard metrics are structurally
blind to the failure.

\medskip
\noindent
\textbf{The unified thesis.}\;
Endogenous constraints---where a distinguished element is selected by
argmax over the solution---create structural failure modes that standard
validity metrics miss.  The context algebra provides the formal
framework to analyse, predict, and prevent these failures.  The absorbing
ideal explains \emph{why} failures are irrecoverable; the tropical lift
explains \emph{when} they become detectable; the streaming analysis
explains \emph{how often} they occur in practice; and the compression
contract explains \emph{what operations} must be guarded to avoid them.

% ══════════════════════════════════════════════════════════════
\section{Research Directions}\label{sec:research-directions}
% ══════════════════════════════════════════════════════════════

The following directions are ordered roughly by distance from the
current results.  Each is labelled with its evidence status.

% ──────────────────────────────────────────────────────────────
\subsection{Set-Valued Algebra / Context Semiring}
\label{sec:future-semiring}
\textsc{Status: Conjectured.}

\medskip
\noindent
The context monoid tracks a single best weight per slot.  When pivot
identity is uncertain---for example, when two focal events have nearly
equal weight---a single-pivot summary discards information that may be
decision-relevant.  A natural extension is to replace the scalar
$\wstar$ with a \emph{set} of candidate pivots, each annotated with its
own feasibility profile.  The resulting structure would be a context
\emph{semiring} rather than a monoid: addition corresponds to
maintaining the non-dominated frontier of pivot candidates, and
multiplication corresponds to block composition.

The tropical context (\cref{ch:tropical-lift}) already moves in this
direction.  The weight vector $W[0..k]$ records the best pivot weight
achievable with exactly $j$ pre-pivot development events, for each $j$.
A full set-valued semiring would generalise this to maintain the
non-dominated Pareto frontier over an arbitrary set of quality
dimensions, not just the development count.

The conjecture is that such a semiring admits an efficient parallel
reduction analogous to the $O(\log n)$ holographic tree, with the
set-valued composition reducing to a tropical matrix product.  No proof
or implementation exists at present.

% ──────────────────────────────────────────────────────────────
\subsection{Agent-Relative Tension Metrics}
\label{sec:future-tension}
\textsc{Status: Proposed.}

\medskip
\noindent
Diana's failures occurred partly because the composite quality metric
$Q$ valued kinetic events---catastrophes, confrontations,
revelations---over epistemic events---observations, realisations,
belief updates.  The $Q$-metric measures global tension: how much the
event changes the state of the world.  An agent-relative metric would
instead measure how much the event changes the \emph{focal agent's
beliefs}, regardless of the event's global salience.

Concretely, define the \emph{agent-relative tension} of event $e$ for
focal actor $a^{\star}$ as the KL divergence between $a^{\star}$'s
belief state before and after $e$.  Turning points would then be
selected by argmax over agent-relative tension rather than over global
kinetic weight.  This would naturally favour events that are
informationally significant to the focal agent, even if they are globally
unremarkable.

No implementation or evaluation exists.  The proposal is motivated by
the observation that Diana's highest-weight events (by the global
$Q$-metric) were protagonist-driven catastrophes that Diana merely
witnessed, while the events most relevant to Diana's own arc were
quieter observations that scored low on global tension.

% ──────────────────────────────────────────────────────────────
\subsection{Quality-Diversity Search via MAP-Elites}
\label{sec:future-map-elites}
\textsc{Status: Proposed.}

\medskip
\noindent
The current solver hierarchy explores the space of narrative arcs by
varying the turning-point candidate (enumerative solver) or by
conditioning on turning-point position (TP-conditioned solver).
Neither method systematically explores the full frontier of
turning-point position versus quality.

MAP-Elites~\citep{mouret2015illuminating} (multi-dimensional archive of phenotypic elites) offers a
principled alternative, drawing on the insight from novelty
search~\citep{lehman2011abandoning} that exploring diverse solutions
can outperform purely objective-driven optimisation.  The behaviour space is discretised into bins
indexed by normalised turning-point position.  Each bin stores the
highest-$Q$ arc whose turning point falls in that bin.  The archive is
populated by mutation and crossover over event-selection vectors.

The expected benefit is diagnostic: the MAP-Elites archive would reveal
whether metric misalignment (the tendency of $Q$ to favour early turning
points) is a \emph{local} phenomenon---affecting only a narrow band of
turning-point positions---or a \emph{global} one that distorts the
entire quality landscape.  If the archive shows high-$Q$ arcs at late
turning-point positions, the problem is local and can be addressed by
reweighting.  If the archive is empty at late positions, the problem is
structural and requires a metric redesign.

% ──────────────────────────────────────────────────────────────
\subsection{Adaptive Temporal Injection Thresholds}
\label{sec:future-adaptive-injection}
\textsc{Status: Partially Validated.}

\medskip
\noindent
The temporal injection filter---which excludes protagonist events
arriving before a normalised position threshold of $0.20$---recovered
7~out of 9 of Diana's failed arcs in the original experiments.  The
threshold $0.20$ was hand-tuned.

A more principled approach would derive the threshold from the pivot
arrival CDF developed in \cref{ch:streaming}.  Specifically, let
$F_{\tp}(t)$ denote the cumulative distribution of the running-max
pivot's arrival position.  The adaptive threshold is then
\[
  t^{\star} \;=\; \inf\bigl\{\,t : F_{\tp}(t) \ge 1 - \epsilon\,\bigr\},
\]
where $\epsilon$ is a tolerance parameter controlling how much early
pivot mass is acceptable.  This ties the injection filter directly to
the record-process analysis: the threshold adapts to the pool's
statistical properties rather than being fixed across all agents.

Partial validation comes from the observation that 7/9 recoveries at
$0.20$ is consistent with the record-process prediction that early
pivots concentrate in the first quintile.  A full validation would
require sweeping $\epsilon$ over a range of agent archetypes and
measuring the recovery rate as a function of the adaptive threshold.

% ──────────────────────────────────────────────────────────────
\subsection{Human Evaluation Gap}
\label{sec:future-human-eval}
\textsc{Status: Acknowledged Limitation.}

\medskip
\noindent
All quality metrics in this work are automated.  The composite
$Q$-score aggregates tension variance, peak tension, trajectory shape,
irony, significance, thematic coherence, and protagonist coverage.
Each component is computed from the event graph without human input.
No human evaluation studies have been conducted.

The correlation between automated $Q$-score and human narrative quality
judgement is unknown.  It is entirely possible that arcs judged
high-quality by the $Q$-metric are perceived as contrived or
incoherent by human readers, or conversely that arcs with low
$Q$-scores are compelling.  This is the single largest open question
for practical deployment of the theory.

A human evaluation study would require: (i)~a corpus of generated arcs
spanning the full range of $Q$-scores and turning-point positions;
(ii)~human raters scoring each arc on narrative coherence, emotional
engagement, and structural satisfaction; and (iii)~a correlation
analysis between human scores and automated metrics, broken down by
pivot preservation status.  The last point is critical: if human raters
reliably distinguish pivot-preserved arcs from mirage arcs, the
validity mirage has perceptual consequences.  If they do not, the
mirage is a formal property with no practical impact.

% ──────────────────────────────────────────────────────────────
\subsection{Extension to LLM Context Management}
\label{sec:future-llm-context}
\textsc{Status: Conjectured.}

\medskip
\noindent
The context algebra's results on compression and commitment timing may
apply directly to several problems in large language model (LLM)
infrastructure.

\paragraph{KV-cache eviction.}
When a transformer's key--value cache exceeds its budget, the system
must evict entries.  The retention sweep experiments in
\cref{ch:mirage} showed that naive eviction (removing low-attention
entries) can silently shift the semantic pivot of the context window.
The compression contract (\cref{ch:absorbing-ideal}) predicts that any
eviction policy lacking a pivot-preservation guard will eventually
produce mirage states.  The connection to attention sink methods (e.g.,
StreamingLLM's~\citep{xiao2023streamingllm} initial-token retention) is suggestive: the ``sink''
tokens may function as pivot anchors, and their retention may be a
special case of the no-absorption contract.

\paragraph{RAG chunk selection.}
Retrieval-augmented generation~\citep{lewis2020rag} selects chunks from a document store to
form the context for a query.  The selected chunks collectively
determine the answer's semantic pivot---the dominant theme or fact
around which the answer is organised.  Chunk selection is thus an
endogenous pivot problem: the ``best'' chunk depends on the answer,
which depends on which chunks are selected.  The context algebra
predicts that greedy chunk selection (top-$k$ by embedding similarity)
will produce mirage states when the highest-similarity chunk anchors
the answer to a subtopic that lacks sufficient supporting detail.

\paragraph{Multi-document summarisation.}
Summarising multiple documents under a token budget requires compressing
context.  The dominant theme of the summary---its pivot---is determined
by the compressed representation.  The algebra predicts that naive
compression (uniform truncation) will shift the pivot toward whichever
document's key sentences survive truncation, regardless of whether that
document is globally dominant.  Contract-guarded compression would
check, after each truncation step, whether the pivot has survived.

These connections are plausible but unproved beyond the KV-cache
experiments reported in \cref{ch:mirage}.

% ══════════════════════════════════════════════════════════════
\section{Connections to Broader Systems}\label{sec:broader-connections}
% ══════════════════════════════════════════════════════════════

The endogenous pivot problem is not specific to narrative generation.
Any system that selects a distinguished element by optimising over its
own output---including generative agent
architectures~\citep{park2023generative}---faces the same structural
coupling.  This section sketches
five domains where the theory applies and states what the algebra
predicts.

% ──────────────────────────────────────────────────────────────
\subsection{LLM Context Management}
\label{sec:conn-llm}

Modern large language models operate under strict context-window
budgets.  When the budget is exceeded, the system must compress:
KV-cache entries are evicted, attention windows are chunked, or
prefixes are summarised.  Each of these operations is a lossy
context reduction.

\paragraph{Endogenous pivot analogue.}
The semantic pivot of the context window is the token (or span of
tokens) that most strongly determines the model's next-token
distribution---the ``attention sink'' in the terminology of
StreamingLLM.  This pivot is endogenous: it depends on the full
context, yet compression modifies the context.

\paragraph{Theory predicts.}
The compression contract (\cref{ch:absorbing-ideal}) predicts that any
eviction policy that does not explicitly check pivot preservation will
produce mirage states: the model's output will remain fluent and
locally coherent (high raw validity) while the semantic anchor of the
generation has silently shifted (low pivot preservation).  Attention
sink methods that retain initial tokens may be understood as an
implicit pivot-preservation heuristic---retaining the tokens most
likely to serve as the global attention anchor.

% ──────────────────────────────────────────────────────────────
\subsection{Incident Triage and Root Cause Analysis}
\label{sec:conn-incident}

Incident investigation---whether in aviation (NTSB reports), software
(post-incident reviews), or industrial safety---requires identifying a
\emph{root cause} from a sequence of events.  The root cause is the
event that, if removed, would have prevented the incident.

\paragraph{Endogenous pivot analogue.}
The root cause is selected by argmax over causal contribution, which
depends on which events are included in the analysis.  When the event
log is truncated (e.g., limited to the final 60~minutes before the
incident), the root cause may shift to a proximate trigger that
happened to fall within the window, even though the true root cause
occurred hours earlier.

\paragraph{Theory predicts.}
The absorbing-state theorem predicts that truncation-induced pivot
shifts are irrecoverable: once the true root cause is outside the
analysis window, no amount of reasoning over the truncated log can
recover it.  The tropical context predicts that maintaining a ranked
list of candidate root causes (not just the top one) provides a natural
defence: if the second-best candidate is qualitatively different from
the best, the analysis is fragile and the window should be extended.

% ──────────────────────────────────────────────────────────────
\subsection{Process Mining}
\label{sec:conn-process-mining}

Process mining~\citep{vanderaalst2016process} discovers workflow models from event logs.  A central
task is \emph{reference activity selection}: choosing the activity that
anchors the alignment of all traces.

\paragraph{Endogenous pivot analogue.}
The reference activity is the process-mining analogue of the turning
point.  It is typically selected as the most frequent or most
informative activity---an argmax over the discovered model.  But the
model depends on how traces are aligned, which depends on the reference
activity.

\paragraph{Theory predicts.}
When event logs are sampled or filtered (e.g., retaining only the most
recent $n$ cases), the reference activity may shift to a different
activity that happens to dominate the filtered sample.  The context
algebra predicts that this shift is absorbing under committed
semantics: once downstream analyses are conditioned on the wrong
reference activity, no subsequent correction can recover the original
alignment without re-processing the full log.

% ──────────────────────────────────────────────────────────────
\subsection{Grammar-Constrained Decoding}
\label{sec:conn-grammar-decoding}

Grammar-constrained decoding systems---such as PICARD~\citep{scholak2021picard} (for SQL),
Outlines~\citep{willard2023outlines} (for structured JSON), and constrained beam search~\citep{gange2020argmax}---guide
language model generation to satisfy a formal grammar.

\paragraph{Endogenous pivot analogue.}
In many structured outputs, a single token or clause serves as the
semantic pivot: the \texttt{WHERE} clause in SQL, the root key in JSON,
the topic sentence in a paragraph.  This pivot is endogenous: it
emerges from the generation process and determines the meaning of
surrounding tokens.

\paragraph{Theory predicts.}
Grammar constraints ensure syntactic validity but cannot enforce
semantic coherence with respect to the pivot.  A grammar-valid SQL
query may satisfy all syntactic rules while selecting the wrong
\texttt{WHERE} predicate---a pivot shift that changes the query's
meaning without triggering a grammar violation.  The theory predicts
that grammar constraints are necessary but not sufficient: they must be
augmented with pivot-aware semantic checks.  This is precisely the
``grammar as regulariser, not validator'' principle from
\cref{ch:manifesto}.

% ──────────────────────────────────────────────────────────────
\subsection{Constrained Beam Search}
\label{sec:conn-beam-search}

Beam search with constraints maintains a set of $B$ partial hypotheses,
pruning at each step to the top-$B$ candidates.  When the output must
satisfy a structural constraint (e.g., balanced parentheses, valid XML),
the beam is further filtered by constraint satisfaction.

\paragraph{Endogenous pivot analogue.}
The semantic pivot of a partial hypothesis is the token or span that
most strongly determines the hypothesis's eventual meaning.  As the
beam evolves, the pivot of the top hypothesis may oscillate---exactly
the streaming oscillation trap of \cref{ch:streaming}.  When the beam
commits to a hypothesis (by pruning all alternatives), the pivot is
locked.

\paragraph{Theory predicts.}
The streaming analysis predicts that beam width $B$ controls the
effective patience parameter $f$: wider beams defer commitment longer
and therefore have lower trap rates.  The tropical context provides a
natural beam-scoring function: instead of scoring hypotheses by
log-likelihood alone, score them by their tropical feasibility
vector, preferring hypotheses that maintain feasibility at multiple
development thresholds.  This is a concrete instantiation of deferred
commitment in the beam-search setting.

% ══════════════════════════════════════════════════════════════
\section{Closing Remarks}\label{sec:closing-remarks}
% ══════════════════════════════════════════════════════════════

The validity mirage is not a bug in a specific system.  It is a
structural property of any system where interpretation depends on
endogenous selection---where the meaning of the output is determined by
an element that is itself selected from the output.  Such systems are
ubiquitous: narrative generators select turning points, incident
analysers select root causes, process miners select reference
activities, retrieval systems select anchor documents, and language
models select attention sinks.  In every case, standard validity
metrics ask ``does the output satisfy the constraints?''\ and miss the
deeper question: ``does the output mean what it was supposed to mean?''

The algebra developed in this book provides tools to see the mirage
(the pivot-preservation metric), measure it (the mirage gap between raw
validity and fixed-pivot feasibility), and prevent it (the compression
contract, the deferred commitment policy, the tropical streaming
algorithm).  The manifesto of \cref{ch:manifesto} distils these tools
into a practitioner's checklist.  The research directions catalogued in
\cref{sec:research-directions} give researchers a roadmap.

The central message is simple.  Validity is necessary but not
sufficient.  Compression is dangerous but manageable.  Commitment is
irreversible but deferrable.  And the algebra is there to tell you
which is which.


\appendix
% ══════════════════════════════════════════════════════════════
%  Appendices
% ══════════════════════════════════════════════════════════════

% ──────────────────────────────────────────────────────────────
\chapter{Notation}\label{app:notation}
% ──────────────────────────────────────────────────────────────

\Cref{tab:notation} collects every symbol used in this book, together
with a brief gloss and the chapter in which it is introduced.

\noindent
\begin{tabular}{@{} l p{8.2cm} l @{}}
\toprule
\textbf{Symbol} & \textbf{Meaning} & \textbf{Introduced} \\
\midrule

$G = (V, E, t, w, a)$
  & Event graph: vertices $V$, edges $E$, timestamp function $t$,
    weight function $w$, actor function $a$
  & \cref{ch:formal-problem} \\

$a^{\star}$
  & Focal actor: the agent whose narrative arc is being extracted
  & \cref{ch:formal-problem} \\

$P$
  & Candidate pool: the set of events involving the focal actor
  & \cref{ch:formal-problem} \\

$\tp(S)$
  & Endogenous turning point of event set $S$: the focal event with
    maximum weight in the selected subsequence
  & \cref{ch:formal-problem} \\

$A = (Q, \Sigma, \delta, q_0, F)$
  & Phase grammar DFA: states $Q$, alphabet $\Sigma$ (phase labels),
    transition function $\delta$, initial state $q_0$, accepting
    states~$F$
  & \cref{ch:formal-problem} \\

$k$
  & Prefix requirement: the minimum number of \textsc{development}
    beats before the turning point
  & \cref{ch:formal-problem} \\

$M$
  & Solver budget: the number of pivot candidates explored by the
    enumerative or TP-conditioned solver
  & \cref{ch:formal-problem} \\

$\jdev$
  & Development-eligible event count: the number of non-focal events
    whose timestamp precedes the argmax-selected pivot
  & \cref{ch:absorbing-states} \\

$C = (\wstar, \dtotal, \dpre)$
  & Context element: a triple summarising the structural state of a
    contiguous block of events
  & \cref{ch:context-algebra} \\

$\opendo$
  & Endogenous composition: the associative binary operation on
    context elements
  & \cref{ch:context-algebra} \\

$e = (-\infty, 0, 0)$
  & Identity element of the context monoid
  & \cref{ch:context-algebra} \\

$T = (W, \dtotal)$
  & Tropical context: a weight vector $W$ paired with a total
    development count
  & \cref{ch:tropical-lift} \\

$W[j]$
  & Best pivot weight achievable with exactly $j$ pre-pivot
    development events
  & \cref{ch:tropical-lift} \\

$\bar{C} = (\wstar, \dtotal, \dpre, \kappa)$
  & Extended context element: a context element augmented with a
    commitment flag
  & \cref{ch:absorbing-ideal} \\

$\opcommit$
  & Committed composition: the composition operation on extended
    context elements that respects commitment
  & \cref{ch:absorbing-ideal} \\

$\kappa$
  & Commitment flag: $\kappa = 0$ (uncommitted, pivot is provisional)
    or $\kappa = 1$ (committed, pivot is locked)
  & \cref{ch:absorbing-ideal} \\

$\absorb$
  & Absorbing predicate: $\dpre < k$, indicating that the context is
    in the absorbing ideal
  & \cref{ch:absorbing-ideal} \\

$\mu$
  & Compression map: a lossy reduction of the context that may shift
    the pivot
  & \cref{ch:absorbing-ideal} \\

$\tp_t$
  & Running-max pivot at step $t$: the highest-weight focal event
    seen in the first $t$ events of the stream
  & \cref{ch:streaming} \\

$\peff$
  & Effective pre-pivot capacity: the number of development-eligible
    events before the running-max pivot in a streaming setting
  & \cref{ch:streaming} \\

$f$
  & Patience parameter: the fraction of the sequence that the deferred
    commitment policy waits before locking the pivot
  & \cref{ch:streaming} \\

$\mingap$
  & Minimum inter-record gap: the smallest number of events between
    consecutive running-max pivot updates
  & \cref{ch:streaming} \\

PPR
  & Pivot preservation rate: the fraction of sequences in which the
    post-compression pivot matches the pre-compression pivot
  & \cref{ch:mirage} \\

FPF
  & Fixed-pivot feasibility: the fraction of sequences that remain
    grammar-valid when evaluated against the original (pre-compression)
    pivot
  & \cref{ch:mirage} \\

SR
  & Semantic regret: the quality loss attributable to pivot shift,
    measured as $Q_{\text{original}} - Q_{\text{shifted}}$
  & \cref{ch:mirage} \\

$Q$
  & Composite quality score: a weighted aggregate of tension variance,
    peak tension, trajectory shape, irony, significance, thematic
    coherence, and protagonist coverage
  & \cref{ch:narrative} \\

VA
  & Validity-adjusted score: a metric that interpolates between raw
    validity and pivot-preserving validity
  & \cref{ch:narrative} \\

$\alpha$
  & Interpolation parameter: controls the weight of pivot preservation
    in the validity-adjusted score
  & \cref{ch:narrative} \\
\bottomrule
\end{tabular}
\label{tab:notation}

% ──────────────────────────────────────────────────────────────
\chapter{Glossary}\label{app:glossary}
% ──────────────────────────────────────────────────────────────

\begin{description}[style=nextline, leftmargin=2.5cm, labelwidth=2.3cm, itemsep=6pt]

\item[Absorbing state]
A state of the product automaton from which no accepting state is
reachable.  Once entered, no continuation can produce a grammar-valid
sequence.

\item[Blast radius]
The number of downstream labels or decisions invalidated by a single
pivot shift.

\item[Candidate pool]
The set of events involving the focal actor that are available for
inclusion in the narrative arc.

\item[Commit-now policy]
A streaming policy that irrevocably assigns phase labels to each event
as it arrives, based on the current running-max pivot.

\item[Committed semantics]
The interpretation regime under which the pivot identity is locked and
cannot be revised by future events.

\item[Compression contract]
A pre/post condition pair that a compression map must satisfy to
guarantee that the pivot is preserved and the context does not enter the
absorbing ideal.

\item[Context element]
A triple $(\wstar, \dtotal, \dpre)$ that summarises the structural
state of a contiguous block of events: the best pivot weight, the total
development capacity, and the pre-pivot development capacity.

\item[Deferred commitment]
A streaming policy that withholds label assignment for the first
fraction~$f$ of the sequence, committing only after the pivot has
stabilised.

\item[Development-eligible]
An event is development-eligible if it is a non-focal event whose
timestamp precedes the current pivot.  Such events can serve as
\textsc{development} beats in the phase grammar.

\item[Endogenous pivot]
A distinguished element of the output that is selected by argmax over
the output itself, creating a circular dependency between selection and
interpretation.

\item[Enumerative solver]
A solver that iterates over the top-$M$ pivot candidates by weight,
attempting grammar-valid extraction for each, and returns the first
success.

\item[Event graph]
The directed graph $G = (V, E, t, w, a)$ encoding all events in the
simulation, with timestamps, weights, and actor assignments.

\item[Feasibility]
A sequence is feasible if it satisfies the phase grammar with at least
$k$ development beats before the turning point selected by the
endogenous pivot function.

\item[Fixed-pivot feasibility (FPF)]
The fraction of compressed sequences that remain grammar-valid when
evaluated against the pre-compression pivot.

\item[Focal actor]
The agent $a^{\star}$ whose narrative arc is being extracted from the
event graph.

\item[Grammar-aware classifier]
A classifier that uses the phase grammar's state to prune or reweight
events during arc extraction, treating the grammar as a regulariser
rather than a post-hoc validator.

\item[Mirage gap]
The quantitative difference between raw validity (which may be $1.0$)
and pivot-preserving metrics such as FPF or PPR.  A large mirage gap
indicates that standard metrics are hiding semantic failures.

\item[Monotonic DFA]
A DFA whose state transitions are monotonically ordered: once a phase
is exited, it cannot be re-entered.

\item[Oscillation trap]
A streaming state in which the running-max pivot has overtaken the
committed pivot, leaving the policy in an absorbing state from which
no grammar-valid continuation exists.

\item[Patience parameter]
The fraction $f \in [0, 1]$ of the sequence that the deferred
commitment policy processes before locking the pivot.

\item[Phase grammar]
A deterministic finite automaton that specifies the legal ordering of
narrative phases: \textsc{setup}, \textsc{development},
\textsc{turning\_point}, \textsc{resolution}.

\item[Pivot preservation (PPR)]
The fraction of sequences in which the post-operation pivot (after
compression, streaming, or other transformation) matches the
pre-operation pivot.

\item[Prefix requirement]
The parameter $k$ specifying the minimum number of \textsc{development}
beats that must precede the turning point for a sequence to be
grammar-valid.

\item[Record process]
The stochastic process formed by the running maximum of pivot weights
in a stream.  Each new record corresponds to a pivot update.

\item[Semantic regret]
The quality loss caused by a pivot shift: the difference in composite
$Q$-score between the arc built around the original pivot and the arc
built around the shifted pivot.

\item[Tropical context]
A pair $(W, \dtotal)$ where $W$ is a weight vector indexed by pre-pivot
development count.  The tropical composition rule is a shift-and-max
operation derived from the tropical semiring.

\item[Turning point]
The event selected by the endogenous pivot function---the single
highest-weight focal event in the chosen subsequence.

\item[Validity mirage]
The phenomenon in which standard validity metrics (grammar satisfaction,
constraint checks) report perfect scores while the semantic content of
the output has been silently corrupted by a pivot shift.

\item[Weight vector]
The vector $W[0..k]$ in a tropical context, where $W[j]$ is the best
pivot weight achievable with exactly $j$ pre-pivot development events.

\end{description}

% ──────────────────────────────────────────────────────────────
\chapter{Determinism Contract}\label{app:determinism}
% ──────────────────────────────────────────────────────────────

Every experimental result in this book is deterministically
reproducible.  This appendix documents the mechanisms that guarantee
reproducibility.

\section{Fixed Random Seeds}\label{sec:det-seeds}

Every random generator in the codebase takes an explicit seed parameter.
No experiment relies on a global random state or on the default seed.
Each test function creates its own \texttt{numpy.random.Generator}
instance initialised with a fixed seed, ensuring that the random state
of one test cannot contaminate another.

\begin{verbatim}
rng = numpy.random.default_rng(seed=42)
\end{verbatim}

\noindent
The seed value is recorded in the test file and in the output CSV.
Re-running the test with the same seed on the same platform produces
bit-identical results.

\section{Deterministic Tie-Breaking}\label{sec:det-tiebreak}

When two events have equal weight, the system breaks ties using a
three-key tuple:
\[
  (\,w,\;\; {-t},\;\; \texttt{event\_id}\,),
\]
where $w$ is the event weight, $t$ is the timestamp (negated so that
earlier events win ties), and \texttt{event\_id} is a unique integer
identifier assigned at event creation.  This ensures a strict total
order on events, eliminating any dependence on hash randomisation or
memory layout.

\section{Stable Sorting}\label{sec:det-sorting}

All sorting operations use a stable sort with an explicit key function.
No code path relies on the default comparison of complex objects or on
the stability guarantees of a particular sorting algorithm beyond the
Python specification that \texttt{list.sort()} and \texttt{sorted()} are
stable.  Where numpy sorting is used, \texttt{kind='stable'} is
specified explicitly.

\section{Isolated Random State}\label{sec:det-isolation}

Each test creates its own \texttt{numpy.random.Generator} with a fixed
seed.  Tests do not share random state, and the order in which tests
are executed does not affect their outputs.  This isolation is
enforced by convention: no test reads from or writes to a global
random state.

\section{Platform Independence}\label{sec:det-platform}

All results reported in this book were produced and verified on the
following platform:
\begin{itemize}[itemsep=2pt]
  \item Python~3.14
  \item NumPy (current stable release)
  \item macOS (Darwin)
\end{itemize}
No platform-specific numerical libraries (e.g., MKL, CUDA) are used
in the core experimental pipeline.  Floating-point operations follow
IEEE~754 double-precision semantics.  No experiment depends on
nondeterministic GPU operations.

% ──────────────────────────────────────────────────────────────
\chapter{Experiment-to-Artifact Map}\label{app:artifact-map}
% ──────────────────────────────────────────────────────────────

\Cref{tab:artifact-map} links every major experimental claim in this
book to the test file that produces it, the CSV artifact that records
the raw data, and the figure (if any) that visualises the result.  All
paths are relative to the repository root.

\noindent
\begin{small}
\begin{tabular}{@{} p{3.0cm} p{1.6cm} p{4.6cm} p{3.4cm} @{}}
\toprule
\textbf{Claim} & \textbf{Test} & \textbf{Artifact CSV} & \textbf{Figure} \\
\midrule

Tropical exactness (0~violations)
  & \texttt{test\_01}
  & \texttt{results/raw/test\_01\_exactness.csv}
  & \texttt{test\_01\_exactness\_heatmap.png} \\

Associativity (0~violations)
  & \texttt{test\_02}
  & \texttt{results/raw/test\_02\_associativity.csv}
  & \texttt{test\_02\_associativity\_heatmap.png} \\

Monoid subsumption
  & \texttt{test\_03}
  & \texttt{results/raw/test\_03\_monoid\_subsumption.csv}
  & \texttt{test\_03\_monoid\_richness.png} \\

Absorbing ideal (0~violations)
  & \texttt{test\_04}
  & \texttt{results/raw/test\_04\_absorbing\_ideal.csv}
  & \texttt{test\_04\_absorption\_escape\_rates.png} \\

Holographic exactness
  & \texttt{test\_05}
  & \texttt{results/raw/test\_05\_holographic\_exactness.csv}
  & --- \\

Incremental consistency
  & \texttt{test\_06}
  & \texttt{results/raw/test\_06\_incremental\_consistency.csv}
  & --- \\

Scaling $O(n \log n)$
  & \texttt{test\_07}
  & \texttt{results/raw/test\_07\_scaling.csv}
  & \texttt{test\_07\_scaling\_loglog.png} \\

Divergence (super-linear)
  & \texttt{test\_08}
  & \texttt{results/raw/test\_08\_divergence\_raw.csv}
  & \texttt{test\_08\_divergence\_loglog.png} \\

Record process
  & \texttt{test\_09}
  & \texttt{results/raw/test\_09\_record\_process\_raw.csv}
  & \texttt{test\_09\_records\_and\_min\_gap.png} \\

Tropical shield
  & \texttt{test\_10}
  & \texttt{results/raw/test\_10\_tropical\_shield\_raw.csv}
  & \texttt{test\_10\_tropical\_shield\_policy\_compare.png} \\

Compression mirage
  & \texttt{test\_11}
  & \texttt{results/raw/test\_11\_mirage\_raw.csv}
  & \texttt{test\_11\_validity\_mirage.png} \\

Deterministic witness
  & \texttt{test\_12}
  & \texttt{results/raw/test\_12\_deterministic\_witness.csv}
  & \texttt{test\_12\_deterministic\_witness.png} \\

Contract compression
  & \texttt{test\_13}
  & \texttt{results/raw/test\_13\_contract\_compression\_raw.csv}
  & \texttt{test\_13\_contract\_vs\_naive.png} \\

Margin correlation
  & \texttt{test\_14}
  & \texttt{results/raw/test\_14\_margin\_correlation\_raw.csv}
  & \texttt{test\_14\_margin\_quartiles.png} \\

Adaptive compression
  & \texttt{test\_15}
  & \texttt{results/raw/test\_15\_adaptive\_compression\_raw.csv}
  & \texttt{test\_15\_adaptive\_vs\_uniform.png} \\

Organic traps (54.9\%)
  & \texttt{test\_16}
  & \texttt{results/raw/test\_16\_organic\_traps.csv}
  & \texttt{test\_16\_trap\_rate\_heatmaps.png} \\

Tropical streaming
  & \texttt{test\_17}
  & \texttt{results/raw/test\_17\_tropical\_streaming.csv}
  & \texttt{test\_17\_policy\_validity\_bar.png} \\

Transition depth
  & \texttt{test\_18}
  & \texttt{results/raw/test\_18\_transition\_vector\_raw.csv}
  & \texttt{test\_18\_transition\_vector\_depth.png} \\

5-model blackbox
  & ---
  & \texttt{release/results/blackbox\_bf16\_5model/*.csv}
  & --- \\

KV-cache eviction
  & ---
  & \texttt{release/results/kv\_cache\_eviction\_llama31\_8b/*.csv}
  & --- \\

NTSB incidents
  & ---
  & \texttt{results/ntsb/*.csv}
  & --- \\
\bottomrule
\end{tabular}
\end{small}
\label{tab:artifact-map}


\backmatter
\bibliographystyle{plainnat}
\bibliography{refs}

\end{document}
