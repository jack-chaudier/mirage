% ══════════════════════════════════════════════════════════════
%  Preface
% ══════════════════════════════════════════════════════════════
\chapter*{Preface}
\addcontentsline{toc}{chapter}{Preface}
\markboth{Preface}{Preface}

% ──────────────────────────────────────────────────────────────
\section*{Why This Book Exists}
% ──────────────────────────────────────────────────────────────

\paragraph{The problem.}
Many sequence-generation systems must select a distinguished element---a
pivot, turning point, or structural anchor---by taking an argmax over the
very solution they are constructing.  This creates an \emph{endogenous
coupling}: the output depends on which pivot is chosen, yet the choice of
pivot depends on the output.  Standard validity metrics ask only ``does the
result satisfy the stated constraints?''\ and therefore miss a deeper
question: does the result \emph{mean what it was supposed to mean}?
Compression, truncation, and premature commitment can each produce outputs
that are technically valid---every constraint is satisfied, every grammar
rule obeyed---while the semantic content is quietly wrong.  The pivot has
shifted, the narrative has changed, and no constraint violation has been
raised.  We call this silent divergence the \emph{validity mirage}.

\paragraph{What we prove.}
This book develops a formal algebra of context that makes the mirage
visible and, in many cases, preventable.  The main results are:
\begin{enumerate}[label=(\alph*),itemsep=2pt]
  \item Under committed semantics, prefix-deficient states are
        \emph{absorbing}---no suffix can repair them
        (\cref{ch:absorbing-states,ch:absorbing-ideal}).
  \item The endogenous context monoid $(\mathcal{C},\opendo)$ is strictly
        associative, which enables $O(\log n)$ parallel reduction of
        context chains (\cref{ch:context-algebra}).
  \item Compression is the unique closure-breaking operation in the
        algebra: it is the only elementary edit that can force a valid
        context out of the feasibility set (\cref{ch:absorbing-ideal}).
  \item Under a commit-now policy, streaming oscillation traps affect
        54.9\% of organic sequences (\cref{ch:streaming}).
  \item Raw validity can remain at $1.0$ while pivot-preservation accuracy
        drops to $0.33$---the quantitative signature of the mirage
        (\cref{ch:mirage}).
\end{enumerate}

\noindent
These results consolidate and extend findings from four companion papers~\citep{gaffney2026absorbing,gaffney2026streaming,gaffney2026mirage,gaffney2026narrative}.

\paragraph{What practitioners should do differently.}
The algebra leads to concrete prescriptions.  First, measure
\emph{pivot preservation} and \emph{fixed-pivot feasibility} alongside raw
validity; without these metrics the mirage is invisible.  Second, use
\emph{contract-guarded compression}: wrap every lossy reduction in a
pre/post contract that checks whether the pivot has survived.  Third,
prefer \emph{deferred commitment} with a commitment fraction around
$f \approx 0.25$ over a commit-now policy when processing streams; the
streaming-trap analysis in \cref{ch:streaming} shows why early commitment
is so destructive.  Fourth, treat grammar constraints as
\emph{regularizers}---they shape the search space---rather than as
validators applied after the fact.  A constraint that fires only at the
end cannot prevent the absorbing states that form in the middle.

% ──────────────────────────────────────────────────────────────
\newpage
\section*{Map of the Ideas}
% ──────────────────────────────────────────────────────────────

The diagram below traces the logical dependencies among the book's central
concepts.  Each node names an idea and the chapter in which it is
developed; each arrow indicates that the source idea is a prerequisite for
the target.

\bigskip
\begin{center}
\begin{tikzpicture}[
    node distance=1.6cm and 2.4cm,
    every node/.style={
      draw,
      rounded corners=4pt,
      text width=3.6cm,
      minimum height=1.1cm,
      align=center,
      font=\small
    },
    arr/.style={
      -{Stealth[length=6pt]},
      thick,
      blue!60!black
    }
  ]

  % --- Row 1 ---
  \node (pivot)
    {Endogenous Pivots\\[2pt]\scriptsize Ch.\,1--2};

  % --- Row 2 ---
  \node (absorb) [below left=of pivot]
    {Absorbing States\\[2pt]\scriptsize Ch.\,3};
  \node (monoid) [below right=of pivot]
    {Context Monoid\\$(\mathcal{C},\opendo)$\\[2pt]\scriptsize Ch.\,5};

  % --- Row 3 ---
  \node (stream) [below=of absorb]
    {Streaming\\Commitment Traps\\[2pt]\scriptsize Ch.\,8};
  \node (ideal)  [below=of monoid]
    {Absorbing Ideal \&\\Compression Breaks\\Closure\\[2pt]\scriptsize Ch.\,7};

  % --- Row 4 ---
  \node (mirage) [below right=of stream, xshift=0.2cm]
    {Validity Mirage\\[2pt]\scriptsize Ch.\,9};

  % --- Arrows ---
  \draw[arr] (pivot)   -- (absorb);
  \draw[arr] (pivot)   -- (monoid);
  \draw[arr] (absorb)  -- (stream);
  \draw[arr] (absorb)  -- (ideal);
  \draw[arr] (monoid)  -- (ideal);
  \draw[arr] (stream)  -- (mirage);
  \draw[arr] (ideal)   -- (mirage);

\end{tikzpicture}
\end{center}
\bigskip

\noindent
Read top-to-bottom: endogenous pivots create the possibility of absorbing
states and motivate the context monoid.  Absorbing states feed into both
the streaming commitment analysis and the algebraic ideal theory.  The
context monoid likewise feeds the ideal theory.  Both the streaming traps
and the closure-breaking role of compression converge on the validity
mirage---the empirical demonstration that standard metrics hide real
failures.

% ──────────────────────────────────────────────────────────────
\section*{Reading Guide}
% ──────────────────────────────────────────────────────────────

The chapters divide naturally into three tracks.  Readers may follow a
single track or weave among them; the dependency diagram above shows which
material is prerequisite for which.

\paragraph{Mathematical core.}
\cref{ch:formal-problem} lays out the formal problem---solution spaces, pivot
functions, committed semantics.  \cref{ch:absorbing-states} proves the absorbing-state
theorem: once a prefix is deficient, no continuation can rescue it.
\cref{ch:context-algebra} introduces the context monoid $(\mathcal{C},\opendo)$,
establishes strict associativity, and derives the logarithmic parallel
reduction.  \cref{ch:tropical-lift} lifts the algebra into the tropical
semiring, connecting our discrete structures to optimization over
real-valued costs.  \cref{ch:absorbing-ideal} completes the algebraic picture by
characterizing the absorbing ideal and proving that compression is the
unique operation that breaks closure.

\paragraph{Empirical and experimental.}
\cref{ch:taxonomy} catalogues the failure modes that the theory predicts
and organizes them into a hierarchy, grounded in experiments on structured
generation tasks.  \cref{ch:streaming} measures oscillation traps in
streaming settings and quantifies the damage caused by premature
commitment.  \cref{ch:mirage} brings the threads together with the
headline experiment: raw validity held at $1.0$ while pivot preservation
collapsed, confirming the mirage across multiple domains.

\paragraph{Applied and motivational.}
\cref{ch:motivation} opens the book with concrete examples of silent
semantic failure---cases where a system ``passed all tests'' yet produced
the wrong answer.  \cref{ch:narrative} tells the narrative origin story of
the research programme, for readers who prefer to see where a theory came
from before studying its proofs.  \cref{ch:manifesto} distils the
practical upshot into a set of design principles.
\cref{ch:discussion} surveys open problems and directions for future work.

\bigskip
\noindent
We hope this book convinces you that validity is necessary but not
sufficient, and that the algebra developed here provides both the
language to articulate what is missing and the tools to fix it.

\vspace{1.5cm}
\begin{flushright}
  \textit{Jack Gaffney}\\
  \textit{February 2026}
\end{flushright}
